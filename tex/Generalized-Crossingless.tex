\documentclass{amsart}   
\usepackage{RepStyle} 
\usepackage{NatMacros} %********REPLACE WITH MilesMacros TO KILL RELIANCE ON \ifnum AND HAVE INCOMPLETE OUTPUT***********bleh
\begin{document}

\title{The Uneven-Height Two-Column Specht Modules of the Hecke Algebra of $S_n$}
\author{Miles Johnson \& Natalie Stewart}
\maketitle

\section{Introduction}
Let $S_{2n+r}$ be the symmetric group on $2n+r$ indices, let $\SH = \SH_{k,q}(S_{2n+r})$ be the corresponding Hecke algebra over field $k$ with parameter $q \in k^\times$, and let $\cbr{T_i}$ be the reflections generating $\SH$.
Let $\brk{m}_q = 1 + q + \dots + q^{m-1}$ be the $q$-number of $m$. 
Let $e$ be the smallest positive integer such that $\brk{e}_q = 0$, and set $e = \infty$ if no such integer exists.
Either $q = 1$ and $e$ is the characteristic of $k$, or $q \neq 1$ and $q$ is a primitive $e$th root of unity.

Let $S^{(n+r,n)'}$ be the Specht module corresponding to the young diagram with two columns with height difference $r$.
The purpose of this writing is to characterize this representation via an isomorphism with another representation of $\SH$.
\begin{definition}
  A \emph{generalized crossingless matching} on $2n+r$ indices with $r$ anchors is a partition of $\cbr{1,\dots,2n+r}$ into $n$ parts of size $2$ and $r$ of size 1 such that no two parts of size two``cross'', i.e. there are no parts $(a,a')$ and $(b,b')$ such that $a < b < a' < b'$, and no parts of size one are ``inside'' of a part of size two, i.e. there are no $c, (a,a')$ such that $a < c < a'$.
  We will call these arcs and anchors, respectively.
  Then, define $W^r_{2n+r}$ to be the $k$-vector space with basis the set of generalized crossingless matchings on $2n+r$ indices with $r$ anchors.

  In order for this to be a $\SH$-module, endow this with the action given by Figure \ref{Action}; if this involves no anchors, act as in $W_{2n}^0$; if it involves one anchor, deform to another generalized crossingless matching and scale by $q^{1/2}$, and otherwise scale by 0.
\end{definition}

\begin{figure}
  \[
    \GeneralizedAction{6}{1/4,2/3}{2}{5/1,6/2}{1}{4}{2/3, 4/5}{1/1,6/2}{q^{1/2}}
    \hspace{50pt}
    \GeneralizedZeroAction{6}{1/4,2/3}{2}{5/1,6/2}{1}{5}
  \]
  \caption{Illustration of the actions $(1 + T_4)w_{\abs{W^2_6}}$ and $(1 + T_2)w_{\abs{W^2_6}}$ in $W^2_6$.
    In general, we act on basis elements away from anchors as we did for $W$, at one anchor we act by deforming and scaling by $q^{1/2}$, and at two anchors we send the element to zero.}
  \label{Action}
\end{figure}

Let the length of an arc $(i,j)$ be $l(i,j) := j - i + 1$.
Note that the crossingless matchings can all be identified with a list of $n$ integers describing the lengths of the arcs from left to right;
using this, we may order the crossingless matchings with 0 hooks in increasing lexicographical order in order to obtain an order on the subbasis containing a particular set of anchors;
let the basis be ordered first by the position of the anchors in increasing lexicographical order, then increasing for the matchings between each anchor.
Let this basis be $\cbr{w_i}$.
This basis is illustrated for $W_{5}^1$ in Figure \ref{S5 Basis}. 

\begin{figure}
  \def\cbasisspacing{5mm}
  $\cbr{
    \begin{gathered}
      \GeneralizedMatching{5}{2/3, 4/5}{1}{1/1}{3/4}, \hspace{\cbasisspacing}
      \GeneralizedMatching{5}{2/5, 3/4}{1}{1/1}{3/4}, \hspace{\cbasisspacing}
      \GeneralizedMatching{5}{1/2, 4/5}{1}{3/1}{3/4}, \hspace{\cbasisspacing}
      \GeneralizedMatching{5}{1/2, 3/4}{1}{5/1}{3/4}, \hspace{\cbasisspacing}
      \GeneralizedMatching{5}{1/4, 2/3}{1}{5/1}{3/4}, \hspace{\cbasisspacing}
     \end{gathered}}$ 
    \caption{The basis for $W_5^1$.}
  \label{S5 Basis}
\end{figure} 

We will prove that $W := W_{2n+r}^r$ and $S := S^{(n+r,n)'}$ are isomorphic as representations in the case that $\SH$ is semisimple.

\newpage
\section{Correspondence} 
We can now begin by proving that $W$ is irreducible;
then $W \simeq S^\lambda$ for some partition $\lambda$ of $2n + r$, and we may use branching rules to determne $W$.
\begin{lemma}
  Every basis vector in $W_{2n + r}^r$ is cyclic.
\end{lemma}
\begin{proof}
  We have already proven this in the $r = 0$ case, so suppose that $r > 0$.

    Note that, between anchors $a<a'$ having no arc $b$ with $a < b < a'$, the $W_{a'-a}^0$ case allows us to generate the vector with all length-2 arcs between $a,a'$ and identical arcs/anchors outside of this sub-matching.\footnote{At the ends, we apply the $W_a^0$ case or the $W_{2n + r - a}^{0}$ case in the same way for the first $a$ or last $2n + r - a$ indices.}

  Applying this between each arc gives us a vector with length-2 arcs and anchors, and we may use the appropriate $(1+T_i)$ to move anchors to any positions, and the reverse process from above to generate the correct matchings between arcs and generate any other basis vector.
\end{proof}

\begin{proposition}
  \;

  \begin{enumerate}[label={(\roman*)}]
    \item The representation $W_{2+r}^r$ is reducible iff $e \mid r+2$.
    \item When $n \neq 1$ and $e > n+1$, the representation $W_{2n + r}^r$ is irreducible.
  \end{enumerate}
\end{proposition}
\begin{proof}
  \def\sectionsep{7pt}
  \textbf{(i)}
  Note that $\ima (1 + T_i)$ is 1-dimensional for each $i$, so it is equivalent that 
  \[K := \bigcap_{i=1}^{r+1} \ker(1 + T_i) = \ker \bigoplus_{i=1}^{r+1} (1 + T_i)\] 
  is trivial via the lemma.
  The transformation $\bigoplus (1 + T_i)$ is a linear operator on $W_{2 + r}^r$ given by the following matrix:
  \newcommand*\bigzero{\makebox(0,15){\text{\Huge0}}}
  \newcommand*\bigzerotwo{\makebox(-20,15){\text{\Huge0}}}
  \[
    A_{r + 1} = 
    \begin{bmatrix}
      \begin{matrix}
      q + 1 & q^{1/2} & \\
      q^{1/2} & q + 1 & q^{1/2}\\
      & q^{1/2} & q + 1 & q^{1/2}\\
      &  & \ddots & \ddots
      \end{matrix}
      &   \bigzero\\
      \bigzerotwo & \begin{matrix}
        q^{1/2} & q+1 & q^{1/2}\\
        &  q^{1/2} & q + 1
       \end{matrix}
    \end{bmatrix}
  \].
  Hence $K$ is trivial iff the determinant $\det A_{r + 1} = \brk{r + 2}_q$ is 0, or equivalently iff $e \mid r + 2$.

  \iffalse % Previous proof, kept for archival purposes
  \vspace{\sectionsep}
  \textbf{(ii)}
  We have already proven this for $r = 0$ and $n = 0$, so suppose $r > 0$ and $n > 1$.
  We will break into case work on $r$;
  suppose first that $r = 1$.
  Then, it is easy to verify that $W_{2n + 1}^1 \simeq \Res W_{2n + 1}^0$, which we have already proven irreducible.
  We may henceforth assume that $r > 1$.

  We will prove the equivalent condition that each vector in $w \in W\backslash \cbr 0$ is \emph{cyclic}, i.e. $\SH w = W$.
  This will proceed in two steps: we will make sure a particular basis vector is represented with the earliest possible position of the second to last anchor $a_{r-1}$, then we will annihilate all other vectors in order to generate a basis vector.
  Then, by the lemma, we will have proven that $w$ is cyclic.

  Overall, we will use induction on $2n + r$;
  this is easily shown via identification with the sign or trivial representation when $2n + r = 2$, so assume that it is true for all $W_{2m + s}^s$ with $2m + s < 2n + r$.

  Let $U_{x_{r-1}}$ be the subspace of $W$ containing at most one anchor at positions $i > x_{r-1}$.
  Order these in increasing order;
  let $U := U_{a_{r-1}}$ be the first of these into which $w$ projects to a nonzero vector.
  Then, by our inductive hypothesis, we may use only actions $T_i$ with $i < a_{r-1}$ to generate a vector $w'$ which projects to a nonzero element in $U$ for which all basis elements represented have anchors $1,\dots,r-1$ and all length-2 arcs at indices $i \leq a_{r-1}$.

  Define the element
  \[
    h = (1 + T_{n + r - 1})(1 + T_{n+r-3})(1 + T_{n+r-2})(1 + T_{n + r - 4})\dots(1 + T_{a_{r-1} +1})(1 + T_{a_{r-1}+2}).
  \]
  Note that $h$ carries a basis element to a multiples of another basis element, and a basis element is in $\ker \, h$ iff it is not in $U$.
  Hence $hw'$ is a nonzero multiple of a basis element, giving $w$ cyclic.
  \fi

  \vspace{\sectionsep}
  \textbf{(ii)}
  We will prove the equivalent condition that each vector in $w \in W\backslash \cbr 0$ is \emph{cyclic}, i.e. $\SH w = W$.
  We will break into case work on $r$;
  suppose first that $r = 1$.
  Then, it is easy to verify that $W_{2n + 1}^1 \simeq \Res W_{2n + 1}^0$, which we have already proven irreducible.
  We may henceforth assume that $r > 1$.
  
  Overall, we will use induction on $2n + r$;
  this is easily shown via identification with the sign or trivial representation when $2n + r = 2$, so assume that it is true for all $W_{2m + s}^s$ with $2m + s = 2n + r - 2$.
  
  The proof will proceed in two steps: first we will make sure a particular basis vector is represented with the earliest possible position of the last anchor $a_r$, then we will use this to generate a nonzero vector representing only vectors with a certain collection of anchors, using the inductive hypothesis to prove that $w$ is cyclic.
  
  \vspace{\sectionsep}
  \textit{Step 1.}
  Let $U_{x_r}$ be the subspace of $W$ containing only anchors at positions $i \leq x_r$.
  Order these in increasing order;
  let $U := U_{a_r}$ be the first of these into which $w$ projects to a nonzero vector.
  If $a_r = n + r$, then $w$ only represents vectors containing anchor $r$;
  then, we may use the inductive hypothesis on the first $n + r - 1$ indices to yield a basis vector, and we are done.

  Henceforth assume $a_r < n + r$.
  Then, by our inductive hypothesis, we may use only actions $T_i$ with $i < a_r$ to generate a vector $w'$ which projects to a vector in $U$ representing basis elements with anchors $1,2,\dots,r$ and all length-2 arcs at indices $r < i \leq a_r$. 

  Now, recall that $\Res_{\SH(S_{2n + r - a_r-1})}^{\SH(S_{2n + r - a_r})} \; W_{2n + r - a_r}^0$ is irreducible;
  hence we may only use actions $T_i$ with $i > a_r+1$ to generate a vector $w''$ which projects to the basis vector $U$ containing anchors $1,\dots,r$ and is otherwise all length-2 arcs.\footnote{Any action by $T_i$ with $i > a_r + 1$ sends basis vectors outside of $U$ to 0 or nonzero vectors outside of $U$, as they cannot generate a vector which doesn't have an anchor in some position $j > a_r$.}

  \vspace{\sectionsep}
  \textit{Step 2.}
  Define an element $h \in \SH$ by
  \[
    h := (1 + T_{n + r - 2})(1 + T_{n+r-1})(1 + T_{n+r-3})(1 + T_{n+r-2})\dots(1 + T_{r+1})(1 + T_{r+2})
  \]
  Then, every basis vector represented in $hw'' \neq 0$ contains anchors $1,\dots,r-1$.
  This is illustrated in Figure \ref{h}.

  Let $U'$ be the subspace of $W$ having anchors $1,\dots,r-1$.
  Note that $hw'' \in U'$.
  Further, $U' \simeq W_{2n + 1}^1$ as vector spaces, and every action in $W_{2n + 1}^1$ is reflected by an action of $\SH$ on $W_{2n + r}^r$.
  Since $r > 1$, we may use the inductive hypothesis to act on indices $i \geq r$ and generate a basis vector, giving $w$ cyclic.
  
  \begin{figure}
    \[
      \GeneralizedNAction{6}{1/2,4/5}{2}{3/1, 6/2}{3/4}{3/1, 2/2}{1/4, 2/3}{5/1, 6/2}{q}
      \hspace{50pt}
      \GeneralizedNAction{6}{1/2,4/5}{2}{3/1, 6/2}{3/4}{4/1, 3/2}{1/2, 3/4}{5/1, 6/2}{x}
    \]
  \caption{
    Demonstration of how the transformation $(1 + T_{i+1})(1 + T_i)$ ``moves'' anchors from positions $i,i+1$ to position $i+3$, with constant $x = q^{1/2}(q+1)$.
    Iterating this across all elements between $r+1$ and $2n + r$ via $h$ concentrates all anchors at the beginning or end; 
    if there are at least two anchors in $w_j$ after $r+1$, then we must act on two anchors eventually, giving $w_j \in \ker \, h$.
  }\label{h}
\end{figure}
\end{proof}

\begin{corollary}
  Other than $W_{3}^1$, the representation $W_{2n + r}^r$ is irreducible when $e > n + 1$.
\end{corollary}

The next piece in our puzzle is to characterize the restrictions of $W$ to $\SH' := \SH_{k,q}(S_{2n + r - 1}) \subset \SH$.
Recall that, when $r,n>0$ and $\SH$ is semisimple,  $\Res S^{(n+r,n)'} \simeq S^{(n+r-1,n)'} \oplus S^{(n+r,n-1)'}$.
Further, note that $S^{(n+r,n)'}$ is the unique irreducible having this restriction.

Next, note that we have already proven the correspondence for $W_{2n}^0$;
for $W_{0 + r}^{r}$, this is the sign representation, which is given correctly by $S^{(r)}$.
Hence, pending information on restrictions, we may prove this via induction on $2n + r$.

\begin{proposition}
  Suppose that $n,r > 0$ and $\SH$ is semisimple.
  Then, $\Res \; W_{2n + r}^r \simeq W_{2n + r - 1}^{r-1} \oplus W_{2n + r - 1}^{r + 1}$.
\end{proposition}
\begin{proof}
  Note that we may identify the subrepresentation of $\Res \; W_{2n + r}^r$ having anchor $n$ with $W_{2n + r - 1}^{r-1}$.
  By semisimplicity, it is sufficient to prove that $U := \Res \; W_{2n + r}^r / W_{2n + r - 1}^{r-1}$ is isomorphic to $W_{2n + r - 1}^{r +1}$.

  Let $\phi:U \rightarrow W_{2n + r - 1}^{r + 1}$ be the $k$-linear map which regards the arc $(i,2n + r)$ in $U$ as an anchor at $i$ in $W_{2n + r - 1}^{r + 1}$.
  It is not hard to verify that this is a well-defined isomorphism of vector spaces, so we must show that it is $\SH$-linear.

  Given a basis vector $w_j$ with arc $(i,2n + r)$, $\phi$ is clearly compatible with $T_{i'}$ with $i' \neq i,i-1$.
  Further, it's easy to verify that $\phi$ is compatible with $T_{i}$ and $T_{i-1}$, as actions on one anchor were designed for this deformation.
  When there are anchors $(i,i+1)$, then $\phi(T_iw_j) = T_i\phi(w_j) = 0$, and similar for $T_{i -1}$.
  Hence $\phi$ is an isomorphism of representations, and the statement is proven.
\end{proof}

\begin{corollary}
  When $\SH$ is semisimple, $W_{2n + r}^r \simeq S^{(n + r,r)'}$.
\end{corollary}
\begin{proof}
  We may argue by induction on $2n + r$, knowing that we have proven the base case $2n + r = 2$.
  Assume that we have proven the isomorphism for all $W_{2n + s}^{s}$ with $2n + s = 2n + r - 2$.
  We have proven the $n = 0$ and $r = 0$ cases already, so assume $n,r > 0$.
  
  Then, $W_{2n + r}^{r}$ is the unique irreducible representation of $\SH$ having restriction $S^{(n+r-1,n)'} \oplus S^{(n + r,n-1)'}$, implying the desired isomorphism.
\end{proof}

\end{document}

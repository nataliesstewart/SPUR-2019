\documentclass{amsart}   

\usepackage{NatMacros}
\usepackage{RepStyle}

\begin{document}
	
\title{A discussion of the intersection of the kernels of each $(1+T_i)$ acting on $M_{2n+r}^r$}
\author{Miles Johnson \& Natalie Stewart}

\maketitle


\section{Introduction} 

Let $\{T_i\}$ be the transpositions generating the Hecke algebra $\SH_{2n+r}(q)$. Let $M_{2n+r}^r$ be the generalized crossingless matchings representation with $2n+r$ nodes, $r$ of which are anchors. Fix the standard basis; we will refer to no other basis in this document. Here we characterize the intersection of the kernels of each $(1+T_i)$, a subrepresentation of $M_{2n+r}^r$. I claim this intersection is trivial for $e>n+r+1$ and one dimensional for $e=n+r+1$.

For compactness, in this document I use $\sim $ to denote "proportional to". For convenience, define $M_0^0$ to be the zero representation.

 \section{restricting the kernel}


\begin{definition}
	
  Fix some basis element $\psi\in M_{2n+r}^r$. For $1 \leq a,b \leq 2n + r$ define $\psi(a):=b$ iff $a$ and $b$ are matched in $\psi$, $\psi(a):=a$ if $a$ is an anchor in $\psi$. Given that $\psi$ has $r'$ anchors in the range $a,..,b$, define a \textbf{sub-matching} $\psi(a,b)$ of $\psi$ to be the basis element $\sigma\in M_{b-a+1}^{r'}$ specified by $\sigma(i)=\psi(i+a-1)-a+1$. This sub-matching is defined for $a<b$ when $\psi(i)\in \{a,a+1,...,b\}$ for all $i\in \{a,a+1,...,b\}$. See \ref{sub-matching example}.
  
  If $b\geq a$ and nodes $1\leq a,...,b\leq 2n+r$, $\psi(a,b)$ will always refer to those nodes and they're matchings, though it may not be a sub-matching unless specified. For any other $a,b$, we define $\psi(a,b)$ to be an element of the representation with no nodes.
  
  \label{sub-matchings}
\end{definition}

\begin{figure} 
	\def\cbasisspacing{5mm}
	$\cbr{
		\begin{gathered}
		\Matching{6}{1/2, 3/6, 4/5}, \hspace{\cbasisspacing}
		\Matching{4}{1/4, 2/3}
		\end{gathered}}$
	\caption{$\psi\in M_6^0$ is pictured on the left, $\sigma\in M_4^0$ is pictured on the right. $\psi(3,6)=\sigma$. $\psi(2,5)$ is not a sub-matching.}
	\label{sub-matching example}
\end{figure}

Define the rainbow element $R\in M_{2n+r}^r$ to be the basis element specified by $R(i)=2n+2r-i+1$ for $i>r$, $R(i)=i$ for $i\leq r$. In other words, the basis element with all anchors to the left then a rainbow.

\begin{proposition}
	Let $w$ be an arbitrary vector in $M_{2n+r}^r$. I claim that if $w\in \cap\text{ker}(1+T_i)$, the coordinate $c$ of the rainbow element $R$ in $w$ is nonzero.
	
	\label{rainbow nonzero}
\end{proposition}	

\begin{proof}
  Let $Y$ be the set of basis elements with nonzero coordinate in $w$. Let $z$ be the greatest integer such that there exists $y\in Y$ where $y(1)-1=...=y(z)-z=0$, and let $U\subset Y$ be the set of such $y$. In other words, $U$ is the set of basis elements in $Y$ which have the most anchors to the far left.
  
  Suppose $z<r$. Then for each $y\in U$ there exists a minimal $i_y>z+2$ such that $y(i_y)=0$. In other words, $i_y$ is the position of the next leftmost anchor in $y$.  Fix $\tilde{y}$ such that $i_{\tilde{y}}\leq i_y$ for all $y$. Then I claim the basis element $y':=q^{-1/2}(1+T_{i_{\tilde{y}}-1})\tilde{y}$ has nonzero coordinate in $(1+T_{i_{\tilde{y}}-1})w$, implying $w\not\in \cap\text{ker}(1+T_i)$. To see this, we can show that $\tilde{y}$ is the only basis element in $Y$ such that $(1+T_{i_{\tilde{y}}-1})\tilde{y}\sim y'$. $y'$ still has $z$ anchors on the left, and $i_{y'}<i_{\tilde{y}}$, so $y'\not\in Y$. If $x\in Y,\not\in U$, the basis element $x'$ proportional to $(1+T_{i_{\tilde{y}}-1})x$ will have $k$ anchors at the far left only if the next anchor is at a position $i_{x'}>i_{\tilde{y}}$, so it cannot be $y'$. If $x\in U$, $x'$ will have anchor at $i_{y'}$ if and only if $i_x=i_{\tilde{y}}$ and $x(i_{\tilde{y}})=\tilde{y}(i_{\tilde{y}})$. Since this is the only match altered by action of $(1+T_{i_{\tilde{y}}-1})$ on $x$, if $(1+T_{i_{\tilde{y}}-1})x\sim y'$ this implies $x=\tilde{y}$. So if $z<r$ $w$ is not in the desired kernel.
  
  Suppose $z=r$ but $R\not\in U$ (so $R\not\in Y$). Let us define a sequence of subsets of $U$ in the following way: $U_0:=U$, $U_{i+1}:=\{u\in U_i| u(r+i+1)=2n+2r-i+1 \}$. Since $R\not\in U$, $\exists t<n-1$ such that $U_{t+1}=\emptyset$. Choose $\tilde{u}\in U_t$ such that $\tilde{u}(r+t+1)\geq u(r+t+1)$ for all $u\in U_t$. Consider the basis element $u':=q^{-1/2}(1+T_{\tilde{u}(r+t+1)})\tilde{u}$. I claim that $\tilde{u}$ is the only element in $Y$ such that $(1+T_{\tilde{u}(r+t+1)})\tilde{u}\sim u'$, again implying that $w$ is not in the desired kernel. $u'$ still has $r$ anchors on the left, $u'(r+i)=2n+2r-i+2$, $1\leq i\leq t$, and $u'(r+t+1)>\tilde{u}(r+t+1)$, so $u'\not\in Y$. If $x\in Y,\not\in U$,  the basis element $x'$ proportional to $(1+T_{\tilde{u}(r+t+1)})x$ will have $r$ leftmost anchors only if $x'(r+t+1)<\tilde{u}(r+t+1)$, so $x'\not=u'$. Similarly, if $x\in U,\not\in U_t$, the basis element $x'$ will have the property $x'(r+i)=2n+2r-i+2$ for all $i\leq t$ only if $x'(r+t+1)<\tilde{u}(r+t+1)$, so $x'\not=u'$. If $x\in U_t$, $x'(r+t+1)=u'(r+t+1)$ if and only if $x(r+t+1)=\tilde{u}(r+t+1)$ and $x(x(r+t+1)+1)=\tilde{u}(\tilde{u}(r+t+1)+1)$ (since $u'\not \in Y$). These are the only matches altered by the action $(1+T_{\tilde{u}(r+t+1)})$, so this implies $x=\tilde{u}$. Thus we have proved that if $R\not\in Y$, $w$ is not in the kernel.
	
\end{proof}

Given a rainbow element $R$, define the basis elements $R_{R,i},R_{L,i}$ to be those where you move the middle hump across $i$ lines to the right or left, respectively. Examples are pictured in \ref{shifted rainbow}. Formally, $R_{R,i}:=q^{-i/2}(1+T_{r+n+i})...(1+T_{r+n+1})R$, $R_{L,i}:=q^{-i/2}(1+T_{r+n-i})...(1+T_{r+n-1})R$.

\begin{figure}
		\def\cbasisspacing{5mm}
	$\cbr{
		\begin{gathered}
		\Matching{8}{1/8, 2/7, 3/6, 4/5}, \hspace{\cbasisspacing}
		\Matching{8}{1/8, 2/7, 3/4, 5/6}, \hspace{\cbasisspacing}
		\Matching{8}{1/8, 2/3, 4/7, 5/6}, \hspace{\cbasisspacing}
		\Matching{8}{1/2,3/8,4/7,5/6}
		\end{gathered}}$
	\caption{$R_{L,0},...,R_{L,3}$ pictured from left to right}
	\label{shifted rainbow}
\end{figure}

 Define $Q_n:=(q^n+...+1)/q^{n/2}(-1)^n$ for $n\in \{0,1,...\}$. The following proposition says that, for any element in the kernel, if some basis element $y$ has coordinate $c$ in that element, and if $y$ has a rainbow sub-matching, the basis elements where you replace that sub-matching by the shifted rainbow matchings $R_{L,i}$ or $R_{R,i}$ both have coordinate $Q_ic$ in the kernel element.
 
\begin{proposition}
	Let $w$ be an element in the kernel intersection $\cap (1+T_z)$ in some generalized crossingless matchings representation. Let $y$ be a basis element with coordinate $c$ in $w$. Suppose $\exists a,b$ such that $y(a,b)=R$, the rainbow element. Define the basis elements $\theta_i,\phi_i$ by $\theta_i(1,a-1)=\phi(1,a-1)=y(1,a-1)$, $\theta_i(b+1,2n)=\phi(b+1,2n)=y(b+1,2n)$, $\theta_i(a,b)=R_{R,i}$, $\phi_i(a,b)=R_{L,i}$ (leave $\theta_i$ or $\phi_i$ undefined for any $i$ where $R_{R,i},R_{L,i}$ are undefined, respectively). The coordinates of $\phi_i$ and $\theta_i$ in $w$ are both $Q_ic$.
	
	\label{shifted rainbow coeffs}
\end{proposition}

Proof of this proposition requires a simple algebraic fact that will be used throughout this document, so I state it as a lemma.

\begin{lemma}
	$Q_1Q_n-Q_{n-1}=Q_{n+1}$
	
	\label{Q alg}
\end{lemma}

\textit{Proof of lemma.}

\begin{align*}
Q_1Q_n-Q_{n-1}=&\frac{-(q+1)}{q^{1/2}}\frac{(-1)^n(q^n+...+1)}{q^{n/2}}-\frac{(-1)^{n-1}(q^{n-1}+...+1)}{q^{(n-1)/2}}\\
=&\frac{(-1)^{n+1}(q^{n+1}+2q^n+...+2q+1)}{q^{(n+1)/2}}-\frac{(-1)^{n+1}(q^n+...+q)}{q^{(n+1)/2}}\\
=&\frac{(-1)^{n+1}(q^{n+1}+...+1)}{q^{(n+1)/2}}=Q_{n+1}
\end{align*}

Now let us prove the proposition.

\begin{proof}
	Consider acting on $w$ by an element $(1+T_z)$. The coordinate of $\phi_i$ in $(1+T_z)w$ will be a linear combination of the coordinates of basis elements sent to $\phi_i$ by the element $(1+T_z)$. Specifically, it will be $(1+q)c\iota+(q^1/2)\sum c_\psi$ where $\iota=1$ if $y(z)=z+1$, $\iota=0$ otherwise, and $c_\psi$ are the coordinates of all basis elements $\psi$ where $(1+T_z)\psi\sim y$.
	
	Let $n:=a+b-1$ and $r$ be the number of anchors in $y(a,b)$. Consider the coordinate of $\phi_i$ in $(1+T_{a-1+r+n/2-i})w$. This is the transposition that acts on the "moved middle hump" in $\phi_i(a,b)=R_{L,i}$, as shown in \ref{shifted relations}. I claim the following:
	
	\begin{center}
		\textit{claim}:
		The only basis elements $\psi$ where $(1+T_{a-1+r+n/2-i})\psi\sim \phi_i$ are $\phi_i$ and $\phi_{i-1},\phi_{i+1}$ when they exist\footnote{we defined $R_{L,i}$ as far out as we can move the hump, so for $0\leq i<n+r$, and take the analogous domain for $\phi_i$}.
		
	\end{center}
	
	Note that the action of any $(1+T_z)$ on a basis element $\psi$ creates exactly two lines: an arc of length two connecting $z$ and $z+1$, and either an anchor or an arc of length $\geq 2$ connecting $\psi(z)$ and $\psi(z+1)$.\footnote{Although this fact follows directly from our definition of the representation, it will be used throughout the document, so it is important that the reader understands it.\label{two humps created}} The easiest way to see the claim is to see that the given transposition is surrounded by arcs on both sides, so any basis element sent to the same element can vary from $\phi_i$ by at most one of those arcs and nothing else. 
	
	Let us prove the claim formally: It is easy to see that the action of $(1+T_{a-1+r+n/2-i})$ will bring $\phi_{i-1},\phi_i,\phi_{i+1}$ to $\sim \phi$, as shown in \ref{shifted relations}. Suppose there was another basis element $\psi$ sent to $\phi_i$ by the given transposition. Me note that if $\psi$ contains the arcs or anchors directly to the right and left of the arc $(a-1+r+n/2-i,a-1+r+n/2-i+1)$ in $\phi_i$ (formally, it contains the arc $(a-1+r+n/2-i-1,a-1+r+n/2+i+2)$ or an anchor at $a-1+r+n/2-i-1$ and the arc $(a-1+r+n/2-i+2,a-1+r+n/2+i+1)$ or an anchor at $a-1+r+n/2-i+2$), it must contain the arc $(a-1+r+n/2-i,a-1+r+n/2-i+1)$ to be a crossingless matching. Thus, if $\psi$ contains both of these arcs/anchors, $(1+T_{a-1+r+n/2-i})$ acts as the constant $(1+q)$, so $(1+T_{a-1+r+n/2-i})\psi\sim \phi_i=>\psi\sim \phi$. If $\psi$ does not contain the left arc/anchor and $(1+T_{a-1+r+n/2-i})\psi\sim \phi_i$, the action of $(1+T_{a-1+r+n/2-i})$ must create that arc/anchor, so $\psi(a-1+r+n/2-i-1)=a-1+r+n/2-i$ and $\psi(a-1+r+n/2-i+1)=a-1+r+n/2+i+2$ in the case of an arc or $a-1+r+n/2-i+1$ is an anchor. All other matchings remain unchanged, so this implies $\psi=\phi_{i+1}$. Likewise, if the right arc $((a+b-1)/2-i+2,(a+b-1)/2+i+1)$ does not exist, $\psi=\phi_{i-1}$. For boundary cases, note that for $\phi_0=\theta_0$, the only other basis element sent to this by the middle transposition is $\phi_1=\theta_1$. Also note that at the edge case $\phi_{n+r-1}$ there is not necessarily a left arc, so other elements may be sent to $\phi_{n+r-1}$ by the given transposition, and this case gives no new information. Lastly, note that our argument was completely symmetric and thus applies to the $\theta_i$ case, except that for $\theta_i$ we do not have to deal with anchors. Thus the claim is proved.
	
	Given this claim and lemma \ref{Q alg}, the proposition follows quickly through induction: 
	
	Acting by $(1+T_{a-1+r+n/2})$ on $w$, the new coordinate of $\phi_0=y$ is $(q+1)c+q^{1/2}c_{\phi_1}$ where $c_{\phi_1}$ is the coordinate of $\phi_1$ in $w$. Since $w$ is in the kernel, we have $(q+1)c+q^{1/2}c_{\phi_1}=0=>c_{\phi_1}=Q_1c$. $\phi_1=\theta_1$ so this gives us all our base cases.
	
	Acting by $(1+T_{a-1+r+n/2-i})$ on $w$, the new coordinate of $\phi_i$ is $q^{1/2}c_{\phi_{i+1}}+q^{1/2}c_{\phi_{i-1}}+(q+1)c_{\phi_{i}}=0$. By the inductive hypothesis, $q^{1/2}c_{\phi_{i+1}}+q^{1/2}Q_{i-1}c+(q+1)Q_i=0$ so $c_{\phi_{i+1}}=Q_1Q_i-Q_{i-1}=Q_{i+1}$ by lemma \ref{Q alg}. $\theta_i$ is an identical proof, so the proposition follows. 
	
	
	\begin{figure}
		\[
		\GeneralizedAction{8}{3/4,5/8,6/7}{2}{1/1,2/2}{1}{3}{3/4,5/8,6/7}{1/1,2/2}{(1+q)}
		\]
		\[
		\GeneralizedAction{8}{2/3,5/8,6/7}{2}{1/1,4/2}{1}{3}{3/4,5/8,6/7}{1/1,2/2}{q^{1/2}}
		\]
		\[
		\GeneralizedAction{8}{3/8,4/5,6/7}{2}{1/1,2/2}{1.25}{3}{3/4,5/8,6/7}{1/1,2/2}{q^{1/2}}
		\]
    \caption{The action of $(1+T_{a-1+r+n/2-i})$ on $\phi_i,\phi_{i=1},\phi_{i+1}$ (ordered from top to bottom), shown as the case where $y$ is the rainbow vector in $M_{8}^2$ and $i=2$.}
		\label{shifted relations}
	\end{figure}
\end{proof}	

Me are now ready to prove our first interesting result. Define $e$ as before.

\begin{proposition}
	Let $M_{2n+r}^r$ be a generalized crossingless matchings representation. Suppose $e$ does not divide $n+r+1$. Then $\cap\text{ker}(1+T_i)=0$.
	
	\label{trivial kernel}
\end{proposition}

\begin{proof}
	Suppose $\cap\text{ker}(1+T_i)=K\not=0$. Take nonzero $w\in K$. By Proposition \ref{rainbow nonzero}, the coordinate of the rainbow vector $R$ is nonzero; suppose the coordinate is $c$. By proposition \ref{shifted rainbow coeffs}, the coordinates of the basis elements $R_{L,n+r-1}$ and $R_{L,n+r-2}$ are $Q_{n+r-1}c$ and $Q_{n+r-2}c$ respectively.
	
	Consider the coordinate of $R_{L,n+r-1}$ in $(1+T_1)w$. Using the same logic as in the proof of proposition \ref{shifted rainbow coeffs}, we note that if a basis element $\psi$ has no anchor at position $3$ and is not equal to $R_{L,n+r-2}$, $(1+T_1)\psi\not\sim R_{L,n+r-1}$. Thus the desired coordinate is equal to $(1+q)Q_{n+r-1}c+q^{1/2}Q_{n+r-2}c=-q^{1/2}Q_{n+r}c$ by lemma \ref{Q alg}. Since $w\in K$, we must have $-q^{1/2}Q_{n+r}c=0$. Me have that $c$ is nonzero, and we assume $q$ nonzero, and $Q_{n+r}$ is zero iff $q$ is a root of $q^{n+r}+...+1$, implying $e|n+r+1$. Thus we have arrived at contradiction, and $K=0$.
\end{proof}


\newpage
\section{Defining the kernel for $e=n+r+1$}

In this section we will fully characterize $\cap\text{ker}(1+T_i)$ when $e=n+r+1$. Me will prove that it is one dimensional and give a basis. Note that we still have not proved the kernel is trivial when $e$ divides but is not equal to $n+r+1$. That proof requires results from this section, and will come next section.

\vspace{5mm}
First let us formalize a useful property of sub-matchings.

\begin{definition}
	Given a basis element $\psi\in M_{2n+r}^r$, specify some sub-matching $\psi(a,b)$. Let $\Res_{\SH_{b-a+1}(q)}^{\SH_{2n+r}(q)}M_{2n+r}^r$ be the restriction to the sub-algebra generated by transpositions $T_a,...,T_{b-1}$. Define $Y_\psi\subset\Res_{\SH_{b-a+1}(q)}^{\SH_{2n+r}(q)}M_{2n+r}^r$ to be the subrepresentation generated by the set of basis elements $\{\sigma| \sigma(1,a-1)=\psi(1,a-1),\ \sigma(b+1,2n+r)=\psi(b+1,2n+r)\}$.
\end{definition}

\begin{lemma}
	Take a basis element $\psi\in M_{2n+r}^r$. Suppose $\psi$ has some sub-matching $\psi(a,b)$ with $r'$ anchors. Define $Y_\psi$ with respect to this sub-matching.
	
	The map $\rho:Y_\psi\rightarrow M_{b-a+1}^{r'}$ defined by $$\rho(\sigma)=\sigma(a,b)$$ is an isomorphism of representations.
	
	\label{sub-matching isomorphism}
\end{lemma}

\begin{proof}
	
	The map is clearly bijective. Thus it is sufficient to prove the following: $$\rho(T_{i+a-1}\sigma)=T_i\rho(\sigma)$$
	
	As mentioned in the previous section, the action of a transposition $T_i$ can change at most 4 nodes, so we need to show that the transpositions end up changing the same nodes in the same way in  $\rho(T_{i+a-1}\sigma)$ and $T_i\rho(\sigma)$.
	
	Suppose $\sigma(i+a-1)=s,\sigma(i+a)=t$. Then $(T_{i+a-1}\sigma)(i+a-1)=i+a$, $(T_{i+a-1}\sigma)(s)=t$, so $\rho(T_{i+a-1}\sigma)(i)=i+1$, $\rho(T_{i+a-1}\sigma)(s-a+1)=t-a+1$. Separately, $\rho(\sigma)(i)=s-a+1$ and $\rho(\sigma)(i+1)=t-a+1$, so $T_i\rho(\sigma)(i)=i+1$ and $T_i\rho(\sigma)(s-a+1)=t-a+1$ as desired. So the map is an isomorphism and the lemma is proved.
\end{proof}

\vspace{5mm}
The lemma above motivates a recursive characterization of the kernel. To do this, it will be convenient to define some notation.

\begin{definition}
	Recall $Q_i:=(q^i+...+q+1)/q^{i/2}(-1)^i$ (lemma \ref{Q alg}). For $a>0$ define $Q(0,b):=1$. For $b>a>0$ define $$\Q_b^a:=\frac{Q_{b-1}...Q_{b-a}}{Q_1...Q_{a-1}}$$
	
\end{definition}

\begin{definition}
	For $\psi \in M_0^0$, define the function $x_\psi(q):=1$.
	
	For all other basis elements $\psi\in M_{2n+r}^r$, we define $x_\psi$ recursively:
	
	$$x_\psi(q):=x_{\psi(2,a-1)}(q)x_{\psi(a+1,2n+r)}(q)\Q^{\lfloor a/2\rfloor}_{n+r}$$
	
	I will refer to $x_\psi$ as the \textbf{coordinate function} of $\psi$.
	
	\label{coeff def}
\end{definition}


\vspace{2mm}
The following proposition states the forward direction of our characterization.
\begin{proposition}
	
	Let $M_{2n+r}^r$ be a crossingless matchings representation, and suppose $Q_1,...Q_{n+r-1},\not=0$. Let $w\in\cap\text{ker}(1+T_i)$. MLOG the rainbow element $R$ has coordinate 1 in $w$ (by proposition \ref{rainbow nonzero}). Then the coordinate of any basis element $\psi\in M_{2n+r}^r$ in $w$ is $x_{\psi}(q)$.
	
	\label{kernel characterization}
	
\end{proposition}

An illustration of this proposition is shown in figure \ref{coeff example}.

	
	\begin{figure}
		\def\cbasisspacing{5mm}
		
		$\cbr{
			\begin{gathered}
			\GeneralizedMatching{11}{1/6, 2/3,4/5,7/10,8/9}{1}{11/1}{11/8}, \hspace{\cbasisspacing}
			\Matching{4}{1/2, 3/4}, 
			\hspace{\cbasisspacing}
			\GeneralizedMatching{5}{1/4, 2/3}{1}{5/1}{3/4}, \hspace{\cbasisspacing}
			\end{gathered}}$ 
		\caption{Suppose the second and third elements have coordinates $x_2(q_1)$ and $x_3(q_2)$ in their respective kernel elements, where for $q_1$, $e=3$ and for $q_2$ $e=4$. The coordinate of the first element is $x(q)=x_2(q)x_3(q)\frac{Q_5Q_4Q_3}{Q_1Q_2}$, where for $q$, $e=7$}
		\label{coeff example}
	\end{figure}


\begin{proof}
	Suppose $\psi(1)=a$. The proof is structured as follows: use proposition \ref{shifted rainbow coeffs} to find the coefficient of the basis element with $\lfloor a/2\rfloor$ humps then a rainbow element; use the same proposition in a reversed manner to find the coefficient of the basis element consisting of the rainbow for the first $a$ nodes, then the rainbow for the final $2n+r-a$ nodes; finally, we finish the proof through induction using lemma \ref{sub-matching isomorphism}.
	
	\vspace{2mm}
	By proposition \ref{shifted rainbow coeffs} the element $R_1:=R_{L,n+r-1}$ has coordinate $Q_{n+r-1}$ in $w$. Then $R_1(3,2n+r)$ is the rainbow element in  $M_{2(n-1)+r}^r$, so the element $R_2$ defined by $R_2(1,2):=R_1(1,2)$, $R_2(3,2n+r):=R_{L,n+r-2}\in M_{2(n-1)+r}^r$ has coordinate $Q_{n-1}Q_{n-2}$. Generally, define $R_i$ by $R_i(1,2(i-1)):=R_{i-1}(1,2(i-1))$, $R_i(2i-1,2n+r):=R_{L,n+r-i}\in M_{2(n-i+1)+r}^r$. Then the coefficient of $R_i$ is $Q_{n+r-1}...Q_{n+r-i}$. These elements are shown in figure 5.
	
	\begin{figure}
		\def\cbasisspacing{5mm}
		
		$\cbr{
			\begin{gathered}
			\GeneralizedMatching{8}{3/8, 4/7,5/6}{2}{1/1,2/2}{5/4}, \hspace{\cbasisspacing}
			\GeneralizedMatching{8}{1/2, 5/8,6/7}{2}{3/1,4/2}{5/4}, 
			\hspace{\cbasisspacing}
			\GeneralizedMatching{8}{1/2, 3/4,7/8}{2}{5/1,6/2}{5/4}, \hspace{\cbasisspacing}
			\end{gathered}}$ 
		\caption{In order, the rainbow element, $R_1$, and $R_2$. The coordinate of the rainbow element is 1. The coordinate of $R_1$ is $Q_{4}$. The coordinate of $R_2$ is $Q_4Q_3$. Generally, $R_i$ is the element with $i$ humps then a rainbow element, and has coordinate $Q_{n+r-1}...Q_{n+r-i}$.}
	\end{figure}
	
	Now define basis elements $E_i$ by $E_i(2i+1,2n+r):=R_i(2i+1,2n+r)$, $E_i(1,2i):=R$, the appropriate rainbow element. By the same argument as above, if $E_i$ has coordinate $c$ in $w$, $R_i$ has coordinate $Q_{i-1}...Q_{1}c$. One way to make this more clear is to consider intermediate basis elements $\sigma_j^{E_i}$ defined by $\sigma_j^{E_i}(2i+1,2n+r):=E_i(2i+1,2n+r)$ and $\sigma_j^{E_i}(1,2i):=R_{L,j}$. Then the coordinates of $\sigma_j^{E_i}(2i+1,2n+r)$ in terms of the coordinate $c$ of $E_i$ are $Q_{i-1}...Q_{i-j}$, and $R_i=\sigma_{i-1}^{E_i}$. 
	
	Since we assume $Q_i\not=0$ for $i< n+r$, this implies the coefficient of $E_i$ is $\frac{Q_{n+r-1}...Q_{n+r-i}}{Q_1...Q_{i-1}}=\Q^i_{n+r}=x_{E_i}$. In particular, returning to our desired basis element $\psi$, the coordinate of $E_{\lfloor a/2\rfloor}$ is $\Q^{\lfloor a/2\rfloor}_{n+r}=x_{E_{\lfloor a/2\rfloor}}$.
	
	Note that the above logic only uses proposition \ref{shifted rainbow coeffs}, which requires only that a sub-matching be a rainbow element. So, suppose some basis element $\sigma$ has sub-matching $\sigma(s,t)=R$ with $n'$ nodes and $r'$ anchors, and that the coordinate of $\sigma$ in $w$ is $c$. Then it follows that the basis element $\theta_i$ defined by $\theta_i(1,s-1):=\sigma(1,s-1)$, $\theta_i(t+1,2n+r):=\sigma(t+1,2n+r)$, and $\theta_i(s,t):=E_i$ has coefficient $\Q^i_{n'+r'}c$. In other words, defining $Y_\psi$ with respect to the sub-matching $\sigma(s,t)$, the operation of finding the coordinate of $E_i$ given the coordinate of $R=\sigma(s,t)$ commutes with the isomorphism to $Y_\psi$.  An example is given in figure \ref{kernel induct characterization example}.
	
	\vspace{2mm}
	The above technique specifies an algorithm for determining the coordinate of $\psi$. 
	
	As a base case, for the zero element have the algorithm return 1.
	
	Suppose inductively that the algorithm returns the coordinate for any $\sigma\in M_{2n'+r'}^{r'}$, $2n'+r'<2n+r$, and that that coordinate is equal to the coordinate function $x_\sigma$. Also suppose that the algorithm commutes with any isomorphism defined by lemma \ref{sub-matching isomorphism}. These statements are clearly true for the base case. 
	
	Given $\psi\in M_{2n+r}^r$, if $\psi(1)=a$, we may find the coordinate of $E_{\lfloor a/2\rfloor}$ as before. Note that this operation commutes with any isomorphism defined by lemma \ref{sub-matching isomorphism}. Me may define $Y_{E_{\lfloor a/2\rfloor}}$ with respect to the sub-matching $E_{\lfloor a/2\rfloor}(2,a-1)$. By the inductive hypothesis, we may apply the algorithm to this sub-matching and commute with the isomorphism with $Y_{E_{\lfloor a/2\rfloor}}$. In this way, we find that the coordinate of $\tilde{\psi}$ defined by $\tilde{\psi}(1,a):=\psi(1,a)$ and $\tilde{\psi}(a+1,2n+r)=R$ is $x_{\psi(2,a-1)}(q)\Q^{\lfloor a/2\rfloor}_{n+r}=x_{\tilde{\psi}}(q)$. Similarly, define $Y_{\tilde{\psi}}$ with respect to the sub-matching $\tilde{\psi}(a+1,2n+r)$, and commute the algorithm with the isomorphism. In the same way, we obtain that the coordinate of $\psi\in Y_{\tilde{\psi}}$ is $x_{\psi(2,a-1)}(q)x_{\psi(a+1,2n+r)}(q)\Q^{\lfloor a/2\rfloor}_{n+r}=x_\psi(q)$ as desired.
	
	Note that we only added a single operation to the algorithm in the inductive step, which also commutes with any isomorphism defined by lemma \ref{sub-matching isomorphism}. Thus the inductive step holds and the proposition is proved.
	
	\begin{figure}
		\def\cbasisspacing{1mm}
		\begin{adjustbox}{width=\textwidth}
			\GeneralizedMatching{14}{3/12,4/11,5/10,6/9,7/8,13/14}{2}{1/1,2/2}{12/4}, \hspace{\cbasisspacing}
			\GeneralizedMatching{14}{1/2,3/4,5/6,9/12,10/11,13/14}{2}{7/1,8/2}{12/4}, \hspace{\cbasisspacing}
			\GeneralizedMatching{14}{1/6,2/5,3/4,9/12,10/11,13/14}{2}{7/1,8/2}{12/4}
		\end{adjustbox}
		
		\caption{The figure on the left has sub-matching $R$ ignoring the last two nodes. The middle figure has submatching $R_3$ ignoring the last two nodes. The figure on the right has sub-matching $E_3$ also ignoring the last two nodes. Since the last two nodes have the same structure for all elements, if the coordinate of the first element is $c$, the coordinate of the second is $Q_6Q_5Q_4c$, and the coordinate of the third is $\frac{Q_6Q_5Q_4}{Q_1Q_2}c$.}
		
		\label{kernel induct characterization example}
	\end{figure}
	
\end{proof}

\vspace{5mm}
The following few corollaries will help to simplify some later arguments.

\begin{corollary}
	Let $w\in\cap\text{ker}(1+T_i)$, $w\not=0$. Suppose $\psi(1,a)$ is a sub-matching with no anchors. Then:
	
	$$x_\psi=x_{\psi(1,a)}(q)x_{\psi(a+1,2n+r)}(q)\Q^{a/2}_{n+r}$$
	
	\label{characterization generalization}
\end{corollary}

\begin{proof}
	Define $a_1=\psi(1)$, $a_i=\psi(a_{i-1}+1)$. Then for some $j$ we have $a_j=a$. If $j=1$, the statement is the same as the proposition. Suppose that the statement is true for any matching with $a_v=a$, $v<j$. Then the statement holds for the sub-matching $\psi(a_1+1,2n+r)$, and we have:
	\begin{align*}
	x_\psi(q)=&x_{\psi(1,a_1)}(q)x_{\psi(a_1+1,2n+r)}(q)\Q^{a_1/2}_{n+r}\\
	=&x_{\psi(1,a_1)}(q)x_{\psi(a_1+1,a)}(q)x_{\psi(a+1,2n+r)}(q)\Q^{a_1/2}_{n+r}\Q_{n+r-a_1/2}^{a/2-a_1/2}\\
	=&x_{\psi(1,a_1)}(q)x_{\psi(a_1+1,a)}(q)x_{\psi(a+1,2n+r)}(q)\Q^{a_1/2}_{n+r}\Q_{n+r-a_1/2}^{a/2-a_1/2}\prn{\frac{Q_{a/2-1}...Q_{a/2-a_1/2}}{Q_{a/2-1}...Q_{a/2-a_1/2}}}\\
	=&x_{\psi(1,a_1)}(q)x_{\psi(a_1+1,a)}(q)x_{\psi(a+1,2n+r)}(q)\Q^{a_1/2}_{a/2}\Q^{a/2}_{n+r}\\
	=&x_{\psi(1,a)}(q)x_{\psi(a+1,2n+r)}(q)\Q^{a/2}_{n+r}
	\end{align*}

\end{proof}

\vspace{5mm}
\begin{corollary}
	If $\psi\in M_{2n+r}^r$, then $x_\psi(q)\not=0$ if $e>n+r$.
	
	\label{coeff nonzero}
\end{corollary}

\begin{proof}
	For our base cases, if $2n+r=2$ all coefficients are 1, which is nonzero for any $q$.
	
	Assume the statement is true for all $2n'+r'<2n+r$. Given $\psi(1)=a$ we have $$x_\psi(q)=x_{\psi(2,a-1)}(q)x_{\psi(a+1,2n+r)}(q)\Q^{\lfloor a/2\rfloor}_{n+r}$$
	
	If $e>n+r$, non of the $Q_i$ term appearing in $\Q^{\lfloor a/2\rfloor}_{n+r}$ are zero, and $n'+r'<n+r<e$ for any of the sub-matchings that appear, so those coordinates are nonzero and the corollary holds.
	
	
\end{proof}


\vspace{5mm}
The proposition fully characterizes any possible kernel element when $Q_1...Q_{n+r-1}\not=0$. In particular, the following corollary holds:

\begin{corollary}
	Mhen $Q_1..Q_{n+r-1}\not=0$ and the kernel is nontrivial, the kernel is one dimensional.
	
	\ref{1-D}
\end{corollary}

This corollary follows from the fact that we may write the coordinate of any basis element as proportional to the coordinate of the rainbow basis element.

\vspace{5mm}
\noindent\rule{16.5cm}{0.4pt}

\vspace{5mm}
The remainder of this section will be used to prove that when $e=n+r+1$, the element specified by proposition \ref{kernel characterization} is indeed an element of the kernel.

To verify the kernel element, we will need to know exactly which basis elements are mapped to a specific basis element by a given $(1+T_i)$. The next two lemmas help address this question.

\begin{lemma}
	Take some basis element $\psi\in M_{2n+r}^r$. 
	
	\begin{enumerate}[label={(\roman*)}]
	
	\item Suppose $\psi(a)=b$ for some $b>a+1$, and that $(1+T_i)\psi=(1+q)\psi$ for some $a<i<b-1$. Me then have a subrepresentation $\psi(a,b)$ and define $Y_\psi$ with respect to this subrepresentation. Then for all basis elements $\sigma$ such that $(1+T_i)\sigma=q^{1/2}\psi$, we have that $$\sigma\in Y_\psi$$.
	
	\item Suppose $\psi$ has some anchor at position $u$, and $(1+T_i)\psi=(1+q)\psi$ for some $i>u$, we again have a subrepresentation $\psi(u,2n+r)$ and define $Y_\psi$ with respect to this subrepresentation. Then for all basis elements $\sigma$ such that $(1+T_i)\sigma=q^{1/2}\psi$, we have that $\sigma\in Y_\psi$ again.
	
	
	\end{enumerate}
	\label{preimage under hump}
\end{lemma}

\begin{proof}
	This lemma follows from an observation I made in section \ref{two humps created}: a transposition can only create two arcs or an arc and an anchor. 
	
	(i) If $\sigma\not\in Y_\psi$ either $\sigma(1,a-1)\not=\psi(1,a-1)$ or $\sigma(b+1,2n+r)\not=\psi(b+1,2n+r)$. Suppose it is the first case. Then for some $s,t\in [1,a-1]$, $s<t$, we have $\psi(s)=t$ and $\sigma(s)\not=t$. To have $(1+T_i)\sigma=q^{1/2}\psi$ we must have $\sigma(t)=i+1$, $\sigma(s)=i$. But then $\sigma(a)\not=b$ and $\sigma(a)\not=i$ or $i+1$, so $((1+T_i)\sigma)(a)\not=b$ and $(1+T_i)\sigma\not=q^{1/2}\psi$. The same argument proves the $\sigma(b+1,2n+r)\not=\psi(b+1,2n+r)$ case. 
	
	(ii) An analogous argument proves the anchor case. Specifically, the anchor cannot exist at position $u$ and is not created by action of $(1+T_i)$ if $\sigma(s)=i$ and $\sigma(t)=i+1$.
	
\end{proof}

\vspace{5mm}
It is important to note that lemma \ref{preimage under hump} only references cases where a transposition acts under an arc or to the right of an anchor. An example is given in figure \ref{preimage under hump example}. 

The next lemma characterizes cases where the transposition is not under any arcs and all anchors are to the right.

\begin{figure}[b]
	\[
	\begin{tabular}{l l l l}
	\GeneralizedAction{13}{1/4,2/3,6/7,8/13,9/10,11/12}{1}{5/1}{5/4}{9}{1/4,2/3,6/7,8/13,9/10,11/12}{5/1}{(1+q)}\\
	
	\GeneralizedAction{13}{1/4,2/3,6/7,8/9,10/13,11/12}{1}{5/1}{5/4}{9}{1/4,2/3,6/7,8/13,9/10,11/12}{5/1}{(q^{1/2})}\\
	
	\GeneralizedAction{13}{1/4,2/3,6/7,8/13,9/10,11/12}{1}{5/1}{5/4}{6}{1/4,2/3,6/7,8/13,9/10,11/12}{5/1}{(1+q)}\\
	
	\GeneralizedAction{13}{1/4,2/3,5/6,8/13,9/10,11/12}{1}{7/1}{5/4}{6}{1/4,2/3,6/7,8/13,9/10,11/12}{5/1}{(q^{1/2})}\\
	
	\end{tabular}
	\]
	
	\caption{In the first line we act under an arc, so if another element without that arc is sent to that element, it must fix the arc as shown in the second line. In the third line we act to the right of an anchor, so if another element without that anchor is sent to that element, it must fix the anchor as shown in the fourth line. 
	}
	\label{preimage under hump example}
\end{figure}

Essentially, this lemma states that the only elements sent to the same element are those which break at most one of the top level arcs to the left of the leftmost anchor, or that break the leftmost anchor. An illustration is given in figure \ref{preimage under nothing example}.

\vspace{5mm}
\begin{lemma}
	Take a basis element $\psi\in M_{2n+r}^r$. Suppose the leftmost anchor in $\psi$ is at index $b$, or let $b=2n+r+1$ if there is no anchor. Define $a_j$ such that $\psi(a_j)=a_{j-1}+1$ and $\psi(a_1)=1$ for all $j$ such that $a_j<b$. 
	
	Suppose $(1+T_i)\psi=(1+q)\psi$ for some $i<b-1$ where $\nexists s,t$ such that $\psi(s)=t$ and $s<i,t>i+1$. Suppose there is some basis element $\sigma$ such that $(1+T_i)\sigma=q^{1/2}\psi$. Then:
	\\
	
	\begin{enumerate}[label={(\roman*)}]
		\item  $\psi(a_{j-1}+2,a_j-1)=\sigma(a_{j-1}+2,a_j-1)$ for all $j$.
		
		\item $\psi(b+1,2n+r)=\sigma(b+1,2n+r)$
		
		\item If $b$ is not an anchor in $\sigma$, $\psi(a_j)=\sigma(a_j)$ for all $j$ such that $a_j\not=i+1$.
		
		\item If $b$ is an anchor in $\sigma$, there exists exactly one value of $j$ such that $\sigma(a_j)\not=\psi(a_j)$ and $a_j\not=i+1$
	\end{enumerate}
	\label{preimage under nothing}
\end{lemma}

\begin{proof}
	
	(i) Suppose that, for some $j$ there exists $s,t\in [a_{j-1}+2,a_j-1]$ such that $\psi(s)=t$ but $\sigma(s)\not=t$. Then if $(1+T_i)\sigma=q^{1/2}\psi$ we must have $\sigma(i)=s$ or $t$ and $\sigma(i+1)=s$ or $t$. But, by definition, $i,i+1\not\in[a_{j-1}+1,a_j]$, so this implies $\sigma(a_j)\not=a_{j-1}+1,i,i+1$, so $((1+T_i)\sigma)(a_j)\not=a_{j-1}+1$ and $(1+T_i)\sigma\not=q^{1/2}\psi$. So (i) is proved.
	
	\vspace{5mm}
	(ii) The proof of (ii) is analogous to the proof of (i). Me cannot have $\psi(b+1,2n+r)\not=\sigma(b+1,2n+r)$ and $\psi(b+1,2n+r)=((1+T_i)\sigma)(b+1,2n+r)$ if $((1+T_i)\sigma)(b)=b$.
	
	\vspace{5mm}
	(iii) If $b$ is not an anchor in $\sigma$ and $(1+T_i)\sigma=q^{1/2}\psi$, we must have $i$ an anchor in $\sigma$, and $\sigma(i+1)=b$. No other nodes in $\sigma$ are changed, so this proves (iii).
	
	\vspace{5mm}
	(iiii) From (i)-(iii) we have that the only remaining matchings that can differ are the $(a_{j-1}+1,a_j)$ matchings. If one of them differs, by the same argument as before it must be fixed by the action of $(1+T_i)$, and no other nodes are changed, so (iiii) is proved.
	
\end{proof}


\begin{figure}[b]
	\[
	\begin{tabular}{l l l l}
	\GeneralizedAction{15}{1/6,2/3,4/5,7/8,9/10,12/15,13/14}{1}{11/1}{5/4}{7}{1/6,2/3,4/5,7/8,9/10,12/15,13/14}{11/1}{(1+q)}\\
	
	\GeneralizedAction{15}{1/8,2/3,4/5,6/7,9/10,12/15,13/14}{1}{11/1}{5/4}{7}{1/6,2/3,4/5,7/8,9/10,12/15,13/14}{11/1}{(q^{1/2})}\\
	
	\GeneralizedAction{15}{1/6,2/3,4/5,7/10,8/9,12/15,13/14}{1}{11/1}{5/4}{7}{1/6,2/3,4/5,7/8,9/10,12/15,13/14}{11/1}{(q^{1/2})}\\
	
	\GeneralizedAction{15}{1/6,2/3,4/5,8/11,9/10,12/15,13/14}{1}{7/1}{5/4}{7}{1/6,2/3,4/5,7/8,9/10,12/15,13/14}{11/1}{(q^{1/2})}\\
	
	\end{tabular}
	\]
	
	\caption{The action of $(1+T_7)$ fixes the first basis element. Shown are all the basis vectors sent to the same element by the same transposition. Note that in all of them nodes 2-5 and 12-15 are the same. This illustrates (i) and (ii) in lemma \ref{preimage under nothing}. Note that in the last case where the anchor is in a different place, 1,6 and 9,10 are still matched. This illustrates (iii). In the middle two cases where the anchor is in the same place, only one of 1,6 or 9,10 are not paired. This illustrates (iiii). 
	}
	\label{preimage under nothing example}
\end{figure}

\vspace{5mm}
Lastly, we will need a small combinatorial result.

\begin{lemma}
	Suppose $n>b\geq a>0$ and $e>n$. Then $$Q_{n-a}Q_b-Q_{n-b-1}Q_{a-1}=Q_{n}Q_{b-a}$$
	
	\label{Q lemma}
\end{lemma}

\begin{proof}
	If $b=1$, the only possibility for $a$ is 1, in which reduces to lemma \ref{Q alg}.
	
	Suppose the lemma is true for all $\tilde{b}<b+1$. Then for $a< b$ we have 
	\begin{align*}	
	Q_{n-a}Q_b -Q_{n-b-1}Q_{a-1}=&Q_{n}Q_{b-a}\\
	Q_1Q_{n-a}Q_b-Q_1Q_{n-b-1}Q_{a-1}=&Q_1Q_{n}Q_{b-a}
	\end{align*}
	
	from lemma \ref{Q alg}, we have	
	$$Q_{n-a}(Q_{b+1}+Q_{b-1})-(Q_{n-b}+Q_{n-b-2})Q_{a-1}=Q_{n}(Q_{b-a-1}+Q_{b-a+1})$$
	
	and from the inductive hypothesis we have
	
	$$Q_{n-a}Q_{b+1}-Q_{n-b}Q_{a-1}=Q_{n}Q_{b-a+1}$$
	
	as desired.
	
	For $a=b$ we have 
	\begin{align*}
	Q_{n-b}Q_b -Q_{n-b-1}Q_{b-1}=&Q_{n}\\
	Q_1Q_{n-b}Q_b-Q_1Q_{n-b-1}Q_{b-1}=&Q_1Q_{n}
	\end{align*}
	
	from lemma \ref{Q alg}, we have
	\begin{align*}
	Q_{n-b}(Q_{b+1}+Q_{b-1})-(Q_{n-b}+Q_{n-b-2})Q_{b-1}=&Q_1Q_{n}\\
	Q_{n-b}Q_{b+1}-Q_{n-b-2}Q_{b-1}=&Q_1Q_{n}\\
	\end{align*}
	
	as desired.
	
	For $a=b+1$, we continue:
	\begin{align*}
	Q_1Q_{n-b}Q_{b+1}-Q_1Q_{n-b-2}Q_{b-1}=&Q_1Q_1Q_{n}\\
	(Q_{n-b-1}+Q_{n-b+1})Q_{b+1}-Q_{n-b-2}(Q_{b}+Q_{b-2})=&(1+Q_2)Q_{n}\\
	\end{align*}
	
	So by the inductive hypothesis
	$$Q_{n-b-1}Q_{b+1}-Q_{n-b-2}Q_{b}=Q_{n}$$
	
	as desired, and the proof is finished by induction.
	
\end{proof}

\vspace{5mm}
Me are now ready to prove existence of a kernel element. To prove this, we will show that if $w\in M_{2n+r}^r$ is as characterized above, the coordinate of any basis element in $(1+T_i)w$ is zero. This will split into various cases related to the previous lemmas.

\vspace{5mm}
\begin{theorem}
	Suppose $e=n+r+1$. Then $\cap\text{ker}(1+T_i)\not=0$.
	\label{kernel existence}
\end{theorem}

\begin{proof}
	
	
	As a base case, when $2n'+r'\leq 2$, the representation is at most one dimensional. If the one basis element has only anchors, it is sent to zero by any $(1+T_i)$, and is in the kernel. If the single basis element is a single arc, it is sent to $(1+q)$ times itself, and we take $e=n+r+1=2$ so $1+q=0$ and the base case holds.
	
	\vspace{5mm}
	
	Assume inductively that the statement holds for all $M_{2n'+r'}^{r'}$ where $2n'+r'<2n+r$. Take $w$ as defined by proposition \ref{kernel characterization}.
	
	Given $\psi\in (1+T_i)M_{2n+r}^r$ let $E_\psi\subset M_{2n+r}^r$ be the pre-image of $\psi$ under the action of $(1+T_i)$. To prove $w$ is in the kernel, we must show the following: 
	\begin{equation}
	(1+q)x_\psi(q)+\sum_{\sigma\in E_\psi,\sigma\not=\psi}q^{1/2}x_\sigma(q)=0\text{ for all basis elements }\psi
	\label{weak kernel}
	\end{equation}
	Inductively, we assume this equation holds for basis elements in smaller representations $M_{2n'+r'}^{r'}$, but only for $q$ such that $e=n'+r'+1$. Clearly this is true in the base case. For the following proof we will need a slightly stronger inductive assumption. Take $\psi'\in M_{2n'+r'}^{r'}$, and suppose either that $\psi'(1)=2n'+r'$, and that $T_i\psi'=(1+q)\psi'$, $1<i<2n'+r'-1$, or that 1 is an anchor in $\psi$. Defining $E_{\psi'}$ as before, we assume
	
	\begin{equation}
		(1+q)x_{\psi'}+\sum_{\sigma\in E_{\psi'},\sigma\not=\psi'}q^{1/2}x_\sigma=0\text{ for any } q \text{ with }e>n'+r'
		\label{strong kernel}
	\end{equation}
	
	Note that \ref{strong kernel} does not apply in the base case. Our proof of the inductive step will be split into cases, and each case will only depend on sub-cases in which certain inductive hypotheses apply, so this will not lead to any problems.
	
	Before exploring the cases, let us formally define $E_\psi$ to be the pre-image of $\psi$ under the action of $(1+T_i)$, and $E_{\psi(a,b)}$ to be the pre-image of $\psi(a,b)$ under action of $(1+T_{i-a+1})$:
	
	\begin{enumerate}[label={case \arabic*:}]
	\item Suppose $\psi\in (1+T_i)M_{2n+r}^r$ for some $i$, and that $\exists s,t$ such that $s<i<t-1$, $s>1$ or $t<2n+r$, and $\psi(s)=t$. Also suppose the leftmost anchor is at some index $u>t$, or that there are no anchors. Then we have a sub-matching $\psi(s,t)$, and by lemma \ref{preimage under hump} $E_\psi\subset Y_\psi$. Then, using corollary \ref{characterization generalization}, the following equality holds:
	
	\begin{align*}
	&(1+q)x_\psi(q)+\sum_{\sigma\in E_\psi,\sigma\not=\psi}q^{1/2}x_\sigma(q)\\
	=&\prn{x_{\psi(1,s-1)}(q)\Q_{n+r}^{(s-1)/2}}\prn{(1+q)x_{\psi(s,2n+r)}(q)+\sum_{\sigma\in E_\psi,\sigma\not=\psi}q^{1/2}x_{\sigma(s,2n+r)}(q)}\\
	=&\prn{x_{\psi(1,s-1)}(q)\Q_{n+r}^{(s-1)/2}}\prn{x_{\psi(t+1,2n+r)}(q)\Q_{n+r-(s-1)/2}^{(t-s+1)/2}}\prn{(1+q)x_{\psi(s,t)}(q)+\sum_{\sigma\in E_\psi,\sigma\not=\psi}q^{1/2}x_{\sigma(s,t)}(q)}
	\end{align*}
	
	Me have that $e>j$ for any $Q_j$ term appearing in the equation above, and $e>n'+r'$ for any sub-matching coordinate appearing above, so by corollary \ref{coeff nonzero}:
	
	$$(1+q)x_\psi(q)+\sum_{\sigma\in E_\psi,\sigma\not=\psi}q^{1/2}x_\sigma(q)=0$$
	
	if and only if
	
	$$(1+q)x_{\psi(s,t)}(q)+\sum_{\sigma\in E_\psi,\sigma\not=\psi}q^{1/2}x_{\sigma(s,t)}(q)=0$$
	
	Note that $(\psi(s,t))(1)=t-s+1$. So by our inductive hypothesis (ii), we have 
	
	$$(1+q)x_{\psi(s,t)}(q)+\sum_{\sigma\in E_{\psi(s,t)},\sigma\not=\psi(s,t)}q^{1/2}x_{\sigma}(q)=0$$
	
	By lemma \ref{sub-matching isomorphism}, if $\sigma\in Y_\psi$, $(1+T_i)\sigma=q^{1/2}\psi$ if and only if $(1+T_{i-s+1})\sigma(s,t)=q^{1/2}\psi(s,t)$, so the previous equation implies $$(1+q)x_{\psi(s,t)}(q)+\sum_{\sigma\in E_\psi,\sigma\not=\psi}q^{1/2}x_{\sigma(s,t)}(q)=0$$
	
	as desired, and this case is proved.
	
	\vspace{5mm}
	\item Again take $\psi\in (1+T_i)M_{2n+r}^r$ for some $i$, but suppose the leftmost anchor is at some position $u$ where $1<u<i$. Then, as before, we have a sub-matching $\psi(u,2n+r)$ and by lemma \ref{preimage under hump} $E_\psi\subset Y_\psi$.
	
	Note that both corollary \ref{characterization generalization} and our inductive hypothesis \ref{strong kernel} still apply in this case, where we consider a left anchor instead of a matching. This allows the exact same logic from the proof of the first case to prove this case.
	\\
	
	It is important to note that, for both case 1 and case 2, the inductive hypothesis depends only on cases in which \ref{strong kernel} holds. Thus, if we show these cases rely on valid base cases, case 1 and 2 follow. This will be done in case 4.
	
	
	\vspace{5mm}
	\item Suppose $\psi\in (1+T_i)M_{2n+r}^r$ for some $i$, the leftmost anchor is at a position $u>i+1$ or there are no anchors, and $\nexists s,t$ such that $\psi(s)=t$ and $s<i<t-1$. Lemma \ref{preimage under nothing} characterizes all $\sigma\in E_\psi$. Me would like to prove the following for arbitrary $q$ where $e>n+r$:
	
	$$(1+q)x_\psi(q)+\sum_{\sigma\in E_\psi,\sigma\not=\psi}q^{1/2}x_\sigma(q)=-q^{1/2}x_{\psi(1,i-1)}(q)x_{\psi(i+2,2n+r)}(q)\Q_{n+r+1}^{(i+1)/2}$$
	
	\begin{figure}[b]
		\[
		\begin{tabular}{l l l l}
		\GeneralizedAction{7}{1/2,3/4,5/6}{1}{7/1}{3/4}{3}{1/2,3/4,5/6}{7/1}{(1+q)}\\
		
		\GeneralizedAction{7}{1/4,2/3,5/6}{1}{7/1}{3/4}{3}{1/2,3/4,5/6}{7/1}{(q^{1/2})}\\
		
		\GeneralizedAction{7}{1/2,3/6,4/5}{1}{7/1}{3/4}{3}{1/2,3/4,5/6}{7/1}{(q^{1/2})}\\
		
		\GeneralizedAction{7}{1/2,4/7,5/6}{1}{3/1}{3/4}{3}{1/2,3/4,5/6}{7/1}{(q^{1/2})}\\
		
		\end{tabular}
		\]
		
		\caption{The four elements sent to the first element by $(1+T_3)$ are listed. The coordinate of the first element is $Q_3Q_2Q_1$. The coordinate of the second is $Q_3Q_2$. The coordinate of the third is $Q_3Q_2$. The coordinate of the fourth is $Q_3$. Call the first element $\psi$. Then $x_{\psi(1,2)}=1$, $x_{\psi(5,7)}=Q_1$, so $-q^{1/2}x_{\psi(1,i-1)}(q)x_{\psi(i+2,2n+r)}(q)\Q_{n+r+1}^{(i+1)/2}=-q^{1/2}Q_4Q_3$. Me also have $(1+q)x_\psi(q)+\sum_{\sigma\in E_\psi,\sigma\not=\psi}q^{1/2}x_\sigma(q)=(q+1)(Q_3Q_2Q_1)+q^{1/2}(Q_3Q_2+Q_3Q_2+Q_3)=-q^{1/2}(Q_3Q_2Q_1^2-2Q_3Q_2+Q_3)=-q^{1/2}Q_4Q_3$ as desired (one can verify the last equality by hand or simplify using lemma \ref{Q lemma}).}
		\label{case 3}
	\end{figure}
	
	
	See figure \ref{case 3} for an example of this equality. Note that if $e=n+r+1$, $Q_{n+r}$ is the only zero component in the right side of this equation, so proving this equation is sufficient to prove case three.
	\\
	
	Me will prove this equality through yet another inductive proof, this time inducting on the number of top level humps, including the leftmost anchor.
	
	Formally, as we have in earlier lemmas, we will define $a_j$ by $a_1:=\psi(1)$, $a_j:=\psi(a_{j-1}+1)$. Then define $b_\psi$ such that $a_{b_\psi}=u$ if there is an anchor or $a_{b_\psi}=2n+r$ otherwise. Me induct on $b_\psi$.
	\\
	
	If $b_\psi=1$, we must be in $M_2^0$ to be in case 3 (otherwise $s<i<t-1$ for some $s,t$ where $\psi(s)=t$), which is trivially satisfied. Thus the base case holds.
	\\
	
	Suppose for all basis elements $\sigma$ such that $b_\sigma<b_\psi$, the equality holds. Suppose $i\not=1$. Then $a_1<i$ and lemma \ref{preimage under nothing} gives that there is a unique $\upsilon\in E_\psi$ such that $\upsilon(1)\not=a_1$. Thus we have the following equality:
	
	\begin{align*}
	&(1+q)x_\psi(q)+\sum_{\sigma\in E_\psi,\sigma\not=\psi,\upsilon}q^{1/2}x_\sigma(q)\\
	=&\prn{x_{\psi(1,a_1)}(q)\Q_{n+r}^{a_1/2}}\prn{(1+q)x_{\psi(a_1+1,2n+r)}(q)+\sum_{\sigma\in E_\psi,\sigma\not=\psi,\upsilon}q^{1/2}x_{\sigma(a_1+1,2n+r)}(q)}
	\end{align*}
	
	Define $Y_\psi$ with respect to the sub-matching $\psi(a_1+1,2n+r)$. Then $\sigma\in E_\psi$, $\sigma\not=\upsilon$ implies $\sigma\in Y_\psi$. By our inductive hypothesis, we have that 
	\begin{align*}
	&(1+q)x_{\psi(a_1+1,2n+r)}(q)+\sum_{\sigma\in E_{\psi(a_1+1,2n+r)},\sigma\not=\psi(a_1+1,2n+r)}q^{1/2}x_{\sigma}(q)\\
	=&-q^{1/2}x_{\psi(a_1+1,i-1)}(q)x_{\psi(i+2,2n+r)}(q)\Q_{n+r-a_1/2+1}^{(i+1-a_1)/2}
	\end{align*}
	
	
	By lemma \ref{sub-matching isomorphism}, $\sigma\subset Y_\psi$, $\sigma\in E_\psi$ if and only if $\sigma(a_1+1,2n+r)\in E_{\psi(a_1+1,2n+r)}$. This implies:
	
	\begin{align*}
	&(1+q)x_{\psi(a_1+1,2n+r)}(q)+\sum_{\sigma\in E_\psi,\sigma\not=\psi,\upsilon}q^{1/2}x_{\sigma(a_1+1,2n+r)}(q)\\
	=&-q^{1/2}x_{\psi(a_1+1,i-1)}(q)x_{\psi(i+2,2n+r)}(q)\Q_{n+r-a_1/2+1}^{(i+1-a_1)/2}
	\end{align*}
	
	So, combining with the aforementioned equality, we have 
	\begin{align*}
	&(1+q)x_\psi(q)+\sum_{\sigma\in E_\psi,\sigma\not=\psi,\upsilon}q^{1/2}x_\sigma(q)\\
	=&\prn{x_{\psi(1,a_1)}(q)\Q_{n+r-a_1/2+1}^{(i+1-a_1)/2}}\prn{-q^{1/2}x_{\psi(a_1+1,i-1)}(q)x_{\psi(i+2,2n+r)}(q)\Q_{n+r-a_1/2+1}^{(i+1-a_1)/2}}\\
	=&\prn{x_{\psi(1,a_1)}(q)\Q_{n+r-a_1/2+1}^{(i+1-a_1)/2}}\prn{-q^{1/2}x_{\psi(a_1+1,i-1)}(q)x_{\psi(i+2,2n+r)}(q)\Q_{n+r-a_1/2+1}^{(i+1-a_1)/2}}\prn{\frac{Q_{(i-1)/2-1}...Q_{(i-1-a_1)/2}}{Q_{(i-1)/2-1}...Q_{(i-1-a_1)/2}}}\\
	=&-q^{1/2}x_{\psi(1,i-1)}(q)x_{\psi(i+2,2n+r)}(q)\Q_{n+r+1}^{(i+1)/2}\prn{\frac{Q_{n+r-a_1/2}Q_{(i-1)/2}}{Q_{n+r}Q_{(i-1-a_1)/2}}}
	\end{align*}
	
	Separately, note that $\upsilon$ is defined by $\upsilon(2,a_1-1)=\psi(2,a_1-1)$, $\upsilon(a_1+1,i-1)=\psi(a_1+1,i-1)$, $\upsilon(i+2,2n+r)=\psi(i+2,2n+r)$, and $\upsilon(1)=i+1$, $\upsilon(a_1)=i$. Thus we may determine $x_\upsilon$, again utilizing corollary \ref{characterization generalization}:
	
	\begin{align*}
	x_\upsilon=&x_{\psi(i+2,2n+r)}x_{\upsilon(2,i)}\Q_{n+r}^{(i+1)/2}\\
	=&x_{\psi(i+2,2n+r)}\prn{x_{\psi(2,a_1-1)}x_{\upsilon(a_1,i)}\Q_{(i-1)/2}^{(a_1-2)/2}}\Q_{n+r}^{(i+1)/2}\\
	=&x_{\psi(i+2,2n+r)}\prn{x_{\psi(1,a_1)}x_{\psi(a_1+1,i-1)}\Q_{(i-1)/2}^{(a_1-2)/2}}\Q_{n+r}^{(i+1)/2}\\
	=&x_{\psi(i+2,2n+r)}x_{\psi(1,i-1)}\Q_{n+r+1}^{(i+1)/2}\prn{\frac{Q_{n+r-(i+1)/2}Q_{a_1/2-1}}{Q_{n+r}Q_{(i-1-a_1)/2}}}
	\end{align*}
	
	Adding this into our previous equation, we have:
	\begin{align*}
	&(1+q)x_\psi(q)+\sum_{\sigma\in E_\psi,\sigma\not=\psi}q^{1/2}x_\sigma(q)\\
	=&-q^{1/2}x_{\psi(1,i-1)}(q)x_{\psi(i+2,2n+r)}(q)\Q_{n+r+1}^{(i+1)/2}\prn{\frac{Q_{n+r-a_1/2}Q_{(i-1)/2}-Q_{n+r-(i+1)/2}Q_{a_1/2-1}}{Q_{n+r}Q_{(i-1-a_1)/2}}}
	\end{align*}
	
	Applying lemma \ref{Q lemma} to the portion of the equation above in parenthesis, the above is equivalent to
	
	$$(1+q)x_\psi(q)+\sum_{\sigma\in E_\psi,\sigma\not=\psi}q^{1/2}x_\sigma(q)=-q^{1/2}x_{\psi(1,i-1)}(q)x_{\psi(i+2,2n+r)}(q)\Q_{n+r+1}^{(i+1)/2}$$
	
	as desired. Note that if $e\geq n+r+1$ the only term above that can be zero is $Q_{n+r}$ (by corollary \ref{coeff nonzero}). Thus we have proved the inductive step for the case where $i\not=1$.
	
	\vspace{5mm}
	If $i=1$, we instead look at the sub-matchings $\psi(1,a_{(b_\psi-1)})$, $\psi(a_{(b_\psi-1)}+1,2n+r)$. Again lemma \ref{preimage under nothing} gives that there is a unique $\upsilon\in E_\psi$ such that $\upsilon(a_{(b_\psi-1)}+1)\not=\psi(a_{(b_\psi-1)}+1)$. Taking $Y_\psi$ with respect to the sub-matching $\psi(a_{(b_\psi-1)}+1,a_{b_\psi})$ again we have that $\sigma\in E_\psi$, $\sigma\not=\upsilon$ implies $\sigma\in Y_\psi$. Thus, following the same logic as before, we arrive at the following equality:
	\begin{align*}
	&(1+q)x_\psi(q)+\sum_{\sigma\in E_\psi,\sigma\not=\psi,\upsilon}q^{1/2}x_\sigma(q)\\
	=&\prn{x_{\psi(a_{(b_\psi-1)}+1,2n+r)}(q)\Q_{n+r}^{a_{(b_\psi-1)}/2}}\prn{-q^{1/2}x_{\psi(3,a_{(b_\psi-1)})}(q)Q_{a_{(b_\psi-1)}/2}}\\
	=&-q^{1/2}x_{\psi(3,2n+r)}\frac{Q_{n+r-1}Q_{a_{(b_\psi-1)}/2}}{Q_{a_{(b_\psi-1)}/2-1}}
	\end{align*}
	
	Again, we know the structure of $\upsilon$ from lemma \ref{preimage under nothing}. Suppose for now that $a_{b_\psi}$ is not an anchor, so it is $2n+r$. Then $\upsilon$ is defined by $\upsilon(3,a_{(b_\psi-1)})=\psi(3,a_{(b_\psi-1)})$, $\upsilon(a_{(b_\psi-1)}+2,2n+r-1)=\psi(a_{(b_\psi-1)}+2,2n+r-1)$, and $\upsilon(1)=2n+r$, $\upsilon(2)=a_{(b_\psi-1)}+1$. So we may again find $x_\upsilon$:
	\begin{align*}
	x_\upsilon=x_{\upsilon(2,2n+r-1)}=&x_{\psi(3,a_{(b_\psi-1)})}x_{\psi(a_{(b_\psi-1)}+2,2n+r-1)}\Q_{n+r-1}^{a_{(b_\psi-1)}/2}\\
	=&x_{\psi(3,a_{(b_\psi-1)})}x_{\psi(a_{(b_\psi-1)}+1,2n+r)}\Q_{n+r-1}^{a_{(b_\psi-1)}/2}\\
	=&x_{\psi(3,2n+r)}\frac{Q_{n-r-1-a_{(b_\psi-1)}/2}}{Q_{a_{(b_\psi-1)}/2-1}}
	\end{align*}
	
	Alternatively, if $a_{b_\psi}$ is an anchor, the definition of $\upsilon$ is now $\upsilon(3,a_{b_\psi}-1)=\psi(3,a_{b_\psi}-1)$, $\upsilon(a_{b_\psi}+1,2n+r)=\psi(a_{b_\psi}+1,2n+r)$, and $\upsilon(1)=1$, $\upsilon(2)=a_{b_\psi}$, so we have:
	
	$$x_\upsilon=x_{\upsilon(2,2n+r)}=x_{\psi(3,a_{(b_\psi-1)})}x_{\psi(a_{b_\psi}+1,2n+r)}\Q_{n+r-1}^{a_{(b_\psi-1)}/2}=x_{\psi(3,2n+r)}\frac{Q_{n-r-1-a_{(b_\psi-1)}/2}}{Q_{a_{(b_\psi-1)}/2-1}}$$
	
	so for our purposes $x_\upsilon$ is the same in either case.
	
	Incorporating into the above equation, we have:
	
	$$(1+q)x_\psi(q)+\sum_{\sigma\in E_\psi,\sigma\not=\psi}q^{1/2}x_\sigma(q)=$$
	
	
	$$-q^{1/2}x_{\psi(3,2n+r)}\frac{Q_{n+r-1}Q_{a_{(b_\psi-1)}/2}-Q_{n-r-1-a_{(b_\psi-1)}/2}}{Q_{a_{(b_\psi-1)}/2-1}}$$	
	
	By lemma \ref{Q lemma}, this is simply $-q^{1/2}x_{\psi(3,2n+r)}Q_{n+r}$ as desired, and we have finished proving case 3.
	\\
	
	\item The only cases we have not yet dealt with are those where either $1$ is an anchor or $\psi(1)=2n+r$. These are those cases related to our inductive hypothesis (ii).
	
	To not be in case 1 or 2, we must have that there are no anchors between index 1 and $i$, and that there is no integer $s$ such that $1<s<i<\psi(s)-1$. It follows from the same argument that proved lemma \ref{preimage under hump} that there exists exactly one $\upsilon\in E_\psi$ such that $\upsilon(1)\not=\psi(1)$. Define $N$ to be $2n+r$ if 1 is an anchor, or $2n+r-1$ if 1 is not an anchor. Then, defining $Y_\psi$ with respect to the sub-matching $\psi(2,N)$, we have that $\sigma\in E_\psi, \sigma\not=\upsilon$ if and only if $\sigma(2,N)\in E_{\psi(2,N)}$. Note that for $E_{\psi(2,N)}$ we may apply the inductive hypothesis from case 3, so we have:
	\begin{align*}
	&(1+q)x_\psi(q)+\sum_{\sigma\in E_\psi,\sigma\not=\psi,\upsilon}q^{1/2}x_\sigma(q)\\
	=&(1+q)x_{\psi(2,N)}(q)+\sum_{\sigma\in E_{\psi(2,N)},\sigma\not=\psi(2,N)}q^{1/2}x_{\sigma}(q)\\
	=&-q^{1/2}x_{\psi(2,i-1)}x_{\psi(i+2,N)}\Q{n+r}^{i/2}
	\end{align*}
	
	As in case 3, we can also determine $x_\upsilon$. $\upsilon$ is defined by $\upsilon(2,i-1)=\psi(2,i-1)$, $\upsilon(i+2,N)=\psi(i+2,N)$, $\upsilon(1)=i$, and $\upsilon(i+1)=2n+r$ if 1 is not an anchor or $i+1$ if 1 is an anchor, and we have:
	
	$$x_\upsilon=x_{\psi(2,i-1)}x_{\upsilon(i+2,N)}\Q{n+r}^{i/2}$$
	
	Thus we have
	
	$$(1+q)x_\psi(q)+\sum_{\sigma\in E_\psi,\sigma\not=\psi}q^{1/2}x_\sigma(q)=$$
	
	$$-q^{1/2}x_{\psi(2,i-1)}x_{\psi(i+2,N)}\Q{n+r}^{i/2}\prn{1-1}=0$$
	
	as desired, and the last case is proved. Note that this only relies on the inductive hypothesis from case 3, for which we showed the base case holds.
	\end{enumerate}

Thus our inductive hypotheses have all been proven, and those that apply in the base case hold in the base case, so by induction the theorem is proved.

\end{proof}

\begin{corollary}
	If $e=n+r+1$, $M_{2n+r}^r$ is reducible, and has a unique sign subrepresentation.
\end{corollary}

\end{document}

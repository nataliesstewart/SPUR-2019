\documentclass{amsart}   
\usepackage{RepStyle}
\begin{document}
\title{There is no Fibonacci Representation of $\SH_{k,q}(S_{n})$}
\author{Miles Johnson \& Natalie Stewart}
\maketitle

Let $k$ be a field, and let $q \neq 0 \in k$ be the parameter of the Hecke algebra $\SH := \SH_{k,q}(S_n)$.
We will show that there is no ``Fibonacci'' representation of $\SH$.

First, let $F$ be the vector space with basis given by the strings of length $n + 1$ with alphabet $\cbr{p,*}$ such that no two $*$ symbols appear in a row.
This is given an action by the braid group that we will try to emulate.
We want an action which is ``local'', i.e. the simple transposition $T_i$ acts on the string from the $i$th to the ($i + 2$)nd symbol, modifying only the middle character, defined by the following rule:
\begin{align}
  \begin{split}
  \widehat{(*pp)} &:= a(*pp)\\
  \widehat{(*p*)} &:= b(*p*)\\
  \widehat{(p*p)} &:= c(p*p) + d(ppp)\\
  \widehat{(pp*)} &:= a(pp*)\\
  \widehat{(ppp)} &:= d(p*p) + e(ppp).
  \end{split}
  \label{Definitions}
 \end{align} 
For suitable constants $a,b,c,d,e \in k$.
\subsection*{Quadratic Relation}
Note that the quadratic relation $T_i^2 = (q -1)T_i + q$ imposes the following restrictions on the constants:
\begin{align}
  \begin{split}
  a^2 &= (q-1)a + q\\
  b^2 &= (q-1)b + q\\
  c^2 + d^2 &= (q -1)c + q\\
  de &= (q-1)d\\
  e^2 + d^2 &= (q -1)e + q\\
  dc &= (q-1)d
\end{split}
  \label{Quadratic}
\end{align}
Note that we immediately have
\begin{align}
  \begin{split}
  a,b &\in \cbr{\frac{q - 1 \pm \sqrt{(q-1)^2 + 4q}}{2}}\\
  &= \cbr{\frac{q - 1 \pm (q + 1)}{2}}\\
  &= \cbr{-1,q}.
\end{split}
\label{ab}
\end{align}
Further, if $d = 0$ then we have $c,e \in \cbr{-1,q}$;
if $d \neq 0$ then we have that $c = e = (q - 1)$, and $d \in \cbr{\pm \sqrt{q}}$.
 
\subsection*{Braid Relations}
Here's where we'll run into some issues.
%First, the braid relation $T_iT_j = T_jT_i$, $\abs{i - j} > 1$ is satisfied, as the action $T_i$ will be entirely unaffected by $T_j$, and one can see that this allows them to easily commute my commitativity of multiplication in $k$.

We must verify the relation $T_iT_{i+1}T_i = T_{i+1}T_iT_{i+1}$.
Let's begin with the case $d \neq 0$, which we can verify on the following strings:
\begin{itemize}
  %\item $(*pp*)$ requires that $a^3 = a^3$, which is guaranteed.
  \item $(*ppp)$ requires that $abd = bcd + ade$ and $a^2e = ae^2 + bd^2$.
    % If $d = 0$, then we require that $ae^2 = 0$ with $a,e \neq 0$, a contradiction.
    % Hence we require that $d^2 = q$ and $c = e = (q-1)$.

    The first of the above equivalently requires
    \begin{align}
      ab &= b(q-1) + a(q-1)
    \label{appp}
    \end{align}
  
    \item $(pppp)$ requires that $acd + de^2 = ade$.
     Equivalently, we require that $e^2 = 0$, i.e. $d = q = 1$.
     Then, by \eqref{appp}, we require that $ab = 0$, but $a,b \neq 0$;
     this is a contradiction, so there are no constants $a,b,c,d,e$ which make this a representation of $\SH$.

     \iffalse
    These together imply that $a = b$, and that $a = 2(q-1)$;
     when $q \neq \frac{1}{2}, 2$ this contradicts \eqref{ab}.
  \item $(*p*p)$ requires that $b^2c = bc^2 + ad^2$ and $abd = bcd + ade$.
    Equivalently, we require that
    \begin{align}
      b^2(d^2-1) &= b(d^2-1)^2 + ad^2 \label{bq}\\
      ab &= b(q-1) + a(q - 1).\nonumber
    \end{align}
    In particular, we have that
    \[
      bd^4 + (a - 2b - b^2)d^2 + b - b^2 = 0 = ad^4 + (b - 2a)d^2 - a^2d + a. 
    \]
    Equivalently, $d$ is a root of the polynomial $(a-b)d^4 + (3b - 3a + b^2)d^2 - a^2d + (2b + b^2)$.
    \fi

      %$(*pp*), $(p*pp)$, $(pp*p)$, $(p*p*)$, $(ppp*)$.
\end{itemize}

Now suppose that $d = 0$.
One thing which is immediately clear is the decomposition into one-dimensional subrepresentations (the spans of each basis vector) if this is a representation.
Note that $(*ppp)$ requires that $a^2e = ae^2$, i.e. $a = e$.
Similarly, $(*p*p)$ requires that $b = c$.
Further, we still satisfy the relations for $(pppp)$, we always satisfy the relations for $(*pp*)$, and the rest of the strings are compatible by symmetry.
All that is left are the relations $T_iT_j = T_jT_i$ for $\abs{i - j} > 1$, which are easy to see.
Hence, for each $q$, there are 4 ``good'' actions on $F$, each of which decomposes into a direct sum of one-dimensional subrepresentations.

\subsection*{A More General Case}
Now, consider a modification of \eqref{Definitions}:
\begin{align}
  \begin{split}
  \widehat{(*pp)} &:= a(*pp)\\
  \widehat{(*p*)} &:= b(*p*)\\
  \widehat{(p*p)} &:= c(p*p) + d(ppp)\\
  \widehat{(pp*)} &:= g(pp*)\\
  \widehat{(ppp)} &:= f(p*p) + e(ppp).
  \end{split}
  \label{General Definitions}
\end{align} 
This time, the quadratic relation reads
\begin{align}
  \begin{split}
  a^2 &= (q-1)a + q\\
  b^2 &= (q-1)b + q\\
  g^2 &= (q-1)g + q\\
  c^2 + df &= (q -1)c + q\\
  de &= (q-1)d\\
  e^2 + df &= (q -1)e + q\\
  fc &= (q-1)f
\end{split}
  \label{General Quadratic}
\end{align}
Notably, we have $a,b,g \in \cbr{-1.q}$ still.
If $d = 0$ or $f = 0$, then $c,e \in \cbr{-1,q}$ still, and if $d = 0$ and $f \neq 0$, then $c = (q-1)$, and hence $c = 1$ and $q = 2$.
If $d \neq 0$, we still have that $c = e = (q-1)$, and we have that $df = q$.

Now, suppose that $d,f \neq 0$.
Now, $(*ppp)$ requires that $abf = bcf + aef$ and $a^2e = ae^2 + bf^2$, so as before we have that $ab = (a+b)(q-1)$.

$(pppp)$ now requires that $aef = acf + e^2f$.
This requires that $e^2 = 0$, so that $q = 1$ and $ab = 0$, a contradiction again.

Now, suppose that $f \neq 0$ and $d = 0$, so that $q = 2$ and $c = 1$.
Then, we have that $a(e-1) = e^2$.
Then, knowing that $e^2 \neq 0$, we have $e = -1$ so $-2a = 1$, a contradiction.
By symmetry, we also arrive at a contradiction if $d \neq 0$ and $f = 0$.

Finally, suppose that $d = f = 0$, and note that we now have $a = e = g$ and $b = c$;
hence our case is precisely the case with $d = f$ and $a = g$, and there are exactly four actions of $\SH$ on $F$ on which each $T_i$ acts analogously on positions $i,i+1,i+2$ as each other, only modifying position $i+1$. 
Each of these actions decomposes into a direct sum of 1-dimensional subrepresentations.

% this is actually wrong, we need that de = dc = 0, so if d is nonzero then c,e are 0 and the braid relations may give some more interesting restrictions?
\iffalse

\subsection*{The Degenerate Case}
For completeness, let's actually consider the case $q = 0$;
then, as seen above, we actually require that $d = 0$, so that $a,b,c,e \in \cbr{-1,0}$. 

Then, the $(*ppp)$ braid relation we require that $ae^2 = 0$, so that at least one of $a,e$ is 0.
As far as I can tell, there are no other restrictions, so there are 12 possible ``Fibonacci'' representations for the case $q = 0$.
\fi
\end{document}

\documentclass{amsart}   
\usepackage{RepStyle} 
\usepackage{NatMacros}

% Emulates \subsection* but doesn't add to ToC (I don't know why amsart is weird 
\newcommand{\fakesubsection}[1]{
  \vspace{7pt}
  \noindent \textbf{#1.}
}

\makeatletter
%Table of Contents
\setcounter{tocdepth}{3}

% Add bold to \section titles in ToC and remove . after numbers
\renewcommand{\tocsection}[3]{%
  \indentlabel{\@ifnotempty{#2}{\bfseries\ignorespaces#1 #2\quad}}\bfseries#3}
% Remove . after numbers in \subsection
\renewcommand{\tocsubsection}[3]{%
  \indentlabel{\@ifnotempty{#2}{\ignorespaces#1 #2\quad}}#3}
\let\tocsubsubsection\tocsubsection% Update for \subsubsection
%...

\newcommand\@dotsep{4.5}
\def\@tocline#1#2#3#4#5#6#7{\relax
  \ifnum #1>\c@tocdepth % then omit
  \else
    \par \addpenalty\@secpenalty\addvspace{#2}%
    \begingroup \hyphenpenalty\@M
    \@ifempty{#4}{%
      \@tempdima\csname r@tocindent\number#1\endcsname\relax
    }{%
      \@tempdima#4\relax
    }%
    \parindent\z@ \leftskip#3\relax \advance\leftskip\@tempdima\relax
    \rightskip\@pnumwidth plus1em \parfillskip-\@pnumwidth
    #5\leavevmode\hskip-\@tempdima{#6}\nobreak
    \leaders\hbox{$\m@th\mkern \@dotsep mu\hbox{.}\mkern \@dotsep mu$}\hfill
    \nobreak
    \hbox to\@pnumwidth{\@tocpagenum{\ifnum#1=1\bfseries\fi#7}}\par% <-- \bfseries for \section page
    \nobreak
    \endgroup
  \fi}

\def\l@subsection{\@tocline{2}{0pt}{2.5pc}{5pc}{}}
\makeatother

\begin{document}

In the case that $e > n + r + 1$, we have confirmed that there exists an iso $\varphi_{2n + r}^r:W \rightarrow V'$ for appropriate subrepresentation $V'$ and quotient $W$ of $M$.
\def\tho{\Phi_3(q^{1/2})}
\def\tht{\brk{3}_{q^{1/2}}}
\def\to{q^{1/2}\brk{2}_{q^{1/2}}}
Define the following coefficients:
\begin{align*}
  \tho &:= q^{3/2} + 1\\
  \tht &:= q^{3/2} + q + q^{1/2} + 1\\
  \to &:= q + q^{1/2}
\end{align*}
Then, we may empirically compute the following:
\begin{align*}
  \varphi_6^0 = \varphi_5^1 &= \begin{bmatrix}
    0 & 0 & -\tho & 0 & 0\\
    0 & -\tho & \tht & 0 & 0\\
    0 & 0 & \tht & 0 & -\tho\\
    -\tht & \tht & \to & 0 & \tht\\
    \tht & 0 & \tht & -\tht & 0
  \end{bmatrix}\\
  \varphi_5^3 &= \begin{bmatrix}
    -\to & \tht & q^{1/2}\\
    q^{1/2} & q^{3/2} & 0\\
    0 & -q - 1
  \end{bmatrix}\\
  \varphi_4^0 &= \begin{bmatrix}
    0 & -\tht\\
    -1 & 1
  \end{bmatrix}\\
  \varphi_4^2 &= \begin{bmatrix}
    0 & \tht & 0\\
    1 & -1 & 0\\
    -1 & 0 & 1
  \end{bmatrix}\\
  \varphi_3^1 &= \begin{bmatrix}
    0 & -\tht\\
    \to & -\to
  \end{bmatrix}
\end{align*}
\iffalse
[0 0 v^3 - 1 0 0]
[0 (v^3 - 1)*L.1 -v^3 + v^2 - v + 1 0 0]
[0 0 -v^3 + v^2 - v + 1 0 (v^3 - 1)*L.1]
[v^3 - v^2 + v - 1 (-v^3 + v^2 - v + 1)*L.1 v^2 - v 0 (-v^3 + v^2 - v + 1)*L.1]
[-v^3 + v^2 - v + 1 0 -v^3 + v^2 - v + 1 (v^3 - v^2 + v - 1)*L.1 0]

[0 0 -v^2 - 1 0 0 0 0 0]
[0 (-v^2 - 1)*L.1 v 0 0 0 0 0]
[0 (v^2 + 1)*L.1 0 -v 0 0 0 0]
[0 0 v 0 0 0 (-v^2 - 1)*L.1 0]
[-v v*L.1 -v^2 + v - 1 0 0 0 v*L.1 0]
[v -v*L.1 0 v^2 - v + 1 0 0 0 -v*L.1]
[v 0 v 0 -v*L.1 0 0 0]
[-v 0 0 0 v*L.1 -v^2 + v - 1 0 v*L.1]

[0 0 v^3 - 1 0 0]
[0 (v^3 - 1)*L.1 -v^3 + v^2 - v + 1 0 0]
[0 0 -v^3 + v^2 - v + 1 0 (v^3 - 1)*L.1]
[v^3 - v^2 + v - 1 (-v^3 + v^2 - v + 1)*L.1 v^2 - v 0 (-v^3 + v^2 - v + 1)*L.1]
[-v^3 + v^2 - v + 1 0 -v^3 + v^2 - v + 1 (v^3 - v^2 + v - 1)*L.1 0]

[                -v^2 + v (-v^3 + v^2 - v + 1)*L.1                   -v*L.1]
[                      -v                 -v^3*L.1                        0]
[                       0           (-v^2 - 1)*L.1                        0]

[                  0 (v^3 - v^2 - 1)*L.1]
[                 -1                 L.1]

[                   0 (-v^3 + v^2 + 1)*L.1                    0]
[                   1                 -L.1                    0]
[                  -1                    0                  L.1]

[                      0 (v^3 - v^2 + v - 1)*L.1]
[                v^2 - v          (-v^2 + v)*L.1]
I really don't know what happened here

[0 0 (-v^3 + 1)*L.1 0 0]
[0 -v^3 + v^2 - v + 1 (v^3 - v^2 + v - 1)*L.1 0 0]
[0 0 (v^3 - v^2 + v - 1)*L.1 0 -v^3 + v^2 - v + 1]
[(-v^3 + v^2 - v + 1)*L.1 v^2 - v (-v^2 + v)*L.1 0 v^2 - v]
[(v^3 - v^2 + v - 1)*L.1 0 (v^3 - v^2 + v - 1)*L.1 -v^2 + v 0]

[0 0 (-v^3 + 1)*L.1 0 0]
[0 -v^3 + v^2 - v + 1 (v^3 - v^2 + v - 1)*L.1 0 0]
[0 0 (v^3 - v^2 + v - 1)*L.1 0 -v^3 + v^2 - v + 1]
[(-v^3 + v^2 - v + 1)*L.1 v^2 - v (-v^2 + v)*L.1 0 v^2 - v]
[(v^3 - v^2 + v - 1)*L.1 0 (v^3 - v^2 + v - 1)*L.1 -v^2 + v 0]


\fi
\setcounter{MaxMatrixCols}{20}
% This replaces $q^{1/2}(q-1)$, v^3 - v^2 + 1, -v^3 + 2v
\def\qqo{\alpha}
\def\tto{\beta}
\def\vtv{\gamma}
\def\fft{\delta}
\def\fto{\varepsilon}

\begin{align*} 
\iota_{3,2,1}&=  
\begin{bmatrix}
  1 & 1
\end{bmatrix}\\
\iota_{3,4,0}&=  
\begin{bmatrix}
  1 & 1
\end{bmatrix}\\
\iota_{3,4,1}&=  
\begin{bmatrix}
  1 & 0 & 0 & 0 & 0\\
  0 & 0 & 1 & 1 & 0\\
  0 & 1 & 0 & 1 & 0\\
  0 & 0 & 0 & 1 & 1
\end{bmatrix}\\
\iota_{3,2,4}&=  
\begin{bmatrix}
  -1 & -1 & 0 & 1 & 1
\end{bmatrix}\\
\iota_{3,6,0}&=  
\begin{bmatrix}
  1 & 0 & 0 & 0 & 0\\
  0 & 0 & 1 & 1 & 0\\
  0 & 1 & 0 & 1 & 0\\
  0 & 0 & 0 & 1 & 1
\end{bmatrix}\\
\iota_{3,4,3}&=  
\begin{bmatrix}
 1 & 0 &-1 &-1 & 1 & 0 & -1 & -1 & 0 & 0 & 0 & 0 & 1 & 1
\end{bmatrix}\\
\iota_{3,6,1}&=  
\begin{bmatrix}
 1 & 0 & 0 & 0 & 0 & 0 & 0 & 0 & 0 & 0 & 0 & 0 & 0 &-1\\
 0 & 1 & 0 & 0 & 0 & 0 & 0 & 0 & 0 & 0 & 0 & 0 & 0 & 1\\
 0 & 0 & 1 & 0 & 0 & 0 & 0 & 0 & 0 & 0 & 0 & 0 & 0 & 1\\
 0 & 0 & 0 & 1 & 0 & 0 & 0 & 0 & 0 & 0 & 0 & 0 & 0 &-1\\
 0 & 0 & 0 & 0 & 1 & 0 & 0 & 0 & 0 & 0 & 0 & 0 & 0 & 1\\
 0 & 0 & 0 & 0 & 0 & 1 & 0 & 0 & 0 & 0 & 0 & 0 & 0 & 1\\
 0 & 0 & 0 & 0 & 0 & 0 & 1 & 0 & 0 & 0 & 0 & 0 & 0 &-1\\
 0 & 0 & 0 & 0 & 0 & 0 & 0 & 1 & 0 & 0 & 0 & 0 & 0 &-1\\
 0 & 0 & 0 & 0 & 0 & 0 & 0 & 0 & 1 & 0 & 0 & 0 & 0 & 1\\
 0 & 0 & 0 & 0 & 0 & 0 & 0 & 0 & 0 & 1 & 0 & 0 & 0 & 1\\
 0 & 0 & 0 & 0 & 0 & 0 & 0 & 0 & 0 & 0 & 1 & 0 & 0 &-1\\
 0 & 0 & 0 & 0 & 0 & 0 & 0 & 0 & 0 & 0 & 0 & 1 & 0 &-1\\
 0 & 0 & 0 & 0 & 0 & 0 & 0 & 0 & 0 & 0 & 0 & 0 & 1 & 1
\end{bmatrix}
\end{align*}

\iffalse

Explicit inclusions given in all cases
e=3      n=2     r=1
[1 1]
e=3      n=4     r=0
[1 1]
e=3      n=4     r=1
[1 0 0 0 1]
[0 0 1 1 0]
[0 1 0 1 0]
[0 0 0 1 1]
e=3      n=2     r=4
[-1 -1  0  1  1]
e=3      n=6     r=0
[1 0 0 0 1]
[0 0 1 1 0]
[0 1 0 1 0]
[0 0 0 1 1]
e=3      n=4     r=3
[ 1  0 -1 -1  1  0 -1 -1  0  0  0  0  1  1]
e=3      n=6     r=1
[ 1  0  0  0  0  0  0  0  0  0  0  0  0 -1]
[ 0  1  0  0  0  0  0  0  0  0  0  0  0  1]
[ 0  0  1  0  0  0  0  0  0  0  0  0  0  1]
[ 0  0  0  1  0  0  0  0  0  0  0  0  0 -1]
[ 0  0  0  0  1  0  0  0  0  0  0  0  0  1]
[ 0  0  0  0  0  1  0  0  0  0  0  0  0  1]
[ 0  0  0  0  0  0  1  0  0  0  0  0  0 -1]
[ 0  0  0  0  0  0  0  1  0  0  0  0  0 -1]
[ 0  0  0  0  0  0  0  0  1  0  0  0  0  1]
[ 0  0  0  0  0  0  0  0  0  1  0  0  0  1]
[ 0  0  0  0  0  0  0  0  0  0  1  0  0 -1]
[ 0  0  0  0  0  0  0  0  0  0  0  1  0 -1]
[ 0  0  0  0  0  0  0  0  0  0  0  0  1  1]
e=4      n=2     r=2
[       1 -v^3 + v        1]
e=4      n=4     r=1
[             1 1/2*(-v^3 + v) 1/2*(-v^3 + v)              1 1/2*(-v^3 + v)]
e=4      n=4     r=2
[1 0 0 0 0 0 -v^3 + v 1 0]
[0 1 0 0 0 -1 -1 v^3 - v 0]
[0 0 1 0 0 -v^3 + v 0 1 0]
[0 0 0 1 0 0 -1 v^3 - v -1]
[0 0 0 0 1 1/2*(-v^3 + v) 1/2*(-v^3 + v) 1 1/2*(-v^3 + v)]
e=4      n=6     r=0
[-v^3 + v        1        1 -v^3 + v        1]
e=4      n=6     r=1
[1 0 0 0 1/2*(-v^3 + v) 0 -1/2 -1/2 0 0 1 1/2 -v^3 + v 1/2]
[0 1 0 0 -1 0 1/2*(-v^3 + v) 1/2*(-v^3 + v) 1 0 v^3 - v 1/2*(-v^3 + v) -1 
    1/2*(v^3 - v)]
[0 0 1 0 -1 0 0 -v^3 + v 1 0 v^3 - v 0 -1 0]
[0 0 0 1 -v^3 + v 0 0 -1 v^3 - v 0 1 -1 0 0]
[0 0 0 0 0 1 1/2*(-v^3 + v) 1/2*(-v^3 + v) 0 0 0 1/2*(v^3 - v) -1 1/2*(v^3 - v)]
[0 0 0 0 0 0 0 0 0 1 1/2*(-v^3 + v) 1/2*(-v^3 + v) 1 1/2*(-v^3 + v)]
e=5      n=2     r=3
[             1 -v^3 + v^2 + 1 -v^3 + v^2 + 1              1]
e=5      n=4     r=2
[-v^3 + v^2 + 1 -v^3 + v^2 + 1 1 1 -v^3 + v^2 + 2 -v^3 + v^2 + 1 -v^3 + v^2 + 1 -v^3 + v^2 + 1 1]
e=5      n=4     r=3
[1 -v^3 + v^2 + 1 -v^3 + v^2 + 1 1 0 0 0 0 0 0 0 0 0 0]
[0 0 0 0 1 -v^3 + v^2 + 1 -v^3 + v^2 + 1 1 0 0 0 0 0 0]
[1 0 0 0 0 0 0 0 -v^3 + v^2 + 1 -v^3 + v^2 + 1 1 0 0 0]
[-v^3 + v^2 + 1 -v^3 + v^2 + 1 1 0 1 -v^3 + v^2 + 2 -v^3 + v^2 + 1 0 -v^3 + v^2  + 1 -v^3 + v^2 + 1 0 1 0 0]
[v^3 - v^2 - 2 v^3 - v^2 - 1 v^3 - v^2 - 1 0 v^3 - v^2 - 1 v^3 - v^2 - 2 v^3 - 
    v^2 - 2 0 v^3 - v^2 - 2 v^3 - v^2 - 2 0 0 1 0]
[0 0 1 0 0 0 -v^3 + v^2 + 1 0 0 -v^3 + v^2 + 1 0 0 0 1]
e=5      n=6     r=1
[1 -v^3 + v^2 -v^3 + v^2 -v^3 + v^2 v^3 - v^2 + 1 -v^3 + v^2 v^3 - v^2 + 1 -v^3 
    + v^2 v^3 - v^2 + 1 1 -v^3 + v^2 -v^3 + v^2 -v^3 + v^2 v^3 - v^2 + 1]
e=6      n=2     r=4
[         1 -v^3 + 2*v          2 -v^3 + 2*v          1]
e=6      n=4     r=3
[-v^3 + 2*v 2 -v^3 + 2*v 1 1 -2*v^3 + 4*v 3 -v^3 + 2*v 2 -2*v^3 + 4*v 2 2 -v^3 +
    2*v 1]
e=7      n=2     r=5
[1 -v^5 + v^4 - v^3 + v^2 + 1 -v^5 + v^2 + 1 -v^5 + v^2 + 1 -v^5 + v^4 - v^3 + 
    v^2 + 1 1]
\fi
\begin{align*}
\iota_{4,2,2}&=  
\begin{bmatrix}
  1 & \qqo & 1
\end{bmatrix}\\
\iota_{4,2,1}&=  
\begin{bmatrix}
  1 & \frac{1}{2}\qqo & \frac{1}{2}\qqo & 1 & \frac{1}{2}\qqo
\end{bmatrix}\\
\iota_{4,4,2} &=
\begin{bmatrix}
  1& 0& 0& 0& 0& 0& \qqo & 1& 0\\
  0& 1& 0& 0& 0& -1& -1& -\qqo &0\\
  0& 0& 1& 0& 0& \qqo& 0 &1 &0\\
  0& 0& 0& 1& 0& 0& -1& -\qqo &-1\\
  0& 0& 0& 0& 1& \frac{1}{2}\qqo& \frac{1}{2}\qqo &1 &\frac{1}{2}\qqo
\end{bmatrix}\\
\iota_{4,6,0} &=
\begin{bmatrix}
  \qqo & 1 & 1 & \qqo & 1
\end{bmatrix}\\
\iota_{4,6,1} &=
\begin{bmatrix}
1& 0& 0& 0& \frac{1}{2}\qqo& 0& -\frac{1}{2}& -\frac{1}{2}& 0& 0& 1& 1/2& \qqo& \frac{1}{2}\\
0& 1& 0& 0& -1& 0& \frac{1}{2}\qqo& \frac{1}{2}\qqo& 1& 0& \qqo& \frac{1}{2}\qqo& -1 & \frac{1}{2}\qqo\\
0& 0& 1& 0& -1& 0& 0& \qqo& 1& 0& -\qqo& 0& -1& 0\\
0& 0& 0& 1& \qqo& 0& 0& -1& -\qqo& 0& 1& -1& 0& 0\\
0& 0& 0& 0& 0& 1& \frac{1}{2}\qqo& \frac{1}{2}\qqo& 0& 0& 0& \frac{1}{2}\qqo& -1& \frac{1}{2}\qqo\\
0& 0& 0& 0& 0& 0& 0& 0& 0& 1& -\frac{1}{2}\qqo& \frac{1}{2}\qqo& 1& \frac{1}{2}\qqo
\end{bmatrix}\\
\iota_{5,2,3} &=
\begin{bmatrix}
  1 & \tto & \tto & 1
\end{bmatrix}\\
\iota_{5,4,2} &=
\begin{bmatrix}
  \tto& \tto& 1& 1& \tto+1& \tto & \tto & \tto & 1
\end{bmatrix}\\
\iota_{5,4,3} &=
\begin{bmatrix}
1& \tto& \tto& 1& 0& 0& 0& 0& 0& 0& 0& 0& 0& 0\\
0& 0& 0& 0& 1& \tto& \tto& 1& 0& 0& 0& 0& 0& 0\\
1& 0& 0& 0& 0& 0& 0& 0& \tto& \tto& 1& 0& 0& 0\\
\tto& \tto& 1& 0& 1& \tto+1& \tto& 0& \tto& \tto& 0& 1& 0& 0\\
-\tto-1& -\tto& -\tto& 0& -\tto& -\tto-1& -\tto-1& 0& -\tto-1& -\tto-1& 0& 0& 1& 0\\
0& 0& 1& 0& 0& 0& \tto& 0& 0& \tto& 0& 0& 0& 1
\end{bmatrix}\\
\iota_{5,6,1} &=
\begin{bmatrix}
1& \qqo& \qqo& \qqo& -\tto& \qqo& -\tto& \qqo& -\tto& 1& \qqo& \qqo& \qqo& -\tto
\end{bmatrix}\\
\iota_{6,2,4} &=
\begin{bmatrix}
  1 & \vtv & 2 & \vtv & 1 
\end{bmatrix}\\
\iota_{6,4,3} &=
\begin{bmatrix}
  \vtv & 2 & \vtv & 1 & 1 & 2\vtv & 3 & \vtv & 2 & 2\vtv & 2 & 2 & \vtv & 1
\end{bmatrix}\\
\iota_{6,4,3} &=
\begin{bmatrix}
  1 & \fft & \fto & \fto & \fft & 1
\end{bmatrix}
\end{align*}

\iffalse
[         1 -v^3 + 2*v          2 -v^3 + 2*v          1]
e=6      n=4     r=3
[-v^3 + 2*v 2 -v^3 + 2*v 1 1 -2*v^3 + 4*v 3 -v^3 + 2*v 2 -2*v^3 + 4*v 2 2 -v^3 +
    2*v 1]
e=7      n=2     r=5
[1 -v^5 + v^4 - v^3 + v^2 + 1 -v^5 + v^2 + 1 -v^5 + v^2 + 1 -v^5 + v^4 - v^3 + 
    v^2 + 1 1]
\fi

The following illustrates triviality of the kernel at various $e$:

\ENRGrid{13/7}{2/1,5/1,8/1,11/1,1/2,4/2,7/2,10/2}

\ENRGrid{13/7}{3/1,7/1,11/1,2/2,6/2,10/2,1/3,5/3,9/3}

\ENRGrid{13/7}{4/1,9/1,3/2,8/2,2/3,7/3,1/4,6/4}

\ENRGrid{13/7}{5/1,11/1,4/2,10/2,3/3,9/3,2/4,1/5}

\ENRGrid{13/7}{6/1,13/1,5/2,4/3,3/4,2/5,1/6}

\ENRGrid{13/7}{7/1,6/2,5/3,4/4,3/5,2/6,1/7}

\ENRGrid{13/7}{8/1,7/2,6/3,5/4,4/5,3/6}

\ENRGrid{13/7}{9/1,8/2,7/3,6/4,5/5}

\ENRGrid{13/7}{10/1,9/2,8/3,7/4}

\ENRGrid{13/7}{11/1,10/2,9/3}

\ENRGrid{13/7}{12/1,11/2}

\end{document}

\documentclass{amsart}   
\usepackage{RepStyle} 
\usepackage{NatMacros}

% Emulates \subsection* but doesn't add to ToC (I don't know why amsart is weird 
\newcommand{\fakesubsection}[1]{
  \vspace{7pt}
  \noindent \textbf{#1.}
}

\makeatletter
%Table of Contents
\setcounter{tocdepth}{3}

% Add bold to \section titles in ToC and remove . after numbers
\renewcommand{\tocsection}[3]{%
  \indentlabel{\@ifnotempty{#2}{\bfseries\ignorespaces#1 #2\quad}}\bfseries#3}
% Remove . after numbers in \subsection
\renewcommand{\tocsubsection}[3]{%
  \indentlabel{\@ifnotempty{#2}{\ignorespaces#1 #2\quad}}#3}
\let\tocsubsubsection\tocsubsection% Update for \subsubsection
%...

\newcommand\@dotsep{4.5}
\def\@tocline#1#2#3#4#5#6#7{\relax
  \ifnum #1>\c@tocdepth % then omit
  \else
    \par \addpenalty\@secpenalty\addvspace{#2}%
    \begingroup \hyphenpenalty\@M
    \@ifempty{#4}{%
      \@tempdima\csname r@tocindent\number#1\endcsname\relax
    }{%
      \@tempdima#4\relax
    }%
    \parindent\z@ \leftskip#3\relax \advance\leftskip\@tempdima\relax
    \rightskip\@pnumwidth plus1em \parfillskip-\@pnumwidth
    #5\leavevmode\hskip-\@tempdima{#6}\nobreak
    \leaders\hbox{$\m@th\mkern \@dotsep mu\hbox{.}\mkern \@dotsep mu$}\hfill
    \nobreak
    \hbox to\@pnumwidth{\@tocpagenum{\ifnum#1=1\bfseries\fi#7}}\par% <-- \bfseries for \section page
    \nobreak
    \endgroup
  \fi}

\def\l@subsection{\@tocline{2}{0pt}{2.5pc}{5pc}{}}
\makeatother

\begin{document}

In the case that $e > n + r + 1$, we have confirmed that there exists an iso $\varphi_{2n + r}^r:W \rightarrow V'$ for appropriate subrepresentation $V'$ and quotient $W$ of $M$.
\def\tho{\Phi_3(q^{1/2})}
\def\tht{\brk{3}_{q^{1/2}}}
\def\to{q^{1/2}\brk{2}_{q^{1/2}}}
Define the following coefficients:
\begin{align*}
  \tho &:= q^{3/2} + 1\\
  \tht &:= q^{3/2} + q + q^{1/2} + 1\\
  \to &:= q + q^{1/2}
\end{align*}
Then, we may empirically compute the following:
\begin{align*}
  \varphi_6^0 = \varphi_5^1 &= \begin{bmatrix}
    0 & 0 & -\tho & 0 & 0\\
    0 & -\tho & \tht & 0 & 0\\
    0 & 0 & \tht & 0 & -\tho\\
    -\tht & \tht & \to & 0 & \tht\\
    \tht & 0 & \tht & -\tht & 0
  \end{bmatrix}\\
  \varphi_5^3 &= \begin{bmatrix}
    -\to & \tht & q^{1/2}\\
    q^{1/2} & q^{3/2} & 0\\
    0 & -q - 1
  \end{bmatrix}\\
  \varphi_4^0 &= \begin{bmatrix}
    0 & -\tht\\
    -1 & 1
  \end{bmatrix}\\
  \varphi_4^2 &= \begin{bmatrix}
    0 & \tht & 0\\
    1 & -1 & 0\\
    -1 & 0 & 1
  \end{bmatrix}\\
  \varphi_3^1 &= \begin{bmatrix}
    0 & -\tht\\
    \to & -\to
  \end{bmatrix}
\end{align*}
\iffalse
[0 0 v^3 - 1 0 0]
[0 (v^3 - 1)*L.1 -v^3 + v^2 - v + 1 0 0]
[0 0 -v^3 + v^2 - v + 1 0 (v^3 - 1)*L.1]
[v^3 - v^2 + v - 1 (-v^3 + v^2 - v + 1)*L.1 v^2 - v 0 (-v^3 + v^2 - v + 1)*L.1]
[-v^3 + v^2 - v + 1 0 -v^3 + v^2 - v + 1 (v^3 - v^2 + v - 1)*L.1 0]

[0 0 v^3 - 1 0 0]
[0 (v^3 - 1)*L.1 -v^3 + v^2 - v + 1 0 0]
[0 0 -v^3 + v^2 - v + 1 0 (v^3 - 1)*L.1]
[v^3 - v^2 + v - 1 (-v^3 + v^2 - v + 1)*L.1 v^2 - v 0 (-v^3 + v^2 - v + 1)*L.1]
[-v^3 + v^2 - v + 1 0 -v^3 + v^2 - v + 1 (v^3 - v^2 + v - 1)*L.1 0]

[                -v^2 + v (-v^3 + v^2 - v + 1)*L.1                   -v*L.1]
[                      -v                 -v^3*L.1                        0]
[                       0           (-v^2 - 1)*L.1                        0]

[                  0 (v^3 - v^2 - 1)*L.1]
[                 -1                 L.1]

[                   0 (-v^3 + v^2 + 1)*L.1                    0]
[                   1                 -L.1                    0]
[                  -1                    0                  L.1]

[                      0 (v^3 - v^2 + v - 1)*L.1]
[                v^2 - v          (-v^2 + v)*L.1]
\fi

The following illustrates triviality of the kernel at various $e$:

\ENRGrid{13/7}{2/1,5/1,8/1,11/1,1/2,4/2,7/2,10/2}

\ENRGrid{13/7}{3/1,7/1,11/1,2/2,6/2,10/2,1/3,5/3,9/3}

\ENRGrid{13/7}{4/1,9/1,3/2,8/2,2/3,7/3,1/4,6/4}

\ENRGrid{13/7}{5/1,11/1,4/2,10/2,3/3,9/3,2/4,1/5}

\ENRGrid{13/7}{6/1,13/1,5/2,4/3,3/4,2/5,1/6}

\ENRGrid{13/7}{7/1,6/2,5/3,4/4,3/5,2/6,1/7}

\ENRGrid{13/7}{8/1,7/2,6/3,5/4,4/5,3/6}

\ENRGrid{13/7}{9/1,8/2,7/3,6/4,5/5}

\ENRGrid{13/7}{10/1,9/2,8/3,7/4}

\ENRGrid{13/7}{11/1,10/2,9/3}

\ENRGrid{13/7}{12/1,11/2}

\end{document}

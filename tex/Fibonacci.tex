\documentclass{amsart}   
\usepackage{RepStyle}
\usepackage{ytableau}

\begin{document}
\title{Whoops, there is a Fibonacci Representation of $\SH_{k,q}(S_{n})$}
\author{Miles Johnson \& Natalie Stewart}
\maketitle


Let $k$ be a field, and let $q \neq 0 \in k$ be the parameter of the Hecke algebra $\SH := \SH_{k,q}(S_n)$.
Let $F^m$ be the vector space with basis given by the strings $\cbr{*,p}^{m+2}$ in which the character $*$ doesn't appear twice in a row.
We will act on this by $\SH$ by defining the action of $T_i$ ``locally'', and changing character $i+1$ only dependent on characters $i,i+1,i+2$.
We will use ``hat'' notation e.g. $(1 + T_1)(pppp) = (\widehat{ppp}p)$.
Then, we may define our action as follows:
\begin{align*}
  \widehat{(*pp)} &:= a(*pp)\\
  \widehat{(*p*)} &:= b(*p*)\\
  \widehat{(p*p)} &:= c(p*p) + d(ppp)\\
  \widehat{(pp*)} &:= a(pp*)\\
  \widehat{(ppp)} &:= d(p*p) + e(ppp).
\end{align*}
for constants
\begin{align*}
  a &= -1\\
  b &= q\\
  c &= \tau(q\tau - 1)\\
  d &= \tau^{3/2}(q + 1)\\
  e &= \tau(q-\tau)\\
  \tau &= \frac{2}{1 + \sqrt 5}.
\end{align*}
It is easy to verify that these are compatible with the quadratic and braid relations, and hence define a representation $F^m$ of $\SH$, henceforth referred to as the Fibonacci representation.
This has 4 subrepresentations dependent on the first and last character.
We will characterize these subrepresentations fully in the following section.

\section{Characterization of $F^m$ via Specht Modules}
Let $F^{m} = F^{m}_{*p} \oplus F^{m}_{p*} \oplus F^m_{**} \oplus F^m_{pp}$ be the decomposition of the Fibonacci representation into the 4 subrepresentations depending on the values of the first and last character.
We'll supress the superscripts when the dimension is clear.

Note that $F^m$ has dimension the $m + 1$st fibonacci number $f_{m+2}$, we have $\dim F^m_{**} = f_{m+1}$, $F^m_{*p} \simeq F^m_{p*}$ has dimension $f_m$, and $\dim F^m_{pp} = f_{n-1}$.
Further, note that $m = 2n$ gives that $\dim F^{2m}_{pp} = \dim D^{(2,\dots,2)}$;
we will prove that these modules are isomorphic via the following propositions:
\begin{enumerate}
  \item $F^m_{*p}$ is irreducible, and $\Res \; F^m_{*p} \simeq F^{m-1}_{pp} \simeq F^{m-1}_{*p} \oplus F^{m-1}_{**}$.
  \item $\Res \; F^m_{**} \simeq F^{m-1}_{*p}$.
  \item $F^m$ decomposes into a direct sum of irreducible representations:
    \[
      F^m \simeq 3 F^m_{*p} \oplus 2F_{**}^m
    \]
  \i tem Let $D_{m,k} := D^{(m,m-k)'}$ be the nearly-two-column Specht module. Then,
    \begin{align*}
      \Res \; D_{m,0} &\simeq D_{m-1,1}\\
      \Res \; D_{m,1} &\simeq D_{m-1,0} \oplus D_{m-1,2}\\
      \Res \; D_{m,2} &\simeq D_{m-1,1} \oplus D_{m-1,3}\\
      \Res \; D_{m,3} &\simeq D_{m-1,2}.
     \end{align*}
  \item The claims are henceforth conjectural: 
    \begin{align*}
      F^{2n}_{**} &\simeq D_{n,0}\\ 
      F^{2n-1}_{**} &\simeq D_{n+1,3}\\
      F^{2n}_{*p} &\simeq D_{n+1,2}\\
      F^{2n-1}_{*p} &\simeq D_{n,1}.
     \end{align*}
     If these are true, then
     \begin{align*}
        F^{2n} &\simeq 3D_{n+1,2} \oplus 2D_{n,0}\\
        F^{2n-1} &\simeq 3D_{n,1} \oplus 2D_{n+1,3}.
     \end{align*}
   \item In the $2n = 8$ case, let $K$ be the intersection of kernels of $(1 + T_i)$ for $W$;
    then, we have $W/K \simeq F_{**}$.
\end{enumerate}

We can start by studying low-dimensional cases.
First, note that $F_{*p}^2$ is the sign representation $D^{(2)}$ and $F_{**}^2$ is the trivial representation $D^{(1)^2}$.

$F_{pp}^2$, which is a 2-dimensional representation of a semisimple commutative algebra, and hence decomposes into a direct sum of two subrepresentations.
In particular, we can use the basis $\cbr{(p*p),(ppp)}$ and explicitly write the matrix
\[
  \rho_{T_1} = \begin{bmatrix}
    c & d\\
    d & e
  \end{bmatrix}
\]
having characteristic polynomial $(c - \lambda)(e - \lambda) - d^2 = \lambda^2 - (c + e)\lambda +(ce - d^2)$.
The reader can verify that this has roots are $-1$ and $q$.
The eigenspaces with eigenvalues $-1$ and $q$ are subrepresentations isomorphic to the sign and trivial representation, hence $F_{pp}$ is isomorphic to a direct sum of the trivial and sign representaitons: $F^2_{pp} \simeq F^2_{*p} \oplus F^2_{**}$. 

Now let's prove that $F^3_{**}$ is irreducible;
this has basis $\cbr{*p*p}, \cbr{*ppp}$, and the following matrices:
\[
  \rho_{T_1} = \begin{bmatrix}
    b & 0\\
    0 & a
  \end{bmatrix}; \hspace{20pt}
  \rho_{T_2} = \begin{bmatrix}
    c & d\\
    d & e
  \end{bmatrix}.
\]
A subrepresentation must be one-dimensional, and hence an eigenspace of each of these matrices;
since $b \neq a$, the first has eigenspaces given by the spans of basis elements, and since $d \neq 0$, these are not eigenspaces of the second.
Hence $F^3_{**}$ is irreducible.
Now we may move on to the general case.
\begin{proposition}
  The representation $F_{*p} := F^m_{*p}$ is irreducible.
\end{proposition}
\begin{proof}
  We will prove this inductively in $m$.
  We've already proven it for $F^3_{*p}$ and $F^4_{*p}$, so suppose that $F^{m-2}_{*p}$ is irreducible.
  
  Let $\cbr{v_i}$ be the basis for $F_{*p}$.
  Then, each $v_i$ is cyclic; indeed, we can transform every basis vector into $(*p\dots p)$ by multiplying by the appropriate $\frac{1}{d-c}(T_i - c)$, and we can transform $(*p\dots p)$ into any basis vector by multiplying be the appropriate $\frac{1}{d-e}(T_i - e)$.
  Hence it is sufficient to show that each $v \in F_{*p}$ generate some basis element.

  Let $v'$ be the basis element $(*p*p\dots p)$, which is many copies of $*p$, followed by an extra $p$ if $m$ is odd.
  We will show that each $v \in F$ generates $v'$.

  Say that a basis element $v_i$ is \emph{represented in $v$} if it has nonzero coefficient in $v$.
  Suppose that no elements beginning $(*p*p)$ are represented in $v_i$;
  then, all such elements are represented in $T_3v$, so we may assume that at least one is represented in $v$.

  Note that $(T_2 - a)v$ is a nonzero element where only elements beginning $(*p*p)$ are represented;
  if $F'$ is the subspace of $F_{*p}$ spanned by $v_i$ beginning $(*p*p)$, then $\Res_{\SH(S_{m-2})}^{\SH(S_m)} F' \simeq F^{m-2}_{*p}$, and $v'$ is mapped to the analogous element in $F^{m-2}_{*p}$.
  Hence irreducibility of $F^{m-2}_{*p}$ implies that $v'$ is generated by $(T_2 - a)$, and $F^m_{*p}$ is irreducible.
\end{proof}

There is a bit of bookkeeping to do;
we've equivocated by saying asserting that the subalgebras $\SH_{k,q}(S_{m-r}) \subset \SH_{k,q}(S_{m})$ are equivalent, inclding restrictions.
This works because they are conjugate, and conjugation gives an isomorphism of the restriction of a representation to separate subalgebras.

Knowing this, the restriction statements are clear;
$\Res F_{*p}^m \simeq F_{pp}^{m-1}$ by considering the last $m-2$ transpositions, and $\Res F_{*p}^{m-1} \simeq F_{*p}^{m-1} \oplus F_{**}^{m-1}$ by considering the first $m-2$.
Similarly, $\Res F_{**}^m \simeq F_{*p}^{m-1}$ by considering the first $m-2$ transpositions.
This gives that $F \simeq 3F_{*p} \oplus 2F_{**}$.

Now we may move on and use Young Tableau to characterize $F$.
Recall that the socle of $D^\lambda$ is given by $\bigoplus\limits_{\mu \xrightarrow{\text{good}} \lambda} D^\mu$,  and that $D^\lambda$ is semisimple iff every $\mu \xrightarrow{\text{normal}} \lambda$ is good.
\begin{proposition}
  The irreducible components of $F$ are given by the following isomorphisms:
    \begin{align*}
      F^{2n}_{**} &\simeq D^{(n,n)'}\\ 
      F^{2n-1}_{**} &\simeq D^{(n+1,n-2)'}\\
      F^{2n}_{*p} &\simeq D^{(n+1,n-1)'}\\
      F^{2n-1}_{*p} &\simeq D^{(n,n-1)'}.
     \end{align*}
\end{proposition}
\begin{proof}
  We will prove this by induction on $n$;
  we have already proven the base case $F^{2}$, so suppose that we have proven these isomorphisms for $F^{2n-2}$.
  We will prove the isomorphisms for $F^{2n-1}$ and $F^{2n}$.

  By semisimplicity, $F^{2n-1}_{**} \simeq D^{\lambda_{**}}$ and $F^{2n-1}_{*p} \simeq D^{\lambda_{*p}}$ for some diagrams $\lambda_{**}$ and $\lambda_{*p}$.
  We will show that $\lambda_{**} = (n+1,n-2)'$ and $\lambda_{*p} = (n+1,n-1)'$.
  
  First, note that we have \[\Res \; D^{\lambda_{**}} \simeq D^{(n,n-2)'} \simeq \Res \; D^{(n+1,n-2)'}\] and \[\Res \; D^{\lambda_{*p}} \simeq D^{(n,n-2)} \oplus D^{(n-1,n-1)} \simeq \Res \; D^{(n,n-1)'}.\]
  By semisimplicity, every normal cell in $\lambda_{**}$ and $\lambda_{*p}$ is good, and every good cell is removed in a summand of the restriction.
  
  In particular, for $\lambda_{**}$, the only tableaus which can remove a cell to yield $D^{(n,n-2)'}$ are $(n+1,n-2)'$, $(n,n-1)'$, and $(n,n-2,1)'$ as illustrated in Figure \ref{OddRes};
  we have already seen that $D^{(n,n-1)'}$ does not have irreducible restriction, so we are left with $(n+1,n-2)'$ and $(n,n-2,1)'$.
  To have irreducible restriction, $\lambda_{**}$ must have 1 as its only normal number;
  we may directly check that $(n,n-2,1)'$ doesn't satisfy this, as we have the following:
  \begin{align*}
    \beta_\lambda(1,2) &= 3 - 2 + (n-2) = n-1\\
    \beta_\lambda(1,3) &= 3 - 1 + n = n+2\\
    \beta_\lambda(2,3) &= 2 - 1 + 3 = 4.
   \end{align*}
  At least one of $\beta(1,2)$ and $\beta(1,3)$ is nonzero, and hence at least one of $M_2$ and $M_3$ is empty.
  Hence at least one of 2 or 3 is normal, and $\lambda_{**} = (n+1,n-2)$.

  For $\lambda_{*p}$, we immediately see from Figure \ref{OddRes} that the only option is $(n,n-1)$.
  \begin{figure}
    \[
      \cbr{
        \begin{gathered}
        \ydiagram{2,2,2,1,1,1}. \; \;
        \ydiagram{2,2,2,2,1}, \; \;
        \ydiagram{3,2,2,1,1}
      \end{gathered} 
    }\longrightarrow \begin{gathered}
      \ydiagram{2,2,2,1,1}
      \end{gathered}
      \hspace{50pt}
      \cbr{
        \begin{gathered}
          \ydiagram{2,2,2,2,1}
        \end{gathered}
      }\longrightarrow
      \begin{gathered}
        \ydiagram{2,2,2,1,1},\;\;
        \ydiagram{2,2,2,2}.
      \end{gathered}
    \]
    \caption{
      Illustration of the partitions of $9$ which can, via row removal, yield $(n,n-2)'$ alone, or both $(n,n-2)'$ and $(n-1,n-1)'$.
    }\label{OddRes}
  \end{figure}
  
  We can perform a similar argument for the $F^{2n}$ case, finding now that 
  \[\Res \; D^{\mu_{**}} \simeq D^{(n,n-1)'} \simeq \Res \; D^{(n,n)'}\] and 
  \[\Res \; D^{\mu_{*p}} \simeq D^{(n,n-1)'} \oplus D^{(n+1,n-2)'} \simeq \Res \; D^{(n+1,n-1)'}.\]
  
  Through a similar process, we see that $\mu_{*p} = (n+1,n-1)'$.
  We narrow down $\mu_{**}$ to one of $(n,n)'$ or $(n,n-1,1)'$, and note that
  \begin{align*}
    \beta_\mu(1,2) &= 3 - 2 + (n-1) = n\\
    \beta_\mu(1,3) &= 3 - 1 + n = n+2\\
    \beta_\mu(2,3) &= 2 - 1 + 2 = 3
  \end{align*}
  and hence at least one of 2 or 3 is normal, $\Res D^{(n,n-1,1)'}$ is not irreducible, and $\mu_{**} = (n,n)'$, finishing our proof.
\end{proof}
Hence $F$ is semisimple, and we have its decomposition into quotients of specht modules.
We've proven almost everything that we've set out to;
all that's left is explicit transition matrices $W \rightarrow F_{**}$.

% this was kept for archival purposes
\iffalse
\newpage
\section{Incorrect Proof of Relations}

First, let $F$ be the vector space with basis given by the strings of length $n + 1$ with alphabet $\cbr{p,*}$ such that no two $*$ symbols appear in a row.
This is given an action by the braid group that we will try to emulate.
We want an action which is ``local'', i.e. the simple transposition $T_i$ acts on the string from the $i$th to the ($i + 2$)nd symbol, modifying only the middle character, defined by the following rule:
\begin{align}
  \begin{split}
  \widehat{(*pp)} &:= a(*pp)\\
  \widehat{(*p*)} &:= b(*p*)\\
  \widehat{(p*p)} &:= c(p*p) + d(ppp)\\
  \widehat{(pp*)} &:= a(pp*)\\
  \widehat{(ppp)} &:= d(p*p) + e(ppp).
  \end{split}
  \label{Definitions}
 \end{align} 
For suitable constants $a,b,c,d,e \in k$.
\subsection{Quadratic Relation}
Note that the quadratic relation $T_i^2 = (q -1)T_i + q$ imposes the following restrictions on the constants:
\begin{align}
  \begin{split}
  a^2 &= (q-1)a + q\\
  b^2 &= (q-1)b + q\\
  c^2 + d^2 &= (q -1)c + q\\
  de &= (q-1)d\\
  e^2 + d^2 &= (q -1)e + q\\
  dc &= (q-1)d
\end{split}
  \label{Quadratic}
\end{align}
Note that we immediately have
\begin{align}
  \begin{split}
  a,b &\in \cbr{\frac{q - 1 \pm \sqrt{(q-1)^2 + 4q}}{2}}\\
  &= \cbr{\frac{q - 1 \pm (q + 1)}{2}}\\
  &= \cbr{-1,q}.
\end{split}
\label{ab}
\end{align}
Further, if $d = 0$ then we have $c,e \in \cbr{-1,q}$;
if $d \neq 0$ then we have that $c = e = (q - 1)$, and $d \in \cbr{\pm \sqrt{q}}$.
   
\subsection{Braid Relations}
Here's where we'll run into some issues.
%First, the braid relation $T_iT_j = T_jT_i$, $\abs{i - j} > 1$ is satisfied, as the action $T_i$ will be entirely unaffected by $T_j$, and one can see that this allows them to easily commute my commitativity of multiplication in $k$.

We must verify the relation $T_iT_{i+1}T_i = T_{i+1}T_iT_{i+1}$.
Let's begin with the case $d \neq 0$, which we can verify on the following strings:
\begin{itemize}
  %\item $(*pp*)$ requires that $a^3 = a^3$, which is guaranteed.
  \item $(*ppp)$ requires that $abd = bcd + ade$ and $a^2e = ae^2 + bd^2$.
    % If $d = 0$, then we require that $ae^2 = 0$ with $a,e \neq 0$, a contradiction.
    % Hence we require that $d^2 = q$ and $c = e = (q-1)$.

    The first of the above equivalently requires
    \begin{align}
      ab &= b(q-1) + a(q-1)
    \label{appp}
    \end{align}
  
    \item $(pppp)$ requires that $acd + de^2 = ade$.
     Equivalently, we require that $e^2 = 0$, i.e. $d = q = 1$.
     Then, by \eqref{appp}, we require that $ab = 0$, but $a,b \neq 0$;
     this is a contradiction, so there are no constants $a,b,c,d,e$ which make this a representation of $\SH$.

     \iffalse
    These together imply that $a = b$, and that $a = 2(q-1)$;
     when $q \neq \frac{1}{2}, 2$ this contradicts \eqref{ab}.
  \item $(*p*p)$ requires that $b^2c = bc^2 + ad^2$ and $abd = bcd + ade$.
    Equivalently, we require that
    \begin{align}
      b^2(d^2-1) &= b(d^2-1)^2 + ad^2 \label{bq}\\
      ab &= b(q-1) + a(q - 1).\nonumber
    \end{align}
    In particular, we have that
    \[
      bd^4 + (a - 2b - b^2)d^2 + b - b^2 = 0 = ad^4 + (b - 2a)d^2 - a^2d + a. 
    \]
    Equivalently, $d$ is a root of the polynomial $(a-b)d^4 + (3b - 3a + b^2)d^2 - a^2d + (2b + b^2)$.
    \fi

      %$(*pp*), $(p*pp)$, $(pp*p)$, $(p*p*)$, $(ppp*)$.
\end{itemize}

Now suppose that $d = 0$.
One thing which is immediately clear is the decomposition into one-dimensional subrepresentations (the spans of each basis vector) if this is a representation.
Note that $(*ppp)$ requires that $a^2e = ae^2$, i.e. $a = e$.
Similarly, $(*p*p)$ requires that $b = c$.
Further, we still satisfy the relations for $(pppp)$, we always satisfy the relations for $(*pp*)$, and the rest of the strings are compatible by symmetry.
All that is left are the relations $T_iT_j = T_jT_i$ for $\abs{i - j} > 1$, which are easy to see.
Hence, for each $q$, there are 4 ``good'' actions on $F$, each of which decomposes into a direct sum of one-dimensional subrepresentations.

\subsection{A More General Case}
Now, consider a modification of \eqref{Definitions}:
\begin{align}
  \begin{split}
  \widehat{(*pp)} &:= a(*pp)\\
  \widehat{(*p*)} &:= b(*p*)\\
  \widehat{(p*p)} &:= c(p*p) + d(ppp)\\
  \widehat{(pp*)} &:= g(pp*)\\
  \widehat{(ppp)} &:= f(p*p) + e(ppp).
  \end{split}
  \label{General Definitions}
\end{align} 
This time, the quadratic relation reads
\begin{align}
  \begin{split}
  a^2 &= (q-1)a + q\\
  b^2 &= (q-1)b + q\\
  g^2 &= (q-1)g + q\\
  c^2 + df &= (q -1)c + q\\
  de &= (q-1)d\\
  e^2 + df &= (q -1)e + q\\
  fc &= (q-1)f
\end{split}
  \label{General Quadratic}
\end{align}
Notably, we have $a,b,g \in \cbr{-1.q}$ still.
If $d = 0$ or $f = 0$, then $c,e \in \cbr{-1,q}$ still, and if $d = 0$ and $f \neq 0$, then $c = (q-1)$, and hence $c = 1$ and $q = 2$.
If $d \neq 0$, we still have that $c = e = (q-1)$, and we have that $df = q$.

Now, suppose that $d,f \neq 0$.
Now, $(*ppp)$ requires that $abf = bcf + aef$ and $a^2e = ae^2 + bf^2$, so as before we have that $ab = (a+b)(q-1)$.

$(pppp)$ now requires that $aef = acf + e^2f$.
This requires that $e^2 = 0$, so that $q = 1$ and $ab = 0$, a contradiction again.

Now, suppose that $f \neq 0$ and $d = 0$, so that $q = 2$ and $c = 1$.
Then, we have that $a(e-1) = e^2$.
Then, knowing that $e^2 \neq 0$, we have $e = -1$ so $-2a = 1$, a contradiction.
By symmetry, we also arrive at a contradiction if $d \neq 0$ and $f = 0$.

Finally, suppose that $d = f = 0$, and note that we now have $a = e = g$ and $b = c$;
hence our case is precisely the case with $d = f$ and $a = g$, and there are exactly four actions of $\SH$ on $F$ on which each $T_i$ acts analogously on positions $i,i+1,i+2$ as each other, only modifying position $i+1$. 
Each of these actions decomposes into a direct sum of 1-dimensional subrepresentations.

\newpage

\fi
\end{document}

\documentclass{amsart}   
\usepackage{RepStyle} 
\usepackage{NatMacros} %********REPLACE WITH MilesMacros TO KILL RELIANCE ON \ifnum AND HAVE INCOMPLETE OUTPUT***********bleh
\begin{document}

\title{Rough Cut of Proven Work on $\SH$.}
\author{Miles Johnson \& Natalie Stewart}
\maketitle

\section{Introduction}
Let $S_{2n+r}$ be the symmetric group on $2n+r$ indices with $2n + r \geq 2$, let $\SH = \SH_{k,q}(S_{2n+r})$ be the corresponding Hecke algebra over field $k$ with parameter $q \in k^\times$ having square root $q^{1/2}$, and let $\cbr{T_i}$ be the reflections generating $\SH$.
Let $\brk{m}_q = 1 + q + \dots + q^{m-1}$ be the $q$-number of $m$. 
Let $e$ be the smallest positive integer such that $\brk{e}_q = 0$, and set $e = \infty$ if no such integer exists.
Either $q = 1$ and $e$ is the characteristic of $k$ (with $0$ replaced by $\infty$), or $q \neq 1$ and $q$ is a primitive $e$th root of unity.

Throughout the text, we will refer to partitions of $2n + r$;
identify each partition with a tuple $\lambda = (\lambda_1^{a_1},\dots,\lambda_l^{a_l})$ having $\lambda_i > \lambda_{i+1}$, $a_i > 0$, and $\sum_i a_i\lambda_i = 2n + r$.
Identify each of these with a subset $\brk \lambda \subset \NN^2$ as defined in Kleshev, and define $\lambda(i) = (\lambda_1^{a_1},\dots,\lambda_{i-1}^{a_{i-1}},\lambda_i^{a_i - 1},\lambda_i-1,\lambda_{i+1}^{a_{i+1}},\dots,\lambda_l^{a_l})$ to be the partition with the $i$th row removed.

Fixing some partition $\lambda$, for $1 \leq i \leq j \leq l$, let $\beta(i,j)$ be the hook length
\[
  \beta(i,j) = \lambda_i - \lambda_j + \sum_{t = i}^j a_t.
\]
Then, adopting Kleshev's terminology, $j$ is normal in $\lambda$ if $\beta(i,j) \not\equiv 0 \pmod e$ for all $i < j$, and $j$ is good if it is the largest normal number (these are stronger conditions than generally necessary). 

Let $S^{(n+r,n)'}$ be the Specht module corresponding to the young diagram with two columns with height difference $r$, and let $D^{(n+r,n)'}$ be the corresponding irreducible quotient.
The purpose of this writing is to characterize these representation via an isomorphism with two graphical representations of $\SH$.

\subsection{Crossingless Matchings}
\begin{definition}
  A \emph{crossingless matching on $2n+r$ indices with $r$ anchors} is a partition of $\cbr{1,\dots,2n+r}$ into $n$ parts of size $2$ and $r$ of size 1 such that no two parts of size two``cross'', i.e. there are no parts $(a,a')$ and $(b,b')$ such that $a < b < a' < b'$, and no parts of size one are ``inside'' of a part of size two, i.e. there are no $c, (a,a')$ such that $a < c < a'$.
  We will call these arcs and anchors, respectively.
  Then, define $W^r_{2n+r}$ to be the $k$-vector space with basis the set of generalized crossingless matchings on $2n+r$ indices with $r$ anchors.

  In order for this to be a $\SH$-module, endow this with the action given by Figure \ref{Action}; 
  if a ``loop'' is created, scale by $q+1$, if a loop is not created and the action involves fewer than 2 anchors, deform into a new crossingless maching and scale by $q^{1/2}$, and if it involves two anchors, scale by 0.
  We verify that this is well-defined in appendix \ref{Cross Relations}.
\end{definition}

\begin{figure}
  \[
    \begin{tabular}{l l}
      \GeneralizedAction{6}{1/4,2/3}{2}{5/1,6/2}{1}{2}{1/4, 2/3}{5/1,6/2}{(1+q)}
      \hspace{20pt}
      &
      \GeneralizedAction{6}{1/4,2/3}{2}{5/1,6/2}{1}{3}{1/2, 3/4}{5/1,6/2}{q^{1/2}}\\
      \GeneralizedAction{6}{1/4,2/3}{2}{5/1,6/2}{1}{4}{2/3, 4/5}{1/1,6/2}{q^{1/2}}
      &
      \GeneralizedZeroAction{6}{1/4,2/3}{2}{5/1,6/2}{1}{5}
    \end{tabular}
  \]
    
  \caption{Illustration of the actions $(1 + T_i)w_{\abs{W^2_6}}$.
    In general, we act by deleting loops, deforming into a new crossingless matching, and scaling by either $q^{1/2}$, $(q + 1)$, or 0.
  }
  \label{Action}
\end{figure}

Let the length of an arc $(i,j)$ be $l(i,j) := j - i + 1$.
Note that the crossingless matchings on $2n$ indices with no anchros can all be identified with a list of $n$ integers describing the lengths of the arcs from left to right;
using this, we may order the crossingless matchings with no anchors in increasing lexicographical order in order to obtain an order on the subbasis containing a particular set of anchors;
let the basis be ordered first by the position of the anchors in decreasing lexicographical order, then increasing for the matchings between each anchor.
Let this basis be $\cbr{w_i}$.
This basis is illustrated for $W_{5}^1$ in Figure \ref{S5 Basis}. 

\begin{figure} 
  \def\cbasisspacing{5mm}
  $\cbr{
    \begin{gathered}
      \GeneralizedMatching{5}{1/2, 3/4}{1}{5/1}{3/4}, \hspace{\cbasisspacing}
      \GeneralizedMatching{5}{1/4, 2/3}{1}{5/1}{3/4}, \hspace{\cbasisspacing}
      \GeneralizedMatching{5}{1/2, 4/5}{1}{3/1}{3/4}, \hspace{\cbasisspacing}
      \GeneralizedMatching{5}{2/3, 4/5}{1}{1/1}{3/4}, \hspace{\cbasisspacing}
      \GeneralizedMatching{5}{2/5, 3/4}{1}{1/1}{3/4}, \hspace{\cbasisspacing}
     \end{gathered}}$ 
    \caption{The basis for $W_5^1$.}
  \label{S5 Basis}
\end{figure} 

We will prove that $W := W_{2n+r}^r$ and $S := S^{(n+r,n)'}$ are isomorphic as representations in the case that $e > n + r + 1$.
Note that, when $r = 0$, these have the same dimension given by the $n$th catalan number $C_n$.
 
\subsection{Fibonacci Representation}
Now suppose that $k = \CC$ and $q = \exp\prn{2 \pi i \ell/5}$ is a primitive 5th root of unity.
Let $V^m$ be a $k$-vector space with basis given by the strings $\cbr{*,p}^{n+1}$ such that the character $*$ never appears twice in a row. 
We will surpress the superscript whenever it is clear from context.

We wish to endow this with a $\SH$-action which acts on a basis vector only dependent on characters $i,i+1,i+2$, sending each basis vector to a combination of the other basis vectors having the same characters $1,\dots,i,i+2,\dots,n+1$ as follows:
\def\vara{\alpha}
\def\varb{\beta}
\def\varc{\gamma}
\def\vard{\delta}
\def\vare{\varepsilon}

\begin{equation}
  \begin{split}
  T_1 \, (*pp) &:= \vara(*pp)\\
  T_1 \, (pp*) &:= \vara(pp*)\\
  T_1 \, (*p*) &:= \varb(*p*)\\
  T_1 \, (p*p) &:= \varc(p*p) + \vard(ppp)\\
  T_1 \, (ppp) &:= \vard(p*p) + \vare(ppp)
\end{split} \label{Fib Action} 
\end{equation}
for constants
\begin{equation}
  \begin{split}
  \vara &= -1\\
  \varb &= q\\
  \varc &= \tau(q\tau - 1)\\
  \vard &= \tau^{3/2}(q + 1)\\
  \vare &= \tau(q-\tau)\\
  \tau &= -1-q^2-q^3
\end{split} \label{Fib Constants} 
\end{equation}
with $T_i$ acting similarly on the substring $i,i+1,i+2$.
We will verify that this is a representation of $\SH$ in Appendix \ref{Fib Relations}

This contains 4 subrepresentations based on the first and last character of the string, which are not modified by $\SH$.
Label the subrepresentation of strings $(*\dots*)$ by $V_{**}$, and similar for the other 3.
It is easy to see that $V_{*p} \simeq V_{p*}$, so that
\[
  V \simeq 2V_{*p} \oplus V_{**} \oplus V_{pp}.
\]
We will show that $V_{pp} \simeq V_{*p} \oplus V_{**}$, and give the following isomorphisms with irreducible quotients of specht modules depending on the parity of the number of indices in $\SH$:
\begin{equation}
  \begin{split}    
    V^{2n}_{**} &\simeq D^{(n,n)'}\\ 
    V^{2n-1}_{**} &\simeq D^{(n+1,n-2)'}\\
    V^{2n}_{*p} &\simeq D^{(n+1,n-1)'}\\
    V^{2n-1}_{*p} &\simeq D^{(n,n-1)'}.
  \end{split} \label{Fib Isos}
\end{equation}
 
\newpage
\section{Crossingless Matchings and Specht Modules}
Our goal is to prove that $W_{2n + r}^r \simeq S^{(n+r,n)'}$ when $e > n + r + 1$.
\begin{proposition}
  \begin{enumerate}[label={(\roman*)}]
    \item
    Suppose that $n,r > 0$.
    Then, a filtration of $\Res W_{2n + r}^r$ is given by
    \begin{equation}
      0 \subset W_{2n + r - 1}^{r-1} \subset \Res \, W_{2n + r}^r \label{Filtration}
    \end{equation}
    with $\Res \, W_{2n + r}^r / W_{2n + r - 1}^{r-1} \simeq W_{2n + r - 1}^{r + 1}$. 
    \item
      We have the following isomorphism of representations:
      \begin{equation}
        W_{2n - 1}^1 \simeq \Res \; W_{2n}^0 \label{0 Restriction}
      \end{equation}
  \end{enumerate}
\end{proposition}
\begin{proof}
  \textbf{(i)}
  Note that we may identify the subrepresentation of $\Res \; W_{2n + r}^r$ having anchor $n$ with $W_{2n + r - 1}^{r-1}$.
  
  Let $U := \Res \; W_{2n + r}^r / W_{2n + r - 1}^{r - 1}$. 
  Let $\phi:U \rightarrow W_{2n + r - 1}^{r + 1}$ be the $k$-linear map which regards the arc $(i,2n + r)$ in $U$ as an anchor at $i$ in $W_{2n + r - 1}^{r + 1}$.
  It is not hard to verify that this is a well-defined isomorphism of vector spaces, so we must show that it is $\SH$-linear.

  Given a basis vector $w_j$ with arc $(i,2n + r)$, $\phi$ is clearly compatible with $T_{i'}$ with $i' \neq i,i-1$.
  Further, it's easy to verify that $\phi$ is compatible with $T_{i}$ and $T_{i-1}$, as actions on one anchor were designed for this deformation.
  When there are anchors $(i,i+1)$, then $\phi(T_iw_j) = T_i\phi(w_j) = 0$, and similar for $T_{i -1}$.
  Hence $\phi$ is an isomorphism of representations, and the statement is proven.

  \textbf{(ii)}
  This follows with the above proof, defining $W_{2n - 1}^{-1} := 0$
\end{proof}

\begin{lemma}
  Every basis vector in $W_{2n + r}^r$ is cyclic.
\end{lemma}
\begin{proof}
  We have already proven this in the $r = 0$ case, so suppose that $r > 0$.

    Note that, between anchors $a<a'$ having no arc $b$ with $a < b < a'$, the $W_{a'-a}^0$ case allows us to generate the vector with all length-2 arcs between $a,a'$ and identical arcs/anchors outside of this sub-matching.\footnote{At the ends, we apply the $W_a^0$ case or the $W_{2n + r - a}^{0}$ case in the same way for the first $a$ or last $2n + r - a$ indices.}

  Applying this between each arc gives us a vector with length-2 arcs and anchors, and we may use the appropriate $(1+T_i)$ to move anchors to any positions, and the reverse process from above to generate the correct matchings between arcs and generate any other basis vector.
\end{proof}

Let $K := \bigcap_{i=1}^{2n+r-1} \ker (1 + T_i) = \ker \bigoplus_{i = 1}^{2n + r - 1} (1 + T_i)$.
This will be a large technical tool in our proof of irreducibility.
\begin{lemma}\label{Represented}
  Let $w_j$ be the basis vector with anchors $1,\dots,r$ and all arcs of maximal length.
  Suppose $w \in K \backslash \cbr 0$.
  Then, $w_j$ is represented in $w$.
\end{lemma}
\begin{proof}
  We will show this in steps; first, we show that, given that a vector is represented with anchors $1,\dots,s$, there must be a vector represented in $w$ with $(s+1)$st anchor, including when $s = 0$;
  this implies that a vector is represented with anchors $1,\dots,r$.
  Then, we will show that, given a vector is represented with anchors $1,\dots,r$ and first $s$ arc-lengths $n,n-2,\dots,n-2s$, there is a vector represented with these and the $(s+1)$st arc-length $n-2s-2$.
  This implies that $w_j$ is represented.
  
  \def\lemsep{7pt}
  \vspace{\lemsep}
  \textit{Step 1.}
  Suppose that $s < r$ is the maximal number such that a vector with anchors $1,\dots,s$ is represented.
  Take the vector $w_i$ which, among vectors represented in $w$ with anchors $1,\dots,s$, has $(s+1)$st anchor at minimal index $t > s + 1$,
  Then, $q^{-1/2}(1 + T_{t-1})w_i$ has anchors $1,\dots,s$ and a earlier index than $t$, so it was not represented before;
  further, for any other basis vector $w_l \neq w_i$ to map onto $q^{-1/2}(1 + T_{t-1})w_i$, we would require that $w_l$ has anchors $1,\dots,s$ and some other anchor at index $t' < t$, so it is not represented.
  Hence $w_i$ is unique among the vectors represented mapping onto $q^{-1/2}(1 + T_{t-1})w_i$, and $(1 + T_{t-1})w$ represents this vector, giving $w \notin \ker (1 + T_{t-1})$.
 
  When $s = 0$, this is similar, and we simply perform this logic on the 1st anchor.
  Each lead to contradiction, so we must have $s = r$. 

  \vspace{\lemsep}
  \textit{Step 2.}
  This step is similar;
  suppose that $s < n$ is the maximal number such that a vector with anchors $1,\dots,r$ and first $s$ arc-lengths $n,\dots,n-2s$ is represented.
  Take the vector $w_i$ which, among vectors represented in $w$ with anchors $1,\dots,r$ and first $s$ arc-lenths $n,\dots,n-2s$, has maximal length $t$ of the arc beginning at index $r + s + 1$.
  Then, $q^{1/2}(1 + T_{r + s + t})w_i$ is mapped to only by $w_i$ and vectors having anchors $1,\dots,r$ and first $s+1$ arc-lengths $n,\dots,n-2s,t'$ with $t' > t$, which are not represented in $w$;
  hence $q^{-1/2}(1 + T_{r + s + t})w_i$ is represented in $(1 + T_{r+s+t})w$, giving $w \notin \ker (1 + T_{r + s + t})$.
  The $s = 0$ case is similar, and implies that $s = n$.
\end{proof}

\begin{lemma}\label{Trivial}
  Suppose $e \nmid n + r + 1$.
  Then, $K = 0$.
\end{lemma}
\begin{proof}
  Consider the matrix $A = \bigoplus (1 + T_i)$ having kernel $K$.
  It is sufficient by lemma \ref{Represented} to show that $A$ includes a row $[0,\dots,0,1,0,\dots,0]$ with a nonzero entry only on the column $j$.

  Now, we may characterize the rows of $A$ as follows;
  if the row corresponding to $(1 + T_i)$ and mapping onto the element $w_l \in W$ is nonzero, then it is of the form $\brk{a_1,\dots,a_{\abs W}}$ where $a_l = 1 + q$, $a_m = q^{1/2}$ whenever $(1 + T_i)w_m = q^{1/2}w_l$, and $a_m = 0$ otherwise.
  
  Seeing this, the row corresponding to $(1 + T_{n + r})$ and $w_j$ has nonzero entries $q^{1/2}$ at $w_j$ and $(1 + q)$ at the vector $w$ agreeing with $w_j$ at all indices except having arcs at $(n+r-1,n+r)$ and $(n+r+1,n+r+2)$.
  Similar justification leads the row corresponding to $(1 + T_{n+r-1})$ at $w$ to have nonzero entries $q^{1/2}$ at $w$ and $(1 + q)$ at $w_j$ and the vector with anchors $1,\dots,r$ , arc $(n+r-3,n+r-2)$, and all other arcs maximum length.
  
  \begin{figure}
   \[
     \Action{10}{1/2,3/10,4/9,5/8,6/7}{2}{1/10,2/3,4/9,5/8,6/7}{q^{1/2}}
   \]
   \[
     \Action{10}{1/10,2/3,4/9,5/8,6/7}{2}{1/10,2/3,4/9,5/8,6/7}{q+1}
   \]
   \[
     \Action{10}{1/10,2/9,3/4,5/8,6/7}{2}{1/10,2/3,4/9,5/8,6/7}{q^{1/2}}
   \]
   \caption{Illustrated is the row constructed for transposition $(1 + T_2)$; clearly these are the only basis elements mapping to multiples of the desired element, and they relate to each other.
     replacing the outermost and/or innermost arc with an anchor typifies the rows constructed with three nonzero coefficients.}
 \label{Submatrix}
 \end{figure}
 
  We may iterate this process as illustrated in Figure \ref{Submatrix}, eventually ending at a row with two nonzero entries, either an arc $(1,2)$ or an arc $(2,3)$, and all anchors otherwise left-aligned and arcs of maximum length.
  These rows together form an $(n + r) \times \abs{W_{2n + r}^r}$ sumbatrix of $A$ which has a nonzero column in the row corresponding to $j$, and has (by removing zero columns) the same column space as the following square matrix:
  \newcommand*\bigzero{\makebox(0,15){\text{\Huge0}}}
  \newcommand*\bigzerotwo{\makebox(-20,15){\text{\Huge0}}}
  \[
    B_{n + r} := \begin{bmatrix}
      \begin{matrix}
      q + 1 & q^{1/2} & \\
      q^{1/2} & q + 1 & q^{1/2}\\
      & q^{1/2} & q + 1 & q^{1/2}\\
      &  & \ddots & \ddots
      \end{matrix}
      &   \bigzero\\
      \bigzerotwo & \begin{matrix}
        q^{1/2} & q+1 & q^{1/2}\\
        &  q^{1/2} & q + 1
       \end{matrix}
    \end{bmatrix}.
  \]
  We will show that this matrix is invertible;
  then, a sequence of elementary row operations will yield the identity, and in particular, when applied to $A$, will yield a row with a nonzero entry only on column $j$, giving $K = 0$.

  We may prove invertibility of this matrix by proving that $\det B_{n + r} = \brk{n + r + 1}_q$ inductively on $n+r$.
  This is satisfied for our base case $n + r = 1$, so supose that it is true for each $m < n + r$.
  Then,
  \begin{align*}
    \det B_{n+r} &= (q + 1) \det B_{n + r - 1} - q \det B_{n + r - 2}\\
    &= (q + 1)(1 + \dots + q^{n + r - 1}) - (q + \dots + q^{n + r - 1})\\
    &= 1 + \dots + q^{n + r}\\
    &= \brk{n + r + 1}_q.
  \end{align*}
  Hence $\det B_{n + r} \neq 0$, and $K = 0$.
\end{proof}

\begin{proposition}
  The representation $W_{2n + r}^r$ is irreducible when $e > n + r + 1$. 
\end{proposition}
\begin{proof}
  We proceed by induction on $2n + r$.
  Note that, by identification with the trivial and sign representations, the base case $2n + r = 2$ is already prove, so suppose we have proven this for each $2m + s < 2n + r$.

  Take some $w \in W$ and some $(1 + T_i)$ not annhialating $w$.
  Note that \[\ima (1 + T_i) = \Span\cbr{w_j \mid w_j \text{ contains arc }(i,i+1)}.\]
  Hence, as vector spaces, there is an isomorphism $\varphi:\ima(1 + T_i) \rightarrow W_{2(n-1) + r}^r$ ``deleting'' the arc $(i,i+1)$.
  This sends every basis vector to another basis vector.

  We will show that, for every action $(1 + T_j') \in \SH(S_{2(n-1) + r})$, there is some action $h_j \in W_{2n + r}^r$ such that the following commutes:
  \[
    \begin{tikzcd}
      \ima (1 + T_i) \arrow[r,"\varphi" above, "\sim" below] \arrow[d,"h_j"] & W_{2(n-1) + r}^r \arrow[d,"1 + T_j"]\\
      \ima (1 + T_i) \arrow[r,"\varphi" above, "\sim" below] & W_{2(n-1) + r}^r
    \end{tikzcd}
  \]
  Indeed, when $i \neq j$ this is given by $h_j = 1 + T_j$, and we have $h_i = q^{-1}(1 + T_i)(1 + T_{i+1})(1 + T_{i-1})$, as given by Figure \ref{bigloop}.
  
  Due to the inductive hypothesis, there is some action $h' \in \SH(S_{2(n-1) + r})$ sending $\varphi((1 + T_i)w)$ to a basis vector;
  then, the action $\SH$ generates the endomorphism $\varphi^{-1}h'\varphi$ sending $(1 + T_i)w$ to a basis vector, giving $w$ cyclic and hence $W_{2n + r}^r$ irreducible.
  \begin{figure}
  \[
    \Action{6}{1/6, 2/5, 3/4}{2}{1/6, 2/3, 4/5 }{q^{1/2}} \hspace{20pt} 
    \IsoAction{8}{1/8, 2/7, 3/4, 5/6}{3}{1/8, 2/5, 3/4, 6/7}{q^{1/2}}
  \]
  \[
    \Action{4}{1/4, 2/3}{2}{1/4,2/3}{(1 + q)} \hspace{20pt}
    \IsoAction{6}{1/6, 2/5, 3/4}{3}{1/6, 2/5, 3/4}{(1 + q)}
  \]
  \caption{The correspondence between the action of $(1 + T_2)$ on $w'_5 \in W^0_6$ and the action of $q^{-1}(1 + T_3)(1 + T_4)(1 + T_2)$ on the corresponding vector in $W^0_8$ having arc $(3,4)$ first, then on $w'_2 \in W^0_4$.
  This demonstrates that the action works with and without creating a loop.
  }
  \label{bigloop}
  \end{figure}
\end{proof}

\iffalse
The next piece in our puzzle is to characterize the restrictions of $W$ to $\SH' := \SH_{k,q}(S_{2n + r - 1}) \subset \SH$.
Recall that, when $r,n>0$ and $\SH$ is semisimple,  $\Res S^{(n+r,n)'} \simeq S^{(n+r-1,n)'} \oplus S^{(n+r,n-1)'}$.
Further, note that $S^{(n+r,n)'}$ is the unique irreducible having this restriction.

Next, note that we have already proven the correspondence for $W_{2n}^0$;
for $W_{0 + r}^{r}$, this is the sign representation, which is given correctly by $S^{(r)}$.
Hence, pending information on restrictions, we may prove this via induction on $2n + r$.
\fi

\begin{corollary}
  Suppose $n,r > 0$ and $e > n + 1$.
  Then, the sequence \eqref{Filtration} is a composition series of $\Res \, W_{2n + r}^r$.\qed
\end{corollary}

\begin{theorem}
  Suppose $e > n + r + 1$, $n > 0$.
  Then, $W_{2n + r}^r \simeq S^{(n+r,n)'}$.
\end{theorem}
\begin{proof}
  By irreducibility, we know that $W_{2n + r}^r \simeq D^\lambda$ for some $e$-restricted partition $\lambda$.
  We will proceed in two steps;
  first we prove that $\lambda = (n+r,n)'$, then we prove that $S^{(n+r,n)'}$ is irreducible.

  This will be done inductively; by identification with the trivial and sign representations, the $2n + r = 2$ caseholds, so suppose this is true for $W_{2m + s}^s$ whenever $2m + s < 2n + r$ and $m + s \leq n + r$ (i.e. $e > m + s + 1$).

  \def\thmsep{7pt}
  \vspace{\thmsep}
  \textit{Step 1.}
  By the inductive hypothesis and irreducibility, we have a composition series given by 
  \begin{equation}
    \label{Composition Series} 0 \longrightarrow D^{(n + r - 1,n)'} \longrightarrow \Res \, D^\lambda \longrightarrow D^{(n+r,n-1)'} \longrightarrow 0
  \end{equation}
  In particular, by the Jordan-H\"older theorem, if we have some module $D^\mu \subset \Res \, D^\lambda$, then $\mu = (n+r-1,n)'$ or $\mu = (n+r,n-1)'$.
  
  By Kleschev, we know that $\soc(D^\lambda) = \bigoplus_{\mu} D^\mu$, where $\mu$ ranges over $\lambda(i)$ for every good number $i$ of $\lambda$.
  Immediately this narrows down $\lambda$ to three options;
  the only $\lambda$ with some $\lambda(i)$ giving $(n + r - 1,n)'$ are $\lambda_1 ;= (n+r-1,n,1)'$, $\lambda_2 = (n+r-1,n+1)'$, and $\lambda_3 = (n+r,n)'$.

  Note numbers $\beta_l(i,j)$ correspond to hook-lengths, and each $\lambda_;$ has maximum hook length $n + r + 1$;
  hence $\beta_l(i,j) \not\equiv 0 \pmod e$ for all $i,j,l$, and 3 is a good number in $\lambda_1$ and 2 in $\lambda_2$.
  This implies $D^{(n+r-2,n,1)'} \subset D^{\lambda_1}$ and $D^{(n+r-2,n+1)'} \subset D^{\lambda_2}$, giving that $\lambda = \lambda_3$ as desired.

  \vspace{\thmsep}
  \textit{Step 2.}
  Let the hook length of node $(a,b)$ in $\brk \lambda$ be written $h_{ab}^\lambda$.
  Let $\nu_p(h_{ab}^\lambda)$ be the $p$-adic valuation of $h_{ab}^\lambda$, and let $\nu_{e,p}(h_{ab}^\lambda)$ be $\nu_p(h^{J})$.
  Mathas Theorem 5.42 says, for $\lambda$ $e$-restricted, $S^\lambda$ is irreducible if $\nu_{e,p}(h_{ab}^\lambda) = \nu_{e,p}(h_{ac}^\lambda)$ for a suitable prime $p$.
  In particular, since $h_{ab}^\lambda < e$ for all $a,b$, we have $\nu_{e,p}(h_{ab}^\lambda) = -1$ for all $a,b$, and $S^\lambda$ is irreducible.
\end{proof}

\iffalse
\begin{proposition}
  Suppose $e > n + r + 1$, and let $\lambda$ be a partition of $2n + r$.
  \begin{enumerate}[label={(\roman*)}]
    \item Suppose $n,r > 0$.
      If $\Res \, D^\lambda$ admits a filtration $0 \subset S^{(n+r-1,n)'} \subset \Res \, S^\lambda$ with $\Res \, S^\lambda / S^{(n + r - 1,n)'} \simeq S^{(n + r,n-1)'}$ then $\lambda = (n+r,n)'$.
    \item Suppose $r = 0$.
      If $\Res \, S^\lambda \simeq S^{(n,n-1)'}$, then $\lambda = (n,n)'$.
  \end{enumerate}
\end{proposition}
\begin{proof}
  \textbf{(i)}
  Recall that the characteristic-free classical branching theorem gives that every specht module $S^{\lambda}$ admits a filtration 
  \begin{equation}
    0 = M_0 \subset \dots \subset M_l = S^{\lambda} \label{filtration}
  \end{equation}  
    with $M_i/M_{i-1} \simeq S^{\lambda(i)}$.
  Hence $\lambda = (n + r, n)'$ satisfies the above formula.

  Note that, when $e > n + r + 1$, this filtration gives a composition series.
  Hence the sequence \eqref{filtration} is at most length-2.

  Suppose it is length-1.
  Then $\lambda$ is rectangle-shaped;
  to have rows removed to the above we then need that that it be $\lambda = \prn{\frac{2n + r}{2}, \frac{2n + r}{2}}'$, so that $\Res S^{\lambda} \simeq S^{\prn{\frac{2n + r}{2}, \frac{2n + r}{2} - 1}'}$.
  However, since $e > n + r + 1 \geq n + \frac{r}{2} + 1$, $\Res \, S^{\lambda}$ is irreducible, contradicting the existence of a length-2 composition series in the hypothesis.

  Now suppose that \eqref{filtration} is length 2, so that it is also a composition series.
  Then, by the Jordan-H\"older theorem, there is a rearranement of the composition factors of \ref{filtration} to give $S^{(n + r - 1,n)'}$ and $S^{(n + r, n - 1)'}$.
  This implies that both $(n + r - 1,n)'$ and $(n + r,n-1)'$ can be acquired by removing rows of $\lambda$, giving that $\lambda = (n+r,n)'$.
  
  \vspace{7pt}
  \textbf{(ii)}
  Note that, since $e > n + 1$, $\Res S^\lambda$ is irreducible.
  Hence the sequence \eqref{filtration} is a composition series of length 1, and $\lambda(1) = (n,n-1)'$.
  The first implies that $\lambda$ is rectangular, and the second implies that $\lambda = (n,n)'$.
\end{proof}
\fi


\section{The Fibonacci Representation and Specht Modules}
We can start our study of $V$ by studying low-dimensional cases.
First, note that $V_{*p}^2$ is the sign representation $D^{(2)}$ and $V_{**}^2$ is the trivial representation $D^{(1)^2}$.

$V_{pp}^2$ is a 2-dimensional representation of a semisimple commutative algebra, and hence decomposes into a direct sum of two subrepresentations.
In particular, we can use the basis $\cbr{(p*p),(ppp)}$ and explicitly write the matrix
\[
  \rho_{T_1} = \begin{bmatrix}
    \varc & \vard\\
    \vard & \vare
  \end{bmatrix}
\]
having characteristic polynomial $(\varc - \lambda)(\vare - \lambda) - \vard^2 = \lambda^2 - (\varc + \vare)\lambda +(\varc\vare - \vard^2)$.
We may verify that, for $\lambda = -1$, this evaluates to
\[
  -((-1 + q + q^2) (1 + q^3 + q^4 + q^5 + 2 q^6 + q^7))\brk{5}_q = 0
\]
and for $\lambda = q$ this evaluates to 
\[
-(q^2 (-1 + q + q^2) (1 + q + q^2 + q^3 + 2 q^4 + q^5)) \brk{5}_q = 0
\]
hence $\rho_{T_1}$ has eigenvalues $-1$ and $q$.

The eigenspaces with eigenvalues $-1$ and $q$ are subrepresentations isomorphic to the sign and trivial representation, hence $V_{pp}$ is isomorphic to a direct sum of the trivial and sign representations: $V^2_{pp} \simeq V^2_{*p} \oplus V^2_{**}$. 

Now let's prove that $V^3_{**}$ is irreducible;
this has basis $\cbr{*p*p}, \cbr{*ppp}$, and the following matrices:
\[
  \rho_{T_1} = \begin{bmatrix}
    \varb & 0\\
    0 & \vara
  \end{bmatrix}; \hspace{20pt}
  \rho_{T_2} = \begin{bmatrix}
    \varc & \vard\\
    \vard & \vare
  \end{bmatrix}.
\]
A subrepresentation must be one-dimensional, and hence an eigenspace of each of these matrices;
since $\varb \neq \vara$, the first has eigenspaces given by the spans of basis elements, and since $\vard \neq 0$, these are not eigenspaces of the second.
Hence $V^3_{**}$ is irreducible.
Now we may move on to the general case.
\begin{proposition}
  The representation $V_{*p} := V^m_{*p}$ is irreducible.
\end{proposition}
\begin{proof}
  We will prove this inductively in $m$.
  We've already proven it for $V^2_{*p}$ and $V^3_{*p}$, so suppose that $V^{m-2}_{*p}$ is irreducible.
  
  Let $\cbr{v_i}$ be the basis for $V_{*p}$.
  Then, each $v_i$ is cyclic; indeed, we can transform every basis vector into $(*p\dots p)$ by multiplying by the appropriate $\frac{1}{\vard-\varc}(T_i - \varc)$, and we can transform $(*p\dots p)$ into any basis vector by multiplying be the appropriate $\frac{1}{\vard-\vare}(T_i - \vare)$.
  Hence it is sufficient to show that each $v \in V_{*p}$ generate some basis element.

  Let $v'$ be the basis element $(*p*p\dots p)$, which is many copies of $*p$, followed by an extra $p$ if $m$ is odd.
  We will show that each $v \in F$ generates $v'$.

  Suppose that no elements beginning $(*p*p)$ are represented in $v_i$;
  then, all such elements are represented in $T_3v$, so we may assume that at least one is represented in $v$.

  Note that $\ima (T_2 - \vara) = \Span\cbr{\text{Basis vectors beginning }(*p*p)}$ and $(T_2 - \vara)v \neq 0$.
  Further, note that $\Res_{\SH(S_{m-2})}^{\SH(S_m)} \ima (T_2 - \vara) \simeq V^{m-2}_{*p}$ as representations.
  Hence irreducibility of $V^{m-2}_{*p}$ implies that $v'$ is generated by $(T_2 - \vara)v$, and $V^m_{*p}$ is irreducible.
\end{proof}

Knowing this, the restriction statements are clear;
$\Res V_{*p}^m \simeq V_{pp}^{m-1}$ by considering the last $m-2$ transpositions, and $\Res V_{*p}^{m-1} \simeq V_{*p}^{m-1} \oplus V_{**}^{m-1}$ by considering the first $m-2$.
Similarly, $\Res V_{**}^m \simeq V_{*p}^{m-1}$ by considering the first $m-2$ transpositions.
This gives that $V \simeq 3V_{*p} \oplus 2V_{**}$.

Now we may move on and use Young Tableau to characterize $V$.
Recall that the socle of $D^\lambda$ is given by $\bigoplus\limits_{\mu \xrightarrow{\text{good}} \lambda} D^\mu$,  and that $D^\lambda$ is semisimple iff every $\mu \xrightarrow{\text{normal}} \lambda$ is good.
\begin{theorem}
  The irreducible components of $V$ are given by the following isomorphisms:
    \begin{align*} 
      V^{2n}_{**} &\simeq D^{(n,n)'}\\ 
      V^{2n-1}_{**} &\simeq D^{(n+1,n-2)'}\\
      V^{2n}_{*p} &\simeq D^{(n+1,n-1)'}\\
      V^{2n-1}_{*p} &\simeq D^{(n,n-1)'}.
     \end{align*}
\end{theorem}
\begin{proof}
  We will prove this by induction on $n$;
  we have already proven the base case $V^{2}$, so suppose that we have proven these isomorphisms for $V^{2n-2}$.
  We will prove the isomorphisms for $V^{2n-1}$ and $V^{2n}$.

  By irreducibility, $V^{2n-1}_{**} \simeq D^{\lambda_{**}}$ and $V^{2n-1}_{*p} \simeq D^{\lambda_{*p}}$ for some diagrams $\lambda_{**}$ and $\lambda_{*p}$.
  We will show that $\lambda_{**} = (n+1,n-2)'$ and $\lambda_{*p} = (n+1,n-1)'$.
  
  First, note that we have \[\Res \; D^{\lambda_{**}} \simeq D^{(n,n-2)'} \simeq \Res \; D^{(n+1,n-2)'}\] and \[\Res \; D^{\lambda_{*p}} \simeq D^{(n,n-2)} \oplus D^{(n-1,n-1)} \simeq \Res \; D^{(n,n-1)'}.\]
  By semisimplicity of $\Res D^{\lambda_{**}}$ and $\Res D^{\lambda_{*p}}$, every normal cell in $\lambda_{**}$ and $\lambda_{*p}$ is good, and every good cell is removed in a summand of the restriction.
  In particular, the only normal number in $\lambda_{**}$ is 1.

  For $\lambda_{**}$, the only tableaux which can remove a cell to yield $D^{(n,n-2)'}$ are $(n+1,n-2)'$, $(n,n-1)'$, and $(n,n-2,1)'$ as illustrated in Figure \ref{OddRes};
  we have already seen that $D^{(n,n-1)'}$ does not have irreducible restriction, so we are left with $(n+1,n-2)'$ and $(n,n-2,1)'$.
  We may directly check that $(n,n-2,1)'$ doesn't satisfy this, as we have the following:
  \begin{align*}
    \beta_\lambda(1,2) &= 3 - 2 + (n-2) = n-1\\
    \beta_\lambda(1,3) &= 3 - 1 + n = n+2\\
    \beta_\lambda(2,3) &= 2 - 1 + 3 = 4.
   \end{align*}
  At least one of $\beta(1,2)$ and $\beta(1,3)$ is nonzero, since $\beta_\lambda(1,3) - \beta_\lambda(1,2) = 3 \not\equiv 0 \pmod e$, and hence at least one of $M_2$ and $M_3$ is empty.
  Hence at least one of 2 or 3 is normal in $(n,n-2,1)'$, and $\lambda_{**} = (n+1,n-2)$.

  For $\lambda_{*p}$, we immediately see from Figure \ref{OddRes} that the only option is $(n,n-1)$.
  \begin{figure}
    \[
      \cbr{
        \begin{gathered}
        \ydiagram{2,2,2,1,1,1}. \; \;
        \ydiagram{2,2,2,2,1}, \; \;
        \ydiagram{3,2,2,1,1}
      \end{gathered} 
    }\longrightarrow \begin{gathered}
      \ydiagram{2,2,2,1,1}
      \end{gathered}
      \hspace{50pt}
      \cbr{
        \begin{gathered}
          \ydiagram{2,2,2,2,1}
        \end{gathered}
      }\longrightarrow
      \begin{gathered}
        \ydiagram{2,2,2,1,1},\;\;
        \ydiagram{2,2,2,2}.
      \end{gathered}
    \]
    \caption{
      Illustration of the partitions of $9$ which can, via row removal, yield $(n,n-2)'$ alone, or both $(n,n-2)'$ and $(n-1,n-1)'$.
    }\label{OddRes}
  \end{figure}
  
  We can perform a similar argument for the $V^{2n}$ case, finding now that 
  \[\Res \; D^{\mu_{**}} \simeq D^{(n,n-1)'} \simeq \Res \; D^{(n,n)'}\] and 
  \[\Res \; D^{\mu_{*p}} \simeq D^{(n,n-1)'} \oplus D^{(n+1,n-2)'} \simeq \Res \; D^{(n+1,n-1)'}.\]
  
  Through a similar process, we see that $\mu_{*p} = (n+1,n-1)'$.
  We narrow down $\mu_{**}$ to one of $(n,n)'$ or $(n,n-1,1)'$, and note that
  \begin{align*}
    \beta_\mu(1,2) &= 3 - 2 + (n-1) = n\\
    \beta_\mu(1,3) &= 3 - 1 + n = n+2\\
    \beta_\mu(2,3) &= 2 - 1 + 2 = 3
  \end{align*}
  and hence at least one of 2 or 3 is normal, $\Res D^{(n,n-1,1)'}$ is not irreducible, and $\mu_{**} = (n,n)'$, finishing our proof.
\end{proof}
\begin{corollary}
  We have the following isomorphisms of representations:
  \begin{align*}
    V^{2n} \simeq 3D^{(n+1,n-1)'} \oplus 2D^{(n,n)'}
    V^{2n - 1} \simeq 3D^{(n,n-1)'} \oplus 2D^{(n+1,n-2)'}
  \end{align*}
\end{corollary}

 
\section{Explicit Relationships}


\newpage
\appendix
\section{Compatibility of Representations with the Relations}
In general, we define representations above for the free algebra on generators $\cbr{T_i}$.
Recall that we may give a presentation of $\SH$ having generators $T_i$ and relations
\begin{align}
  (T_i - q)(T_i + 1) &= 0 \label{quadratic}\\
  T_iT_{i+1}T_i &= T_{i+1}T_iT_{i+1} \label{braid1}\\ 
  T_iT_j &= T_jT_i \hspace{40pt} \abs{i - j} > 1. \label{braid2}
\end{align}
We call \eqref{quadratic} the \emph{quadratic relation} and \eqref{braid1}, \eqref{braid2} the \emph{braid relations}.
It is easily seen that a representation of $\SH$ is equivalent to a representation of the free algebra $k\langle T_i \rangle$ which acts as 0 on the relations (henceforth referred to as \emph{compatibility} with the relations).
We will prove in the following sections that $V$ and $W$ are compatible with the Hecke algebra relations.

\subsection{The Crossingless Matchings Representaiton}
\label{Cross Relations}
Take some basis vector $w_i$.
We will first check \eqref{quadratic} by case work:
\begin{itemize}
  \item Suppose there is an arc $(i,i+1)$.
    Then, $(T_i-q)(T_i + 1)w = (1 + q)\brk{(1 + T_i)w - (1 + q)w} = 0$, giving \eqref{quadratic}.

 
  \item Suppose there is no arc $(i,i+1)$ and $i,i+1$ do not both have anchors;
    then $(T_i +  1)w = q^{1/2}w''$ for some basis vector $w'$ having arc $(i,i+1)$, and the computation follows as above for \eqref{quadratic}.
  \item Suppose $i,i+1$ are anchrors;
    then $(T_i + 1)w = 0$, giving \eqref{quadratic}.
\end{itemize}
   
\vspace{5pt}
Now we verify \eqref{braid1}.
Let $h := (1 + T_i)(1 + T_{i+1})(1+T_i)$, and let $g := (1 + T_{i+1})(1 + T_i)(1 + T_{i+1})$.
Note the following expansion:  
  \begin{align*}
      hw
      &= 1 + 2T_i + T_i^2 + T_{i+1} + T_iT_{i+1} + T_{i+1}T_i + T_iT_{i+1}T_i\\
      &= 1 + (1+q)T_i + T_{i+1} + T_iT_{i+1} + T_{i+1}T_i + T_iT_{i+1}T_i.
    \end{align*}
    An analogous formula gives an analogous equality in $g$.
    Hence we have
    \[
      (h-g)w = q(T_i - T_{i+1}) + T_iT_{i+1}T_i - T_{i+1}T_iT_{i+1}.
    \]
    Hence we may equivalently check that $(h-g)w = q(T_i - T_{i+1})$.
    This is illustrated in Figure \ref{braid1arc}.
    \begin{figure}[b]
  \[
    \GeneralizedNAction{5}{1/2,3/4}{1}{5/1}{3/4}{2/1, 1/2,2/3}{1/4,2/3}{5/1}{q^{3/2}}
    \hspace{20pt}
    \GeneralizedNAction{5}{1/2,3/4}{1}{5/1}{3/4}{1/1, 2/2,1/3}{1/2,3/4}{5/1}{q(q+1)}
  \]
  \[
    \NAction{6}{1/6,2/5,3/4}{1/1,2/2,1/3}{1/2,3/4,5/6}{q^{3/2}}
    \hspace{20pt}
    \NAction{6}{1/6,2/5,3/4}{2/1,1/2,2/3}{1/6,2/3,4/5}{q^{3/2}}
  \]
  \[
    \GeneralizedZeroAction{4}{3/4}{2}{1/1,2/2}{.5}{1}
    \hspace{20pt}
    \GeneralizedNAction{4}{3/4}{2}{1/1,2/2}{.5}{2/1, 1/2,2/3}{2/3}{1/1,4/2}{q(q+1)}
  \]
  \caption{
    Here we verify in small cases that $hw = qT_i$ and $gw = qT_{i+1}$.
    These 6 cases cover the situations that there is an arc among the indices $i,i+1,i+2$, that there isn't and there are not two arcs, and that there are two arcs.
  }
  \label{braid1arc}
\end{figure}

Lastly, we have the equation
\[
  (1 + T_i)(1 + T_j) - (1 + T_j)(1 + T_i) = T_iT_j - T_jT_i
\]
and hence we simply need to verify that $(1 + T_i)$ and $(1 + T_j)$ commute, which the reader may easily check.

\subsection{The Fibonacci Representation} 
\label{Fib Relations}
Similar to before, the reader may verify that \eqref{braid2} follows easily, and the others may be verified on strings of length 3 and 4.
By considering the coefficients in order of \eqref{Fib Action}, the quadratic relation \eqref{quadratic} gives the following quadratics:
\begin{equation}
  \begin{split}
    (\vara - q)(\vara + 1) &= 0\\
    (\varb - q)(\varb + 1) &= 0\\
    \varc\vard + \vard\vare &= (q-1)\vard\\
    \varc^2 + \vard^2 &= (q -1)\varc + q\\
    \vare^2 + \vard^2 &= (q -1)\vare + q
  \end{split}
\end{equation}
The first two of these are easily verified for any $q$.
Since $\vard \neq 0$, the third is equivalently given by
\[
  (q - 1) = \varc + \vare = t(q\tau - 1 + q - \tau) = (\tau^2 + \tau)(q - 1)
\]
or that $\prn{\tau^2 + \tau - 1}(q-1) = 0$.
One may verify that \[\tau^2 + \tau - 1 = q^6 + 2q^5 + q^4 + q^3 + q^2 - 1 = (-1+q+q^2)\brk{5}_q = 0.\]

The fourth is given by the quadratic
\[
  \tau^2\brk{(q\tau-1)^2 - \tau(q+1)} = \tau(q-1)(q\tau - 1) + q
\]
or equivalently,
\[
  (\tau^2 + \tau - 1)\brk{q\prn{qt^2 + 1} + t} = 0
\]
which is true for every $q$.

The fifth is similarly given by
\[
  (\tau^2 + \tau - 1)\brk{q\prn{qt + 1} + t^2} = 0 
\]
which is true for every $q$.

\vspace{7pt}
We now verify \eqref{braid1}.
We may order the basis for $V^4$ as follows:
\[
  \cbr{(pppp),(*pp*),(ppp*),(*ppp),(*p*p),(p*p*),(pp*p),(p*pp)}.
\]
Then, in verifying the braid relation \eqref{braid1} in this order, we encounter the following quadratics (with tautologies and repetitions omitted):
\begin{align*}
    \vara\vare^2 + \varb\vard^2 &= \vara^2 \vare\\
    \vara\vard\vare + \varb\varc\vard &= \vara\varb\vard\\
    \varb\varc^2 + \vara\vard^2 &= \varb^2\varc\\
    \vara\varc^2 + \vard^2\vare &= \vara^2\varc\\
    \vard\vare^2 + \vara\varc\vard &= \vara\vard\vare
\end{align*}
Substituting in $\tau$ and dividing by $\delta$ whenever possible, these are equivalent to the vanishing of the following polynomials in $q$:
\begin{align*}
  -q (1 + q) (1 + q^2 + q^3) (2 + q + 3 q^2 + 2 q^3) \brk{5}_q &= 0\\
  (1 + 2 q + q^3 + q^4) \brk{5}_q &= 0\\
  (1+q)^2 (1+q^2+q^3) (1+3q^3 - q^4 + q^6)\brk{5}_q &= 0\\
  (1+q)^2 (1+q^2+q^3) (1+5q+5q^2+3q^3+3q^4+3q^5+q^6)\brk{5}_q &= 0\\
  (1+q) (1+q^2+q^3) (-1+2q+q^2+q^3+q^4)\brk{5}_q &= 0.
\end{align*}
Notably, each of these vanish when $e = 5$.

\section{Miscellaneous Algebra Facts}
Throughout the text, for some representation $V$, we refer to $\Res_{\SH(S_{l})}^{\SH(S_m)} V$ without specifying exactly which subalgebra $\SH(S_{l})$.
\begin{proposition}
  Suppose $B,B'$ are subalgebras of the $k$-algebra $A$ with $B = uB'u^{-1}$, and let $V$ be a representation of $A$.
  Then, the linear isomorphism $V \xrightarrow{\phi} V$ given by $v \mapsto uv$ causes the following to commute for any $b \in B$:
  \[
    \begin{tikzcd}
      V \arrow[r,"\phi"] \arrow[d,"b"] & V \arrow[d,"ubu^{-1}"]\\
      V \arrow[r,"\phi"] & V
    \end{tikzcd}
  \]
  Hence, through the identification of $B$ and $B'$ via conjugation, we have $\Res_{B}^A V \simeq \Res_{B'}^A V$
\end{proposition}
\begin{proof}
  This is simply given by $(ubu^{-1})uv = ubv$.
\end{proof}

\begin{corollary}
  Suppose $\SH',\SH''$ are two subalgebras of $\SH(S_m)$ generated by $l$ reflections and $V$ is a representation of $\SH$.
  Then, $\Res_{\SH'}^{\SH} V \simeq \Res_{\SH''}^\SH V$.
\end{corollary}
\begin{proof}
  Let $\SH'$ and $\SH''$ be the subalgebras of $\SH(S_m)$ generated by the reflections $\cbr{T_{i_1},\dots,T_{i_l}}$ and $\cbr{T_{i_1},\dots,T_{i_{j-1}},T_{i_j + 1},T_{i_{j+1}},\dots,T_{i_l}}$ for $1 \leq i_1 < \dots < i_{j-1} < i_j + 1 < i_{j+1} < \dots < i_l \leq n$.
  It is sufficient to prove that $\SH'$ and $\SH''$ are conjugate;
  then transitivity gives conjugacy of any $S_l \subset S_m$, and the previous proposition gives isomorphisms of the representations.
  
  In fact, the reader can verify that $\SH'' = T_{i_j}\SH'T_{i_j}^{-1}$.
\end{proof}


\end{document}

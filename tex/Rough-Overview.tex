\documentclass{amsart}   
\usepackage{RepStyle} 
\usepackage{NatMacros}

% Emulates \subsection* but doesn't add to ToC (I don't know why amsart is weird 
\newcommand{\fakesubsection}[1]{
  \vspace{7pt}
  \noindent \textbf{#1.}
}

\makeatletter
%Table of Contents
\setcounter{tocdepth}{3}

% Add bold to \section titles in ToC and remove . after numbers
\renewcommand{\tocsection}[3]{%
  \indentlabel{\@ifnotempty{#2}{\bfseries\ignorespaces#1 #2\quad}}\bfseries#3}
% Remove . after numbers in \subsection
\renewcommand{\tocsubsection}[3]{%
  \indentlabel{\@ifnotempty{#2}{\ignorespaces#1 #2\quad}}#3}
\let\tocsubsubsection\tocsubsection% Update for \subsubsection
%...

\newcommand\@dotsep{4.5}
\def\@tocline#1#2#3#4#5#6#7{\relax
  \ifnum #1>\c@tocdepth % then omit
  \else
    \par \addpenalty\@secpenalty\addvspace{#2}%
    \begingroup \hyphenpenalty\@M
    \@ifempty{#4}{%
      \@tempdima\csname r@tocindent\number#1\endcsname\relax
    }{%
      \@tempdima#4\relax
    }%
    \parindent\z@ \leftskip#3\relax \advance\leftskip\@tempdima\relax
    \rightskip\@pnumwidth plus1em \parfillskip-\@pnumwidth
    #5\leavevmode\hskip-\@tempdima{#6}\nobreak
    \leaders\hbox{$\m@th\mkern \@dotsep mu\hbox{.}\mkern \@dotsep mu$}\hfill
    \nobreak
    \hbox to\@pnumwidth{\@tocpagenum{\ifnum#1=1\bfseries\fi#7}}\par% <-- \bfseries for \section page
    \nobreak
    \endgroup
  \fi}

\def\l@subsection{\@tocline{2}{0pt}{2.5pc}{5pc}{}}
\makeatother

\begin{document}

\title[Some Graphical Realizations of Two-Row Specht Modules of Hecke Algebras]{Some Graphical Realizations of Two-Row Specht Modules of Iwahori-Hecke Algebras of the Symmetric Group}
\author[Miles Johnson \& Natalie Stewart]{Miles Johnson \& Natalie Stewart\\
  Mentor Oron Propp\\
Project Suggested by Roman Bezrukavnikov\\ \; \\
July 31, 2019
}

 \begin{titlepage}
  \maketitle
  \begin{abstract}
    We consider the Iwahori-Hecke algebra of the symmetric group on $2n + r$ letters with parameter $q \in k^\times$.
    Let $e$ be the smallest integer such that $\brk{e}_q = 0$, or set $e = \infty$ if none exist.
    We modify Khovanov's Crossingless Matchings to include $2n$ nodes and $r$ anchors, and prove that the corresponding module is isomorphic to the Specht module $S^{(n+r,n)}$ when the specht module is irreducible and $e > n$.
    Additionally, we prove heuristics in support of the general case.
    Lastly, when $e = 5$, we prove an isomorphism between $D^{(n+r,n)}$ with $r \leq 3$ and a subrepresentations of Jordan-Shor's fibonacci representation.
  \end{abstract}

\tableofcontents

\end{titlepage}

\section{Introduction}
Let $S_{2n+r}$ be the symmetric group on $2n+r$ indices with $2n + r \geq 2$, let $\SH = \SH_{k,q}(S_{2n+r})$ be the corresponding Hecke algebra over field $k$ with parameter $q \in k^\times$ having square root $q^{1/2}$, and let $\cbr{T_i}$ be the reflections generating $\SH$.
Let $\brk{m}_q = 1 + q + \dots + q^{m-1}$ be the $q$-number of $m$. 
Let $e$ be the smallest positive integer such that $\brk{e}_q = 0$, and set $e = \infty$ if no such integer exists.
Either $q = 1$ and $e$ is the characteristic of $k$ (with $0$ replaced by $\infty$), or $q \neq 1$ and $q$ is a primitive $e$th root of unity.

Throughout the text, we will refer to partitions of $2n + r$;
identify each partition with a tuple $\lambda = (\lambda_1^{a_1},\dots,\lambda_l^{a_l})$ having $\lambda_i > \lambda_{i+1}$, $a_i > 0$, and $\sum_i a_i\lambda_i = 2n + r$.
Identify each of these with a subset $\brk \lambda \subset \NN^2$ as defined in Kleshchev, and define $\lambda(i) = (\lambda_1^{a_1},\dots,\lambda_{i-1}^{a_{i-1}},\lambda_i^{a_i - 1},\lambda_i-1,\lambda_{i+1}^{a_{i+1}},\dots,\lambda_l^{a_l})$ to be the partition with the $i$th row removed.

Fixing some partition $\lambda$, for $1 \leq i \leq j \leq l$, let $\beta(i,j)$ be the hook length
\[
  \beta(i,j) = \lambda_i - \lambda_j + \sum_{t = i}^j a_t.
\]
Then, adopting Kleshchev's terminology, $j$ is normal in $\lambda$ if $\beta(i,j) \not\equiv 0 \pmod e$ for all $i < j$, and $j$ is good if it is the largest normal number (these are stronger conditions than generally necessary). 

Let $S^{(n+r,n)'}$ be the Specht module corresponding to the young diagram with two columns with height difference $r$, and let $D^{(n+r,n)'}$ be the corresponding irreducible quotient.
The purpose of this writing is to characterize these representation via an isomorphism with two graphical representations of $\SH$.

\fakesubsection{Crossingless Matchings}

\begin{definition}
  A \emph{crossingless matching on $2n+r$ indices with $r$ anchors} is a partition of $\cbr{1,\dots,2n+r}$ into $n$ parts of size $2$ and $r$ of size 1 such that no two parts of size two``cross'', i.e. there are no parts $(a,a')$ and $(b,b')$ such that $a < b < a' < b'$, and no parts of size one are ``inside'' of a part of size two, i.e. there are no $c, (a,a')$ such that $a < c < a'$.
  We will call these arcs and anchors, respectively.
  Then, define $W^r_{2n+r}$ to be the $k$-vector space with basis the set of generalized crossingless matchings on $2n+r$ indices with $r$ anchors.

  In order for this to be a $\SH$-module, endow this with the action given by Figure \ref{Action}; 
  if a ``loop'' is created, scale by $q+1$, if a loop is not created and the action involves fewer than 2 anchors, deform into a new crossingless maching and scale by $q^{1/2}$, and if it involves two anchors, scale by 0.
  We verify that this is well-defined in appendix \ref{Cross Relations}.
\end{definition}

\begin{figure}[b]
  \[
    \begin{tabular}{l l}
      \GeneralizedAction{6}{1/4,2/3}{2}{5/1,6/2}{1}{2}{1/4, 2/3}{5/1,6/2}{(1+q)}
      \hspace{20pt}
      &
      \GeneralizedAction{6}{1/4,2/3}{2}{5/1,6/2}{1}{3}{1/2, 3/4}{5/1,6/2}{q^{1/2}}\\
      \GeneralizedAction{6}{1/4,2/3}{2}{5/1,6/2}{1}{4}{2/3, 4/5}{1/1,6/2}{q^{1/2}}
      &
      \GeneralizedZeroAction{6}{1/4,2/3}{2}{5/1,6/2}{1}{5}
    \end{tabular}
  \]
    
  \caption{Illustration of the actions $(1 + T_i)w_{\abs{W^2_6}}$.
    In general, we act by deleting loops, deforming into a new crossingless matching, and scaling by either $q^{1/2}$, $(q + 1)$, or 0.
  }
  \label{Action}
\end{figure}

Let the length of an arc $(i,j)$ be $l(i,j) := j - i + 1$.
Note that the crossingless matchings on $2n$ indices with no anchros can all be identified with a list of $n$ integers describing the lengths of the arcs from left to right;
using this, we may order the crossingless matchings with no anchors in increasing lexicographical order in order to obtain an order on the subbasis containing a particular set of anchors;
let the basis be ordered first by the position of the anchors in decreasing lexicographical order, then increasing for the matchings between each anchor.
Let this basis be $\cbr{w_i}$.
This basis is illustrated for $W_{5}^1$ in Figure \ref{S5 Basis}. 

\begin{figure} 
  \def\cbasisspacing{5mm}
  $\cbr{
    \begin{gathered}
      \GeneralizedMatching{5}{1/2, 3/4}{1}{5/1}{3/4}, \hspace{\cbasisspacing}
      \GeneralizedMatching{5}{1/4, 2/3}{1}{5/1}{3/4}, \hspace{\cbasisspacing}
      \GeneralizedMatching{5}{1/2, 4/5}{1}{3/1}{3/4}, \hspace{\cbasisspacing}
      \GeneralizedMatching{5}{2/3, 4/5}{1}{1/1}{3/4}, \hspace{\cbasisspacing}
      \GeneralizedMatching{5}{2/5, 3/4}{1}{1/1}{3/4}, \hspace{\cbasisspacing}
     \end{gathered}}$ 
    \caption{The basis for $W_5^1$.}
  \label{S5 Basis}
\end{figure} 

Note that the representations $W_{0 + r}^r$ and $S^{(r)}$ are isomorphic to the sign representation; 
we will prove that $W := W_{2n+r}^r$ and $S := S^{(n+r,n)'}$ are isomorphic as representations in the case that $e > n$ and $S^{(n+r,n)'}$ is irreducible.
Note that, when $r = 0$, these have the same dimension given by the $n$th catalan number $C_n$.
  
\def\vara{\alpha_1}
\def\varb{\alpha_2}
\def\varc{\varepsilon_1}
\def\vard{\delta}
\def\vare{\varepsilon_2}
\def\vs{\texttt{*}}
\def\vp{\texttt{0}}
\fakesubsection{Fibonacci Representation}
Now suppose that $e = 5$ and $k$ contains the algebraic number $(-1 - q^2 - q^3)^{3/2}$.
Let $V^m$ be a $k$-vector space with basis given by the strings $\cbr{\vs,\vp}^{n+1}$ such that the character $\vs$ never appears twice in a row. 
We will suppress the superscript whenever it is clear from context.

We wish to endow this with a $\SH$-action which acts on a basis vector only dependent on characters $i,i+1,i+2$, sending each basis vector to a combination of the other basis vectors having the same characters $1,\dots,i,i+2,\dots,n+1$ as follows:

\begin{equation} 
  \begin{split}
    T_1 \, \prn{\vs\vp\vp} &:= \vara\prn{\vs\vp\vp}\\
    T_1 \, \prn{\vp\vp\vs} &:= \vara\prn{\vp\vp\vs}\\
    T_1 \, \prn{\vs\vp\vs} &:= \varb\prn{\vs\vp\vs}\\
    T_1 \, \prn{\vp\vs\vp} &:= \varc\prn{\vp\vs\vp} + \vard\prn{\vp\vp\vp}\\
    T_1 \, \prn{\vp\vp\vp} &:= \vard\prn{\vp\vs\vp} + \vare\prn{\vp\vp\vp}
\end{split} \label{Fib Action} 
\end{equation}
for constants
\begin{equation}
  \begin{split}
  \vara &= -1\\
  \varb &= q\\
  \varc &= \tau(q\tau - 1)\\
  \vard &= \tau^{3/2}(q + 1)\\
  \vare &= \tau(q-\tau)\\
  \tau &= -1-q^2-q^3
\end{split} \label{Fib Constants} 
\end{equation}
with $T_i$ acting similarly on the substring $i,i+1,i+2$.
We will verify that this is a representation of $\SH$ in Appendix \ref{Fib Relations}

This contains 4 subrepresentations based on the first and last character of the string, which are not modified by $\SH$.
Label the subrepresentation spanned by strings $(\vs\dots\vs)$ by $V_{\vs\vs}$, and similar for the other 3.
It is easy to see that $V_{\vs\vp} \simeq V_{\vp\vs}$, so that
\[
  V \simeq 2V_{\vs\vp} \oplus V_{\vs\vs} \oplus V_{\vp\vp}.
\]
We will show that $V_{\vp\vp} \simeq V_{\vs\vp} \oplus V_{\vs\vs}$, and give the following isomorphisms with irreducible quotients of specht modules depending on the parity of the number of indices in $\SH$:
\begin{equation}
  \begin{split}    
    V^{2n}_{\vs\vs} &\simeq D^{(n,n)'}\\ 
    V^{2n-1}_{\vs\vs} &\simeq D^{(n+1,n-2)'}\\
    V^{2n}_{\vs\vp} &\simeq D^{(n+1,n-1)'}\\
    V^{2n-1}_{\vs\vp} &\simeq D^{(n,n-1)'}.
  \end{split} \label{Fib Isos}
\end{equation}

% Acknowledgements should be given section-level heading in toc, subsection-level in text.
\addcontentsline{toc}{section}{\textbf{Acknowledgements}}
\fakesubsection{Acknowledgements}

\newpage
 
\section{Preliminaries on Specht Modules}
\subsection{Irreducibility of $S^\lambda$}
Let $k$ have characteristic $\ell$;
then, set
\[
  p := \begin{cases}
    \ell & \ell > 0\\
    \infty & \ell = 0
  \end{cases}.
\]  
For $h$ a natural number, let $\nu_p(h)$ be the $p$-adic evaluation of $h$.
As a convention, set $\nu_\infty(h) = 0$ for all $h$.
Define the function $\nu_{e,p} : \NN \rightarrow \cbr{-1} \cup \NN$ by
\[
  \nu_{e,p}(h) := \begin{cases}
    \nu_p(h) & e \mid h\\
    -1 & e \nmid h
  \end{cases}.
\]
Lastly, let $h_{ab}^\lambda$ be the hook length of node $(a,b)$ in $\brk \lambda$.

\begin{theorem}[James-Mathas]
  The following are equivalent:
  \begin{enumerate}
    \item $S^{\lambda} \simeq D^{\lambda}$.
    \item $\lambda$ is $e$-restricted and $S^\lambda$ is irreducible.
    \item $\nu_{e,p}\prn{h_{ab}^\lambda} = \nu_{e,p}\prn{h_{ac}^\lambda}$ for all nodes $(a,b)$ and $(a,c)$ in $\brk \lambda$.
  \end{enumerate}\qed
\end{theorem}


\begin{corollary}\label{S irreducibility}
  Assume $e > 2$ and $n > 0$.
  
  \begin{enumerate}[label={(\roman*)}]
    \item Suppose $e > n$.
      Then, $S^{(n+r,n)'}$ is irreducible iff $e \nmid l$ for all $r + 2 \leq l \leq n + r + 1$.
    \item Suppose $e \leq n$.
      Then, $S^{(n+r,n)'}$ is irreducible iff $e | r + 1$ and $\nu_{e,p}(h_{ab}^\lambda)$ acquires exactly two values over rows $a \leq n$.
  \end{enumerate}  
\end{corollary}
\begin{proof}
  \textbf{(i)}
  Note that $\nu_p(l) \neq -1$ for all $l$ and only hook lengths in the first column may vanish mod $e$;
  hence we may equivalently prove that $e$ divides no hook lengths in the first $n$ rows of the first column.
  These hook lengths are precisely $r + 2,\dots,n+r+1$.

  \textbf{(ii)}
  Note that we have $\nu_{e,p}\prn{h^\lambda_{n-e+1,2}} \neq -1$.
  Suppose that $e \nmid r + 1$.
  Then, \[\nu_{e,p}\prn{h^\lambda_{n-e+1,1}} = \nu_{e,p}\prn{h^\lambda_{n-e,2} + r + 1} = -1,\]
  giving $S^{(n+r,n)'}$ reducible.

  Note that $\nu$ acquires at least two values since $e$ is a hook length.
  Now, suppose that \[0 \leq \nu_{e,p}(h^{\lambda}_{ab}) < \nu_{e,p}\prn{h^{\lambda}_{a'b'}}\] and $S^{(n+r,n)'}$ is irreducible.
  If $a=a'$ then we have reducibility, so assume $a \neq a'$.
  Further, if $b = b'$, then we may replace $b$ with the other column;
  hence we may assume WLOG that $b \neq b'$ as well.

  Note that $p$-adic valuation is monotonic, so $h^\lambda_{ab} < h^\lambda_{a'b'}$.
  If $b < b'$, then $\nu_p(h_{ab}) = \nu_p(h_{ab'}) < \nu(h_{a'b'})$ while $h_{ab'} > h_{a'b'}$, a contradiction.
  If $b > b'$, then $\nu_p(h_{ab}) = \nu_p(h_{ab'})$ and $h_{ab} < h_{ab'}$, so there is some $c < b$ with $\nu_{e,b}(h_{ab}) = \nu_{e,p}(h_{ac})$; we may replace $b$ with $c$, and repeat until $b = b'$ to reach contradiction.

  Finally, assume $e | r + 1$ and $\nu_{e,p}$ acquires two values over the top $n$ rows.
  Since $e | r + 1$, $e \mid h_{a1}^\lambda$ iff $e \mid h_{a2}$;
  since $\nu_{e,p}$ acquires only two values, this proves that $\nu_{e,p}(h_{a1}) = \nu_{e,p}(h_{a2})$ across these rows, giving the lemma.
\end{proof}

In the case that $\lambda = (n+r,n)$ and $e$ satisfy hypothesis \eqref{S irreducibility}.(i), say that $\lambda$ is \emph{$e$-top-indivisible}.

This is complicated and annoying, and the second amounts to a condition on $p$.
We will illustrate that irreducibility interpolates between $p = 2$ and $p$ large:
\begin{corollary}
  Assume $2 < e \leq n$.
  \begin{enumerate}[label={(\roman*)}]
    \item Suppose $p > n + r + 1$. Then, $S^{(n + r,n)'}$ is irreducible iff $e | r + 1$.
    \item Suppose $p = 2$. Then, $S^{(n+r,n)'}$ is not irreducible.
  \end{enumerate}
\end{corollary}
\begin{proof}
  (i) is clear.
  (ii) is given by noting that $e + r + 1 \geq 2e$;
  then, $h_{n-e+1,1} - h_{n-e+1,2} \geq 1$, giving reducibility.
\end{proof}

\subsection{Branching}
\begin{theorem}[Kleshchev-Brundan]
  We have the following isomorphisms of vector spaces
  \begin{align*}
    \Homarg{\SH'}{S^\mu,\Res D^\lambda} 
    &\simeq \begin{cases}
      k & \mu \xrightarrow{\text{normal}} \lambda\\
      0 & \text{otherwise}
    \end{cases}\\
    \Homarg{\SH'}{D^\mu,\Res D^\lambda} 
    &\simeq \begin{cases}
      k & \mu \xrightarrow{\text{good}} \lambda\\
      0 & \text{otherwise}
    \end{cases}
  \end{align*}
  and $\Res D^\lambda$ is semisimple if and only if every normal number in $\lambda$ is good.\qed
\end{theorem}

\begin{corollary}
  Suppose $n,r > 0$.
  Then, we may characterize the socle of $\Res \, D^\lambda$ as follows:
  \[
    \soc\prn{\Res \, D^{(n+r,n)'}} \simeq \begin{cases}
      D^{(n + r - 1,n)'} & e \mid r+2\\
      D^{(n + r,n-1)'} & e \nmid r+2, \; e \mid r\\
      D^{(n+r-1,n)'} \oplus D^{(n+r,n-1)} & e \nmid r+2,r
    \end{cases}
  \]
\end{corollary}
\begin{proof}
  This amounts to computations of the hook lengths $\beta(1,2)$ and $\gamma(1,2)$:
  \begin{align*}
    \beta_\lambda(1,2) &= r + 2\\
    \gamma_\lambda(1,2) &= r
  \end{align*}
  Since $2$ is the largest removable number, $D^{(n+r,n-1)'} \subset D^{(n+r,n)'}$ iff $e \nmid r + 2$.
  Further, if $e \nmid r + 2$, then $D^{(n+r-1,n)'} \subset D^{(n+r,n)'}$ iff 1 is good iff $e \nmid r$
\end{proof}


\begin{proposition}\label{Combinatorics}
  Suppose that $e,n,r$ satisfy criterion (i) of Proposition \ref{S irreducibility}.
  Let $\lambda$ be a partition of $2n + r$.
  \begin{enumerate}[label={(\roman*)}]
    \item Suppose $r > 0$.
      If $D^\lambda$ has the composition series 
      \begin{equation}\label{D Composition Series}
        0 \subset D^{(n+r-1,n)'} \subset \Res \, D^{\lambda}
      \end{equation}
      with factor $\Res \, D^{\lambda} / D^{(n+r-1,n)'} \simeq D^{(n+r,n)'}$, then $\lambda = (n+r,n)'$.
    \item Suppose $r = 0$ and $D^{(n,n-1)'} \simeq \Res \, D^\lambda$.
      Then $\lambda = (n,n)'$.
  \end{enumerate}
\end{proposition}
\begin{proof}
  \textbf{(i)}
  Let $\varpi := (n+r-1,n,1)$, let $\varsigma := (n+r-1,n+1)'$, and let $\mu := (n + r,n)'$.
  Since $D^{(n+r-1,n)'} \subset \Res \, D^\lambda$, we have $(n + r - 1,n)' \longrightarrow \lambda$, implying $\lambda = \varpi,\varsigma,\mu$.
  
  We will show that $\varpi, \varsigma$ do not have socle compatible with \eqref{D Composition Series};
  then, we will have $\lambda = \mu$.

  First, we focus on $\varpi$;
  Suppose $r > 1$ and $e \mid n + 1$;
  then, $\beta_{\varpi}(2,3) = n + 1 \equiv 0 \pmod e$, so $3$ is not normal and $D^{(n+r-1,n)'}$ is not in the socle of $D^\varpi$, giving $\lambda \neq \varpi$.
  
  Now suppose $r > 1$ and $e \nmid n + 1$.
  Then $S^{(n+r-2,n,1)'} \simeq D^{(n+r-2,n,1)'}$.
  Then, we have $D^{(n + r - 2,n,1)'} \simeq S^{(n + r - 2,n,1} \subset \Res \, D^{\varpi}$ since 1 is normal;
  this is not a composition factor in \eqref{D Composition Series}, so $\lambda \neq \varpi$.
  

  Similarly, assume $r = 1$;
  then, by the same logic, $D^{(n+r-1,n)'} \not\subset \Res \, D^\varpi$ or $D^{(n,n-1,1)} \subset \Res \, D^\varpi$, also implying $\lambda \neq \varpi$.
  \iffalse % keeping just in case
  \begin{align*}
    \beta_{\varpi}(1,2)     &= r + 1 \\
    \beta_{\varpi}(1,3)     &= n + r + 1 \\
    \beta_{\varpi}(2,3)     &= n + 1\\
    \gamma_{\varpi}(1,2)    &= r\\
    \gamma_{\varpi}(1,3)    &= n + r\\
    \gamma_{\varpi}(2,3)    &= n\\
  \end{align*}
  \fi

  \vspace{5pt}
  Now, focus on $\varsigma$;
  if $r < 2$, then $\varsigma$ is not a partition.
  If $r = 2$, then $\Res \, D^\varsigma \simeq D^{(n+1,n)'}$ is irreducible, contradicting \eqref{D Composition Series}. 
  If $r > 2$, then $D^{(n + r - 2,n+1)'}$ is irreducible, so $D^{(n+r-2,n+1)'} \simeq S^{(n+r-2,n+1)'} \subset D^\varsigma$, contradicting \eqref{D Composition Series}.

  \textbf{(ii)}
  This result is fundamentally similar to the previous result;
  since the socle of $D^\lambda$ is irreducible, we require that 1 is the only normal number, which pins $\lambda$.
\end{proof}

\iffalse
\begin{proposition}
  Suppose that $n > 0$ and $e \nmid n + r$.
  \begin{enumerate}[label={(\roman*)}]
    \item Suppose $r > 0$, $n > 1$, and either $e \nmid n + 1$ or $e \nmid n + r + 1, n + r - 1$.
      If $D^\lambda$ has the composition series 
      \begin{equation}\label{D Composition Series}
        0 \subset D^{(n+r-1,n)'} \subset \Res \, D^{\lambda}
      \end{equation}
      with factor $\Res \, D^{\lambda} / D^{(n+r-1,n)'} \simeq D^{(n+r,n)'}$, then $\lambda = (n+r,n)'$.
    \item Suppose $n = 1$ and $e \nmid r+2$, and suppose that we have composition series \eqref{D Composition Series}.
      Then $\lambda=(r+1,r+2)'$.
    \item Suppose $r = 0$, either $e \nmid n+2$ or $e > 3$, and $D^{(n,n-1)'} \simeq \Res \, D^\lambda$.
      Then $\lambda = (n,n)'$.
  \end{enumerate}
  The above conditions are unilaterally met whenever $e > n + r + 1$ and $(n,r) \neq (1,0)$.
\end{proposition}
\begin{proof}
  \textbf{(i)}
  Let $\lambda_1 := (n+r-1,n,1)$, let $\lambda_2 := (n+r-1,n+1)'$, and let $\lambda_3 := (n + r,n)'$.
  Since $D^{(n+r-1,n)'} \subset \Res \, D^\lambda$, we have $(n + r - 1,n)' \longrightarrow \lambda$, implying $\lambda = \lambda_i$ for some $i = 1,2,3$.
  We will show that $\lambda_1,\lambda_2$ each have good numbers other than 1, leading to elements of their socle that are not factors in \eqref{D Composition Series};
  then, we will have $\lambda = \lambda_3$.

  We may straightforwardly compute this.
  First assume $n \geq 2$, so that
  \begin{align*}
    \beta_{\lambda_1}(1,2) &= n + 1\\
    \beta_{\lambda_1}(1,3) &= n + r + 1\\
    \beta_{\lambda_1}(2,3) &= n + r - 1\\
    \beta_{\lambda_2}(1,2) &= n + r
  \end{align*}

  \textbf{(ii)}
  The choices are the same as above, but now $\lambda_1$ only has two removable nodes, so that
  \begin{align*}
    \beta_{\lambda_1}(1,3) &= r + 2\\
    \beta_{\lambda_2}(1,2) &= r + 1
  \end{align*}

  \textbf{(iii)}
  Let $\lambda_1 := (n,n-1,1)'$, let $\lambda_2 := (n+1,n-1)'$, and let $\lambda_3 := (n,n)'$;
  as before we know that $\lambda = \lambda_i$ for $i = 1,2,3$, so we will eliminate $\lambda_1,\lambda_2$.
  First, we will show that $\lambda_1$ has a good number other than 1:
  \begin{align*}
    \beta_{\lambda_1}(1,2) &= n\\
    \beta_{\lambda_1}(1,3) &= n + 2\\
    \beta_{\lambda_1}(2,3) &= n\\
  \end{align*}
  Hence we may rule out $\lambda_1$ when $e | n$.


  Suppose $D^{\lambda_2}$ satisfies the isomorphism;
  then the number 2 is normal.
  We may make the following computations:
  \begin{align*}
    \beta_{\lambda_2}(1,2) &= n + 2\\
    \gamma_{\lambda_2}(1,2) &= 3.
  \end{align*}
  When $e | n + 2$, this brings contradiction;
  when $e \nmid n + 2$ and $e > 3$, 2 is a good number (recalling that $e > 2$), giving non-semisimple restriction, a contradiction.
  Hence $\lambda = \lambda_3$ and the proposition is proven.
\end{proof}
\fi

\section{Crossingless Matchings and Specht Modules}
\subsection{Irreducibility of $M$}
\begin{lemma}
  \label{Cyclic}
  Every basis vector in $W_{2n + r}^r$ is cyclic.
\end{lemma}
\begin{proof}
  We have already proven this in the $r = 0$ case, so suppose that $r > 0$.

    Note that, between anchors $a<a'$ having no arc $b$ with $a < b < a'$, the $W_{a'-a}^0$ case allows us to generate the vector with all length-2 arcs between $a,a'$ and identical arcs/anchors outside of this sub-matching.\footnote{At the ends, we apply the $W_a^0$ case or the $W_{2n + r - a}^{0}$ case in the same way for the first $a$ or last $2n + r - a$ indices.}

  Applying this between each arc gives us a vector with length-2 arcs and anchors, and we may use the appropriate $(1+T_i)$ to move anchors to any positions, and the reverse process from above to generate the correct matchings between arcs and generate any other basis vector.
\end{proof}

Let $K := \bigcap_{i=1}^{2n+r-1} \ker (1 + T_i) = \ker \bigoplus_{i = 1}^{2n + r - 1} (1 + T_i)$.
This will be a large technical tool in our proof of irreducibility.

\iffalse
\begin{lemma}\label{Represented}
  Let $w_j$ be the basis vector with anchors $1,\dots,r$ and all arcs of maximal length.
  Suppose $w \in K \backslash \cbr 0$.
  Then, $w_j$ is represented in $w$.
\end{lemma}
\begin{proof}
  We will show this in steps; first, we show that, given that a vector is represented with anchors $1,\dots,s$, there must be a vector represented in $w$ with $(s+1)$st anchor, including when $s = 0$;
  this implies that a vector is represented with anchors $1,\dots,r$.
  Then, we will show that, given a vector is represented with anchors $1,\dots,r$ and first $s$ arc-lengths $n,n-2,\dots,n-2s$, there is a vector represented with these and the $(s+1)$st arc-length $n-2s-2$.
  This implies that $w_j$ is represented.
  
  \def\lemsep{7pt}
  \vspace{\lemsep}
  \textit{Step 1.}
  Suppose that $s < r$ is the maximal number such that a vector with anchors $1,\dots,s$ is represented.
  Take the vector $w_i$ which, among vectors represented in $w$ with anchors $1,\dots,s$, has $(s+1)$st anchor at minimal index $t > s + 1$,
  Then, $q^{-1/2}(1 + T_{t-1})w_i$ has anchors $1,\dots,s$ and a earlier index than $t$, so it was not represented before;
  further, for any other basis vector $w_l \neq w_i$ to map onto $q^{-1/2}(1 + T_{t-1})w_i$, we would require that $w_l$ has anchors $1,\dots,s$ and some other anchor at index $t' < t$, so it is not represented.
  Hence $w_i$ is unique among the vectors represented mapping onto $q^{-1/2}(1 + T_{t-1})w_i$, and $(1 + T_{t-1})w$ represents this vector, giving $w \notin \ker (1 + T_{t-1})$.
 
  When $s = 0$, this is similar, and we simply perform this logic on the 1st anchor.
  Each lead to contradiction, so we must have $s = r$. 

  \vspace{\lemsep}
  \textit{Step 2.}
  This step is similar;
  suppose that $s < n$ is the maximal number such that a vector with anchors $1,\dots,r$ and first $s$ arc-lengths $n,\dots,n-2s$ is represented.
  Take the vector $w_i$ which, among vectors represented in $w$ with anchors $1,\dots,r$ and first $s$ arc-lenths $n,\dots,n-2s$, has maximal length $t$ of the arc beginning at index $r + s + 1$.
  Then, $q^{1/2}(1 + T_{r + s + t})w_i$ is mapped to only by $w_i$ and vectors having anchors $1,\dots,r$ and first $s+1$ arc-lengths $n,\dots,n-2s,t'$ with $t' > t$, which are not represented in $w$;
  hence $q^{-1/2}(1 + T_{r + s + t})w_i$ is represented in $(1 + T_{r+s+t})w$, giving $w \notin \ker (1 + T_{r + s + t})$.
  The $s = 0$ case is similar, and implies that $s = n$.
\end{proof}
\fi

\begin{lemma}\label{Trivial}
  Suppose $e \nmid n + r + 1$.
  Then, $K = 0$.
\end{lemma}
\begin{proof}[Proof 1.]
  Consider the matrix $A = \bigoplus (1 + T_i)$ having kernel $K$.
  It is sufficient by lemma \ref{Represented} to show that $A$ includes a row $[0,\dots,0,1,0,\dots,0]$ with a nonzero entry only on the column $j$.

  Now, we may characterize the rows of $A$ as follows;
  if the row corresponding to $(1 + T_i)$ and mapping onto the element $w_l \in W$ is nonzero, then it is of the form $\brk{a_1,\dots,a_{\abs W}}$ where $a_l = 1 + q$, $a_m = q^{1/2}$ whenever $(1 + T_i)w_m = q^{1/2}w_l$, and $a_m = 0$ otherwise.
  
  Seeing this, the row corresponding to $(1 + T_{n + r})$ and $w_j$ has nonzero entries $q^{1/2}$ at $w_j$ and $(1 + q)$ at the vector $w$ agreeing with $w_j$ at all indices except having arcs at $(n+r-1,n+r)$ and $(n+r+1,n+r+2)$.
  Similar justification leads the row corresponding to $(1 + T_{n+r-1})$ at $w$ to have nonzero entries $q^{1/2}$ at $w$ and $(1 + q)$ at $w_j$ and the vector with anchors $1,\dots,r$ , arc $(n+r-3,n+r-2)$, and all other arcs maximum length.
  
  \begin{figure}
   \[
     \Action{10}{1/2,3/10,4/9,5/8,6/7}{2}{1/10,2/3,4/9,5/8,6/7}{q^{1/2}}
   \]
   \[
     \Action{10}{1/10,2/3,4/9,5/8,6/7}{2}{1/10,2/3,4/9,5/8,6/7}{q+1}
   \]
   \[
     \Action{10}{1/10,2/9,3/4,5/8,6/7}{2}{1/10,2/3,4/9,5/8,6/7}{q^{1/2}}
   \]
   \caption{Illustrated is the row constructed for transposition $(1 + T_2)$; clearly these are the only basis elements mapping to multiples of the desired element, and they relate to each other.
     replacing the outermost and/or innermost arc with an anchor typifies the rows constructed with three nonzero coefficients.}
 \label{Submatrix}
\end{figure}
 
  We may iterate this process as illustrated in Figure \ref{Submatrix}, eventually ending at a row with two nonzero entries, either an arc $(1,2)$ or an arc $(2,3)$, and all anchors otherwise left-aligned and arcs of maximum length.
  These rows together form an $(n + r) \times \abs{W_{2n + r}^r}$ submatrix of $A$ which has a nonzero column in the row corresponding to $j$, and has (by removing zero columns) the same column space as the following square matrix:
  \newcommand*\bigzero{\makebox(0,15){\text{\Huge0}}}
  \newcommand*\bigzerotwo{\makebox(-20,15){\text{\Huge0}}}
  \begin{equation}\label{Bnr}
    B_{n + r} := \begin{bmatrix}
      \begin{matrix}
      q + 1 & q^{1/2} & \\
      q^{1/2} & q + 1 & q^{1/2}\\
      & q^{1/2} & q + 1 & q^{1/2}\\
      &  & \ddots & \ddots
      \end{matrix}
      &   \bigzero\\
      \bigzerotwo & \begin{matrix}
        q^{1/2} & q+1 & q^{1/2}\\
        &  q^{1/2} & q + 1
       \end{matrix}
    \end{bmatrix}.
  \end{equation}
  We will show that this matrix is invertible;
  then, a sequence of elementary row operations will yield the identity, and in particular, when applied to $A$, will yield a row with a nonzero entry only on column $j$, giving $K = 0$.

  We may prove invertibility of this matrix by proving that $\det B_{n + r} = \brk{n + r + 1}_q$ inductively on $n+r$.
  This is satisfied for our base case $n + r = 1$, so suppose that it is true for each $m < n + r$.
  Then,
  \begin{align*}
    \det B_{n+r} &= (q + 1) \det B_{n + r - 1} - q \det B_{n + r - 2}\\
    &= (q + 1)(1 + \dots + q^{n + r - 1}) - (q + \dots + q^{n + r - 1})\\
    &= 1 + \dots + q^{n + r}\\
    &= \brk{n + r + 1}_q.
  \end{align*}
  Hence $\det B_{n + r} \neq 0$, and $K = 0$.
\end{proof}
\begin{proposition}
  The representation $W_{2n + r}^r$ is irreducible when $e > n + r + 1$. 
\end{proposition}
\begin{proof}
  We proceed by induction on $2n + r$.
  Note that, by identification with the trivial and sign representations, the base case $2n + r = 2$ is already prove, so suppose we have proven this for each $2m + s < 2n + r$.

  Take some $w \in W$ and some $(1 + T_i)$ not annhialating $w$.
  Note that \[\ima (1 + T_i) = \Span\cbr{w_j \mid w_j \text{ contains arc }(i,i+1)}.\]
  Hence, as vector spaces, there is an isomorphism $\varphi:\ima(1 + T_i) \rightarrow W_{2(n-1) + r}^r$ ``deleting'' the arc $(i,i+1)$.
  This sends every basis vector to another basis vector.

  We will show that, for every action $(1 + T_j') \in \SH(S_{2(n-1) + r})$, there is some action $h_j \in W_{2n + r}^r$ such that the following commutes:
  \[
    \begin{tikzcd}
      \ima (1 + T_i) \arrow[r,"\varphi" above, "\sim" below] \arrow[d,"h_j"] & W_{2(n-1) + r}^r \arrow[d,"1 + T_j"]\\
      \ima (1 + T_i) \arrow[r,"\varphi" above, "\sim" below] & W_{2(n-1) + r}^r
    \end{tikzcd}
  \]
  Indeed, when $i \neq j$ this is given by $h_j = 1 + T_j$, and we have $h_i = q^{-1}(1 + T_i)(1 + T_{i+1})(1 + T_{i-1})$, as given by Figure \ref{bigloop}.
  
  Due to the inductive hypothesis, there is some action $h' \in \SH(S_{2(n-1) + r})$ sending $\varphi((1 + T_i)w)$ to a basis vector;
  then, the action $\SH$ generates the endomorphism $\varphi^{-1}h'\varphi$ sending $(1 + T_i)w$ to a basis vector, giving $w$ cyclic and hence $W_{2n + r}^r$ irreducible.
  \begin{figure}
  \[
    \Action{6}{1/6, 2/5, 3/4}{2}{1/6, 2/3, 4/5 }{q^{1/2}} \hspace{20pt} 
    \IsoAction{8}{1/8, 2/7, 3/4, 5/6}{3}{1/8, 2/5, 3/4, 6/7}{q^{1/2}}
  \]
  \[
    \Action{4}{1/4, 2/3}{2}{1/4,2/3}{(1 + q)} \hspace{20pt}
    \IsoAction{6}{1/6, 2/5, 3/4}{3}{1/6, 2/5, 3/4}{(1 + q)}
  \]
  \caption{The correspondence between the action of $(1 + T_2)$ on $w'_5 \in W^0_6$ and the action of $q^{-1}(1 + T_3)(1 + T_4)(1 + T_2)$ on the corresponding vector in $W^0_8$ having arc $(3,4)$ first, then on $w'_2 \in W^0_4$.
  This demonstrates that the action works with and without creating a loop.
  }
  \label{bigloop}
  \end{figure}
\end{proof}

\subsection{Correspondence}
\begin{proposition}\;
  
  \begin{enumerate}[label={(\roman*)}]
    \item
    Suppose that $n,r > 0$.
    Then, a filtration of $\Res W_{2n + r}^r$ is given by
    \begin{equation}
      0 \subset W_{2n + r - 1}^{r-1} \subset \Res \, W_{2n + r}^r \label{Filtration}
     \end{equation}
     with $\Res \, W_{2n + r}^r / W_{2n + r - 1}^{r-1} \simeq W_{2n + r - 1}^{r + 1}$. 
    \item
      We have the following isomorphism of representations:
      \begin{equation}
        W_{2n - 1}^1 \simeq \Res \; W_{2n}^0 \label{0 Restriction}
       \end{equation}
  \end{enumerate}
    When the case is type (i) from the irreduibility lemma, this is a composition series.
\end{proposition}
\begin{proof}
  \textbf{(i)}
  Note that we may identify the subrepresentation of $\Res \; W_{2n + r}^r$ having anchor $n$ with $W_{2n + r - 1}^{r-1}$.
  
  Let $U := \Res \; W_{2n + r}^r / W_{2n + r - 1}^{r - 1}$. 
  Let $\phi:U \rightarrow W_{2n + r - 1}^{r + 1}$ be the $k$-linear map which regards the arc $(i,2n + r)$ in $U$ as an anchor at $i$ in $W_{2n + r - 1}^{r + 1}$.
  It is not hard to verify that this is a well-defined isomorphism of vector spaces, so we must show that it is $\SH$-linear.

  Given a basis vector $w_j$ with arc $(i,2n + r)$, $\phi$ is clearly compatible with $T_{i'}$ with $i' \neq i,i-1$.
  Further, it's easy to verify that $\phi$ is compatible with $T_{i}$ and $T_{i-1}$, as actions on one anchor were designed for this deformation.
  When there are anchors $(i,i+1)$, then $\phi(T_iw_j) = T_i\phi(w_j) = 0$, and similar for $T_{i -1}$.
  Hence $\phi$ is an isomorphism of representations, and the statement is proven.

  \textbf{(ii)}
  This follows with the above proof, defining $W_{2n - 1}^{-1} := 0$
\end{proof}

\begin{theorem}
  Suppose $e > n + r + 1$.
  Then, $W_{2n + r}^r \simeq S^{(n+r,n)'}$.
\end{theorem}
\begin{proof}
  The case $n = 0$ is already proven, so suppose $n > 0$.

  By irreducibility, we know that $W_{2n + r}^r \simeq D^\lambda$ for some $e$-restricted partition $\lambda$.
  We will prove this inductively; by identification with the trivial and sign representations, the $2n + r = 2$ caseholds, so suppose this is true for $W_{2m + s}^s$ whenever $2m + s < 2n + r$ and $m + s \leq n + r$ (i.e. $e > m + s + 1$).

  By the inductive hypothesis and irreducibility, we have a composition series given by 
  \begin{equation}
    \label{Composition Series} 0 \longrightarrow D^{(n + r - 1,n)'} \longrightarrow \Res \, D^\lambda \longrightarrow D^{(n+r,n-1)'} \longrightarrow 0
   \end{equation}
  Hence the theorem is given by Proposition \ref{Combinatorics}.
\end{proof}

\iffalse
We may reuse the above structure of proof as above;
suppose $W_{2n + r}^r$ and $S^{(n+r,n)'}$ are irreducible, and $\lambda = (n+r,n)'$ is the unique diagram having composition series \eqref{Composition Series}.
Then, $W_{2n + r}^r \simeq S^{(n+r,n)'}$.

We have covered all irreducible cases for $r = 0$ and $n = 0$ above.
For example, another case where we may show all irreducible cases is $n = 1$, which we will do now.
\begin{lemma}\label{n=1 irreducibility}
    Set $n = 1$. 
    Then $W_{2 + r}^r$ is irreducible iff $e \nmid r + 2$.
\end{lemma}
\begin{proof}
  A basis element $w_i \in W$ is determined entirely by the position of the arc, of which there are $r + 1$ options.
  Hence $\dim W = r + 1$.

  Further, $\dim\prn{\ima (1 + T_i)} = 1$ for all $i$;
  by lemma \ref{Cyclic} $W$ is irreducible iff $K := \bigcap_{i = 1}^{r+1} \ker (1 + T_i) = \ker \bigoplus (1 + T_i)$ is trivial.
  By \ref{Trivial}, this is true iff $e \nmid r + 2$.
\end{proof}

\begin{theorem}
  Set $n = 1$ and suppose $e \nmid r + 2$.
  Then $W_{2 + r}^r \simeq S^{(1+r,1)'}$.
\end{theorem}
\begin{proof}
  Using lemma \ref{n=1 irreducibility} and \ref{S irreducibility} it is sufficient to prove that, for $D^\lambda$ having composition series \eqref{Composition Series}, we have $\lambda = (1+r,1)'$.
  Similar to before, we consider the options $\lambda_1 := (r,1,1)'$, $\lambda_2 := (r,2)'$, and $\lambda_3 = (1+r,1);$.
  We may straightforwardly check that $\lambda_1$ and $\lambda_2$ have socle containing partitions other than the composition factors in \eqref{Composition Series} by showing that each have a good number 2 as follows:
  \begin{align*}
    \beta_{\lambda_1}(1,2) &= 3 - 1 + r = r+2\\
    \beta_{\lambda_2}(1,2) &= 2 - 1 + r + 1 = r+2.
  \end{align*}
  These are each nonvanishing mod $e$, so they are good and $\lambda = \lambda_3$, giving the theorem.
\end{proof}
\fi

\subsection{Kernels and Further Work}
\begin{lemma}\label{Represented}
  Let $w_j$ be the basis vector with anchors $1,\dots,r$ and all arcs of maximal length.
  Suppose $w \in K \backslash \cbr 0$.
  Then, $w_j$ is represented in $w$.
\end{lemma}
\begin{proposition}
  Suppose $e = n + r + 1$.
  Then $K$ is nontrivial.
\end{proposition}

\section{The Fibonacci Representation and Quotients of Specht Modules}
We can start our study of $V$ by studying low-dimensional cases.
First, note that $V_{\vs\vp}^2$ is the sign representation $D^{(2)}$ and $V_{\vs\vp}^2$ is the trivial representation $D^{(1)^2}$.

$V_{\vp\vp}^2$ is a 2-dimensional representation of a semisimple commutative algebra, and hence decomposes into a direct sum of two subrepresentations.
In particular, we can use the basis $\cbr{(\vp\vs\vp),(\vp\vp\vp)}$ and explicitly write the matrix
\[
  \rho_{T_1} = \begin{bmatrix}
    \varc & \vard\\
    \vard & \vare
  \end{bmatrix}
\]
having characteristic polynomial $(\varc - \lambda)(\vare - \lambda) - \vard^2 = \lambda^2 - (\varc + \vare)\lambda +(\varc\vare - \vard^2)$.
We may verify that, for $\lambda = -1$, this evaluates to
\[
  -((-1 + q + q^2) (1 + q^3 + q^4 + q^5 + 2 q^6 + q^7))\brk{5}_q = 0
\]
and for $\lambda = q$ this evaluates to 
\[
-(q^2 (-1 + q + q^2) (1 + q + q^2 + q^3 + 2 q^4 + q^5)) \brk{5}_q = 0
\]
hence $\rho_{T_1}$ has eigenvalues $-1$ and $q$.

The eigenspaces with eigenvalues $-1$ and $q$ are subrepresentations isomorphic to the sign and trivial representation, hence $V_{\vp\vp}$ is isomorphic to a direct sum of the trivial and sign representations: $V^2_{\vp\vp} \simeq V^2_{\vs\vp} \oplus V^2_{\vs\vs}$. 

Now let's prove that $V^3_{\vs\vs}$ is irreducible;
this has basis $\cbr{\vs\vp\vs\vp}, \cbr{\vs\vp\vp\vp}$, and the following matrices:
\[
  \rho_{T_1} = \begin{bmatrix}
    \varb & 0\\
    0 & \vara
  \end{bmatrix}; \hspace{20pt}
  \rho_{T_2} = \begin{bmatrix}
    \varc & \vard\\
    \vard & \vare
  \end{bmatrix}.
\]
A subrepresentation must be one-dimensional, and hence an eigenspace of each of these matrices;
since $\varb \neq \vara$, the first has eigenspaces given by the spans of basis elements, and since $\vard \neq 0$, these are not eigenspaces of the second.
Hence $V^3_{\vs\vs}$ is irreducible.
Now we may move on to the general case.
\begin{proposition}
  The representation $V_{\vs\vp} := V^m_{\vs\vp}$ is irreducible.
\end{proposition}
\begin{proof}
  We will prove this inductively in $m$.
  We've already proven it for $V^2_{\vs\vp}$ and $V^3_{\vs\vp}$, so suppose that $V^{m-2}_{\vs\vp}$ is irreducible.
  
  Let $\cbr{v_i}$ be the basis for $V_{\vs\vp}$.
  Then, each $v_i$ is cyclic; indeed, we can transform every basis vector into $(\vs\vp \dots \vp)$ by multiplying by the appropriate $\frac{1}{\vard-\varc}(T_i - \varc)$, and we can transform $(\vs\vp \dots \vp)$ into any basis vector by multiplying be the appropriate $\frac{1}{\vard-\vare}(T_i - \vare)$.
  Hence it is sufficient to show that each $v \in V_{\vs\vp}$ generate some basis element.

  Let $v'$ be the basis element $(\vs\vp\vs\vp\dots \vp)$, which is many copies of $\vs\vp$, followed by an extra $\vp$ if $m$ is odd.
  We will show that each $v \in F$ generates $v'$.

  Suppose that no elements beginning $(\vs\vp\vs\vp)$ are represented in $v_i$;
  then, all such elements are represented in $T_3v$, so we may assume that at least one is represented in $v$.

  Note that $\ima (T_2 - \vara) = \Span\cbr{\text{Basis vectors beginning }(\vs\vp\vs\vp)}$ and $(T_2 - \vara)v \neq 0$.
  Further, note that $\Res_{\SH(S_{m-2})}^{\SH(S_m)} \ima (T_2 - \vara) \simeq V^{m-2}_{\vs\vp}$ as representations.
  Hence irreducibility of $V^{m-2}_{\vs\vp}$ implies that $v'$ is generated by $(T_2 - \vara)v$, and $V^m_{\vs\vp}$ is irreducible.
\end{proof}

Knowing this, the restriction statements are clear;
$\Res V_{\vs\vp}^m \simeq V_{\vp\vp}^{m-1}$ by considering the last $m-2$ transpositions, and $\Res V_{\vs\vp}^{m-1} \simeq V_{\vs\vp}^{m-1} \oplus V_{\vs\vs}^{m-1}$ by considering the first $m-2$.
Similarly, $\Res V_{\vs\vs}^m \simeq V_{\vs\vp}^{m-1}$ by considering the first $m-2$ transpositions.
This gives that $V \simeq 3V_{\vs\vp} \oplus 2V_{\vs\vs}$.

Now we may move on and use Young Tableau to characterize $V$.
Recall that the socle of $D^\lambda$ is given by $\bigoplus\limits_{\mu \xrightarrow{\text{good}} \lambda} D^\mu$,  and that $D^\lambda$ is semisimple iff every $\mu \xrightarrow{\text{normal}} \lambda$ is good.
\begin{theorem}
  The irreducible components of $V$ are given by the following isomorphisms:
     \begin{align*}  
      V^{2n}_{\vs\vs} &\simeq D^{(n,n)'}\\ 
      V^{2n-1}_{\vs\vs} &\simeq D^{(n+1,n-2)'}\\
      V^{2n}_{\vs\vp} &\simeq D^{(n+1,n-1)'}\\
      V^{2n-1}_{\vs\vp} &\simeq D^{(n,n-1)'}.
     \end{align*}
\end{theorem}
\begin{proof}
  We will prove this by induction on $n$;
  we have already proven the base case $V^{2}$, so suppose that we have proven these isomorphisms for $V^{2n-2}$.
  We will prove the isomorphisms for $V^{2n-1}$ and $V^{2n}$.

  By irreducibility, $V^{2n-1}_{\vs\vs} \simeq D^{\lambda_{\vs\vs}}$ and $V^{2n-1}_{\vs\vp} \simeq D^{\lambda_{\vs\vp}}$ for some diagrams $\lambda_{\vs\vs}$ and $\lambda_{\vs\vp}$.
  We will show that $\lambda_{\vs\vs} = (n+1,n-2)'$ and $\lambda_{\vs\vp} = (n+1,n-1)'$.
  
  First, note that we have \[\Res \; D^{\lambda_{\vs\vs}} \simeq D^{(n,n-2)'} \simeq \Res \; D^{(n+1,n-2)'}\] and \[\Res \; D^{\lambda_{\vs\vp}} \simeq D^{(n,n-2)} \oplus D^{(n-1,n-1)} \simeq \Res \; D^{(n,n-1)'}.\]
  By semisimplicity of $\Res D^{\lambda_{\vs\vs}}$ and $\Res D^{\lambda_{\vs\vp}}$, every normal cell in $\lambda_{\vs\vs}$ and $\lambda_{\vs\vp}$ is good.
  In particular, the only normal number in $\lambda_{\vs\vs}$ is 1.

  For $\lambda_{\vs\vs}$, the only tableaux which can remove a cell to yield $D^{(n,n-2)'}$ are $(n+1,n-2)'$, $(n,n-1)'$, and $(n,n-2,1)'$ as illustrated in Figure \ref{OddRes};
  we have already seen that $D^{(n,n-1)'}$ does not have irreducible restriction, so we are left with $(n+1,n-2)'$ and $\lambda = (n,n-2,1)'$.
  We may directly check that $\lambda$ doesn't satisfy this, as we have the following:
  \begin{align*} 
    \beta_\lambda(1,2) &= 3 - 2 + (n-2) = n-1\\
    \beta_\lambda(1,3) &= 3 - 1 + n = n+2\\
    \beta_\lambda(2,3) &= 2 - 1 + 3 = 4.
   \end{align*} 
  At least one of $\beta_\lambda(1,2)$ and $\beta_\lambda(1,3)$ is nonzero, since $\beta_\lambda(1,3) - \beta_\lambda(1,2) = 3 \not\equiv 0 \pmod e$, and hence at least one of $M_2$ and $M_3$ is empty.
  Hence at least one of 2 or 3 is normal in $(n,n-2,1)'$, and $\lambda_{\vs\vs} = (n+1,n-2)$.

  For $\lambda_{\vs\vp}$, we immediately see from Figure \ref{OddRes} that the only option is $(n,n-1)$.
  \begin{figure}
    \begin{align*}
      \cbr{
        \begin{gathered}
        \ydiagram{6,3}. \; \;
        \ydiagram{5,4}, \; \;
        \ydiagram{5,3,1}
      \end{gathered} 
    }&\longrightarrow \begin{gathered}
      \ydiagram{5,3}
      \end{gathered}\\
      \cbr{
        \begin{gathered}
          \ydiagram{5,4}
        \end{gathered}
      }&\longrightarrow
      \begin{gathered}
        \ydiagram{5,3},\;\;
        \ydiagram{4,4}.
      \end{gathered}
    \end{align*}
    \caption{
      Illustration of the partitions of $9$ which can, via row removal, yield $(n,n-2)'$ alone, or both $(n,n-2)'$ and $(n-1,n-1)'$.
    }\label{OddRes}
  \end{figure}  
  
  We can perform a similar argument for the $V^{2n}$ case, finding now that 
  \[\Res \; D^{\mu_{\vs\vs}} \simeq D^{(n,n-1)'} \simeq \Res \; D^{(n,n)'}\] and 
  \[\Res \; D^{\mu_{\vs\vp}} \simeq D^{(n,n-1)'} \oplus D^{(n+1,n-2)'} \simeq \Res \; D^{(n+1,n-1)'}.\]
  
  Through a similar process, we see that $\mu_{\vs\vp} = (n+1,n-1)'$.
  We narrow down $\mu_{\vs\vs}$ to one of $(n,n)'$ or $\mu := (n,n-1,1)'$, and note that
  \begin{align*} 
    \beta_\mu(1,2) &= 3 - 2 + (n-1) = n\\
    \beta_\mu(1,3) &= 3 - 1 + n = n+2\\
    \beta_\mu(2,3) &= 2 - 1 + 2 = 3
  \end{align*}
  and hence at least one of 2 or 3 is normal, $\Res D^{(n,n-1,1)'}$ is not irreducible, and $\mu_{\vs\vs} = (n,n)'$, finishing our proof.
\end{proof}
\begin{corollary}
  We have the following isomorphisms of representations:
  \begin{align*} 
    V^{2n} &\simeq 3D^{(n+1,n-1)'} \oplus 2D^{(n,n)'}\\
    V^{2n - 1} &\simeq 3D^{(n,n-1)'} \oplus 2D^{(n+1,n-2)'}
  \end{align*}
\end{corollary}


\section{Conjecture}
Recall that $K_{2n + r}^r := K$ is the direct sum of all copies of the sign representation in $W$.
Hence the following characterises sign subrepresentations of $W$ completely:
\begin{proposition}
  $K \subset W_{2n + r}^r$ is trivial when $e \neq n + r + 1$, and $\dim \, K = 1$ when $e = n + r + 1$.\qed
\end{proposition} 
\begin{proposition}
  Suppose $e < n + r + 1$, and suppose $n'$ is such that $e = n' + r + 1$.
  Note that $h := (1 + T_1)(1 + T_3)\dots(1 + T_{n - n'})$ maps $W_{2n + r}^r$ onto $W_{2n' + r}^r$.
  Then, the preimage $h^{-1}(K_{2n + r}^r)$ is a subrepresentation of $W_{2n + r}^r$, and the series
  \[
    0 \subset h^{-1}(K_{2n + r}^r) \subset W_{2n + r}^r
  \]
  is a composition series of $W_{2n + r}^r$.\qed
\end{proposition}
\begin{proposition}
  Denote the composition factor $W_{2n + r}^r / h^{-1}(K_{2n + r}^r)$ by $U_{2n + r}^r$.
  Then, there exist some naturals $m,s$ satisfying $2m + s = 2n + r$ and $m + s > n + r$ such that the following is an isomorphism of $\SH$-modules \[h^{-1}\prn{K_{2n + r}^r} \simeq U_{2m + s}^s\].
  \qed
\end{proposition}
\begin{proposition}
  For the same $m,s$ as above, we have the following composition series of specht modules:
  \[
    0 \longrightarrow D^{(m+s,m)'} \longrightarrow S^{(n+r,n)'} \longrightarrow D^{(n+r,n)'} \longrightarrow 0.
  \]
\end{proposition}
\begin{proposition}
  $W_{2n + r}^r \simeq S^{(n+r,n)'}$ and $U_{2m +s}^s \simeq D^{(m+s,m)'}$.
\end{proposition}

\section{Empirical Results}


\bibliography{RepBib}
\iffalse
\begin{thebibliography}{9}

\bibitem{Brundan}
  Brundan citation here

\bibitem{Etingof}
  Etingof citation here

\bibitem{Kleschev}
  Kleschev citation here

\bibitem{Mathas Book}
  Mathas book citation here

\bibitem{Mathas}
  Mathas article citation here

\bibitem{Shor}
  Shor citation here

\end{thebibliography}
\fi

\newpage
\appendix
\section{Compatibility of Representations with the Relations}
In general, we define representations above for the free algebra on generators $\cbr{T_i}$.
Recall that we may give a presentation of $\SH$ having generators $T_i$ and relations
\begin{align}
  (T_i - q)(T_i + 1) &= 0 \label{quadratic}\\
  T_iT_{i+1}T_i &= T_{i+1}T_iT_{i+1} \label{braid1}\\ 
  T_iT_j &= T_jT_i \hspace{40pt} \abs{i - j} > 1. \label{braid2}
\end{align}
We call \eqref{quadratic} the \emph{quadratic relation} and \eqref{braid1}, \eqref{braid2} the \emph{braid relations}.
It is easily seen that a representation of $\SH$ is equivalent to a representation of the free algebra $k\langle T_i \rangle$ which acts as 0 on the relations (henceforth referred to as \emph{compatibility} with the relations).
We will prove in the following sections that $V$ and $W$ are compatible with the Hecke algebra relations.

\subsection{The Crossingless Matchings Representaiton}
\label{Cross Relations}
Take some basis vector $w_i$.
We will first check \eqref{quadratic} by case work:
\begin{itemize}
  \item Suppose there is an arc $(i,i+1)$.
    Then, $(T_i-q)(T_i + 1)w = (1 + q)\brk{(1 + T_i)w - (1 + q)w} = 0$, giving \eqref{quadratic}.

 
  \item Suppose there is no arc $(i,i+1)$ and $i,i+1$ do not both have anchors;
    then $(T_i +  1)w = q^{1/2}w''$ for some basis vector $w'$ having arc $(i,i+1)$, and the computation follows as above for \eqref{quadratic}.
  \item Suppose $i,i+1$ are anchrors;
    then $(T_i + 1)w = 0$, giving \eqref{quadratic}.
\end{itemize}
   
\vspace{5pt}
Now we verify \eqref{braid1}.
Let $h := (1 + T_i)(1 + T_{i+1})(1+T_i)$, and let $g := (1 + T_{i+1})(1 + T_i)(1 + T_{i+1})$.
Note the following expansion:  
  \begin{align*}
      hw
      &= 1 + 2T_i + T_i^2 + T_{i+1} + T_iT_{i+1} + T_{i+1}T_i + T_iT_{i+1}T_i\\
      &= 1 + (1+q)T_i + T_{i+1} + T_iT_{i+1} + T_{i+1}T_i + T_iT_{i+1}T_i.
    \end{align*}
    An analogous formula gives an analogous equality in $g$.
    Hence we have
    \[
      (h-g)w = q(T_i - T_{i+1}) + T_iT_{i+1}T_i - T_{i+1}T_iT_{i+1}.
    \]
    Hence we may equivalently check that $(h-g)w = q(T_i - T_{i+1})$.
    This is illustrated in Figure \ref{braid1arc}.
    \begin{figure}
  \[
    \GeneralizedNAction{5}{1/2,3/4}{1}{5/1}{3/4}{2/1, 1/2,2/3}{1/4,2/3}{5/1}{q^{3/2}}
    \hspace{20pt}
    \GeneralizedNAction{5}{1/2,3/4}{1}{5/1}{3/4}{1/1, 2/2,1/3}{1/2,3/4}{5/1}{q(q+1)}
  \]
  \[
    \NAction{6}{1/6,2/5,3/4}{1/1,2/2,1/3}{1/2,3/4,5/6}{q^{3/2}}
    \hspace{20pt}
    \NAction{6}{1/6,2/5,3/4}{2/1,1/2,2/3}{1/6,2/3,4/5}{q^{3/2}}
  \]
  \[
    \GeneralizedZeroAction{4}{3/4}{2}{1/1,2/2}{.5}{1}
    \hspace{20pt}
    \GeneralizedNAction{4}{3/4}{2}{1/1,2/2}{.5}{2/1, 1/2,2/3}{2/3}{1/1,4/2}{q(q+1)}
  \]
  \caption{
    Here we verify in small cases that $hw = qT_i$ and $gw = qT_{i+1}$.
    These 6 cases cover the situations that there is an arc among the indices $i,i+1,i+2$, that there isn't and there are not two arcs, and that there are two arcs.
  }
  \label{braid1arc}
\end{figure}

Lastly, we have the equation
\[
  (1 + T_i)(1 + T_j) - (1 + T_j)(1 + T_i) = T_iT_j - T_jT_i
\]
and hence we simply need to verify that $(1 + T_i)$ and $(1 + T_j)$ commute, which the reader may easily check.

\subsection{The Fibonacci Representation} 
\label{Fib Relations}
Similar to before, the reader may verify that \eqref{braid2} follows easily, and the others may be verified on strings of length 3 and 4.
By considering the coefficients in order of \eqref{Fib Action}, the quadratic relation \eqref{quadratic} gives the following quadratics:
\begin{equation}
  \begin{split}
    (\vara - q)(\vara + 1) &= 0\\
    (\varb - q)(\varb + 1) &= 0\\
    \varc\vard + \vard\vare &= (q-1)\vard\\
    \varc^2 + \vard^2 &= (q -1)\varc + q\\
    \vare^2 + \vard^2 &= (q -1)\vare + q
  \end{split}
\end{equation}
The first two of these are easily verified for any $q$.
Since $\vard \neq 0$, the third is equivalently given by
\[
  (q - 1) = \varc + \vare = t(q\tau - 1 + q - \tau) = (\tau^2 + \tau)(q - 1)
\]
or that $\prn{\tau^2 + \tau - 1}(q-1) = 0$.
One may verify that \[\tau^2 + \tau - 1 = q^6 + 2q^5 + q^4 + q^3 + q^2 - 1 = (-1+q+q^2)\brk{5}_q = 0.\]

The fourth is given by the quadratic
\[
  \tau^2\brk{(q\tau-1)^2 - \tau(q+1)} = \tau(q-1)(q\tau - 1) + q
\]
or equivalently,
\[
  (\tau^2 + \tau - 1)\brk{q\prn{qt^2 + 1} + t} = 0
\]
which is true for every $q$.

The fifth is similarly given by
\[
  (\tau^2 + \tau - 1)\brk{q\prn{qt + 1} + t^2} = 0 
\]
which is true for every $q$.

\vspace{7pt}
We now verify \eqref{braid1}.
We may order the basis for $V^4$ as follows:
\[
  \cbr{(\vp\vp\vp\vp),(\vs\vp\vp\vs),(\vp\vp\vp\vs),(\vs\vp\vp\vp),(\vs\vp\vs\vp),(\vp\vs\vp\vs),(\vp\vp\vs\vp),(\vp\vs\vp\vp)}.
\]
Then, in verifying the braid relation \eqref{braid1} in this order, we encounter the following quadratics (with tautologies and repetitions omitted):
\begin{align*}
    \vara\vare^2 + \varb\vard^2 &= \vara^2 \vare\\
    \vara\vard\vare + \varb\varc\vard &= \vara\varb\vard\\
    \varb\varc^2 + \vara\vard^2 &= \varb^2\varc\\
    \vara\varc^2 + \vard^2\vare &= \vara^2\varc\\
    \vard\vare^2 + \vara\varc\vard &= \vara\vard\vare
\end{align*}
Substituting in $\tau$ and dividing by $\delta$ whenever possible, these are equivalent to the vanishing of the following polynomials in $q$:
\begin{align*}
  -q (1 + q) (1 + q^2 + q^3) (2 + q + 3 q^2 + 2 q^3) \brk{5}_q &= 0\\
  (1 + 2 q + q^3 + q^4) \brk{5}_q &= 0\\
  (1+q)^2 (1+q^2+q^3) (1+3q^3 - q^4 + q^6)\brk{5}_q &= 0\\
  (1+q)^2 (1+q^2+q^3) (1+5q+5q^2+3q^3+3q^4+3q^5+q^6)\brk{5}_q &= 0\\
  (1+q) (1+q^2+q^3) (-1+2q+q^2+q^3+q^4)\brk{5}_q &= 0.
\end{align*}
Notably, each of these vanish when $e = 5$.

\section{Miscellaneous Algebra Facts}
Throughout the text, for some representation $V$, we refer to $\Res_{\SH(S_{l})}^{\SH(S_m)} V$ without specifying exactly which subalgebra $\SH(S_{l})$.
\begin{proposition}
  Suppose $B,B'$ are subalgebras of the $k$-algebra $A$ with $B = uB'u^{-1}$, and let $V$ be a representation of $A$.
  Then, the linear isomorphism $V \xrightarrow{\phi} V$ given by $v \mapsto uv$ causes the following to commute for any $b \in B$:
  \[
    \begin{tikzcd}
      V \arrow[r,"\phi"] \arrow[d,"b"] & V \arrow[d,"ubu^{-1}"]\\
      V \arrow[r,"\phi"] & V
    \end{tikzcd}
  \]
  Hence, through the identification of $B$ and $B'$ via conjugation, we have $\Res_{B}^A V \simeq \Res_{B'}^A V$
\end{proposition}
\begin{proof}
  This is simply given by $(ubu^{-1})uv = ubv$.
\end{proof}

\begin{corollary}
  Suppose $\SH',\SH''$ are two subalgebras of $\SH(S_m)$ generated by $l$ reflections and $V$ is a representation of $\SH$.
  Then, $\Res_{\SH'}^{\SH} V \simeq \Res_{\SH''}^\SH V$.
\end{corollary}
\begin{proof}
  Let $\SH'$ and $\SH''$ be the subalgebras of $\SH(S_m)$ generated by the reflections $\cbr{T_{i_1},\dots,T_{i_l}}$ and $\cbr{T_{i_1},\dots,T_{i_{j-1}},T_{i_j + 1},T_{i_{j+1}},\dots,T_{i_l}}$ for $1 \leq i_1 < \dots < i_{j-1} < i_j + 1 < i_{j+1} < \dots < i_l \leq n$.
  It is sufficient to prove that $\SH'$ and $\SH''$ are conjugate;
  then transitivity gives conjugacy of any $S_l \subset S_m$, and the previous proposition gives isomorphisms of the representations.
  
  In fact, the reader can verify that $\SH'' = T_{i_j}\SH'T_{i_j}^{-1}$.
\end{proof}


\end{document}

\documentclass{amsart}   
\usepackage{RepStyle} 
\usepackage{NatMacros} %********REPLACE WITH MilesMacros TO KILL RELIANCE ON \ifnum AND HAVE INCOMPLETE OUTPUT***********bleh
\begin{document}

\title{Rough Cut of Proven Work on $\SH$.}
\author{Miles Johnson \& Natalie Stewart}
\maketitle

\section{Introduction}
Let $S_{2n+r}$ be the symmetric group on $2n+r$ indices with $2n + r \geq 2$, let $\SH = \SH_{k,q}(S_{2n+r})$ be the corresponding Hecke algebra over field $k$ with parameter $q \in k^\times$ having square root $q^{1/2}$, and let $\cbr{T_i}$ be the reflections generating $\SH$.
Let $\brk{m}_q = 1 + q + \dots + q^{m-1}$ be the $q$-number of $m$. 
Let $e$ be the smallest positive integer such that $\brk{e}_q = 0$, and set $e = \infty$ if no such integer exists.
Either $q = 1$ and $e$ is the characteristic of $k$, or $q \neq 1$ and $q$ is a primitive $e$th root of unity.

Let $S^{(n+r,n)'}$ be the Specht module corresponding to the young diagram with two columns with height difference $r$, and let $D^{(n+r,n)'}$ be the corresponding irreducible quotient.
The purpose of this writing is to characterize these representation via an isomorphism with two graphical representations of $\SH$.

\subsection{Crossingless Matchings}
\begin{definition}
  A \emph{crossingless matching on $2n+r$ indices with $r$ anchors} is a partition of $\cbr{1,\dots,2n+r}$ into $n$ parts of size $2$ and $r$ of size 1 such that no two parts of size two``cross'', i.e. there are no parts $(a,a')$ and $(b,b')$ such that $a < b < a' < b'$, and no parts of size one are ``inside'' of a part of size two, i.e. there are no $c, (a,a')$ such that $a < c < a'$.
  We will call these arcs and anchors, respectively.
  Then, define $W^r_{2n+r}$ to be the $k$-vector space with basis the set of generalized crossingless matchings on $2n+r$ indices with $r$ anchors.

  In order for this to be a $\SH$-module, endow this with the action given by Figure \ref{Action}; 
  if a ``loop'' is created, scale by $q+1$, if a loop is not created and the action involves fewer than 2 anchors, deform into a new crossingless maching and scale by $q^{1/2}$, and if it involves two anchors, scale by 0.
  We verify that this is well-defined in appendix \ref{Cross Relations}.
\end{definition}

\begin{figure}
  \[
    \begin{tabular}{l l}
      \GeneralizedAction{6}{1/4,2/3}{2}{5/1,6/2}{1}{2}{1/4, 2/3}{5/1,6/2}{(1+q)}
      \hspace{20pt}
      &
      \GeneralizedAction{6}{1/4,2/3}{2}{5/1,6/2}{1}{3}{1/2, 3/4}{5/1,6/2}{q^{1/2}}\\
      \GeneralizedAction{6}{1/4,2/3}{2}{5/1,6/2}{1}{4}{2/3, 4/5}{1/1,6/2}{q^{1/2}}
      &
      \GeneralizedZeroAction{6}{1/4,2/3}{2}{5/1,6/2}{1}{5}
    \end{tabular}
  \]
    
  \caption{Illustration of the actions $(1 + T_i)w_{\abs{W^2_6}}$.
    In general, we act by deleting loops, deforming into a new crossingless matching, and scaling by either $q^{1/2}$, $(q + 1)$, or 0.
  }
  \label{Action}
\end{figure}

Let the length of an arc $(i,j)$ be $l(i,j) := j - i + 1$.
Note that the crossingless matchings on $2n$ indices with no anchros can all be identified with a list of $n$ integers describing the lengths of the arcs from left to right;
using this, we may order the crossingless matchings with no anchors in increasing lexicographical order in order to obtain an order on the subbasis containing a particular set of anchors;
let the basis be ordered first by the position of the anchors in increasing lexicographical order, then increasing for the matchings between each anchor.
Let this basis be $\cbr{w_i}$.
This basis is illustrated for $W_{5}^1$ in Figure \ref{S5 Basis}. 

\begin{figure}
  \def\cbasisspacing{5mm}
  $\cbr{
    \begin{gathered}
      \GeneralizedMatching{5}{2/3, 4/5}{1}{1/1}{3/4}, \hspace{\cbasisspacing}
      \GeneralizedMatching{5}{2/5, 3/4}{1}{1/1}{3/4}, \hspace{\cbasisspacing}
      \GeneralizedMatching{5}{1/2, 4/5}{1}{3/1}{3/4}, \hspace{\cbasisspacing}
      \GeneralizedMatching{5}{1/2, 3/4}{1}{5/1}{3/4}, \hspace{\cbasisspacing}
      \GeneralizedMatching{5}{1/4, 2/3}{1}{5/1}{3/4}, \hspace{\cbasisspacing}
     \end{gathered}}$ 
    \caption{The basis for $W_5^1$.}
  \label{S5 Basis}
\end{figure} 

We will prove that $W := W_{2n+r}^r$ and $S := S^{(n+r,n)'}$ are isomorphic as representations in the case that $e > n + r + 1$.
Note that, when $r = 0$, these have the same dimension given by the $n$th catalan number $C_n$.

\subsection{Fibonacci Representation}
Now suppose that $k = \CC$ and $q = \exp\prn{2 \pi i \ell/5}$ is a primitive 5th root of unity.
Let $V^m$ be a $k$-vector space with basis given by the strings $\cbr{*,p}^{n+1}$ such that the character $*$ never appears twice in a row. 
We will surpress the superscript whenever it is clear from context.

We wish to endow this with a $\SH$-action which acts on a basis vector only dependent on characters $i,i+1,i+2$, sending each basis vector to a combination of the other basis vectors having the same characters $1,\dots,i,i+2,\dots,n+1$ as follows:
\def\vara{\alpha}
\def\varb{\beta}
\def\varc{\gamma}
\def\vard{\delta}
\def\vare{\varepsilon}

\begin{equation}
  \begin{split}
  T_1 \, (*pp) &:= \vara(*pp)\\
  T_1 \, (pp*) &:= \vara(pp*)\\
  T_1 \, (*p*) &:= \varb(*p*)\\
  T_1 \, (p*p) &:= \varc(p*p) + \vard(ppp)\\
  T_1 \, (ppp) &:= \vard(p*p) + \vare(ppp)
\end{split} \label{Fib Action} 
\end{equation}
for constants
\begin{equation}
  \begin{split}
  \vara &= -1\\
  \varb &= q\\
  \varc &= \tau(q\tau - 1)\\
  \vard &= \tau^{3/2}(q + 1)\\
  \vare &= \tau(q-\tau)\\
  \tau &= \begin{cases}
    \frac{1}{2}\prn{\sqrt 5 - 1} & \ell \equiv 1,4 \pmod 5\\
    \frac{1}{2}\prn{\sqrt 5 + 1} & \ell \equiv 2,3 \pmod 5
  \end{cases}
\end{split} \label{Fib Constants} 
\end{equation}
with $T_i$ acting similarly on the substring $i,i+1,i+2$.
We will verify that this is a representation of $\SH$ in Appendix \ref{Fib Relations}

This contains 4 subrepresentations based on the first and last character of the string, which are not modified by $\SH$.
Label the subrepresentation of strings $(*\dots*)$ by $V_{**}$, and similar for the other 3.
It is easy to see that $V_{*p} \simeq V_{p*}$, so that
\[
  V \simeq 2V_{*p} \oplus V_{**} \oplus V_{pp}.
\]
We will show that $V_{pp} \simeq V_{*p} \oplus V_{**}$, and give the following isomorphisms with irreducible quotients of specht modules depending on the parity of the number of indices in $\SH$:
\begin{equation}
  \begin{split}    
    F^{2n}_{**} &\simeq D^{(n,n)'}\\ 
    F^{2n-1}_{**} &\simeq D^{(n+1,n-2)'}\\
    F^{2n}_{*p} &\simeq D^{(n+1,n-1)'}\\
    F^{2n-1}_{*p} &\simeq D^{(n,n-1)'}.
  \end{split} \label{Fib Isos}
\end{equation}
 
\section{Crossingless Matchings and Specht Modules}
\section{The Fibonacci Representation and Specht Modules}
\section{Explicit Relationships}


\newpage
\appendix
 \section{Compatibility of Representations with the Relations}
In general, we define representations above for the free algebra on generators $\cbr{T_i}$.
Recall that we may give a presentation of $\SH$ having generators $T_i$ and relations
\begin{align}
  (T_i - q)(T_i + 1) &= 0 \label{quadratic}\\
  T_iT_{i+1}T_i &= T_{i+1}T_iT_{i+1} \label{braid1}\\ 
  T_iT_j &= T_jT_i \hspace{40pt} \abs{i - j} > 1. \label{braid2}
\end{align}
We call \eqref{quadratic} the \emph{quadratic relation} and \eqref{braid1}, \eqref{braid2} the \emph{braid relations}.
It is easily seen that a representation of $\SH$ is equivalent to a representation of the free algebra $k\langle T_i \rangle$ which acts as 0 on the relations (henceforth referred to as \emph{compatibility} with the relations).
We will prove in the following sections that $V$ and $W$ are compatible with the Hecke algebra relations.

\subsection{The Crossingless Matchings Representaiton}
\label{Cross Relations}
Take some basis vector $w_i$.
We will first check \eqref{quadratic} by case work:
\begin{itemize}
  \item Suppose there is an arc $(i,i+1)$.
    Then, $(T_i-q)(T_i + 1)w = (1 + q)\brk{(1 + T_i)w - (1 + q)w} = 0$, giving \eqref{quadratic}.

 
  \item Suppose there is no arc $(i,i+1)$ and $i,i+1$ do not both have anchors;
    then $(T_i +  1)w = q^{1/2}w''$ for some basis vector $w'$ having arc $(i,i+1)$, and the computation follows as above for \eqref{quadratic}.
  \item Suppose $i,i+1$ are anchrors;
    then $(T_i + 1)w = 0$, giving \eqref{quadratic}.
\end{itemize}
   
\vspace{5pt}
Now we verify \eqref{braid1}.
Let $h := (1 + T_i)(1 + T_{i+1})(1+T_i)$, and let $g := (1 + T_{i+1})(1 + T_i)(1 + T_{i+1})$.
Note the following expansion:  
  \begin{align*}
      hw
      &= 1 + 2T_i + T_i^2 + T_{i+1} + T_iT_{i+1} + T_{i+1}T_i + T_iT_{i+1}T_i\\
      &= 1 + (1+q)T_i + T_{i+1} + T_iT_{i+1} + T_{i+1}T_i + T_iT_{i+1}T_i.
    \end{align*}
    An analogous formula gives an analogous equality in $g$.
    Hence we have
    \[
      (h-g)w = q(T_i - T_{i+1}) + T_iT_{i+1}T_i - T_{i+1}T_iT_{i+1}.
    \]
    Hence we may equivalently check that $(h-g)w = q(T_i - T_{i+1})$.
    This is illustrated in Figure \ref{braid1arc}.
    \begin{figure}[b]
  \[
    \GeneralizedNAction{5}{1/2,3/4}{1}{5/1}{3/4}{2/1, 1/2,2/3}{1/4,2/3}{5/1}{q^{3/2}}
    \hspace{20pt}
    \GeneralizedNAction{5}{1/2,3/4}{1}{5/1}{3/4}{1/1, 2/2,1/3}{1/2,3/4}{5/1}{q(q+1)}
  \]
  \[
    \NAction{6}{1/6,2/5,3/4}{1/1,2/2,1/3}{1/2,3/4,5/6}{q^{3/2}}
    \hspace{20pt}
    \NAction{6}{1/6,2/5,3/4}{2/1,1/2,2/3}{1/6,2/3,4/5}{q^{3/2}}
  \]
  \[
    \GeneralizedZeroAction{4}{3/4}{2}{1/1,2/2}{.5}{1}
    \hspace{20pt}
    \GeneralizedNAction{4}{3/4}{2}{1/1,2/2}{.5}{2/1, 1/2,2/3}{2/3}{1/1,4/2}{q(q+1)}
  \]
  \caption{
    Here we verify in small cases that $hw = qT_i$ and $gw = qT_{i+1}$.
    These 6 cases cover the situations that there is an arc among the indices $i,i+1,i+2$, that there isn't and there are not two arcs, and that there are two arcs.
  }
  \label{braid1arc}
\end{figure}

Lastly, we have the equation
\[
  (1 + T_i)(1 + T_j) - (1 + T_j)(1 + T_i) = T_iT_j - T_jT_i
\]
and hence we simply need to verify that $(1 + T_i)$ and $(1 + T_j)$ commute, which the reader may easily check.

\subsection{The Fibonacci Representation} 
\label{Fib Relations}
Similar to before, the reader may verify that \eqref{braid2} follows easily, and the others may be verified on strings of length 3 and 4.
By considering the coefficients in order of \eqref{Fib Action}, the quadratic relation \eqref{quadratic} gives the following quadratics:
\begin{equation}
  \begin{split}
    (\vara - q)(\vara + 1) &= 0\\
    (\varb - q)(\varb + 1) &= 0\\
    \varc\vard + \vard\vare &= (q-1)\vard\\
    \varc^2 + \vard^2 &= (q -1)\varc + q\\
    \vare^2 + \vard^2 &= (q -1)\vare + q
  \end{split}
\end{equation}
The first two of these are easily verified for any $q$.
Since $\vard \neq 0$, the third is equivalently given by
\[
  (q - 1) = \varc + \vare = t(q\tau - 1 + q - \tau) = (\tau^2 + \tau)(q - 1)
\]
or that $\prn{\tau^2 + \tau - 1}(q-1) = 0$.
The reader may verify that $\tau^2 + \tau - 1 = 0$, so this is true for every $q$.

The fourth is given by the quadratic
\[
  \tau^2\brk{(q\tau-1)^2 - \tau(q+1)} = \tau(q-1)(q\tau - 1) + q
\]
or equivalently,
\[
  (\tau^2 + \tau - 1)\brk{q\prn{qt^2 + 1} + t} = 0
\]
which is true for every $q$.

The fifth is similarly given by
\[
  (\tau^2 + \tau - 1)\brk{q\prn{qt + 1} + t^2} = 0 
\]
which is true for every $q$.

\vspace{7pt}
We now verify \eqref{braid1}.
We may order the basis for $V^4$ as follows:
\[
  \cbr{(pppp),(*pp*),(ppp*),(*ppp),(*p*p),(p*p*),(pp*p),(p*pp)}.
\]
Then, in verifying the braid relation \eqref{braid1} in this order, we encounter the following quadratics (with tautologies and repetitions omitted):
\begin{align*}
    \vara\vare^2 + \varb\vard^2 &= \vara^2 \vare\\
    \vara\vard\vare + \varb\varc\vard &= \vara\varb\vard\\
    \varb\varc^2 + \vara\vard^2 &= \varb^2\varc\\
    \vara\varc^2 + \vard^2\vare &= \vara^2\varc\\
    \vard\vare^2 + \vara\varc\vard &= \vara\vard\vare
\end{align*}
The reader may verify that each of these are satisfied for $q$ a primitive 5th root of unity and $\tau$ as defined.

This highlights the difficulty with deforming our module to $q = 1$ at any field;
the quadratic relations require that $(\tau^2 + \tau - 1)(\tau^2 + \tau + 1) = 0$, but neither of these appear in the first braid relation, which reads $\tau (\tau^3 - 6 \tau^2 + 1) = 0$.
If we have $\tau^2 + \tau \pm 1 = 0$, then $\tau \neq 0$ and $-7\tau^2 \pm \tau + 1 = 0$.
Hence $(7 \pm 1)\tau + (1 \pm 1) = 0$, implying that $\tau = \frac{1}{4},0$, neither of which satisfy $\tau^2 + \tau \pm 1 = 0$, a contradiction.

To attempt to deform this to $q = 1$ would require that we rewrite $\varc,\vard,\vare$ entirely, rather than simply modifying $\tau$. 

\end{document}

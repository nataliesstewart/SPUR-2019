  \documentclass{amsart}   
  \usepackage{../RepStyle} 
  \usepackage{../NatMacros}
  % Colors links
  \hypersetup{
  colorlinks,
  citecolor=blue,
  linkcolor=orange!50!red,
  urlcolor=blue}
  \def\thmcolor{black!60!orange}
  

  % Emulates \subsection* but doesn't add to ToC (I don't know why amsart is weird 
  \newcommand{\fakesubsection}[1]{
    \vspace{7pt}
    \noindent \textbf{#1.}
  }

\makeatletter
%Table of Contents
\setcounter{tocdepth}{3}

% Add bold to \section titles in ToC and remove . after numbers
\renewcommand{\tocsection}[3]{%
  \indentlabel{\@ifnotempty{#2}{\bfseries\ignorespaces#1 #2\quad}}\bfseries#3}
% Remove . after numbers in \subsection
\renewcommand{\tocsubsection}[3]{%
  \indentlabel{\@ifnotempty{#2}{\ignorespaces#1 #2\quad}}#3}
%\let\tocsubsubsection\tocsubsection% Update for \subsubsection
%...

\newcommand\@dotsep{4.5}
\def\@tocline#1#2#3#4#5#6#7{\relax
  \ifnum #1>\c@tocdepth % then omit
  \else
    \par \addpenalty\@secpenalty\addvspace{#2}%
    \begingroup \hyphenpenalty\@M
    \@ifempty{#4}{%
      \@tempdima\csname r@tocindent\number#1\endcsname\relax
    }{%
      \@tempdima#4\relax
    }%
    \parindent\z@ \leftskip#3\relax \advance\leftskip\@tempdima\relax
    \rightskip\@pnumwidth plus1em \parfillskip-\@pnumwidth
    #5\leavevmode\hskip-\@tempdima{#6}\nobreak
    \leaders\hbox{$\m@th\mkern \@dotsep mu\hbox{.}\mkern \@dotsep mu$}\hfill
    \nobreak
    \hbox to\@pnumwidth{\@tocpagenum{\ifnum#1=1\bfseries\fi#7}}\par% <-- \bfseries for \section page
    \nobreak
    \endgroup
  \fi}
\AtBeginDocument{%
\expandafter\renewcommand\csname r@tocindent0\endcsname{0pt}
}
\def\l@subsection{\@tocline{2}{0pt}{2.5pc}{5pc}{}}
\makeatother

  \begin{document}

  \title[Some Graphical Realizations of Two-Row Specht Modules of Hecke Algebras]{Some Graphical Realizations of Two-Row Specht Modules of Iwahori--Hecke Algebras of the Symmetric Group}
  \author[Miles Johnson and Natalie Stewart]{Miles Johnson and Natalie Stewart\\
    Mentor : Oron Propp\\
  Project Suggested by : Roman Bezrukavnikov\\ \; \\
  July 31, 2019
  }

   \begin{titlepage}
    \maketitle
    \begin{abstract}
      We consider the Iwahori--Hecke algebra of the symmetric group on $2n + r$ letters with parameter $q$.
      Let $e$ be the smallest positive integer such that the $q$-number $\brk{e}_q = 0$, or set $e = \infty$ if none exist.
      We modify Khovanov's crossingless matchings to include $2n$ ``nodes'' and $r$ ``anchors,'' and prove in the case $e > n + r + 1$ that the associated module is isomorphic to the Specht module $S^{(n+r,n)}$ which corresponds to the partition $(n + r,n) \vdash 2n + r$.
      We then give heuristics in support of the general case, including explicit composition series for $e = n + r + 1$ and for $2n + r \leq 7$. 
      Lastly, when $e = 5$, we prove an isomorphism between the irreducible quotient $D^{(n+r,n)}$ with $r \leq 3$ and some subrepresentations of Jordan--Shor's Fibonacci representation.
      We provide explicit transition matrices between this representation and the crossingless matchings representation for $2n + r \leq 6$.
    \end{abstract}

  \begingroup
  \hypersetup{linkcolor=black} % no colored hyperlinks in toc
  \tableofcontents
  \endgroup

  \end{titlepage}

\section{Introduction} 
  Let $S_{2n+r}$ be the symmetric group on $2n+r$ letters with $2n + r \geq 2$.
  Let $\SH := \SH_{k,q}(S_{2n+r})$ be the corresponding Iwahori-Hecke algebra (henceforth simply Hecke algebra) over a field $k$ with parameter $q \in k^\times$ having a fixed square root $q^{1/2}$.
  Let $\cbr{T_1,\dots,T_{2n + r - 1}}$ be the simple reflections generating $\SH$.
   
  Let $\brk{m}_q = 1 + q + \dots + q^{m-1}$ be the $q$-number of the positive integer $m$. 
  Let $e$ be the smallest positive integer such that $\brk{e}_q = 0$, and set $e = \infty$ if no such integer exists.
  Note that either $q = 1$ and $e$ is the characteristic of $k$ (with $0$ replaced by $\infty$), or $q \neq 1$ and $q$ is a primitive $e$th root of unity.

  \vspace{5pt}
  When $q = 1$, the Hecke algebra $\SH$ is isomorphic to the group algebra $k\brk{S_{2n+r}}$;
  hence the representation theory of $\SH$ generalizes the representation theory of the symmetric group.
  The Hecke algebra is also well-known to be connected to the representation theory of the general linear group over a finite field \cite{Mathas-book}.
  It is a classical result that $\SH$ is semisimple precisely when $e > 2n + r$, in which case the irreducible representations of $\SH$ are given by \emph{Specht modules} $S^\lambda$, which are indexed by the partitions $\lambda$ of $2n + r$ \cite{Mathas-book}.
  
  For all $e$, $\SH$ admits a cellular basis with cell modules given by $S^\lambda$.
  In particular, these admit quotients $D^\lambda$ such that the modules $\cbr{D^\lambda \mid D^\lambda \neq 0, \lambda \vdash n}$ are a pairwise-nonisomorphic list of all irreducible $\SH$-modules.
  This set is indexed by the partitions $\lambda \vdash n$ which are $e$-regular \cite{Murphy1,Murphy2}.
  These representations $D^\lambda$ have explicit constructions, but many of their properties are unknown.
  For instance, the dimension of $D^\lambda$ is unknown outside of some special cases \cite{Mathas-book}.
  However, there does exist an algorithm due to Lascoux--Leclerc--Thibon--Ariki which computes the decomposition matrices of the Specht modules of the Hecke algebra $\SH_{\CC,q}(S_{2n + r})$ for $q$ an $e$th root of unity \cite{LLT,Ariki}.
  
  The cellular basis for $S^\lambda$ and associated basis for $D^\lambda$ are complicated and often computationally intractable.
  We aim to give simple graphical realizations of $S^\lambda$ and $D^\lambda$ in some cases that $\lambda = (n+r,n)$ is a partition of two parts (henceforth called \emph{two-row partitions};
  These realizations behave in an intuitive and computationally simple way.

  \begin{remark}
    We follow the convention of Murphy--Kleshchev concerning the correspondence $\lambda \leftrightarrow S^\lambda$, which is dual to the conventions of Dipper--James--Mathas;
    one may translate our results to the latter convention by transposing all partitions \cite{Kleshchev,Mathas-book,Murphy1}. 
    For instance, we refer to the \emph{sign representation} as $S^{(1^{2n + r})}$.
  \end{remark}

  We will use the \emph{crossingless matchings representation}, first defined by Khovanov \cite{Khovanov, Khovanov-tangle, Khovanov-abelian}, to realize the Specht modules of two-row partitions.
  These crossingless matchings representations have found applications in both knot theory not geometric representation theory.  

  In the seminal paper \cite{Khovanov-tangle}, Khovanov constructs a categorification of the crossingless matchings representation as modules over a certain \emph{arc algebra} given by applying a certain 2-dimensional topological quantum field theory (specified via the cohomology of the 2-sphere) to all possible concatenations of crossingless matchings in the plane.
  Khovanov then uses this to categorify the Jones polynomial of a tangle by associating to any tangle the chain homotopy class of a certain complex of bimodules over the arc algebra.
  A generalization of the arc algebra was further studied by Brundan--Stroppel in \cite{Brundan-1,Brundan-2,Brundan-3,Brundan-4}, yielding similar results to the original case.
  For a survey of abelian categorification, including the Khovanov's categorification of the crossingless matchings representation, see \cite{Khovanov-abelian}.
  
  Further, in \cite{Khovanov} Khovanov proves that the center of the arc algebra is isomorphic as a graded ring to the cohomology of the Springer fiber of complete flags in $\CC^{2n}$ stabilized by a fixed nilpotent operator with two Jordan blocks of size $n$ (corresponding to the two rows of the partition $(n,n)$).
  Moreover, this isomorphism is equivariant with respect to the natural actions of the symmetric group $S_{2n}$ on each ring.
  In \cite{Stroppel}, Stroppel extended this to an isomorphism between the entire arc algebra and the endomorphism ring of a full projective-injective module in the principal block of the parabolic category $\mathcal{O}$ corresponding to the same two-part partition $(n,n) \vdash 2n$.
  Stroppel also showed in \cite{Stroppel} that the crossingles matchings on $2n$ points and no anchors naturally label label indecomposable self-dual projective modules in the principal block of the parabolic category $\mathcal{0}$ corresponding to $(n,n)$.

  Additionally, in \cite{Anno-t,Anno-tangles}, Anno--Nandakumar proved that the irreducible objects of the heart of the exotic $t$-structure on the derived category of coherent sheaves on a two-block springer fiber are naturally inexed by crossingless matchings on an Annulus.
  The authors conjecture an ``annular'' analog of Khovanov's arc algebra as a description of the $\mathrm{Ext}$-algebra between these irreducible objects.  

  \begin{remark}
    Khovanov considers only the partition $(n,n) \vdash 2n$ in his categorification above.
    We will consider the general two-row partition $(n+r,n)$ in this paper.
    A possible extension of our and Khovanov's work is to categorify the $r > 0$ case of our crossingless matchings representations as in \cite{Khovanov-tangle} and study whether Khovanov's work generalizes to the cohomology of the Springer fiber of complete flags in $\CC^{2n + r}$ stabilized by an analogous operator with Jordan blocks of size $n+r$ and $n$.
  \end{remark}

  \begin{definition}
    Define a \emph{crossingless matching on $2n + r$ nodes and $r$ anchors} to be an isotopy class of $n + r$ non-intersecting paths in the slice $\RR \times \brk{0,1}$ connecting $2n + r$ distinct points of $\RR \times \cbr{0}$ and $r$ points of $\RR \times \cbr{1}$ such that none of the latter points are connected. 
  Let $M_{2n + r}^r$ be the $k$-vector space with a basis given by these matchings.
  This is illustrated in Figure \ref{S5 Basis}.
 
  Order the points on $\RR \times \cbr{0}$ via the order $<$ on $\RR$, and refer to these as \emph{nodes}.
  Refer to a path connecting the $a$th and $b$th node as an \emph{arc} $(a,b)$, and refer to a path connecting node $a$ to a point in $\RR \times \cbr{1}$ as an \emph{anchor}.
  Let the length of an arc $(i,j)$ be $j - i + 1$.


  We endow $M_{2n + r}^r$ with an $\SH$-action by specifying $(1 + T_i)w_j$ for any basis element $w_j$ of $M_{2n + r}$, as illustrated in Figure \ref{Action}.
  We do so by concatenating in ``vertical lines'' below each point other than the $i$th and $(i+1)$st, concatenating paths between the $i$th and $(i+1)$st points as well as points below them, removing any ``loops'' this forms, and taking the isotopy class of the resultant diagram;
  if this is not the isotopy class of a crossingless matching, then there are anchors at $i,i+1$ and we set $(1 + T_i)w_j := 0$;
  if this is the isotopy class of a crossingless matching $w_l$ and there is a ``loop,'' set $(1 + T_i)w_j := (1 + q)w_j$ and otherwise set $(1 + T_i)w_j := q^{1/2}w_l$.
  
  \end{definition}
  \begin{figure} 
    \def\cbasisspacing{5mm}
    \begin{adjustbox}{width=\textwidth}
    $\cbr{
      \begin{gathered}
        \GeneralizedMatching{5}{1/2, 3/4}{1}{5/1}{3/4}, \hspace{\cbasisspacing}
        \GeneralizedMatching{5}{1/4, 2/3}{1}{5/1}{3/4}, \hspace{\cbasisspacing}
        \GeneralizedMatching{5}{1/2, 4/5}{1}{3/1}{3/4}, \hspace{\cbasisspacing}
        \GeneralizedMatching{5}{2/3, 4/5}{1}{1/1}{3/4}, \hspace{\cbasisspacing}
        \GeneralizedMatching{5}{2/5, 3/4}{1}{1/1}{3/4}, \hspace{\cbasisspacing}
       \end{gathered}}$ 
     \end{adjustbox}
       \caption{The basis for $M_5^1$.}
    \label{S5 Basis}
   \end{figure} 

  \begin{figure}
    \begin{adjustbox}{width=\textwidth}
      \begin{tabular}{l l}
        \GeneralizedAction{6}{1/4,2/3}{2}{5/1,6/2}{1}{2}{1/4, 2/3}{5/1,6/2}{(1+q)}
        \hspace{20pt}
        &
        \GeneralizedAction{6}{1/4,2/3}{2}{5/1,6/2}{1}{3}{1/2, 3/4}{5/1,6/2}{q^{1/2}}\\
        \GeneralizedAction{6}{1/4,2/3}{2}{5/1,6/2}{1}{4}{2/3, 4/5}{1/1,6/2}{q^{1/2}}
        &
        \GeneralizedZeroAction{6}{1/4,2/3}{2}{5/1,6/2}{1}{5}
      \end{tabular}
  \end{adjustbox}

    \caption{Illustration of the actions $(1 + T_i)w_{\abs{M^2_6}}$.
      In general, we act by deleting loops, taking an isotopy onto a new crossingless matching, and scaling by either $q^{1/2}$, $(q + 1)$, or 0.
    }
    \label{Action}
   \end{figure}
  
  A more explicit definition for $M_{2n + r}^r$ is given in Appendix \ref{Explicit Definition} and we verify that this is well-defined in Appendix \ref{Cross Relations}.
  In Section \ref{Crossingless Matchings Section}, we will prove the following theorem on irreducibility of $M_{2n + r}^r$.
  \begingroup
  \hypersetup{linkcolor=\thmcolor} % no orange theorem label
  \begin{customthm}{\ref{Irreducibility Theorem}}
    Suppose $e > n$ and $S^{(n+r,n)}$ is irreducible.
    Then $M_{2n + r}^r$ is irreducible.
  \end{customthm}
  \endgroup
  Note that the representations $M_{0 + r}^r$ and $S^{(r)}$ are both isomorphic to the sign representation.
  This and Theorem \ref{Irreducibility Theorem} are suggestive;
  in fact, we will prove the following.
  \begingroup
  \hypersetup{linkcolor=\thmcolor}
  \begin{customthm}{\ref{Correspondence Theorem}}
    If $r = 0$, suppose $e > 2$.
    Suppose $e > n + r + 1$.
    Then, $M_{2n + r}^r \cong S^{(n+r,n)}$.
  \end{customthm}
  \endgroup
  Each of these are powerful characterizations of the module $M_{2n + r}^r$ in the case that it and $S^{(n+r,n)}$ are irrducible.
  We will prove the following theorem which characterizes $M_{2n + r}^r$ in the reducible case:
  \begingroup
  \hypersetup{linkcolor=\thmcolor}
  \begin{customthm}{\ref{kernel existence}}
    Suppose $e = n + r + 1$.
    Then, $M_{2n + r}^r$ contains a subrepresentation isomorphic to the sign representation. 
  \end{customthm}
  \endgroup
  We will go on to prove Corollary \ref{n+r+1 composition series}, which specifies a composition series for $M_{2n + r}^r$ for the case $e = n + r + 1$.

  This proves a graphical characterization of $S^\lambda \cong D^\lambda$ in many cases.
  However, when $S^\lambda$ is reducible, the crossingless matchings representations cannot provide a graphical realization of the irreducible quotients $D^\lambda$;
  our next goal is to provide a similar graphical realization of the modules $D^{(n+r,n)}$ with $r \leq 3$ when $e=5$, using a modification of Shor--Jordan's Fibonacci representation of the braid group \cite{Shor}.
  It is possible that similar graphical representations can be constructed for other $r$ and $e$, but we do not attempt to do so here.
  
  \def\vara{\alpha_1}
  \def\varb{\alpha_2}
  \def\varc{\varepsilon_1}
  \def\vard{\delta}
  \def\vare{\varepsilon_2}
  \def\vs{\texttt{*}}
  \def\vp{\texttt{0}}
  Henceforth assume that $e = 5$ and $k$ contains the algebraic number $\sqrt{q + q^4}$ (for reasons which will be apparent soon).
  For convenience set $m := 2n + r$.
  
  In \cite{Shor}, Shor--Jordan originally defined a rescaled version of the following representation, called the \emph{Fibonacci representation}, and showed that the trace of a subrepresentation of this is the Jones polynomial of the trace closure of a braid on $m$ strands evaluated at a fifth root of unity;
  they used this to show that evaluating a certain approximation of this polynomial is a complete problem for the ``cone clean quibit'' complexity class of quantum computers.
  The Shor--Jordan fibonacci representation is itself a rescaling of a more general representation of the braid group, named the \emph{path model representation}, defined in \cite{Aharonov}.
  
  \begin{remark}
    The path model representation of \cite{Aharonov} is defined for $\ell$th any primitive root of unity $\text{exp}\prn{2 \pi i/\ell} \in \CC$.
    This gives a potential extension of our work to include realizations of $D^{(n+r,n)}$ at more general $e$ and $r$.
  \end{remark}

  \begin{definition}\label{Fib definition}
  Let $V^m$ be the $k$-vector space with basis given by the strings $\cbr{\vs,\vp}^{m+1}$ such that the character $\vs$ never appears twice consecutively. 
  We will refer to $V^m$ as the \emph{Fibonacci representation}.

  We endow $V^m$ with an $\SH$-action which acts on a basis vector in a manner depending only on characters $i,i+1,i+2$, sending each basis vector to a combination of other basis vectors agreeing on characters $1,\dots,i,i+2,\dots,n+1$ as follows:

  \begin{equation} 
    \begin{split}
      T_1 \, \prn{\vs\vp\vp} &:= \vara\prn{\vs\vp\vp},\\
      T_1 \, \prn{\vp\vp\vs} &:= \vara\prn{\vp\vp\vs},\\
      T_1 \, \prn{\vs\vp\vs} &:= \varb\prn{\vs\vp\vs},\\
      T_1 \, \prn{\vp\vs\vp} &:= \varc\prn{\vp\vs\vp} + \vard\prn{\vp\vp\vp},\\
      T_1 \, \prn{\vp\vp\vp} &:= \vard\prn{\vp\vs\vp} + \vare\prn{\vp\vp\vp}.
  \end{split} \label{Fib Action} 
  \end{equation}
  for constants
  \begin{equation}
    \begin{split}
    \tau  &:= q + q^4,\\
    \vara &:= -1,\\
    \varb &:= q,\\
    \varc &:= \tau(q\tau - 1),\\
    \vard &:= \tau^{3/2}(q + 1),\\
    \vare &:= \tau(q-\tau).
  \end{split} \label{Fib Constants} 
  \end{equation}
  with $T_i$ acting analogously on a basis element dependent on the substring $i,i+1,i+2$.
  We verify that $V^m$ is a representation of $\SH$ in Appendix \ref{Fib Relations}
  \end{definition}

  Note that the action $T_i$ does not modify characters $i,i+2$, so characters $1$ and $n+1$ are preserved by $\SH$.
  Hence the representation $V^m$ contains four subrepresentations spanned by strings beginning and ending with specified characters.
  Label the subrepresentation spanned by strings $(\vs\dots\vs)$ by $V_{\vs\vs}$, and similar for $V_{\vs\vp}$, $V_{\vp\vs}$, and $V_{\vp\vp}$.
  We prove the following theorem.
  \begingroup
  \hypersetup{linkcolor=\thmcolor}
  \begin{customthm}{\ref{Fibonacci Theorem}}
    We have the following isomorphisms:
    \begin{align*}  
      \emph{$V^{2n}_{\vs\vs}$} &\cong D^{(n,n)},\\ 
      \emph{$V^{2n-1}_{\vs\vs}$} &\cong D^{(n+1,n-2)},\\
      \emph{$V^{2n}_{\vs\vp}$} &\cong D^{(n+1,n-1)},\\
      \emph{$V^{2n-1}_{\vs\vp}$} &\cong D^{(n,n-1)}.
    \end{align*}
  \end{customthm}
  \endgroup
  This provides a graphical characterization of $D^{(n+r,n)}$ for $e = 5$, $r \leq 3$, as well as a combinatorial characterization of the Fibonacci representation in \cite{Shor}.

  \fakesubsection{Overview of paper}
  
  In Section \ref{Specht Modules Section} we give corollaries to standard theorems concerning Specht modules.
  First, James-Mathas provide a sharp characterization of the irreducibility $S^\lambda$ for $\lambda \vdash 2n + r$ an $e$-regular partition, called the \emph{Carter criterion} \cite[Thm.~5.42]{Mathas-book}.
  We specialize this to the case that $\lambda = (n+r,n)$ to give a combinatorial condition for irreducibility of $S^{(n+r,n)}$.
  We note that this irreducibility depends only on $e$ when $e > n$;
  otherwise it depends on both $e$ and the characteristic of $k$.
  Further, we use Kleshchev--Brundan's modular branching rules to prove our first significant statement: if $S^\lambda \cong D^\lambda$ and $e > n$, then a particular length-2 composition series uniquely determines $\lambda$;
  further, an irreducible restriction to $D^{(n,n-1)}$ determines $\lambda$ as well \cite{Kleshchev, Brundan}.

  \vspace{5pt}
  In Section \ref{Crossingless Matchings Section}, we begin by proving Proposition \ref{trivial kernel} concerning sign subrepresentations of $M_{2n + r}^r$ when $e \nmid n + r + 1$;
  this allows us to prove Theorem \ref{Irreducibility Theorem}.
  Following this, we prove the existence of a particular filtration with factors isomorphic to other crossingless matchings representations;
  using irreducibility, this becomes a composition series.
  This combined with an inductive argument and the branching of Section \ref{Specht Modules Section} allow us to prove Theorem \ref{Correspondence Theorem}.
  
  \vspace{5pt}
  In Section \ref{Sign section}, we begin by determining an explicit basis for the direct sum $K_{2n + r}^r$ of all sign subrepresentations of $M_{2n + r}^r$ in the case $e = n + r + 1$.
  We prove in Theorem \ref{kernel existence} that such $K_{2n + r}^r$ is nontrivial when $e = n + r + 1$, and thereby provide an explicit composition series for such $M$ in Corollary \ref{n+r+1 composition series}.
  We finish the section by providing several corollaries concerning the structure of $M_{2n + r}^r$ at irreducible cases with $e < n + r + 1$.

  \vspace{5pt}
  In Section \ref{Fibonacci Section}, we begin by establishing the $2n = 2$ case of Theorem \ref{Fibonacci Theorem}, as well as irreducibility of $V^3_{\vs\vp}$.
  We then use these cases to prove that $V^m_{\vs\vp}$ and $V^m_{\vs\vs}$ are irreducible for all $m$.
  From this, we inductively prove Theorem \ref{Fibonacci Theorem}.

  \vspace{5pt}
  In Appendix \ref{Compatibility Section}, we begin by giving a precise definition of $M_{2n + r}^r$.
  Then, we verify that the crossingless matchings and Fibonacci representations are compatible with the Hecke algebra relations.
  In Appendix \ref{Algebra}, we prove a lemma concerning restrictions to various subalgebras of the Hecke algebra.
  In Appendix \ref{Empirics Section}, we give explicit data both supporting the conjectures laid out in Section \ref{Conjecture Section} and giving explicit transitions between $M_{2n + r}^r$ and $V^{2n + r}$ and composition series of $M_{2n + r}^r$ in the case that $2n + r \leq 7$.

  % Acknowledgements should be given section-level heading in toc, subsection-level in text.
  \addcontentsline{toc}{section}{\textbf{Acknowledgements}}
  \fakesubsection{Acknowledgements}
  The authors thank Prof. Roman Bezrukavnikov for suggesting this project, as well as Dr. Slava Gerovitch for organizing the SPUR+ program.
  We would also like to thank Profs. David Jerison and Ankur Moitra for their role in SPUR+ as well as their general advice.
  We would also like to thank Professor Alexander Kleshchev for helpful conversations concerning branching theorems.
  Lastly, we would like to express our gratitude to our mentor Oron Propp for his help and advice in both acquiring background knowledge and in executing the mathematics in this paper, as well as his comments on early drafts of this paper;
  this project would not be possible without him.
   
\section{Preliminaries on Specht modules}\label{Specht Modules Section}
  For this section and the rest of the paper, assume $n > 0$ unless stated otherwise.
  Throughout the rest of the paper, it will be useful to have precise notation for partitions;
  identify each partition $\lambda \vdash 2n + r$ with a tuple $\lambda := (\lambda_1^{a_1},\dots,\lambda_l^{a_l})$ having $\lambda_i > \lambda_{i+1}$, $a_i > 0$, and $\sum_i a_i\lambda_i = 2n + r$.
  Identify each of these with a subset $\brk \lambda \subset \NN^2$ as in \cite{Kleshchev}, and define $\lambda(i) = (\lambda_1^{a_1},\dots,\lambda_{i-1}^{a_{i-1}},\lambda_i^{a_i - 1},\lambda_i-1,\lambda_{i+1}^{a_{i+1}},\dots,\lambda_l^{a_l})$ to be the partition with the $i$th row removed.
  Say that $\lambda$ is \emph{$e$-regular} if $\lambda_i - \lambda_{i+1} < e$ for all $i$ and $\lambda_l < e$.

  In the following subsection, we cite a theorem of James--Mathas which precisely characterizes the irreducibility of $S^\lambda$ in the case that $\lambda$ is $e$-regular, and we specialize this result to the case of two-row Specht modules.
  This falls into two cases: either $e > n$, where $S^{(n+r,n)}$ is irreducible iff $e \nmid r+2,\dots,n+r+1$, or $e \leq n$ where the irreducibility of $S^{(n+r,n)}$ is more complicated and depends also on the characteristic of $k$.
  We will focus primarily on the former case.

  Following this, we reproduce the branching theorems of Kleshchev--Brundan, which allow us to fully characterize the socle of $\Res D^{\lambda}$.
  This and some combinatorial arguments yield the main result of this section, which allows us to determine certain $D^{(n+r,n)}$ via their composition series.
  This will be instrumental later for characterizing the crossingless matchings representation $M_{2n + r}^r$ as a Specht module, and it will extend to all cases with $e > n + r + 1$.

  \subsection{Irreducibility of Specht modules}
  Let $\ell$ be the characteristic of $k$;
  then, set
  \[
    p := \begin{cases}
      \ell & \text{if }\ell > 0,\\
      \infty & \text{if }\ell = 0.
    \end{cases}
  \]
  Note that $p = e$ when $q = 1$.
  For $h$ a natural number, let $\nu_p(h)$ be the $p$-adic valuation of $h$.
  By convention, set $\nu_\infty(h) = 0$ for all $h$.
  Define the function $\nu_{e,p} : \NN \rightarrow \cbr{-1} \cup \NN$ by
  \[
    \nu_{e,p}(h) := \begin{cases}
      \nu_p(h) & \text{if }e \mid h,\\
      -1 & \text{if }e \nmid h.
    \end{cases}
  \]
  Lastly, let $h_{ab}^\lambda$ be the hook length of node $(a,b)$ in $\brk \lambda$ as defined in \cite{Kleshchev}.
  With this language, we may express the following theorem, parts (ii)-(iii) of which are known as the \emph{Carter criterion} in the symmetric group case, due to James--Mathas \cite{Mathas-book}.
  \begin{theorem}[James--Mathas]\label{Mathas Irreducibility}
    The following are equivalent:
    \begin{enumerate}[label={(\roman*)}]
      \item $S^{\lambda} \cong D^{\lambda}$.
      \item $\lambda$ is $e$-regular and $S^\lambda$ is irreducible.
      \item $\nu_{e,p}\prn{h_{ab}^\lambda} = \nu_{e,p}\prn{h_{ac}^\lambda}$ for all nodes $(a,b)$ and $(a,c)$ in $\brk \lambda$.
    \end{enumerate}
  \end{theorem}
  \begin{proof}
    See \cite[Thm~5.42]{Mathas-book}.
  \end{proof}
  This result gives information solely on $e$-regular partitions, and the general irreducibility of $S^\lambda$ away from $p=2$ is not well understood.
  We will henceforth specialize slightly to the case that $(n+r,n)$ is $e$-regular. 

  \begin{corollary}\label{S irreducibility}
    If $r = 0$, assume $e > 2$.
    \begin{enumerate}[label={(\roman*)}]
      \item Suppose $e > n$.
        Then, $S^{(n+r,n)}$ is irreducible iff $e \nmid r + 2,r+3,\dots,n + r + 1$.
      \item Suppose $e \leq n$.
        If $S^{(n+r,n)}$ is irreducible, then $e \mid r + 1$.
    \end{enumerate}  
  \end{corollary}
  Note that the condition $e \nmid r+2,r+3,\dots,n+r+1$ implies that $e > n$.
  \begin{proof}
    Our initial assumption on $e$ implies that $\lambda$ is $e$-regular, which enables us to use Theorem \ref{Mathas Irreducibility} below.

    \textbf{(i)}
    Note that $\nu_p(h) \neq -1$ for all natural numbers $h$ and only hook lengths in the top row may vanish mod $e$ by Figure \ref{Hooks};
    hence we may equivalently prove that $e$ divides no hook lengths in the leftmost $n$ columns of the top row by Theorem \ref{Mathas Irreducibility}.
    These hook lengths are precisely $r + 2,\dots,n+r+1$.

    \textbf{(ii)}
    Suppose that $e \nmid r + 1$.
    Then, \[\nu_{e,p}\prn{h^\lambda_{1,n-e+1}} = \nu_{e,p}\prn{h^\lambda_{2,n-e+1} + r + 1} = -1,\]
    giving $S^{(n+r,n)}$ reducible by Theorem \ref{Mathas Irreducibility}.
  \end{proof}

  \begin{figure}
  \ytableausetup{mathmode,boxsize=4.5em}
  \begin{ytableau}
    n+r+1 & n + r & \dots & r + 2 & r & r-1 & \dots & 1\\ 
    n & n-1 & \dots & 1
  \end{ytableau}
  \ytableausetup{mathmode,boxsize=normal}
  \caption{The young diagram corresponding to the partition $(n+r,n)$. The hook lengths are in the center of the corresponding cells.}\label{Hooks}
\end{figure}

  From part (i) we see that irreducibility at $e > n$ is not dependent on $p$, and we may cover many modular cases without reference to the characteristic of $k$.
  We will finish our discussion of irreducibility of $S^\lambda$ via sharp characterization of the $e \leq n$ case for large $p$.
  \begin{corollary}
    If $r = 0$, assume $e > 2$.
    Suppose $e \leq n$, and suppose $p > n + r + 1$.
    Then, $S^{(n+r,n)}$ is irreducible if and only if $e \mid r+1$.
  \end{corollary}
  \begin{proof}
    This follows from the proof of Corollary \ref{S irreducibility} part (ii) and the fact that $\nu_p(h) = \nu_p(h')$ for all natural numbers $h,h'$. 
  \end{proof}

  \subsection{Branching theorems for Specht modules}
  In this section as well as later sections, we will consider the restriction of representations of $\SH$ to particular subalgebras isomorphic to $\SH_{k,q}\prn{S_{2n + r - 1}}$.
  We verify in Appendix \ref{Algebra} that any two subalgebras of $\SH$ generated by $2n + r - 2$ simple transpositions are canonically isomorphic, and the corresponding restrictions are canonically isomorphic via this isomorphism of algebras.
  We will hence abuse notation, pick one such subalgebra $\SH'$, and notate $\Res_{\SH'}^\SH W$ by $\Res W$ for any $\SH$-module $W$.
  
  Fixing some partition $\lambda \vdash 2n + r$, for $1 \leq i \leq j \leq l$, let $\beta_\lambda(i,j)$ and $\gamma_\lambda$ be the quantities
  \begin{align*}
    \beta_\lambda(i,j) &= \lambda_i - \lambda_j + \sum_{t = i}^j a_t,\\
    \gamma_\lambda(i,j) &= \lambda_i - \lambda_j + \sum_{t = i+1}^j a_t.
  \end{align*}
  Note that $\beta_\lambda(i,j)$ is the hook length of cell $(a_1 + \dots + a_{i-1}+1,\lambda_j)$.

  Results due to Kleshchev and Brundan refer to \emph{normal} and \emph{good} numbers;
  for these, we will use the facts that 1 is always normal and that $j$ is normal when $\beta_\lambda(i,j) \not\equiv 0 \pmod e$ for all $i \leq j$.
  Further, we will use the definition that $j$ is good if and only if $j$ is normal and $\gamma_\lambda(j,j') \not\equiv 0 \pmod e$ for all $j' \geq j$ normal \cite{Brundan,Kleshchev}.
  When $\lambda(i) = \mu$ for $i$ normal, write $\mu \xrightarrow{\text{normal}} \lambda$, and similar in the good case.

  The following statements, collectively known as \emph{modular branching rules} of $D^\lambda$, were originally written by Kleshchev for Specht modules of the group algebra $k\brk{S_n}$, then generalized to the Hecke algebra case by Brundan \cite{Kleshchev,Brundan}. 
  They combinatorially characterize the socle and semisimplicity of $\Res D^\lambda$.
  \begin{theorem}[Kleshchev-Brundan]
    We have the following isomorphisms of vector spaces
    \begin{align*}
      \Homarg{\SH'}{S^\mu,\Res D^\lambda} 
      &\cong \begin{cases}
        k & \text{if }\mu \xrightarrow{\text{normal}} \lambda,\\
        0 & \text{otherwise}.
      \end{cases}\\
      \Homarg{\SH'}{D^\mu,\Res D^\lambda} 
      &\cong \begin{cases}
        k & \text{if } \mu \xrightarrow{\text{good}} \lambda,\\
        0 & \text{otherwise}.
      \end{cases}
    \end{align*}
    and $\Res D^\lambda$ is semisimple if and only if every normal number in $\lambda$ is good.\qed
  \end{theorem}

  Using this, we immediately see that, for any rectangular partition $(m^\ell)$, we have
  \[
    \Res D^{(m^\ell)} \cong D^{(m^{\ell-1},m-1)}.
  \]
  The non-rectangular two-row case is more complicated, but we may still describe it fully as follows.
  \begin{corollary}\label{D Restrictions}
    Suppose $r > 0$.
    Then, we may characterize the socle of $\Res \, D^\lambda$ as follows:
    \[
      \soc\prn{\Res \, D^{(n+r,n)}} \cong \begin{cases}
        D^{(n + r - 1,n)} & \text{if }e \mid r+2\\
        D^{(n + r,n-1)} &e \text{if }e\nmid r+2, \; e \mid r\\
        D^{(n+r-1,n)} \oplus D^{(n+r,n-1)} & \text{if } e \nmid r+2,r
      \end{cases}
    \]
    Further, when $e \nmid r$ or $e \mid r + 2$, $\Res D^{(n+r,n)}$ is semisimple.
  \end{corollary}
  \begin{proof}
    This amounts to computations of the hook lengths $\beta(1,2)$ and $\gamma(1,2)$:
    \begin{align*}
      \beta_\lambda(1,2) &= r + 2\\
      \gamma_\lambda(1,2) &= r
    \end{align*}
    Since $2$ is the largest removable number, $D^{(n+r,n-1)} \subset D^{(n+r,n)}$ if and only if $e \nmid r + 2$.
    Further, if $e \nmid r + 2$, then $D^{(n+r-1,n)} \subset D^{(n+r,n)}$ if and only if 1 is good;
    this is equivalent to $e \nmid r$.
  \end{proof}

  Now that we've characterized how $D^\lambda$ restrict, we can describe how strongly these restrictions characterize irreducibles.
  Namely, we will prove that some $D^\lambda$ having the same composition series as  $D^{(n+r,n)}$ is sufficient to determine that $\lambda = (n+r,n)$ in a slightly more restricted case than (i) of Corollary \ref{S irreducibility}.
  \newpage
  \begin{proposition}\label{Combinatorics}
    Let $\lambda$ be an $e$-regular partition of $2n + r$.
    \begin{enumerate}[label={(\roman*)}]
      \item Suppose $r > 0$, suppose $e \nmid r+1, r + 2,\dots,n + r + 1$, and suppose either $e \mid r$ or $e \nmid r-2$.
        If $D^\lambda$ has the composition series 
        \begin{equation}\label{D Composition Series}
          0 \subset D^{(n+r-1,n)} \subset \Res \, D^{\lambda}
        \end{equation}
        with factor $\Res \, D^{\lambda} / D^{(n+r-1,n)} \cong D^{(n+r,n)}$, then $\lambda = (n+r,n)$.
      \item Suppose $r = 0$, suppose $e \nmid 4$, and suppose $D^{(n,n-1)} \cong \Res \, D^\lambda$.
        Then $\lambda = (n,n)$. 
    \end{enumerate}
  \end{proposition}
  \begin{proof}
    Note that $e > n$.
    Further, note that the above characterizations are necessary regardless of $e$-regularity;
    in the case that $\mu$ below fails too be $e$-regular, this proposition will prove that $\lambda$ does not satisfy \ref{D Composition Series}, a contradiction.

    \textbf{(i)}
    Let $\varpi := (n+r-1,n,1)$, let $\varsigma := (n+r-1,n+1)$, and let $\mu := (n + r,n)$.
    Since $D^{(n+r-1,n)} \subset \Res \, D^\lambda$, we have $(n + r - 1,n) \longrightarrow \lambda$, implying $\lambda \in \cbr{\varpi,\varsigma,\mu}$.
    We will show that $\varpi, \varsigma$ do not have socle compatible with \eqref{D Composition Series}, allowing us to conclude $\lambda = \mu$.
    
    If $\varpi$ or $\varsigma$ are not $e$-regular, then $D^\varpi \cong 0$ or $D^\varsigma \cong 0$, and we may immediately rule these out;
    henceforth assume that these are each $e$-regular.
    First suppose that that $\lambda = \varpi$.
    We will break into cases with $r$.
    \begin{enumerate}[label={\textit{Case \arabic*.}}]
      \item Suppose that $r > 1$.
        Note that $e \nmid r + 1 = \beta_\varpi(1,2)$, so $2$ is normal.
        Further, \[\gamma_\varpi(2,3) = n \not\equiv 0 \pmod e,\] so 2 is good and $D^{(n+r-1,n-1,1)} \subset D^\varpi$, which is not a composition factor in \eqref{D Composition Series}.
        Hence, by the Jordan-H\"older theorem \cite[Thm.~3.7.1]{Etingof}, we have $\lambda \neq \varpi$.

      \item Suppose that $r = 1$.
        Then, $\varpi = (n,n,1)$ has \[\gamma_{\varpi}(1,2) = n \not\equiv 0 \pmod e,\] giving $D^{(n,n-1,1)} \subset D^\varpi$ and hence $\lambda \neq \varpi$ as in the previous case.
    \end{enumerate} 

  \vspace{5pt}
  Now suppose that $\lambda = \varsigma$.
  Note that $\varsigma$ is not a partition when $r < 2$, so we may assume that $r \geq 2$.
  We further break into cases with $r$:
  \begin{enumerate}[label={\textit{Case \arabic*.}}]
    \item Suppose $r > 2$.
      Then, by Corollary \ref{D Restrictions}, we require that $e \nmid r$ and $e \mid r - 2$;
      these are not satisfied, so $\lambda \neq \varsigma$.
    \item Suppose $r = 2$.
      Then, we have the restriction \[\Res \, D^\varsigma \cong D^{(n+1,n)},\] as $\varsigma = (n+1,n+1)$ has two rows of the same length.
      This contradicts \eqref{D Composition Series}. 
  \end{enumerate}
  Hence $\lambda = \mu$, completing the proof.

  \textbf{(ii)}
  Since the socle of $D^\lambda$ is irreducible, we require that 1 is the only normal number and $\lambda(1) = (n,n-1)$.
  This reduces to the cases of $\varsigma := (n + 1, n - 1)$ and $\mu := (n,n)$;
  if $\lambda = \varsigma$, then we have that \[\beta_\varsigma(1,2) = 4 \equiv 0 \pmod e,\] a contradiction.
  Hence $\lambda = \mu$, completing the proof.
\end{proof}
This will be an important technical tool in proving the correspondence between the crossingless matchings representation $M_{2n + r}^r$ and the Specht module $S^{(n+r,n)}$ in the following section. 

\section{Crossingless matchings and Specht modules}\label{Crossingless Matchings Section}
In this section, we analyze the crossingless matchings representation $M := M_{2n + r}^r$ with the goal of proving $M_{2n + r}^r \cong S^{(n+r,n)}$ under certain conditions in $e$.
We begin by proving that $M_{2n + r}^r$ contains no subrepresentations isomorphic to the sign representation (henceforth referred to as \emph{sign subrepresentations}) when $e \nmid n + r + 1$.
Using this, we prove irreducibility of $M_{2n + r}^r$ whenever $e \nmid r+2,r+3,\dots,n+r+1$;
when $e>n$, this is true if and only if $S^{(n+r,n)}$ is irreducible by Corollary \ref{S irreducibility}.

We will use the following base case to the correspondence throughout:
\begin{lemma}\label{Base cases}
  Note that $S^{(1^n)}$ is the sign representation and $S^{(n)}$ the trivial representation.
  We have the following isomorphisms
  \begin{enumerate}[label={(\roman*)}]
    \item $M_2^0 \cong S^{(2)}$,
    \item $M_r^r \cong S^{(1^r)}$.
  \end{enumerate}
  Each of these are 1-dimensional, so they are irreducible.
\end{lemma}
\begin{proof}
  Note that, for any $w \in M_2^0$, we have $(1 + T_1)w = (1 + q)w$, implying $T_1w = qw$ and proving (i).
  Similarly, for any $w \in M_2^2$, we have $(1 + T_1)w = 0$, implying $T_1w = -w$ and proving (ii).
\end{proof}
In particular, Lemma \ref{Base cases} is the base case in an inductive proof that $M_{2n + r}^r \cong S^{(n+r,n)}$ whenever $e > n + r + 1$.
Before arguing this, we begin by analyzing sign subrepresentations of $M_{2n + r}^r$.

\subsection{Sign subrepresentations for $e\nmid n+r+1$}
\label{kernel}

Define $K_{2n+r}^r := \bigcap_{i=1}^{2n+r-1} \ker (1 + T_i) = \ker \bigoplus_{i = 1}^{2n + r - 1} (1 + T_i)$. When the limits are clear, we will simply denote this kernel by $K$. Note that $K$ is the direct sum of all sign subrepresentations of $M_{2n+r}^r$. In this section, we prove $K$ is trivial for $e \nmid n + r + 1$, thus $M_{2n+r}^r$ has no sign subrepresentations.

For compactness, in this section we use $\sim $ to denote ``proportional to." For convenience, define $M_0^0$ and $M_1^1$ to be the zero representation. Note that in this section, as well as later sections characterizing the sign subrepresentation, we do not assume $n>0$.

\begin{definition}
	We will use the following notation extensively while exploring the structure of the kernel. Examples are given in figure \ref{sub-matching example}. Fix some basis element $\psi\in M_{2n+r}^r$. 
	
	
	\begin{itemize}
		\item For $1 \leq a,b \leq 2n + r$ define $\psi(a):=b$ if $a$ and $b$ are matched in $\psi$, and $\psi(a):=a$ if $a$ is an anchor in $\psi$. 
		
		
		\item Let $r'$ be the number of anchors in $\psi$ in the range $a,\ldots,b$. Suppose $1\leq a\leq b\leq 2n+r$ and $\psi(i)\in \{a,\ldots,b\}$ for all $i\in \{a,\ldots,b\}$. Define a \emph{sub-matching} $\psi(a,b)$ of $\psi$ to be the basis element $\sigma\in M_{b-a+1}^{r'}$ specified by $\sigma(i)=\psi(i+a-1)-(a-1)$.
		
		\item For $a,b$ satisfying $1\leq a\leq b\leq 2n+r$, if $\psi(a,b)$ is not a sub-matching, define it to be the ordered set of nodes $(\psi(a),\ldots,\psi(b))$. For any other $a,b$, we define $\psi(a,b)$ to be an element of the zero representation $M_0^0$.
		
		\item Define the \emph{rainbow element} $R\in M_{2n+r}^r$ to be the basis element specified by $R(i)=2n+2r-i+1$ for $i>r$, $R(i)=i$ for $i\leq r$. In other words, the basis element with all anchors to the left followed by a ``rainbow" to the right.
	\end{itemize}
	
	\label{sub-matchings}
\end{definition}

\begin{figure} 
	\def\cbasisspacing{5mm}
	$\cbr{
		\begin{gathered}
		\GeneralizedMatching{8}{3/8, 4/7, 5/6}{2}{1/1,2/2}{5/4}, \hspace{\cbasisspacing}
		\Matching{4}{1/4, 2/3}
		\end{gathered}}$
	\caption{The rainbow element $R\in M_8^2$ is pictured on the left, $\sigma\in M_4^0$ is pictured on the right. The nodes $R(4)=7$, $R(5)=6$, so $R(4,7)$ defines a sub-matching. Specifically, $R(4,7)=\sigma$. Alternatively, $R(3)=8$, so, for instance $R(2,7)$ is not a sub-matching, and refers to the ordered set $(2,8,7,6,5,4)$. $R(2,1)$ is an element of the zero representation since $2>1$.}
	\label{sub-matching example}
\end{figure}

The following proposition lets us begin to characterize the coordinate vector of any element in $K$, and will serve as the starting point for all following characterizations.


\begin{proposition}
	Let $w\in M_{2n+r}^r$. If $w\in K$ is nonzero, then the coordinate of the rainbow element $R$ in $w$ is nonzero.
	
	\label{rainbow nonzero}
\end{proposition}	

\begin{proof}
	Let $W$ be the set of basis elements with nonzero coordinate in $w$. Let $z$ be the maximal number of anchors to the far left in any element $\psi\in W$. Formally, $z$ is the greatest positive integer such that there exists some element $\psi\in W$ satisfying $\psi(1)-1=...=\psi(z)-z=0$. For some positive integer $j$ define $W^j\subset W$ to be the set of all basis elements in $W$ with $j$ anchors to the far left.
	
	Suppose $z<r$. Then for each $\psi\in W^z$, exactly $z$ anchors are positioned to the far left, so we may define $i_\psi$ to be the position of the next leftmost anchor in $\psi$. Fix some $\widehat{\psi}\in W^z$ such that $i_{\widehat{\psi}}\leq i_\psi$ for all $\psi$.
	
	Define the basis element $\psi':=q^{-1/2}(1+T_{i_{\widehat{\psi}}-1})\widehat{\psi}$. First, note that $i_\psi>z+1$, so $i_\psi-1$ is not an anchor and $\psi'\not=0$. This follows from the definition of $z$ and $W^z$, which require that the node $z+1$ is not an anchor. We will prove that $\psi'$ has nonzero coordinate in $(1+T_{i_{\widehat{\psi}}-1})w$, implying $w\not\in K$. 
	
	It suffices to show that $\widehat{\psi}$ is the only basis element in $W$ brought to $\psi'$ by the action of $(1+T_{i_{\widehat{\psi}}-1})$. Formally, we will show that if $\sigma\in W$ and $(1+T_{i_{\widehat{\psi}}-1})\sigma\sim \psi'$, then $\sigma=\widehat{\psi}$. 
	
	Immediately, we may intuit a few properties of $\psi'$:
	\begin{enumerate}[label={(\roman*)}]
		\item $\psi'$ still has $z$ anchors on the left.
		\item Defining $i_{\psi'}$ to be the next leftmost anchor, $i_{\psi'}<i_{\widehat{\psi}}$
	\end{enumerate}
	
	Together, (i) and (ii) imply $\psi'\not\in W$. So, if $(1+T_{i_{\widehat{\psi}}-1})\sigma\sim \psi'$, $q^{-1/2}(1+T_{i_{\widehat{\psi}}-1})\sigma=\psi'$.
	
	Define $\sigma':=q^{-1/2}(1+T_{i_{\widehat{\psi}}-1})\sigma$.If $\sigma\not\in W^z$, $\sigma'$ will have $z$ anchors at the far left only if $\sigma'\in W^{z-1}$ and the $z$th leftmost anchor is at position $i_{\widehat{\psi}}$. But then the position of the $z+1$st anchor is unchanged by the action, and must be at position greater than $i_{\widehat{\psi}}$, so from (ii) $\sigma'\not=\psi'$. If $\sigma\in W^z$, $\sigma'$ will have anchor at $i_{\psi'}$ if and only if $i_\sigma=i_{\widehat{\psi}}$ and $\sigma(i_{\widehat{\psi}}-1)=\widehat{\psi}(i_{\widehat{\psi}}-1)$. Since these are the only three indices altered by action of $(1+T_{i_{\widehat{\psi}}-1})$ on $\sigma$, if $(1+T_{i_{\widehat{\psi}}-1})\sigma\sim \psi'$ this implies $\sigma=\widehat{\psi}$ as desired. So if $z<r$ $w$ is not in the desired kernel.
	
	\vspace{1mm}
	We have proven that there exists an element in $W$ with all anchors to the far left, or equivalently that $W^r$ is nonempty. Now we must prove that $R\in W^r$. The proof is analogous to the previous case. First we define subsets $W^r_j\subset W^r$ that have the top $j$ arcs of the rainbow, and define $y$ to be the most arcs any element has. In the previous proof, we showed that if we move the next left-most anchor further left, the image of $w$ under that transposition will be nonzero. In this proof, we will show that if we expand the next-largest arc, the image of $w$ under that transposition will be nonzero.
	
	\vspace{1mm}
	Suppose $z=r$ but $R\not\in W^r$ (so $R\not\in W$). Let us define a sequence of subsets $W^r_i\subset W^r$ to be the set of elements in $W^r$ with the top $i$ humps of the rainbow. Formally, $W^r_0:=W^r$, $W^r_{i+1}:=\{\upsilon\in W^r_i| \upsilon(r+i+1)=2n+r-i \}$. Since $R\not\in W^r$,  there exists some positive integer $y$ less than $n-1$ such that $W^r_{y+1}=\emptyset$; $y$ is the largest number of contiguous top level rainbow humps that any element in $W^r$ has. 
	
	
	Choose $\widehat{\upsilon}\in W^r_t$ such that $\widehat{\upsilon}(r+t+1)\geq \upsilon(r+t+1)$ for all $\upsilon\in W^r_t$. Define the basis element $\upsilon':=q^{-1/2}(1+T_{\widehat{\upsilon}(r+t+1)})\widehat{\upsilon}$. 
	
	First, note that $\widehat{\upsilon}(r+t+1)<2n+r-t$, so:
	\begin{enumerate}[label={[\roman*]}]
		\item $\upsilon'(r+t+1)>\widehat{\upsilon}(r+t+1)$ 
		\item For $0\leq j\leq t$, $\upsilon'(r+j)=\widehat{\upsilon}(r+j)=2n+r-j+1$.
	\end{enumerate}
	
	These properties are analogous to properties (i) and (ii) used earlier in this proof. As before, these imply that $\upsilon'\not\in W$, so, for $\sigma\in W$, if $(1+T_{\widehat{\upsilon}(r+t+1)})\sigma\sim \upsilon'$, $q^{-1/2}(1+T_{\widehat{\upsilon}(r+t+1)})\sigma=\upsilon'$.
	
	We will prove that $\upsilon'$ has nonzero coordinate in $(1+T_{\widehat{\upsilon}(r+t+1)})w$, implying $w\not\in K$. Again, it is sufficient to show that, for $\sigma\in W$, $q^{-1/2}(1+T_{\widehat{\upsilon}(r+t+1)})\sigma=\upsilon'$ implies $\sigma=\widehat{\upsilon}$.
	
	Define $\sigma':=q^{-1/2}(1+T_{\widehat{\upsilon}(r+t+1)})\sigma$. Suppose $\sigma\not\in W^r_t$. To satisfy [ii], we must have $\sigma\in W^r_{t-1}$ and $\sigma(r+t)=\widehat{\upsilon}(r+t+1)$. But then $\sigma(r+t+1)<\widehat{\upsilon}(r+t+1)$ is unchanged by action of $(1+T_{\widehat{\upsilon}(r+t+1)})$, and $\sigma'$ does not satisfy [i], so $\sigma'\not=\upsilon'$. If $\sigma\in W^r_t$, to satisfy $\sigma'(r+t+1)=\upsilon'(r+t+1)$, we must have $\sigma(r+t+1)=\widehat{\upsilon}(r+t+1)$ and $\sigma(\widehat{\upsilon}(r+t+1)+1)=\widehat{\upsilon}(\widehat{\upsilon}(r+t+1)+1)$. Since these are the only two matchings altered by the transposition, if $q^{-1/2}(1+T_{\widehat{\upsilon}(r+t+1)})\sigma=\upsilon'$ we must have that $\sigma=\widehat{\upsilon}$ as desired. Thus, if $w\in K$ is nonzero, $R$ has nonzero coordinate in $w$.
\end{proof}

Given a rainbow element $R$, define the basis elements $R_{R,i},R_{L,i}$ to be those where you move the middle hump across $i$ lines to the right or left, respectively. Examples are pictured in \ref{shifted rainbow}. Formally, $R_{R,i}:=q^{-i/2}(1+T_{r+n+i})...(1+T_{r+n+1})R$, $R_{L,i}:=q^{-i/2}(1+T_{r+n-i})...(1+T_{r+n-1})R$.

\begin{figure}
	\def\cbasisspacing{5mm}
	$\cbr{
		\begin{gathered}
		\Matching{8}{1/8, 2/7, 3/6, 4/5}, \hspace{\cbasisspacing}
		\Matching{8}{1/8, 2/7, 3/4, 5/6}, \hspace{\cbasisspacing}
		\Matching{8}{1/8, 2/3, 4/7, 5/6}, \hspace{\cbasisspacing}
		\Matching{8}{1/2,3/8,4/7,5/6}
		\end{gathered}}$
	\caption{$R_{L,0},...,R_{L,3}$ pictured from left to right}
	\label{shifted rainbow}
\end{figure}

Define $Q_n:=(q^n+...+1)/q^{n/2}(-1)^n$ for $n\in \{0,1,...\}$. The following proposition says that, for any element in the kernel, if some basis element $y$ has coordinate $c$ in that element, and if $y$ has a rainbow sub-matching, the basis elements where you replace that sub-matching by the shifted rainbow matchings $R_{L,i}$ or $R_{R,i}$ both have coordinate $Q_ic$ in the kernel element.

\begin{proposition}
	Let $w$ be an element in the kernel intersection $\cap (1+T_z)$ in some generalized crossingless matchings representation. Let $y$ be a basis element with coordinate $c$ in $w$. Suppose $\exists a,b$ such that $y(a,b)=R$, the rainbow element. Define the basis elements $\theta_i,\phi_i$ by $\theta_i(1,a-1)=\phi(1,a-1)=y(1,a-1)$, $\theta_i(b+1,2n)=\phi(b+1,2n)=y(b+1,2n)$, $\theta_i(a,b)=R_{R,i}$, $\phi_i(a,b)=R_{L,i}$ (leave $\theta_i$ or $\phi_i$ undefined for any $i$ where $R_{R,i},R_{L,i}$ are undefined, respectively). The coordinates of $\phi_i$ and $\theta_i$ in $w$ are both $Q_ic$.
	
	\label{shifted rainbow coeffs}
\end{proposition}

Proof of this proposition requires a simple algebraic fact that will be used throughout this document, so I state it as a lemma.

\begin{lemma}
	$Q_1Q_n-Q_{n-1}=Q_{n+1}$
	
	\label{Q alg}
\end{lemma}

\textit{Proof of lemma.}

\begin{align*}
Q_1Q_n-Q_{n-1}=&\frac{-(q+1)}{q^{1/2}}\frac{(-1)^n(q^n+...+1)}{q^{n/2}}-\frac{(-1)^{n-1}(q^{n-1}+...+1)}{q^{(n-1)/2}}\\
=&\frac{(-1)^{n+1}(q^{n+1}+2q^n+...+2q+1)}{q^{(n+1)/2}}-\frac{(-1)^{n+1}(q^n+...+q)}{q^{(n+1)/2}}\\
=&\frac{(-1)^{n+1}(q^{n+1}+...+1)}{q^{(n+1)/2}}=Q_{n+1}
\end{align*}

Now let us prove the proposition.

\begin{proof}
	Consider acting on $w$ by an element $(1+T_z)$. The coordinate of $\phi_i$ in $(1+T_z)w$ will be a linear combination of the coordinates of basis elements sent to $\phi_i$ by the element $(1+T_z)$. Specifically, it will be $(1+q)c\iota+(q^1/2)\sum c_\psi$ where $\iota=1$ if $y(z)=z+1$, $\iota=0$ otherwise, and $c_\psi$ are the coordinates of all basis elements $\psi$ where $(1+T_z)\psi\sim y$.
	
	Let $n:=a+b-1$ and $r$ be the number of anchors in $y(a,b)$. Consider the coordinate of $\phi_i$ in $(1+T_{a-1+r+n/2-i})w$. This is the transposition that acts on the "moved middle hump" in $\phi_i(a,b)=R_{L,i}$, as shown in \ref{shifted relations}. I claim the following:
	
	\begin{center}
		\textit{claim}:
		The only basis elements $\psi$ where $(1+T_{a-1+r+n/2-i})\psi\sim \phi_i$ are $\phi_i$ and $\phi_{i-1},\phi_{i+1}$ when they exist\footnote{we defined $R_{L,i}$ as far out as we can move the hump, so for $0\leq i<n+r$, and take the analogous domain for $\phi_i$}.
		
	\end{center}
	
	Note that the action of any $(1+T_z)$ on a basis element $\psi$ creates exactly two lines: an arc of length two connecting $z$ and $z+1$, and either an anchor or an arc of length $\geq 2$ connecting $\psi(z)$ and $\psi(z+1)$.\footnote{Although this fact follows directly from our definition of the representation, it will be used throughout the document, so it is important that the reader understands it.\label{two humps created}} The easiest way to see the claim is to see that the given transposition is surrounded by arcs on both sides, so any basis element sent to the same element can vary from $\phi_i$ by at most one of those arcs and nothing else. 
	
	Let us prove the claim formally: It is easy to see that the action of $(1+T_{a-1+r+n/2-i})$ will bring $\phi_{i-1},\phi_i,\phi_{i+1}$ to $\sim \phi$, as shown in \ref{shifted relations}. Suppose there was another basis element $\psi$ sent to $\phi_i$ by the given transposition. Me note that if $\psi$ contains the arcs or anchors directly to the right and left of the arc $(a-1+r+n/2-i,a-1+r+n/2-i+1)$ in $\phi_i$ (formally, it contains the arc $(a-1+r+n/2-i-1,a-1+r+n/2+i+2)$ or an anchor at $a-1+r+n/2-i-1$ and the arc $(a-1+r+n/2-i+2,a-1+r+n/2+i+1)$ or an anchor at $a-1+r+n/2-i+2$), it must contain the arc $(a-1+r+n/2-i,a-1+r+n/2-i+1)$ to be a crossingless matching. Thus, if $\psi$ contains both of these arcs/anchors, $(1+T_{a-1+r+n/2-i})$ acts as the constant $(1+q)$, so $(1+T_{a-1+r+n/2-i})\psi\sim \phi_i=>\psi\sim \phi$. If $\psi$ does not contain the left arc/anchor and $(1+T_{a-1+r+n/2-i})\psi\sim \phi_i$, the action of $(1+T_{a-1+r+n/2-i})$ must create that arc/anchor, so $\psi(a-1+r+n/2-i-1)=a-1+r+n/2-i$ and $\psi(a-1+r+n/2-i+1)=a-1+r+n/2+i+2$ in the case of an arc or $a-1+r+n/2-i+1$ is an anchor. All other matchings remain unchanged, so this implies $\psi=\phi_{i+1}$. Likewise, if the right arc $((a+b-1)/2-i+2,(a+b-1)/2+i+1)$ does not exist, $\psi=\phi_{i-1}$. For boundary cases, note that for $\phi_0=\theta_0$, the only other basis element sent to this by the middle transposition is $\phi_1=\theta_1$. Also note that at the edge case $\phi_{n+r-1}$ there is not necessarily a left arc, so other elements may be sent to $\phi_{n+r-1}$ by the given transposition, and this case gives no new information. Lastly, note that our argument was completely symmetric and thus applies to the $\theta_i$ case, except that for $\theta_i$ we do not have to deal with anchors. Thus the claim is proved.
	
	Given this claim and lemma \ref{Q alg}, the proposition follows quickly through induction: 
	
	Acting by $(1+T_{a-1+r+n/2})$ on $w$, the new coordinate of $\phi_0=y$ is $(q+1)c+q^{1/2}c_{\phi_1}$ where $c_{\phi_1}$ is the coordinate of $\phi_1$ in $w$. Since $w$ is in the kernel, we have $(q+1)c+q^{1/2}c_{\phi_1}=0=>c_{\phi_1}=Q_1c$. $\phi_1=\theta_1$ so this gives us all our base cases.
	
	Acting by $(1+T_{a-1+r+n/2-i})$ on $w$, the new coordinate of $\phi_i$ is $q^{1/2}c_{\phi_{i+1}}+q^{1/2}c_{\phi_{i-1}}+(q+1)c_{\phi_{i}}=0$. By the inductive hypothesis, $q^{1/2}c_{\phi_{i+1}}+q^{1/2}Q_{i-1}c+(q+1)Q_i=0$ so $c_{\phi_{i+1}}=Q_1Q_i-Q_{i-1}=Q_{i+1}$ by lemma \ref{Q alg}. $\theta_i$ is an identical proof, so the proposition follows. 
	
	
	\begin{figure}
		\[
		\GeneralizedAction{8}{3/4,5/8,6/7}{2}{1/1,2/2}{1}{3}{3/4,5/8,6/7}{1/1,2/2}{(1+q)}
		\]
		\[
		\GeneralizedAction{8}{2/3,5/8,6/7}{2}{1/1,4/2}{1}{3}{3/4,5/8,6/7}{1/1,2/2}{q^{1/2}}
		\]
		\[
		\GeneralizedAction{8}{3/8,4/5,6/7}{2}{1/1,2/2}{1.25}{3}{3/4,5/8,6/7}{1/1,2/2}{q^{1/2}}
		\]
		\caption{The action of $(1+T_{a-1+r+n/2-i})$ on $\phi_i,\phi_{i=1},\phi_{i+1}$ (ordered from top to bottom), shown as the case where $y$ is the rainbow vector in $M_{8}^2$ and $i=2$.}
		\label{shifted relations}
	\end{figure}
\end{proof}	

We are now ready to prove the central proposition of this section.


\begin{proposition}
	Let $W_{2n+r}^r$ be a generalized crossingless matchings representation. Suppose $e$ does not divide $n+r+1$. Then $K=0$.
	
	\label{trivial kernel}
\end{proposition}

\begin{proof}
	Suppose $K\not=0$. Take nonzero $w\in K$. By Proposition \ref{rainbow nonzero}, the coordinate of the rainbow vector $R$ is nonzero; suppose the coordinate is $c$. By proposition \ref{shifted rainbow coeffs}, the coordinates of the basis elements $R_{L,n+r-1}$ and $R_{L,n+r-2}$ are $Q_{n+r-1}c$ and $Q_{n+r-2}c$ respectively.
	
	Consider the coordinate of $R_{L,n+r-1}$ in $(1+T_1)w$. Using the same logic as in the proof of proposition \ref{shifted rainbow coeffs}, we note that if a basis element $\psi$ has no anchor at position $3$ and is not equal to $R_{L,n+r-2}$, $(1+T_1)\psi\not\sim R_{L,n+r-1}$. Thus the desired coordinate is equal to $(1+q)Q_{n+r-1}c+q^{1/2}Q_{n+r-2}c=-q^{1/2}Q_{n+r}c$ by lemma \ref{Q alg}. Since $w\in K$, we must have $-q^{1/2}Q_{n+r}c=0$. Me have that $c$ is nonzero, and we assume $q$ nonzero, and $Q_{n+r}$ is zero iff $q$ is a root of $q^{n+r}+...+1$, implying $e|n+r+1$. Thus we have arrived at contradiction, and $K=0$.
\end{proof}

\subsection{Irreducibility of the crossingless matchings representation}
In this subsection, we will first give a lemma on cyclic vectors.
Then, we will use triviality of $K$ to ``project down'' onto a copy of $M := M_{2n + r}^r$ with strictly fewer nodes;
this will prove inductively that $M$ is irreducible when $e > n + r + $.

We refer to a vector $w \in M$ satisfying $\SH w = M$ as \emph{cyclic}.
It is a classical result that a representation $M$ is irreducible if and only if every nonzero element of $M$ is cyclic \cite{Etingof}.
We will prove irreducibility by showing that every nonzero $w \in M$ is cyclic;
the following lemma is plays a key role in showing this.

\begin{lemma}
  \label{Cyclic}
  Every basis vector in $M_{2n + r}^r$ is cyclic.
\end{lemma}
\begin{proof}
  Take some basis vector $w_j \in M_{2n + r}^r$.
  We will first prove that $w_j$ is cyclic in the case $r = 0$, then move on to the case $r > 0$.

  Suppose $r = 0$ and $w$ has arcs $(1,b)$ and $(a,b-1)$.
  Then, $q^{-1/2}(1 + T_{b-1})w_j$ is a basis vector containing arc $(1,a)$ with $a < b$ as by the left mapping as follows:
  \[
    \Action{6}{1/4, 2/3, 5/6}{3}{1/2, 3/4, 5/6}{q^{1/2}}; \hspace{20pt} 
    \Action{6}{1/4, 2/3, 5/6}{4}{1/6, 2/3, 4/5}{q^{1/2}}. 
  \]
  Similarly, if $w$ has arc $(b+1,c)$, $q^{-1/2}(1 + T_b)w_j$ is a basis vector containing $(1,c)$ with $c > b$ by the right mapping above.
  Using these, we may iteratively ``reduce'' each arc to have length 2 and thus generate the vector with all length-2 arcs.
  Then, we may ``expand'' each arc to generate an arbitrary crossingless matching;
  this generates the basis of $M_{2n + r}^r$, so it generates all of $M_{2n + r}^r$.

  Now, suppose $r > 0$.
  Note that, between anchors at indices $a<a'$ having no anchor at index $b$ with $a < b < a'$, the $M_{a'-a}^0$ case allows us to generate the basis vector with all length-2 arcs between $a,a'$ and identical arcs/anchors outside of this sub-matching.
  At the ends, we apply the $M_a^0$ case or the $M_{2n + r - a}^{0}$ case in the same way for the first $a$ or last $2n + r - a$ indices.

  Applying this between each arc gives us a vector with anchors and length-2 arcs, and we may use the appropriate $(1+T_i)$ to move anchors to any positions.
  Then, we may use the reverse process from above to generate the correct matchings between arcs and generate any other basis vector.
\end{proof}

Now we will prove irreducibility.

\begin{theorem}\label{Irreducibility Theorem}
  Suppose that $e \nmid r+2,r+3,\dots,n+r+1$.
  Then the representation $M_{2n + r}^r$ is irreducible. 
\end{theorem}
\begin{proof}
  We proceed by induction on $n$.
  The base cases $n = 0, r \neq 0$ and $n = 1, r = 0$ follow from Lemma \ref{Base cases}.

  Take an arbitrary vector $w \in M$.
  By Proposition \ref{trivial kernel} there exists some $(1 + T_i) \in \SH$ such that $(1 + T_i)w \neq 0$.
  Note that \[\ima (1 + T_i) = \Span\cbr{w_j \mid w_j \text{ contains the arc }(i,i+1)}.\]
  Hence, as vector spaces, there is an isomorphism $\varphi:\ima(1 + T_i) \rightarrow M_{2(n-1) + r}^r$ ``deleting'' the arc $(i,i+1)$.

  We will show that, for every element $(1 + T_j') \in \SH(S_{2(n-1) + r})$, there is some element $h_j \in \SH(S_{2n + r})$ such that the following commutes:
  \begin{equation}\label{Irreducibility commutative diagram}
    \begin{tikzcd}
      \ima (1 + T_i) \arrow[r,"\varphi" above, "\sim" below] \arrow[d,"h_j"] & M_{2(n-1) + r}^r \arrow[d,"1 + T_j'"]\\
      \ima (1 + T_i) \arrow[r,"\varphi" above, "\sim" below] & M_{2(n-1) + r}^r
    \end{tikzcd}
  \end{equation}
  Indeed, when $i+1 \neq j$ this is given by $h_j = 1 + T_j$, and we have $h_{i+1} = q^{-1}(1 + T_i)(1 + T_{i+1})(1 + T_{i-1})$.
  This is illustrated by Figure \ref{bigloop}, and can alternately be seen as a consequence of the following isotopy.
  \[
    \begin{gathered}\CapNAction{4}{1/1,3/2,2/3}{2}\end{gathered}
    \longsquigrightarrow
    \begin{gathered}\BigNAction\end{gathered}
  \]
  This may be intuitively viewed as a ``large'' version of the action connecting nodes $i-1$ and $i+2$;
  it preserves the arc $(i,i+1)$ and the factor $q$ in $h_{i+1}$ gives the correct scaling.

  Note that, by the hypothesis of the proposition, $e \nmid r+2,\dots,n+r$ and hence $(n+r-1,n-1)$ satisfies the hypotheses of the proposition as well.
  Then, by the inductive hypothesis, there is some element $h' \in \SH(S_{2(n-1) + r})$ sending $\varphi((1 + T_i)w)$ to the image of a basis vector of $M_{2n + r}^r$ via $\varphi$;
  then, by \eqref{Irreducibility commutative diagram} the action $\SH$ generates the endomorphism $\varphi^{-1}h'\varphi$ of $M$, which sends $(1 + T_i)w$ to a basis vector in $M_{2n + r}^r$.
  This implies that $w$ is cyclic, and hence $M_{2n + r}^r$ is irreducible.
  \begin{figure}
  \begin{adjustbox}{width=\textwidth}
    \Action{6}{1/6, 2/5, 3/4}{2}{1/6, 2/3, 4/5 }{q^{1/2}} \hspace{20pt} 
    \IsoAction{8}{1/8, 2/7, 3/4, 5/6}{3}{1/8, 2/5, 3/4, 6/7}{q^{1/2}}
  \end{adjustbox}
  \begin{adjustbox}{width=\textwidth}
    \Action{4}{1/4, 2/3}{2}{1/4,2/3}{(1 + q)} \hspace{20pt}
    \IsoAction{6}{1/6, 2/5, 3/4}{3}{1/6, 2/5, 3/4}{(1 + q)}
\end{adjustbox}
  \caption{The correspondence between the action of $(1 + T_2)$ on $w'_5 \in M^0_6$ and the action of $q^{-1}(1 + T_3)(1 + T_4)(1 + T_2)$ on the corresponding vector in $M^0_8$ having arc $(3,4)$ first, then on $w'_2 \in M^0_4$.
  This demonstrates that the action works with or without creating a loop.
  }
  \label{bigloop}
  \end{figure}
\end{proof}

\subsection{Correspondence with Specht modules}
The following theorem due to Mathas \cite[Thm.~5.5]{Mathas-article} generalizes the classical branching theorem of the symmetric group.
This result is not be necessary for our proof of the correspondence in the case $e > n + r + 1$, but the analogy with $M$ is suggestive.
\begin{theorem}[Characteristic-free classical branching theorem]
  Let $\lambda$ be a partition of $m$ with $\ell$ removable nodes.
  Then, $\Res S^\lambda$ has an $\SH_{k,q}(S_{m-1})$-module filtration
  \[
    0 = S^{0,\lambda} \subset S^{1,\lambda} \subset \cdots \subset S^{\ell,\lambda} = \Res S^\lambda
  \]
  such that $S^{t,\lambda} / S^{t-1,\lambda} \cong S^{\lambda(t)}$ for all $1 \leq t \leq \ell$.\qed
\end{theorem}
In particular, this holds in cases where $S^\lambda$ fails to be irreducible.
If we replace $S^\lambda$ with the appropriate $M_{2n + r}^r$ above, we find the statement of the following proposition.
\begin{proposition}Suppose that $n > 0$.
  
  \begin{enumerate}[label={(\roman*)}]
    \item
    Suppose that $r > 0$.
    Then, a filtration of $\Res M_{2n + r}^r$ is given by
    \begin{equation}
      0 \subset M_{2n + r - 1}^{r-1} \subset \Res M_{2n + r}^r, \label{Filtration}
     \end{equation}
     with $\Res M_{2n + r}^r / M_{2n + r - 1}^{r-1} \cong M_{2n + r - 1}^{r + 1}$. 
    \item
      We have the following isomorphism of representations:
      \begin{equation}
        M_{2n - 1}^1 \cong \Res M_{2n}^0. \label{0 Restriction}
       \end{equation}
  \end{enumerate}
  In the case that $e \nmid r+1,\dots,n + r + 1$, \eqref{Filtration} and \eqref{0 Restriction} are composition series.
\end{proposition}
\begin{proof}
  \textbf{(i)}
  Note that the span of the matchings having an anchor at index $2n + r$ is a subrepresentation of $\Res M_{2n + r}^r$;
  this subrepresentation is easily verified to be isomorphic to $M_{2n + r - 1}^{r-1}$, and we will henceforth refer to it as such. 
  
  Let $U := \Res M_{2n + r}^r / M_{2n + r - 1}^{r - 1}$, and let $\pi: \Res M_{2n + r}^r \twoheadrightarrow U$ be the associated projection to $U$.
  Let $\phi:U \rightarrow M_{2n + r - 1}^{r + 1}$ be the $k$-linear map which regards the arc $(i,2n + r)$ in $U$ as an anchor at $i$ in $M_{2n + r - 1}^{r + 1}$.
  It is not hard to verify that this is a well-defined isomorphism of vector spaces, so we must show that it is $\SH$-linear.

  We begin by showing commutativity of the following diagram:
  \begin{equation}\label{Branching commutative diagram}
    \begin{tikzcd}
      M_{2n + r}^r \arrow[r,"1 + T_i"] \arrow[d,"\phi \circ \pi"] & M_{2n + t}^r \arrow[d,"\phi \circ \pi"]\\
      M_{2(n-1) + r + 1}^{r+1} \arrow[r,"1 + T_i"] &  M_{2(n-1) + r + 1}^{r+1}
    \end{tikzcd}
  \end{equation}
  We illustrate this compatibility in the following graphic.
  \[
    \QuotDiagramOne
    \hspace{50pt}
    \QuotDiagramTwo
  \]
  In both the left and right square, we see that $\phi$ may be treated as an isotopy which ``unfolds'' the last node into an anchor.
  This isotopy may equivalently be pre- or post-composed with the isotopy defining the action of $1 + T_i$, implying commutativity of \eqref{Branching commutative diagram}. 

  Now, using \eqref{Branching commutative diagram}, we have
  \begin{align*}
    \phi(T_jw_i) 
    &= \phi(-w_j + (1+T_j)w)\\
    &= -\phi(w_j) + (1 + T_j)\phi(w_j)\\
    &= T_j\phi(w_j).
  \end{align*}
  Hence $\phi$ is an isomorphism of representations, and the statement is proven.

  \textbf{(ii)}
  This follows from an analogous proof:
  now, $\phi:\Res M_{2n}^0 \rightarrow M_{2(n-1)+1}^1$ is an isomorphism of representations, which is proven to be $\SH$-linear by the same logic.
\end{proof}

We've now assembled the basic pieces necessary to prove our correspondence in the case $e > n + r + 1$.

\begin{theorem}\label{Correspondence Theorem}
  If $r = 0$ then suppose $e > 2$.
  Suppose $e > n + r + 1$.
  Then, $M_{2n + r}^r \cong S^{(n+r,n)}$.
\end{theorem}
\begin{proof}
  The case $n = 0$ is already proven via lemma \ref{Base cases}, so suppose $n > 0$.
  In order to use Proposition \ref{Combinatorics}, suppose for now that either $e \nmid 4$ or $r > 0$.

  We will prove this inductively; the base case $2n + r = 2$ is implied by Lemma \ref{Base cases}, so suppose that $M_{2n + s}^s \cong S^{(m+s,s)}$ whenever $2m + s < 2n + r$ and $m + s \leq n + r$.
  Note that $e > m + s + 1$.

  Suppose $r > 0$.
  By Theorem \ref{Irreducibility Theorem}, we know that $M_{2n + r}^r \cong D^\lambda$ for some $e$-regular partition $\lambda$.
  By the inductive hypothesis and Corollary \ref{S irreducibility}, we have a composition series given by the short exact sequence
  \begin{equation}
    \label{Composition Series} 0 \longrightarrow D^{(n + r - 1,n)} \longrightarrow \Res \, D^\lambda \longrightarrow D^{(n+r,n-1)} \longrightarrow 0
   \end{equation}
  Hence the theorem is given by Proposition \ref{Combinatorics} (i).

  Now suppose $r = 0$ and $e \neq 4$.
  Similarly, by Theorem \ref{Irreducibility Theorem}, we know that $M_{2n + r}^r \cong D^\lambda$ for some $e$-regular partition $\lambda$, and by the inductive hypothesis and Corollary \ref{S irreducibility}, we have the irreducible restriction $\Res D^\lambda \cong D^{(n,n-1)}$.
  Then, the theorem is given by Proposition \ref{Combinatorics} (ii).

  Now, suppose $e = 4$ and $r = 0$;
  then $4 > n + 1$, so $n \leq 2$.
  We've already proven the $n = 1$ case via the trivial representation, so suppose $n = 2$.
  Then, from the proof of Proposition \ref{Combinatorics}, we know that $M_{4}^0 \cong D^\lambda$, where $\lambda \in \cbr{(n,n), (n+1,n-1)}$.
  We have already proven that $M_4^2 \cong D^{(n+1,n-1)}$, and we may verify that $\dim M_4^0 = 2 \neq 3 = \dim M_4^2$, so we have that $\lambda = (n,n)$ and the theorem is proven for $e = 4$.
\end{proof}

  This entirely characterizes $M_{2n + r}^r$ in the case that $e > n + r + 1$.
  In the next section, we will give weaker characterizations of $M_{2n + r}^r$ in the case $e = n + r + 1$, where the representation s $M_{2n + r}^r$ and $S^{(n+r,n)}$ are reducible.

 \section{Sign subrepresentations}\label{Sign section}
 Recall $K:=K^r_{2n+r}$, defined in section \ref{kernel}, is the direct sum of all sign subrepresentations. In the previous section, we were not able to prove $M_{2n+r}^r\cong S^{(n+r,n)}$ for $e=n+r+1$ because we could not assume $K=0$. In particular, our proof of irreducibility failed.
 
 In this section, we prove that there is exactly one sign subrepresentation when $e=n+r+1$, thus $M_{2n+r}^r$ is reducible. We give an explicit basis for $K$, and explore how our approach may generalize to finding subrepresentations other $e$.
 
 
 \subsection{Kernel basis}
 
 
 Here we determine an explicit basis for $K$ when $e=n+r+1$, assuming $K\not=0$. In the next section, we prove $K\not=0$.
 
 First let us formalize a useful property of sub-matchings.
 
 \begin{definition}
 	Given a basis element $\psi\in M_{2n+r}^r$, specify some sub-matching $\psi(a,b)$. Let $\Res_{\SH_{b-a+1}(q)}^{\SH_{2n+r}(q)}M_{2n+r}^r$ be the restriction to the sub-algebra generated by transpositions $T_a,...,T_{b-1}$. Define $Y_\psi\subset\Res_{\SH_{b-a+1}(q)}^{\SH_{2n+r}(q)}M_{2n+r}^r$ to be the subrepresentation generated by the set of basis elements $\{\sigma| \sigma(1,a-1)=\psi(1,a-1),\ \sigma(b+1,2n+r)=\psi(b+1,2n+r)\}$.
 \end{definition}
 
 \begin{lemma}
 	Take a basis element $\psi\in M_{2n+r}^r$. Suppose $\psi$ has some sub-matching $\psi(a,b)$ with $r'$ anchors. Define $Y_\psi$ with respect to this sub-matching.
 	
 	The map $\rho:Y_\psi\rightarrow M_{b-a+1}^{r'}$ defined by $$\rho(\sigma)=\sigma(a,b)$$ is an isomorphism of representations.
 	
 	\label{sub-matching isomorphism}
 \end{lemma}
 
 \begin{proof}
 	
 	The map is clearly bijective. Thus it is sufficient to prove the following: $$\rho(T_{i+a-1}\sigma)=T_i\rho(\sigma)$$
 	
 	As mentioned in the previous section, the action of a transposition $T_i$ can change at most 4 nodes, so we need to show that the transpositions end up changing the same nodes in the same way in  $\rho(T_{i+a-1}\sigma)$ and $T_i\rho(\sigma)$.
 	
 	Suppose $\sigma(i+a-1)=s,\sigma(i+a)=t$. Then $(T_{i+a-1}\sigma)(i+a-1)=i+a$, $(T_{i+a-1}\sigma)(s)=t$, so $\rho(T_{i+a-1}\sigma)(i)=i+1$, $\rho(T_{i+a-1}\sigma)(s-a+1)=t-a+1$. Separately, $\rho(\sigma)(i)=s-a+1$ and $\rho(\sigma)(i+1)=t-a+1$, so $T_i\rho(\sigma)(i)=i+1$ and $T_i\rho(\sigma)(s-a+1)=t-a+1$ as desired. So the map is an isomorphism and the lemma is proved.
 \end{proof}
 
 \vspace{5mm}
 The lemma above motivates a recursive characterization of the kernel. To do this, it will be convenient to define some notation.
 
 \begin{definition}
 	Recall $Q_i:=(q^i+...+q+1)/q^{i/2}(-1)^i$ (lemma \ref{Q alg}). For $a>0$ define $Q(0,b):=1$. For $b>a>0$ define $$\Q_b^a:=\frac{Q_{b-1}...Q_{b-a}}{Q_1...Q_{a-1}}$$
 	
 \end{definition}
 
 \begin{definition}
 	For $\psi \in M_0^0$, define the function $x_\psi(q):=1$.
 	
 	For all other basis elements $\psi\in M_{2n+r}^r$, we define $x_\psi$ recursively:
 	
 	$$x_\psi(q):=x_{\psi(2,a-1)}(q)x_{\psi(a+1,2n+r)}(q)\Q^{\lfloor a/2\rfloor}_{n+r}$$
 	
 	I will refer to $x_\psi$ as the \textbf{coordinate function} of $\psi$.
 	
 	\label{coeff def}
 \end{definition}
 
 
 \vspace{2mm}
 The following proposition states the forward direction of our characterization.
 \begin{proposition}
 	
 	Let $M_{2n+r}^r$ be a crossingless matchings representation, and suppose $Q_1,...Q_{n+r-1},\not=0$. Let $w\in\cap\ker(1+T_i)$. MLOG the rainbow element $R$ has coordinate 1 in $w$ (by proposition \ref{rainbow nonzero}). Then the coordinate of any basis element $\psi\in M_{2n+r}^r$ in $w$ is $x_{\psi}(q)$.
 	
 	\label{kernel characterization}
 	
 \end{proposition}
 
 An illustration of this proposition is shown in figure \ref{coeff example}.
 
 
 \begin{figure}
 	\def\cbasisspacing{5mm}
 	
 	$\cbr{
 		\begin{gathered}
 		\GeneralizedMatching{11}{1/6, 2/3,4/5,7/10,8/9}{1}{11/1}{11/8}, \hspace{\cbasisspacing}
 		\Matching{4}{1/2, 3/4}, 
 		\hspace{\cbasisspacing}
 		\GeneralizedMatching{5}{1/4, 2/3}{1}{5/1}{3/4}, \hspace{\cbasisspacing}
 		\end{gathered}}$ 
 	\caption{Suppose the second and third elements have coordinates $x_2(q_1)$ and $x_3(q_2)$ in their respective kernel elements, where for $q_1$, $e=3$ and for $q_2$ $e=4$. The coordinate of the first element is $x(q)=x_2(q)x_3(q)\frac{Q_5Q_4Q_3}{Q_1Q_2}$, where for $q$, $e=7$}
 	\label{coeff example}
 \end{figure} 
 
 
 \begin{proof}
 	Suppose $\psi(1)=a$. The proof is structured as follows: use proposition \ref{shifted rainbow coeffs} to find the coefficient of the basis element with $\lfloor a/2\rfloor$ humps then a rainbow element; use the same proposition in a reversed manner to find the coefficient of the basis element consisting of the rainbow for the first $a$ nodes, then the rainbow for the final $2n+r-a$ nodes; finally, we finish the proof through induction using lemma \ref{sub-matching isomorphism}.
 	
 	\vspace{2mm}
 	By proposition \ref{shifted rainbow coeffs} the element $R_1:=R_{L,n+r-1}$ has coordinate $Q_{n+r-1}$ in $w$. Then $R_1(3,2n+r)$ is the rainbow element in  $M_{2(n-1)+r}^r$, so the element $R_2$ defined by $R_2(1,2):=R_1(1,2)$, $R_2(3,2n+r):=R_{L,n+r-2}\in M_{2(n-1)+r}^r$ has coordinate $Q_{n-1}Q_{n-2}$. Generally, define $R_i$ by $R_i(1,2(i-1)):=R_{i-1}(1,2(i-1))$, $R_i(2i-1,2n+r):=R_{L,n+r-i}\in M_{2(n-i+1)+r}^r$. Then the coefficient of $R_i$ is $Q_{n+r-1}...Q_{n+r-i}$. These elements are shown in figure 5.
 	
 	\begin{figure}
 		\def\cbasisspacing{5mm}
 		
 		$\cbr{
 			\begin{gathered}
 			\GeneralizedMatching{8}{3/8, 4/7,5/6}{2}{1/1,2/2}{5/4}, \hspace{\cbasisspacing}
 			\GeneralizedMatching{8}{1/2, 5/8,6/7}{2}{3/1,4/2}{5/4}, 
 			\hspace{\cbasisspacing}
 			\GeneralizedMatching{8}{1/2, 3/4,7/8}{2}{5/1,6/2}{5/4}, \hspace{\cbasisspacing}
 			\end{gathered}}$ 
 		\caption{In order, the rainbow element, $R_1$, and $R_2$. The coordinate of the rainbow element is 1. The coordinate of $R_1$ is $Q_{4}$. The coordinate of $R_2$ is $Q_4Q_3$. Generally, $R_i$ is the element with $i$ humps then a rainbow element, and has coordinate $Q_{n+r-1}...Q_{n+r-i}$.}
 	\end{figure}
 	
 	Now define basis elements $E_i$ by $E_i(2i+1,2n+r):=R_i(2i+1,2n+r)$, $E_i(1,2i):=R$, the appropriate rainbow element. By the same argument as above, if $E_i$ has coordinate $c$ in $w$, $R_i$ has coordinate $Q_{i-1}...Q_{1}c$. One way to make this more clear is to consider intermediate basis elements $\sigma_j^{E_i}$ defined by $\sigma_j^{E_i}(2i+1,2n+r):=E_i(2i+1,2n+r)$ and $\sigma_j^{E_i}(1,2i):=R_{L,j}$. Then the coordinates of $\sigma_j^{E_i}(2i+1,2n+r)$ in terms of the coordinate $c$ of $E_i$ are $Q_{i-1}...Q_{i-j}$, and $R_i=\sigma_{i-1}^{E_i}$. 
 	
 	Since we assume $Q_i\not=0$ for $i< n+r$, this implies the coefficient of $E_i$ is $\frac{Q_{n+r-1}...Q_{n+r-i}}{Q_1...Q_{i-1}}=\Q^i_{n+r}=x_{E_i}$. In particular, returning to our desired basis element $\psi$, the coordinate of $E_{\lfloor a/2\rfloor}$ is $\Q^{\lfloor a/2\rfloor}_{n+r}=x_{E_{\lfloor a/2\rfloor}}$.
 	
 	Note that the above logic only uses proposition \ref{shifted rainbow coeffs}, which requires only that a sub-matching be a rainbow element. So, suppose some basis element $\sigma$ has sub-matching $\sigma(s,t)=R$ with $n'$ nodes and $r'$ anchors, and that the coordinate of $\sigma$ in $w$ is $c$. Then it follows that the basis element $\theta_i$ defined by $\theta_i(1,s-1):=\sigma(1,s-1)$, $\theta_i(t+1,2n+r):=\sigma(t+1,2n+r)$, and $\theta_i(s,t):=E_i$ has coefficient $\Q^i_{n'+r'}c$. In other words, defining $Y_\psi$ with respect to the sub-matching $\sigma(s,t)$, the operation of finding the coordinate of $E_i$ given the coordinate of $R=\sigma(s,t)$ commutes with the isomorphism to $Y_\psi$.  An example is given in figure \ref{kernel induct characterization example}.
 	
 	\vspace{2mm}
 	The above technique specifies an algorithm for determining the coordinate of $\psi$. 
 	
 	As a base case, for the zero element have the algorithm return 1.
 	
 	Suppose inductively that the algorithm returns the coordinate for any $\sigma\in M_{2n'+r'}^{r'}$, $2n'+r'<2n+r$, and that that coordinate is equal to the coordinate function $x_\sigma$. Also suppose that the algorithm commutes with any isomorphism defined by lemma \ref{sub-matching isomorphism}. These statements are clearly true for the base case. 
 	
 	Given $\psi\in M_{2n+r}^r$, if $\psi(1)=a$, we may find the coordinate of $E_{\lfloor a/2\rfloor}$ as before. Note that this operation commutes with any isomorphism defined by lemma \ref{sub-matching isomorphism}. Me may define $Y_{E_{\lfloor a/2\rfloor}}$ with respect to the sub-matching $E_{\lfloor a/2\rfloor}(2,a-1)$. By the inductive hypothesis, we may apply the algorithm to this sub-matching and commute with the isomorphism with $Y_{E_{\lfloor a/2\rfloor}}$. In this way, we find that the coordinate of $\widehat{\psi}$ defined by $\widehat{\psi}(1,a):=\psi(1,a)$ and $\widehat{\psi}(a+1,2n+r)=R$ is $x_{\psi(2,a-1)}(q)\Q^{\lfloor a/2\rfloor}_{n+r}=x_{\widehat{\psi}}(q)$. Similarly, define $Y_{\widehat{\psi}}$ with respect to the sub-matching $\widehat{\psi}(a+1,2n+r)$, and commute the algorithm with the isomorphism. In the same way, we obtain that the coordinate of $\psi\in Y_{\widehat{\psi}}$ is $x_{\psi(2,a-1)}(q)x_{\psi(a+1,2n+r)}(q)\Q^{\lfloor a/2\rfloor}_{n+r}=x_\psi(q)$ as desired.
 	
 	Note that we only added a single operation to the algorithm in the inductive step, which also commutes with any isomorphism defined by lemma \ref{sub-matching isomorphism}. Thus the inductive step holds and the proposition is proved.
 	
 	\begin{figure}
 		\def\cbasisspacing{1mm}
 		\begin{adjustbox}{width=\textwidth}
 			\GeneralizedMatching{14}{3/12,4/11,5/10,6/9,7/8,13/14}{2}{1/1,2/2}{12/4}, \hspace{\cbasisspacing}
 			\GeneralizedMatching{14}{1/2,3/4,5/6,9/12,10/11,13/14}{2}{7/1,8/2}{12/4}, \hspace{\cbasisspacing}
 			\GeneralizedMatching{14}{1/6,2/5,3/4,9/12,10/11,13/14}{2}{7/1,8/2}{12/4}
 		\end{adjustbox}
 		
 		\caption{The figure on the left has sub-matching $R$ ignoring the last two nodes. The middle figure has submatching $R_3$ ignoring the last two nodes. The figure on the right has sub-matching $E_3$ also ignoring the last two nodes. Since the last two nodes have the same structure for all elements, if the coordinate of the first element is $c$, the coordinate of the second is $Q_6Q_5Q_4c$, and the coordinate of the third is $\frac{Q_6Q_5Q_4}{Q_1Q_2}c$.}
 		
 		\label{kernel induct characterization example}
 	\end{figure}
 	
 \end{proof}
 
 \vspace{5mm}
 The following few corollaries will help to simplify some later arguments.
 
 \begin{corollary}
 	Let $w\in\cap\ker(1+T_i)$, $w\not=0$. Suppose $\psi(1,a)$ is a sub-matching with no anchors. Then:
 	
 	$$x_\psi=x_{\psi(1,a)}(q)x_{\psi(a+1,2n+r)}(q)\Q^{a/2}_{n+r}$$
 	
 	\label{characterization generalization}
 \end{corollary}
 
 \begin{proof}
 	Define $a_1=\psi(1)$, $a_i=\psi(a_{i-1}+1)$. Then for some $j$ we have $a_j=a$. If $j=1$, the statement is the same as the proposition. Suppose that the statement is true for any matching with $a_v=a$, $v<j$. Then the statement holds for the sub-matching $\psi(a_1+1,2n+r)$, and we have:
 	\begin{align*}
 	x_\psi(q)=&x_{\psi(1,a_1)}(q)x_{\psi(a_1+1,2n+r)}(q)\Q^{a_1/2}_{n+r}\\
 	=&x_{\psi(1,a_1)}(q)x_{\psi(a_1+1,a)}(q)x_{\psi(a+1,2n+r)}(q)\Q^{a_1/2}_{n+r}\Q_{n+r-a_1/2}^{a/2-a_1/2}\\
 	=&x_{\psi(1,a_1)}(q)x_{\psi(a_1+1,a)}(q)x_{\psi(a+1,2n+r)}(q)\Q^{a_1/2}_{n+r}\Q_{n+r-a_1/2}^{a/2-a_1/2}\prn{\frac{Q_{a/2-1}...Q_{a/2-a_1/2}}{Q_{a/2-1}...Q_{a/2-a_1/2}}}\\
 	=&x_{\psi(1,a_1)}(q)x_{\psi(a_1+1,a)}(q)x_{\psi(a+1,2n+r)}(q)\Q^{a_1/2}_{a/2}\Q^{a/2}_{n+r}\\
 	=&x_{\psi(1,a)}(q)x_{\psi(a+1,2n+r)}(q)\Q^{a/2}_{n+r}
 	\end{align*}
 	
 \end{proof}
 
 \vspace{5mm}
 \begin{corollary}
 	If $\psi\in M_{2n+r}^r$, then $x_\psi(q)\not=0$ if $e>n+r$.
 	
 	\label{coeff nonzero}
 \end{corollary}
 
 \begin{proof}
 	For our base cases, if $2n+r=2$ all coefficients are 1, which is nonzero for any $q$.
 	
 	Assume the statement is true for all $2n'+r'<2n+r$. Given $\psi(1)=a$ we have $$x_\psi(q)=x_{\psi(2,a-1)}(q)x_{\psi(a+1,2n+r)}(q)\Q^{\lfloor a/2\rfloor}_{n+r}$$
 	
 	If $e>n+r$, non of the $Q_i$ term appearing in $\Q^{\lfloor a/2\rfloor}_{n+r}$ are zero, and $n'+r'<n+r<e$ for any of the sub-matchings that appear, so those coordinates are nonzero and the corollary holds.
 	
 	
 \end{proof}
 
 
 \vspace{5mm}
 The proposition fully characterizes any possible kernel element when $Q_1...Q_{n+r-1}\not=0$. In particular, the following corollary holds:
 
 \begin{corollary}
 	When $Q_1..Q_{n+r-1}\not=0$ and the kernel is nontrivial, the kernel is one dimensional.
 	
 	\label{1-D}
 \end{corollary}
 
 This corollary follows from the fact that we may write the coordinate of any basis element as proportional to the coordinate of the rainbow basis element.
 
 
 \subsection{Nontrivial kernel}
 
 
 To verify the kernel element, we will need to know exactly which basis elements are mapped to a specific basis element by a given $(1+T_i)$. The next two lemmas help address this question.
 
 {\color{magenta} Move lemmas to a new appendix $\backslash$section.}
 \begin{lemma}
 	Take some basis element $\psi\in M_{2n+r}^r$. 
 	
 	\begin{enumerate}[label={(\roman*)}]
 		
 		\item Suppose $\psi(a)=b$ for some $b>a+1$, and that $(1+T_i)\psi=(1+q)\psi$ for some $a<i<b-1$. Me then have a subrepresentation $\psi(a,b)$ and define $Y_\psi$ with respect to this subrepresentation. Then for all basis elements $\sigma$ such that $(1+T_i)\sigma=q^{1/2}\psi$, we have that $$\sigma\in Y_\psi$$.
 		
 		\item Suppose $\psi$ has some anchor at position $u$, and $(1+T_i)\psi=(1+q)\psi$ for some $i>u$, we again have a subrepresentation $\psi(u,2n+r)$ and define $Y_\psi$ with respect to this subrepresentation. Then for all basis elements $\sigma$ such that $(1+T_i)\sigma=q^{1/2}\psi$, we have that $\sigma\in Y_\psi$ again.
 		
 		
 	\end{enumerate}
 	\label{preimage under hump}
 \end{lemma}
 
 \begin{proof}
 	This lemma follows from an observation I made in section \ref{two humps created}: a transposition can only create two arcs or an arc and an anchor. 
 	
 	(i) If $\sigma\not\in Y_\psi$ either $\sigma(1,a-1)\not=\psi(1,a-1)$ or $\sigma(b+1,2n+r)\not=\psi(b+1,2n+r)$. Suppose it is the first case. Then for some $s,t\in [1,a-1]$, $s<t$, we have $\psi(s)=t$ and $\sigma(s)\not=t$. To have $(1+T_i)\sigma=q^{1/2}\psi$ we must have $\sigma(t)=i+1$, $\sigma(s)=i$. But then $\sigma(a)\not=b$ and $\sigma(a)\not=i$ or $i+1$, so $((1+T_i)\sigma)(a)\not=b$ and $(1+T_i)\sigma\not=q^{1/2}\psi$. The same argument proves the $\sigma(b+1,2n+r)\not=\psi(b+1,2n+r)$ case. 
 	
 	(ii) An analogous argument proves the anchor case. Specifically, the anchor cannot exist at position $u$ and is not created by action of $(1+T_i)$ if $\sigma(s)=i$ and $\sigma(t)=i+1$.
 	
 \end{proof}
 
 \vspace{5mm}
 It is important to note that lemma \ref{preimage under hump} only references cases where a transposition acts under an arc or to the right of an anchor. An example is given in figure \ref{preimage under hump example}. 
 
 The next lemma characterizes cases where the transposition is not under any arcs and all anchors are to the right.
 
 \begin{figure}[b]
 	\[
 	\begin{tabular}{l l l l}
 	\GeneralizedAction{13}{1/4,2/3,6/7,8/13,9/10,11/12}{1}{5/1}{5/4}{9}{1/4,2/3,6/7,8/13,9/10,11/12}{5/1}{(1+q)}\\
 	
 	\GeneralizedAction{13}{1/4,2/3,6/7,8/9,10/13,11/12}{1}{5/1}{5/4}{9}{1/4,2/3,6/7,8/13,9/10,11/12}{5/1}{(q^{1/2})}\\
 	
 	\GeneralizedAction{13}{1/4,2/3,6/7,8/13,9/10,11/12}{1}{5/1}{5/4}{6}{1/4,2/3,6/7,8/13,9/10,11/12}{5/1}{(1+q)}\\
 	
 	\GeneralizedAction{13}{1/4,2/3,5/6,8/13,9/10,11/12}{1}{7/1}{5/4}{6}{1/4,2/3,6/7,8/13,9/10,11/12}{5/1}{(q^{1/2})}\\
 	
 	\end{tabular}
 	\]
 	
 	\caption{In the first line we act under an arc, so if another element without that arc is sent to that element, it must fix the arc as shown in the second line. In the third line we act to the right of an anchor, so if another element without that anchor is sent to that element, it must fix the anchor as shown in the fourth line. 
 	}
 	\label{preimage under hump example}
 \end{figure}
 
 Essentially, this lemma states that the only elements sent to the same element are those which break at most one of the top level arcs to the left of the leftmost anchor, or that break the leftmost anchor. An illustration is given in figure \ref{preimage under nothing example}.
 
 \vspace{5mm}
 \begin{lemma}
 	Take a basis element $\psi\in M_{2n+r}^r$. Suppose the leftmost anchor in $\psi$ is at index $b$, or let $b=2n+r+1$ if there is no anchor. Define $a_j$ such that $\psi(a_j)=a_{j-1}+1$ and $\psi(a_1)=1$ for all $j$ such that $a_j<b$. 
 	
 	Suppose $(1+T_i)\psi=(1+q)\psi$ for some $i<b-1$ where $\nexists s,t$ such that $\psi(s)=t$ and $s<i,t>i+1$. Suppose there is some basis element $\sigma$ such that $(1+T_i)\sigma=q^{1/2}\psi$. Then:
 	\\
 	
 	\begin{enumerate}[label={(\roman*)}]
 		\item  $\psi(a_{j-1}+2,a_j-1)=\sigma(a_{j-1}+2,a_j-1)$ for all $j$.
 		
 		\item $\psi(b+1,2n+r)=\sigma(b+1,2n+r)$
 		
 		\item If $b$ is not an anchor in $\sigma$, $\psi(a_j)=\sigma(a_j)$ for all $j$ such that $a_j\not=i+1$.
 		
 		\item If $b$ is an anchor in $\sigma$, there exists exactly one value of $j$ such that $\sigma(a_j)\not=\psi(a_j)$ and $a_j\not=i+1$
 	\end{enumerate}
 	\label{preimage under nothing}
 \end{lemma}
 
 \begin{proof}
 	
 	(i) Suppose that, for some $j$ there exists $s,t\in [a_{j-1}+2,a_j-1]$ such that $\psi(s)=t$ but $\sigma(s)\not=t$. Then if $(1+T_i)\sigma=q^{1/2}\psi$ we must have $\sigma(i)=s$ or $t$ and $\sigma(i+1)=s$ or $t$. But, by definition, $i,i+1\not\in[a_{j-1}+1,a_j]$, so this implies $\sigma(a_j)\not=a_{j-1}+1,i,i+1$, so $((1+T_i)\sigma)(a_j)\not=a_{j-1}+1$ and $(1+T_i)\sigma\not=q^{1/2}\psi$. So (i) is proved.
 	
 	\vspace{5mm}
 	(ii) The proof of (ii) is analogous to the proof of (i). Me cannot have $\psi(b+1,2n+r)\not=\sigma(b+1,2n+r)$ and $\psi(b+1,2n+r)=((1+T_i)\sigma)(b+1,2n+r)$ if $((1+T_i)\sigma)(b)=b$.
 	
 	\vspace{5mm}
 	(iii) If $b$ is not an anchor in $\sigma$ and $(1+T_i)\sigma=q^{1/2}\psi$, we must have $i$ an anchor in $\sigma$, and $\sigma(i+1)=b$. No other nodes in $\sigma$ are changed, so this proves (iii).
 	
 	\vspace{5mm}
 	(iiii) From (i)-(iii) we have that the only remaining matchings that can differ are the $(a_{j-1}+1,a_j)$ matchings. If one of them differs, by the same argument as before it must be fixed by the action of $(1+T_i)$, and no other nodes are changed, so (iiii) is proved.
 	
 \end{proof}
 
 
 \begin{figure}[b]
 	\[
 	\begin{tabular}{l l l l}
 	\GeneralizedAction{15}{1/6,2/3,4/5,7/8,9/10,12/15,13/14}{1}{11/1}{5/4}{7}{1/6,2/3,4/5,7/8,9/10,12/15,13/14}{11/1}{(1+q)}\\
 	
 	\GeneralizedAction{15}{1/8,2/3,4/5,6/7,9/10,12/15,13/14}{1}{11/1}{5/4}{7}{1/6,2/3,4/5,7/8,9/10,12/15,13/14}{11/1}{(q^{1/2})}\\
 	
 	\GeneralizedAction{15}{1/6,2/3,4/5,7/10,8/9,12/15,13/14}{1}{11/1}{5/4}{7}{1/6,2/3,4/5,7/8,9/10,12/15,13/14}{11/1}{(q^{1/2})}\\
 	
 	\GeneralizedAction{15}{1/6,2/3,4/5,8/11,9/10,12/15,13/14}{1}{7/1}{5/4}{7}{1/6,2/3,4/5,7/8,9/10,12/15,13/14}{11/1}{(q^{1/2})}\\
 	
 	\end{tabular}
 	\]
 	
 	\caption{The action of $(1+T_7)$ fixes the first basis element. Shown are all the basis vectors sent to the same element by the same transposition. Note that in all of them nodes 2-5 and 12-15 are the same. This illustrates (i) and (ii) in lemma \ref{preimage under nothing}. Note that in the last case where the anchor is in a different place, 1,6 and 9,10 are still matched. This illustrates (iii). In the middle two cases where the anchor is in the same place, only one of 1,6 or 9,10 are not paired. This illustrates (iiii). 
 	}
 	\label{preimage under nothing example}
 \end{figure}
 
 \vspace{5mm}
 Lastly, we will need a small combinatorial result.
 
 \begin{lemma}
 	Suppose $n>b\geq a>0$ and $e>n$. Then $$Q_{n-a}Q_b-Q_{n-b-1}Q_{a-1}=Q_{n}Q_{b-a}$$
 	
 	\label{Q lemma}
 \end{lemma}
 
 \begin{proof}
 	If $b=1$, the only possibility for $a$ is 1, in which reduces to lemma \ref{Q alg}.
 	
 	Suppose the lemma is true for all $\widehat{b}<b+1$. Then for $a< b$ we have 
 	\begin{align*}	
 	Q_{n-a}Q_b -Q_{n-b-1}Q_{a-1}=&Q_{n}Q_{b-a}\\
 	Q_1Q_{n-a}Q_b-Q_1Q_{n-b-1}Q_{a-1}=&Q_1Q_{n}Q_{b-a}
 	\end{align*}
 	
 	from lemma \ref{Q alg}, we have	
 	$$Q_{n-a}(Q_{b+1}+Q_{b-1})-(Q_{n-b}+Q_{n-b-2})Q_{a-1}=Q_{n}(Q_{b-a-1}+Q_{b-a+1})$$
 	
 	and from the inductive hypothesis we have
 	
 	$$Q_{n-a}Q_{b+1}-Q_{n-b}Q_{a-1}=Q_{n}Q_{b-a+1}$$
 	
 	as desired.
 	
 	For $a=b$ we have 
 	\begin{align*}
 	Q_{n-b}Q_b -Q_{n-b-1}Q_{b-1}=&Q_{n}\\
 	Q_1Q_{n-b}Q_b-Q_1Q_{n-b-1}Q_{b-1}=&Q_1Q_{n}
 	\end{align*}
 	
 	from lemma \ref{Q alg}, we have
 	\begin{align*}
 	Q_{n-b}(Q_{b+1}+Q_{b-1})-(Q_{n-b}+Q_{n-b-2})Q_{b-1}=&Q_1Q_{n}\\
 	Q_{n-b}Q_{b+1}-Q_{n-b-2}Q_{b-1}=&Q_1Q_{n}\\
 	\end{align*}
 	
 	as desired.
 	
 	For $a=b+1$, we continue:
 	\begin{align*}
 	Q_1Q_{n-b}Q_{b+1}-Q_1Q_{n-b-2}Q_{b-1}=&Q_1Q_1Q_{n}\\
 	(Q_{n-b-1}+Q_{n-b+1})Q_{b+1}-Q_{n-b-2}(Q_{b}+Q_{b-2})=&(1+Q_2)Q_{n}\\
 	\end{align*}
 	
 	So by the inductive hypothesis
 	$$Q_{n-b-1}Q_{b+1}-Q_{n-b-2}Q_{b}=Q_{n}$$
 	
 	as desired, and the proof is finished by induction.
 	
 \end{proof}
 
 \vspace{5mm}
 Me are now ready to prove existence of a kernel element. To prove this, we will show that if $w\in M_{2n+r}^r$ is as characterized above, the coordinate of any basis element in $(1+T_i)w$ is zero. This will split into various cases related to the previous lemmas.
 
 \vspace{5mm}
 \begin{theorem}
 	Suppose $e=n+r+1$. Then $\cap\ker(1+T_i)\not=0$.
 	\label{kernel existence}
 \end{theorem}
 
 \begin{proof}
 	
 	
 	As a base case, when $2n'+r'\leq 2$, the representation is at most one dimensional. If the one basis element has only anchors, it is sent to zero by any $(1+T_i)$, and is in the kernel. If the single basis element is a single arc, it is sent to $(1+q)$ times itself, and we take $e=n+r+1=2$ so $1+q=0$ and the base case holds.
 	
 	\vspace{5mm}
 	
 	Assume inductively that the statement holds for all $M_{2n'+r'}^{r'}$ where $2n'+r'<2n+r$. Take $w$ as defined by proposition \ref{kernel characterization}.
 	
 	Given $\psi\in (1+T_i)M_{2n+r}^r$ let $E_\psi\subset M_{2n+r}^r$ be the pre-image of $\psi$ under the action of $(1+T_i)$. To prove $w$ is in the kernel, we must show the following: 
 	\begin{equation}
 	(1+q)x_\psi(q)+\sum_{\sigma\in E_\psi,\sigma\not=\psi}q^{1/2}x_\sigma(q)=0\text{ for all basis elements }\psi
 	\label{weak kernel}
 	\end{equation}
 	Inductively, we assume this equation holds for basis elements in smaller representations $M_{2n'+r'}^{r'}$, but only for $q$ such that $e=n'+r'+1$. Clearly this is true in the base case. For the following proof we will need a slightly stronger inductive assumption. Take $\psi'\in M_{2n'+r'}^{r'}$, and suppose either that $\psi'(1)=2n'+r'$, and that $T_i\psi'=(1+q)\psi'$, $1<i<2n'+r'-1$, or that 1 is an anchor in $\psi$. Defining $E_{\psi'}$ as before, we assume
 	
 	\begin{equation}
 	(1+q)x_{\psi'}+\sum_{\sigma\in E_{\psi'},\sigma\not=\psi'}q^{1/2}x_\sigma=0\text{ for any } q \text{ with }e>n'+r'
 	\label{strong kernel}
 	\end{equation}
 	
 	Note that \ref{strong kernel} does not apply in the base case. Our proof of the inductive step will be split into cases, and each case will only depend on sub-cases in which certain inductive hypotheses apply, so this will not lead to any problems.
 	
 	Before exploring the cases, let us formally define $E_\psi$ to be the pre-image of $\psi$ under the action of $(1+T_i)$, and $E_{\psi(a,b)}$ to be the pre-image of $\psi(a,b)$ under action of $(1+T_{i-a+1})$:
 	
 	\begin{enumerate}[label={case \arabic*:}]
 		\item Suppose $\psi\in (1+T_i)M_{2n+r}^r$ for some $i$, and that $\exists s,t$ such that $s<i<t-1$, $s>1$ or $t<2n+r$, and $\psi(s)=t$. Also suppose the leftmost anchor is at some index $u>t$, or that there are no anchors. Then we have a sub-matching $\psi(s,t)$, and by lemma \ref{preimage under hump} $E_\psi\subset Y_\psi$. Then, using corollary \ref{characterization generalization}, the following equality holds:
 		
 		\begin{align*}
 		&(1+q)x_\psi(q)+\sum_{\sigma\in E_\psi,\sigma\not=\psi}q^{1/2}x_\sigma(q)\\
 		=&\prn{x_{\psi(1,s-1)}(q)\Q_{n+r}^{(s-1)/2}}\prn{(1+q)x_{\psi(s,2n+r)}(q)+\sum_{\sigma\in E_\psi,\sigma\not=\psi}q^{1/2}x_{\sigma(s,2n+r)}(q)}\\
 		=&\prn{x_{\psi(1,s-1)}(q)\Q_{n+r}^{(s-1)/2}}\prn{x_{\psi(t+1,2n+r)}(q)\Q_{n+r-(s-1)/2}^{(t-s+1)/2}}\prn{(1+q)x_{\psi(s,t)}(q)+\sum_{\sigma\in E_\psi,\sigma\not=\psi}q^{1/2}x_{\sigma(s,t)}(q)}
 		\end{align*}
 		
 		Me have that $e>j$ for any $Q_j$ term appearing in the equation above, and $e>n'+r'$ for any sub-matching coordinate appearing above, so by corollary \ref{coeff nonzero}:
 		
 		$$(1+q)x_\psi(q)+\sum_{\sigma\in E_\psi,\sigma\not=\psi}q^{1/2}x_\sigma(q)=0$$
 		
 		if and only if
 		
 		$$(1+q)x_{\psi(s,t)}(q)+\sum_{\sigma\in E_\psi,\sigma\not=\psi}q^{1/2}x_{\sigma(s,t)}(q)=0$$
 		
 		Note that $(\psi(s,t))(1)=t-s+1$. So by our inductive hypothesis (ii), we have 
 		
 		$$(1+q)x_{\psi(s,t)}(q)+\sum_{\sigma\in E_{\psi(s,t)},\sigma\not=\psi(s,t)}q^{1/2}x_{\sigma}(q)=0$$
 		
 		By lemma \ref{sub-matching isomorphism}, if $\sigma\in Y_\psi$, $(1+T_i)\sigma=q^{1/2}\psi$ if and only if $(1+T_{i-s+1})\sigma(s,t)=q^{1/2}\psi(s,t)$, so the previous equation implies $$(1+q)x_{\psi(s,t)}(q)+\sum_{\sigma\in E_\psi,\sigma\not=\psi}q^{1/2}x_{\sigma(s,t)}(q)=0$$
 		
 		as desired, and this case is proved.
 		
 		\vspace{5mm}
 		\item Again take $\psi\in (1+T_i)M_{2n+r}^r$ for some $i$, but suppose the leftmost anchor is at some position $u$ where $1<u<i$. Then, as before, we have a sub-matching $\psi(u,2n+r)$ and by lemma \ref{preimage under hump} $E_\psi\subset Y_\psi$.
 		
 		Note that both corollary \ref{characterization generalization} and our inductive hypothesis \ref{strong kernel} still apply in this case, where we consider a left anchor instead of a matching. This allows the exact same logic from the proof of the first case to prove this case.
 		\\
 		
 		It is important to note that, for both case 1 and case 2, the inductive hypothesis depends only on cases in which \ref{strong kernel} holds. Thus, if we show these cases rely on valid base cases, case 1 and 2 follow. This will be done in case 4.
 		
 		
 		\vspace{5mm}
 		\item Suppose $\psi\in (1+T_i)M_{2n+r}^r$ for some $i$, the leftmost anchor is at a position $u>i+1$ or there are no anchors, and $\nexists s,t$ such that $\psi(s)=t$ and $s<i<t-1$. Lemma \ref{preimage under nothing} characterizes all $\sigma\in E_\psi$. Me would like to prove the following for arbitrary $q$ where $e>n+r$:
 		
 		$$(1+q)x_\psi(q)+\sum_{\sigma\in E_\psi,\sigma\not=\psi}q^{1/2}x_\sigma(q)=-q^{1/2}x_{\psi(1,i-1)}(q)x_{\psi(i+2,2n+r)}(q)\Q_{n+r+1}^{(i+1)/2}$$
 		
 		\begin{figure}[b]
 			\[
 			\begin{tabular}{l l l l}
 			\GeneralizedAction{7}{1/2,3/4,5/6}{1}{7/1}{3/4}{3}{1/2,3/4,5/6}{7/1}{(1+q)}\\
 			
 			\GeneralizedAction{7}{1/4,2/3,5/6}{1}{7/1}{3/4}{3}{1/2,3/4,5/6}{7/1}{(q^{1/2})}\\
 			
 			\GeneralizedAction{7}{1/2,3/6,4/5}{1}{7/1}{3/4}{3}{1/2,3/4,5/6}{7/1}{(q^{1/2})}\\
 			
 			\GeneralizedAction{7}{1/2,4/7,5/6}{1}{3/1}{3/4}{3}{1/2,3/4,5/6}{7/1}{(q^{1/2})}\\
 			
 			\end{tabular}
 			\]
 			
 			\caption{The four elements sent to the first element by $(1+T_3)$ are listed. The coordinate of the first element is $Q_3Q_2Q_1$. The coordinate of the second is $Q_3Q_2$. The coordinate of the third is $Q_3Q_2$. The coordinate of the fourth is $Q_3$. Call the first element $\psi$. Then $x_{\psi(1,2)}=1$, $x_{\psi(5,7)}=Q_1$, so $-q^{1/2}x_{\psi(1,i-1)}(q)x_{\psi(i+2,2n+r)}(q)\Q_{n+r+1}^{(i+1)/2}=-q^{1/2}Q_4Q_3$. Me also have $(1+q)x_\psi(q)+\sum_{\sigma\in E_\psi,\sigma\not=\psi}q^{1/2}x_\sigma(q)=(q+1)(Q_3Q_2Q_1)+q^{1/2}(Q_3Q_2+Q_3Q_2+Q_3)=-q^{1/2}(Q_3Q_2Q_1^2-2Q_3Q_2+Q_3)=-q^{1/2}Q_4Q_3$ as desired (one can verify the last equality by hand or simplify using lemma \ref{Q lemma}).}
 			\label{case 3}
 		\end{figure}
 		
 		
 		See figure \ref{case 3} for an example of this equality. Note that if $e=n+r+1$, $Q_{n+r}$ is the only zero component in the right side of this equation, so proving this equation is sufficient to prove case three.
 		\\
 		
 		Me will prove this equality through yet another inductive proof, this time inducting on the number of top level humps, including the leftmost anchor.
 		
 		Formally, as we have in earlier lemmas, we will define $a_j$ by $a_1:=\psi(1)$, $a_j:=\psi(a_{j-1}+1)$. Then define $b_\psi$ such that $a_{b_\psi}=u$ if there is an anchor or $a_{b_\psi}=2n+r$ otherwise. Me induct on $b_\psi$.
 		\\
 		
 		If $b_\psi=1$, we must be in $M_2^0$ to be in case 3 (otherwise $s<i<t-1$ for some $s,t$ where $\psi(s)=t$), which is trivially satisfied. Thus the base case holds.
 		\\
 		
 		Suppose for all basis elements $\sigma$ such that $b_\sigma<b_\psi$, the equality holds. Suppose $i\not=1$. Then $a_1<i$ and lemma \ref{preimage under nothing} gives that there is a unique $\upsilon\in E_\psi$ such that $\upsilon(1)\not=a_1$. Thus we have the following equality:
 		
 		\begin{align*}
 		&(1+q)x_\psi(q)+\sum_{\sigma\in E_\psi,\sigma\not=\psi,\upsilon}q^{1/2}x_\sigma(q)\\
 		=&\prn{x_{\psi(1,a_1)}(q)\Q_{n+r}^{a_1/2}}\prn{(1+q)x_{\psi(a_1+1,2n+r)}(q)+\sum_{\sigma\in E_\psi,\sigma\not=\psi,\upsilon}q^{1/2}x_{\sigma(a_1+1,2n+r)}(q)}
 		\end{align*}
 		
 		Define $Y_\psi$ with respect to the sub-matching $\psi(a_1+1,2n+r)$. Then $\sigma\in E_\psi$, $\sigma\not=\upsilon$ implies $\sigma\in Y_\psi$. By our inductive hypothesis, we have that 
 		\begin{align*}
 		&(1+q)x_{\psi(a_1+1,2n+r)}(q)+\sum_{\sigma\in E_{\psi(a_1+1,2n+r)},\sigma\not=\psi(a_1+1,2n+r)}q^{1/2}x_{\sigma}(q)\\
 		=&-q^{1/2}x_{\psi(a_1+1,i-1)}(q)x_{\psi(i+2,2n+r)}(q)\Q_{n+r-a_1/2+1}^{(i+1-a_1)/2}
 		\end{align*}
 		
 		
 		By lemma \ref{sub-matching isomorphism}, $\sigma\subset Y_\psi$, $\sigma\in E_\psi$ if and only if $\sigma(a_1+1,2n+r)\in E_{\psi(a_1+1,2n+r)}$. This implies:
 		
 		\begin{align*}
 		&(1+q)x_{\psi(a_1+1,2n+r)}(q)+\sum_{\sigma\in E_\psi,\sigma\not=\psi,\upsilon}q^{1/2}x_{\sigma(a_1+1,2n+r)}(q)\\
 		=&-q^{1/2}x_{\psi(a_1+1,i-1)}(q)x_{\psi(i+2,2n+r)}(q)\Q_{n+r-a_1/2+1}^{(i+1-a_1)/2}
 		\end{align*}
 		
 		So, combining with the aforementioned equality, we have 
 		\begin{align*}
 		&(1+q)x_\psi(q)+\sum_{\sigma\in E_\psi,\sigma\not=\psi,\upsilon}q^{1/2}x_\sigma(q)\\
 		=&\prn{x_{\psi(1,a_1)}(q)\Q_{n+r-a_1/2+1}^{(i+1-a_1)/2}}\prn{-q^{1/2}x_{\psi(a_1+1,i-1)}(q)x_{\psi(i+2,2n+r)}(q)\Q_{n+r-a_1/2+1}^{(i+1-a_1)/2}}\\
 		=&\prn{x_{\psi(1,a_1)}(q)\Q_{n+r-a_1/2+1}^{(i+1-a_1)/2}}\prn{-q^{1/2}x_{\psi(a_1+1,i-1)}(q)x_{\psi(i+2,2n+r)}(q)\Q_{n+r-a_1/2+1}^{(i+1-a_1)/2}}\prn{\frac{Q_{(i-1)/2-1}...Q_{(i-1-a_1)/2}}{Q_{(i-1)/2-1}...Q_{(i-1-a_1)/2}}}\\
 		=&-q^{1/2}x_{\psi(1,i-1)}(q)x_{\psi(i+2,2n+r)}(q)\Q_{n+r+1}^{(i+1)/2}\prn{\frac{Q_{n+r-a_1/2}Q_{(i-1)/2}}{Q_{n+r}Q_{(i-1-a_1)/2}}}
 		\end{align*}
 		
 		Separately, note that $\upsilon$ is defined by $\upsilon(2,a_1-1)=\psi(2,a_1-1)$, $\upsilon(a_1+1,i-1)=\psi(a_1+1,i-1)$, $\upsilon(i+2,2n+r)=\psi(i+2,2n+r)$, and $\upsilon(1)=i+1$, $\upsilon(a_1)=i$. Thus we may determine $x_\upsilon$, again utilizing corollary \ref{characterization generalization}:
 		
 		\begin{align*}
 		x_\upsilon=&x_{\psi(i+2,2n+r)}x_{\upsilon(2,i)}\Q_{n+r}^{(i+1)/2}\\
 		=&x_{\psi(i+2,2n+r)}\prn{x_{\psi(2,a_1-1)}x_{\upsilon(a_1,i)}\Q_{(i-1)/2}^{(a_1-2)/2}}\Q_{n+r}^{(i+1)/2}\\
 		=&x_{\psi(i+2,2n+r)}\prn{x_{\psi(1,a_1)}x_{\psi(a_1+1,i-1)}\Q_{(i-1)/2}^{(a_1-2)/2}}\Q_{n+r}^{(i+1)/2}\\
 		=&x_{\psi(i+2,2n+r)}x_{\psi(1,i-1)}\Q_{n+r+1}^{(i+1)/2}\prn{\frac{Q_{n+r-(i+1)/2}Q_{a_1/2-1}}{Q_{n+r}Q_{(i-1-a_1)/2}}}
 		\end{align*}
 		
 		Adding this into our previous equation, we have:
 		\begin{align*}
 		&(1+q)x_\psi(q)+\sum_{\sigma\in E_\psi,\sigma\not=\psi}q^{1/2}x_\sigma(q)\\
 		=&-q^{1/2}x_{\psi(1,i-1)}(q)x_{\psi(i+2,2n+r)}(q)\Q_{n+r+1}^{(i+1)/2}\prn{\frac{Q_{n+r-a_1/2}Q_{(i-1)/2}-Q_{n+r-(i+1)/2}Q_{a_1/2-1}}{Q_{n+r}Q_{(i-1-a_1)/2}}}
 		\end{align*}
 		
 		Applying lemma \ref{Q lemma} to the portion of the equation above in parenthesis, the above is equivalent to
 		
 		$$(1+q)x_\psi(q)+\sum_{\sigma\in E_\psi,\sigma\not=\psi}q^{1/2}x_\sigma(q)=-q^{1/2}x_{\psi(1,i-1)}(q)x_{\psi(i+2,2n+r)}(q)\Q_{n+r+1}^{(i+1)/2}$$
 		
 		as desired. Note that if $e\geq n+r+1$ the only term above that can be zero is $Q_{n+r}$ (by corollary \ref{coeff nonzero}). Thus we have proved the inductive step for the case where $i\not=1$.
 		
 		\vspace{5mm}
 		If $i=1$, we instead look at the sub-matchings $\psi(1,a_{(b_\psi-1)})$, $\psi(a_{(b_\psi-1)}+1,2n+r)$. Again lemma \ref{preimage under nothing} gives that there is a unique $\upsilon\in E_\psi$ such that $\upsilon(a_{(b_\psi-1)}+1)\not=\psi(a_{(b_\psi-1)}+1)$. Taking $Y_\psi$ with respect to the sub-matching $\psi(a_{(b_\psi-1)}+1,a_{b_\psi})$ again we have that $\sigma\in E_\psi$, $\sigma\not=\upsilon$ implies $\sigma\in Y_\psi$. Thus, following the same logic as before, we arrive at the following equality:
 		\begin{align*}
 		&(1+q)x_\psi(q)+\sum_{\sigma\in E_\psi,\sigma\not=\psi,\upsilon}q^{1/2}x_\sigma(q)\\
 		=&\prn{x_{\psi(a_{(b_\psi-1)}+1,2n+r)}(q)\Q_{n+r}^{a_{(b_\psi-1)}/2}}\prn{-q^{1/2}x_{\psi(3,a_{(b_\psi-1)})}(q)Q_{a_{(b_\psi-1)}/2}}\\
 		=&-q^{1/2}x_{\psi(3,2n+r)}\frac{Q_{n+r-1}Q_{a_{(b_\psi-1)}/2}}{Q_{a_{(b_\psi-1)}/2-1}}
 		\end{align*}
 		
 		Again, we know the structure of $\upsilon$ from lemma \ref{preimage under nothing}. Suppose for now that $a_{b_\psi}$ is not an anchor, so it is $2n+r$. Then $\upsilon$ is defined by $\upsilon(3,a_{(b_\psi-1)})=\psi(3,a_{(b_\psi-1)})$, $\upsilon(a_{(b_\psi-1)}+2,2n+r-1)=\psi(a_{(b_\psi-1)}+2,2n+r-1)$, and $\upsilon(1)=2n+r$, $\upsilon(2)=a_{(b_\psi-1)}+1$. So we may again find $x_\upsilon$:
 		\begin{align*}
 		x_\upsilon=x_{\upsilon(2,2n+r-1)}=&x_{\psi(3,a_{(b_\psi-1)})}x_{\psi(a_{(b_\psi-1)}+2,2n+r-1)}\Q_{n+r-1}^{a_{(b_\psi-1)}/2}\\
 		=&x_{\psi(3,a_{(b_\psi-1)})}x_{\psi(a_{(b_\psi-1)}+1,2n+r)}\Q_{n+r-1}^{a_{(b_\psi-1)}/2}\\
 		=&x_{\psi(3,2n+r)}\frac{Q_{n-r-1-a_{(b_\psi-1)}/2}}{Q_{a_{(b_\psi-1)}/2-1}}
 		\end{align*}
 		
 		Alternatively, if $a_{b_\psi}$ is an anchor, the definition of $\upsilon$ is now $\upsilon(3,a_{b_\psi}-1)=\psi(3,a_{b_\psi}-1)$, $\upsilon(a_{b_\psi}+1,2n+r)=\psi(a_{b_\psi}+1,2n+r)$, and $\upsilon(1)=1$, $\upsilon(2)=a_{b_\psi}$, so we have:
 		
 		$$x_\upsilon=x_{\upsilon(2,2n+r)}=x_{\psi(3,a_{(b_\psi-1)})}x_{\psi(a_{b_\psi}+1,2n+r)}\Q_{n+r-1}^{a_{(b_\psi-1)}/2}=x_{\psi(3,2n+r)}\frac{Q_{n-r-1-a_{(b_\psi-1)}/2}}{Q_{a_{(b_\psi-1)}/2-1}}$$
 		
 		so for our purposes $x_\upsilon$ is the same in either case.
 		
 		Incorporating into the above equation, we have:
 		
 		$$(1+q)x_\psi(q)+\sum_{\sigma\in E_\psi,\sigma\not=\psi}q^{1/2}x_\sigma(q)=$$
 		
 		
 		$$-q^{1/2}x_{\psi(3,2n+r)}\frac{Q_{n+r-1}Q_{a_{(b_\psi-1)}/2}-Q_{n-r-1-a_{(b_\psi-1)}/2}}{Q_{a_{(b_\psi-1)}/2-1}}$$	
 		
 		By lemma \ref{Q lemma}, this is simply $-q^{1/2}x_{\psi(3,2n+r)}Q_{n+r}$ as desired, and we have finished proving case 3.
 		\\
 		
 		\item The only cases we have not yet dealt with are those where either $1$ is an anchor or $\psi(1)=2n+r$. These are those cases related to our inductive hypothesis (ii).
 		
 		To not be in case 1 or 2, we must have that there are no anchors between index 1 and $i$, and that there is no integer $s$ such that $1<s<i<\psi(s)-1$. It follows from the same argument that proved lemma \ref{preimage under hump} that there exists exactly one $\upsilon\in E_\psi$ such that $\upsilon(1)\not=\psi(1)$. Define $N$ to be $2n+r$ if 1 is an anchor, or $2n+r-1$ if 1 is not an anchor. Then, defining $Y_\psi$ with respect to the sub-matching $\psi(2,N)$, we have that $\sigma\in E_\psi, \sigma\not=\upsilon$ if and only if $\sigma(2,N)\in E_{\psi(2,N)}$. Note that for $E_{\psi(2,N)}$ we may apply the inductive hypothesis from case 3, so we have:
 		\begin{align*}
 		&(1+q)x_\psi(q)+\sum_{\sigma\in E_\psi,\sigma\not=\psi,\upsilon}q^{1/2}x_\sigma(q)\\
 		=&(1+q)x_{\psi(2,N)}(q)+\sum_{\sigma\in E_{\psi(2,N)},\sigma\not=\psi(2,N)}q^{1/2}x_{\sigma}(q)\\
 		=&-q^{1/2}x_{\psi(2,i-1)}x_{\psi(i+2,N)}\Q_{n+r}^{i/2}
 		\end{align*}
 		
 		As in case 3, we can also determine $x_\upsilon$. $\upsilon$ is defined by $\upsilon(2,i-1)=\psi(2,i-1)$, $\upsilon(i+2,N)=\psi(i+2,N)$, $\upsilon(1)=i$, and $\upsilon(i+1)=2n+r$ if 1 is not an anchor or $i+1$ if 1 is an anchor, and we have:
 		
 		$$x_\upsilon=x_{\psi(2,i-1)}x_{\upsilon(i+2,N)}\Q_{n+r}^{i/2}$$
 		
 		Thus we have
 		
 		$$(1+q)x_\psi(q)+\sum_{\sigma\in E_\psi,\sigma\not=\psi}q^{1/2}x_\sigma(q)=$$
 		
 		$$-q^{1/2}x_{\psi(2,i-1)}x_{\psi(i+2,N)}\Q_{n+r}^{i/2}\prn{1-1}=0$$
 		
 		as desired, and the last case is proved. Note that this only relies on the inductive hypothesis from case 3, for which we showed the base case holds.
 	\end{enumerate}
 	
 	Thus our inductive hypotheses have all been proven, and those that apply in the base case hold in the base case, so by induction the theorem is proved.
 	
 \end{proof}
 
 \begin{corollary}
 	If $e=n+r+1$, $M_{2n+r}^r$ is reducible, and has a unique sign subrepresentation.
 \end{corollary}
 
 Due to the argument present in Theorem \ref{Irreducibility Theorem}, sign subrepresentations provide a surprisingly strong characterization of some $M_{2n + r}^r$ as follows.
 \begin{proposition}\label{Preimage of K}
 	Fix some $M := M_{2n + r}^r$.
 	Suppose there is some natural number $n'$ with $n' \leq n$ such that $K_{2n' + r}^r \neq 0$.
 	Let $\pi:M_{2n + r}^r \twoheadrightarrow M_{2n' + r}^r$ be the linear map
 	\[
 	\pi := \begin{cases}
 	(1 + T_1)(1 + T_3)\cdots(1 + T_{n-n'}) & \text{if } n < n',\\
 	\id{M} & \text{if } n = n',
 	\end{cases}
 	\]
  where $\id M$ is the identity map on $M_{2n + r}^r$.
  Then, $\pi^{-1}(K_{2n' + r}^r)$ contains any proper subrepresentations of $M_{2n + r}^r$.
 \end{proposition}
 \begin{proof}
   Set $M := M_{2n + r}$ and $K := K_{2n' + r}^r$.
  It suffices to show that any vector $w \in M - \pi^{-1}(K)$ is cyclic;
 	then, any subrepresentation containing $w$ also contains all of $M$, implying the proposition.
 	
 	In fact, since $\pi(w) \notin K$, there is some $(1 + T_i) \in \SH$ such that $(1 + T_i)\pi(w) \neq 0$.
 	Note that $e > (n'-1) + r + 1$, so $M_{2(n-1) + r}^r$ is irreducible;
 	then, we may use diagram \eqref{Irreducibility commutative diagram} and an analogous argument to Theorem \ref{Irreducibility Theorem} to argue that $(1 + T_i)\pi(w)$ is cyclic, and hence $w$ is cyclic.
 \end{proof}
 \begin{corollary}\label{n+r+1 composition series}
 	Suppose $e = n + r + 1$.
 	Then $M_{2n + r}^r$ has a composition series given by
 	\[
    0 \subset K_{2n + r}^r \subset M_{2n + r}^r.
 	\]\qed
 \end{corollary}

 \begin{remark}
   In the case $n < n'$, we may replace $\pi$ with a product of $n-n'$ elements $(1 + T_{z_1})\cdots(1 + T_{z_{n-n'}}) \in \SH$ such that $\abs{z_i - z_j} \geq 2$ for each $1 \leq i < j \leq n-n'$.
   Then, an analogous statement to Proposition \ref{Preimage of K} follows from an analogous proof.
 \end{remark}
 
 Proposition \ref{Preimage of K} and Corollary \ref{n+r+1 composition series} are suggestive;
 the preimage $\pi^{-1}(K)$ may play a key role in characterizing the representation $M_{2n + r}^r$ in other reducible cases.
 The following section poses several conjectures involving this structure.
  
 \subsection{Conjectures on sign subrepresentations}\label{Conjecture Section}
 If $\pi^{-1}(K)$ is itself a subrepresentation of $M$, then we will have identified the radical of $M_{2n + r}^r$ via Proposition \ref{Preimage of K}.
 Observing decomposition matrices for $S_{(n+r,n)}$ for small $e$ and $n$, we note that $S^{(n+r,n)}$ appears to have composition series of length at-most two \cite[~Appendix~B]{Mathas-book};
 we give heuristic support for an analogous statement on $M_{2n + r}^r$ in Appendix \ref{Empirics Section}.
 This motivates our first conjecture:
 \begin{conjecture}\label{Preimage subrepresentation}
   In Proposition \ref{Preimage of K}, $\pi^{-1}(K_{2n' + r}^r)$ is an irreducible subrepresentation of $M_{2n + r}^r$.
 \end{conjecture}
 This has been verified for $\pi:M_6^0 \twoheadrightarrow M_4^0$.
 In general, this is surprising;
 $\pi$ is a-priori only a linear map, so we do not expect a-priori that the preimage of a subrepresentation via $\pi$ is a subrepresentation of $M_{2n + r}^r$.
  
  \begin{remark}
    As a corollary to \ref{Preimage subrepresentation}, the quotient $M_{2n + r}^r / \pi^{-1}(K_{2n' + r}^r)$ is irreducible \cite[Thm.~10.4]{Dummit}.
  \end{remark}

  We can posit more structure than this;
 by observing Figure \ref{Empirical Kernel}, we note that the kernel is nontrivial in more cases than $e = n + r + 1$, and appears to follow a pattern: 
 \begin{conjecture}\label{Sharp kernel}
 	Suppose $p = \infty$.
  Then, $K_{2n + r}^r \subset M_{2n + r}^r$ is nontrivial if and only if $e \mid n + r + 1$ and $e > n$, in which case it is one-dimensional.
 \end{conjecture} 
 Then, assuming that Conjectures \ref{Preimage subrepresentation} and \ref{Sharp kernel} are true, we reach the following corollary.
 \begin{corollary}
  Assume Conjectures \ref{Preimage subrepresentation} and \ref{Sharp kernel}.
 	Suppose $e,n,r$ are such that $M_{2n + r}^r$ is irreducible and $e \mid n' + r + 1$ for some $e < n' < n$.
 	Then, $M := M_{2n + r}^r$ has a composition series given by \begin{equation}\label{Conjecture filtration}
    0 \subset \pi^{-1}(K_{2n' + r}^r) \subset M_{2n + r}^r.\end{equation}
 \end{corollary}
 \begin{proof}
 	By Conjecture \ref{Sharp kernel}, we have the filtration \eqref{Conjecture filtration}.
 	By Conjecture \ref{Preimage subrepresentation}, the first factor is irreducible, and by \ref{Preimage of K}, the second factor is irreducible.
 \end{proof}
 
 These conjectures would characterize much of the structure of $M_{2n + r}^r$.
 We will pose a final conjecture as follows.
 \begin{conjecture}\label{Final conjecture}
 	For any $e,n,r$, we have $M_{2n + r}^r \cong S^{(n+r,n)}$.
 \end{conjecture}
 The support for this includes the special case $e > n+r+1$ case proven in Theorem \ref{Correspondence Theorem}, as well as the previous contents of this section.
 Further, note that the basis of $M_{2n + r}^r$ is defined analogously for all $e$;
 it is well-known that a basis for $S^\lambda$ is indexed by the standard young tableaux of $\lambda$, which is also independent of $e$ \cite[Prop.~3.22]{Mathas-book}.
 Hence Theorem \ref{Correspondence Theorem} implies that $M_{2n + r}^r$ and $S^{(n+r,n)}$ have the same dimension. 
  These facts give heuristic support for Conjecture \ref{Final conjecture}
 
\section{Fibonacci representations and quotients of Specht modules}\label{Fibonacci Section}
Irreducibility of $S^{(n+r,n)}$ plays a central role in the arguments of Section \ref{Crossingless Matchings Section};
without irreducibility of $M_{2n + r}^r$ (which is conjecturally equivalent to irreducibility of $S^{(n+r,n)}$), $M$ is not necessarily isomorphic to any Specht module $S^\lambda$ or quotient $D^\lambda$, preventing the arguments of Section \ref{Specht Modules Section} from being used.
Further, when $S^{(n+r,n)}$ is reducible, we have $e \mid l$ for some $r+2 \leq l \leq n+r+1$, and hence the filtration provided in Proposition \ref{Filtration} fails to be a composition series.
In summary, for $S^{(n+r,n)}$, nearly every argument in Sections \ref{Specht Modules Section} and \ref{Crossingless Matchings Section} break down, and hence it is difficult to characterize the Specht module $S^{(n+r,n)}$ via crossingless matchings.

Due to these difficulties, we additionally seek a graphical realization of the irreducible quotient $D^{(n+r,n)}$.
In doing so, we may study their branching via Corollary \ref{D Restrictions}, which gives an interesting recurrence in their dimension. 
\begin{example}
Suppose $e = 5$. 
If $m$ is even, let $t$ be such that $2t = m$;
if $m$ is odd, let $t'$ be such that $2t' + 3 = m$.
Then, we may define the following quantity:
\[
  d_{m}^{0,3} := \begin{cases}
    \dim D^{(t,t)} & \text{if } m \text{ is even,}\\
    \dim D^{(t'+3,t')} & \text{if } m \text{ is odd.}
  \end{cases}
\]
Define $d_m^{1,2}$ similarly.
Then, by Corollary \ref{D Restrictions}, we have
\begin{align*}
    d_m^{0,3} &= d_{m-1}^{1,2},\\
    d_m^{1,2} &= d_{m-1}^{1,2} + d_{m-1}^{0,3}\\
    &= d_{m-1}^{1,2} + d_{m-2}^{1,2}.
\end{align*}
Further, by Lemma \ref{Base cases}, we have that $d_2^{0,3} = d_3^{1,2} = 1$;
hence $d_m^{1,2} = d_{m+1}^{0,3} = f_m$, where $f_m$ is the $n$th \emph{Fibonacci number}.
\end{example}

In this section, we henceforth restrict to the case $e = 5$ and $r \leq 3$.
We note that Shor--Jordan \cite{Shor} have conveniently used complex representations of the braid group on $m$ strands in \cite{Shor} having dimensions $f_m$ and $f_{m-1}$.
In fact, the \emph{Fibonacci representation} of Definition \ref{Fib definition} with $k = \CC$ and $q = e^{-3\pi i /5}$ is a rescaling of Shor--Jordan's Fibonacci representation.

Note that Shor--Jordan does not characterize this representation any more than the definition and decomposition into four subrepresentations;
we will give a characterization of this representation which is stronger than presented in \cite{Shor} or its generalization in \cite{Aharonov}.

We will start our study of the Fibonacci representation $V := V^m$ by studying low-dimensional cases.
Recall that we have decomposed $V$ into a direct sum of the subrepresentations $V_{\vs\vs}$, $V_{\vs\vp}$, $V_{\vp\vs}$, and $V_{\vp\vp}$, indexed by the first and last character of the strings in $V$.
\begin{proposition}\label{2 correspondence}
  We have the following isomorphisms of representations:
  \begin {align*}
    \emph{$V_{\vs\vs}^2$} &\cong D^{(1^2)}\\
    \emph{$V_{\vs\vp}^2$} &\cong D^{(2)}\\
    \emph{$V_{\vp\vp}^2$} &\cong \emph{$V_{\vs\vs}^2 \oplus V_{\vs\vp}$}.
  \end{align*}
\end{proposition}
\begin{proof}
  The first isomorphism follows via identification with the trivial representation, and the second with the sign representation.

  Further, $V_{\vp\vp}^2$ is a 2-dimensional representation of a semisimple commutative algebra, and hence decomposes into a direct sum of two 1-dimensional subrepresentations.
  In particular, we may fix basis $\cbr{(\vp\vs\vp),(\vp\vp\vp)}$ for $V_{\vp\vp}^2$ and note that $T_1$ acts by the matrix
  \[
    \rho_{T_1} = \begin{bmatrix}
      \varc & \vard\\
      \vard & \vare
    \end{bmatrix},
  \]
  which has characteristic polynomial \[\lambda^2 - (\varc + \vare)\lambda +(\varc\vare - \vard^2).\]
  We may verify that, for $\lambda = -1$, this evaluates to
  \[
    -((-1 + q + q^2) (1 + q^3 + q^4 + q^5 + 2 q^6 + q^7))\brk{5}_q = 0,
  \]
  and for $\lambda = q$ this evaluates to 
  \[
  -(q^2 (-1 + q + q^2) (1 + q + q^2 + q^3 + 2 q^4 + q^5)) \brk{5}_q = 0.
  \]
  Hence $\rho_{T_1}$ has eigenvalues $-1$ and $q$.

  The eigenspaces with eigenvalues $-1$ and $q$ are subrepresentations isomorphic to the sign and trivial representation, hence $V_{\vp\vp}$ is isomorphic to a direct sum of the trivial and sign representations, as required.
\end{proof}

We may further study the low-dimensional Fibonacci representations with the following proposition:
\begin{proposition}\label{3 irreducibility}
  The representation \emph{$V^3_{\vs\vp}$} is irreducible.
\end{proposition}
\begin{proof}
  Fix the basis $\cbr{\prn{\vs\vp\vs\vp}, \prn{\vs\vp\vp\vp}}$ for $V^3_{\vs\vp}$.
  Then, $T_1$ and $T_2$ act by the following matrices:
\[
  \rho_{T_1} = \begin{bmatrix}
    \varb & 0\\
    0 & \vara
  \end{bmatrix}; \hspace{20pt}
  \rho_{T_2} = \begin{bmatrix}
    \varc & \vard\\
    \vard & \vare
  \end{bmatrix}.
\]
  A proper nontrivial subrepresentation of $V_{\vs\vp}^3$ must be one-dimensional, and hence an eigenspace of each of these matrices;
  since $\varb \neq \vara$, the $\rho_{T_1}$ has two independent eigenspaces given by the spans each basis element; since $\vard \neq 0$, neither basis element is an eigenvector of $\rho_{T_2}$.
  Hence $V^3_{\vs\vp}$ is irreducible.
\end{proof}

These propositions establish the low-dimensional behavior of $V^m$ that we will use in our analysis of general $V^m$ below.
We will proceed first by proving that $V^m_{\vs\vp}$ and $V^m_{\vs\vs}$ are irreducible;
then, we will use combinatorial arguments to prove that these are isomorphic to the desired irreducible quotients two-row Specht modules.

\begin{proposition}\label{Fib irreducibility}
  The representation \emph{$V^m_{\vs\vp}$} is irreducible.
\end{proposition}
\begin{proof}
  We will prove this inductively in $m$.
  We've already proven irreducibility of $V^2_{\vs\vp}$ and $V^3_{\vs\vp}$ via Propositions \ref{2 correspondence} and \ref{3 irreducibility}, so suppose that $V^{m-2}_{\vs\vp}$ is irreducible.
  
  Let $\cbr{v_i}$ be the basis for $V_{\vs\vp}$.
  Then each $v_i$ is cyclic; indeed, we can transform every basis vector into $(\vs\vp \dots \vp)$ via action by the appropriate $\frac{1}{\vard}(T_i - \varc)$, and we can transform $(\vs\vp \dots \vp)$ into any basis vector via action by the appropriate $\frac{1}{\vard}(T_i - \vare)$.
  Hence it is sufficient to show that each $v \in V_{\vs\vp}$ generates some basis element.

  Let $v_j$ be the basis element $(\vs\vp\vs\vp\dots \vp)$, which is many copies of $\vs\vp$, followed by an extra $\vp$ if $m$ is odd.
  We will show that each $v \in V^m_{\vs\vp}$ generates $v_j$.
  Then, each $v$ will be cyclic, implying the proposition.

  Suppose that no basis elements beginning $(\vs\vp\vs\vp)$ have nonzero coefficient in $v$;
  then, there is some basis element $v_i$ beginning $(\vs\vp\vp\vp)$ having nonzero coefficient in $v$, and the basis element having all other characters identical to $v_i$ except for the beginning $(\vs\vp\vs\vp)$ has nonzero coefficient in $T_3v$.
  Hence we may assume that at least one element beginning $(\vs\vp\vs\vp)$ has nonzero coefficient in $v$.

  Note that \[\ima (T_1 - \vara) = \Span\cbr{\text{Basis vectors beginning }(\vs\vp\vs\vp)}\] and $(T_1 - \vara)v \neq 0$.
  Further, note that we may consider $\ima (T_1 - \vara)$ to be a subrepresentation of $\Res_{\SH(S_{m-2})}^{\SH(S_m)} V^m_{\vs\vp}$;
  this yields that $\ima (T_1 - \vara) \cong V^{m-2}_{\vs\vp}$ as representations.
  Hence irreducibility of $V^{m-2}_{\vs\vp}$ implies that $v_j$ is generated by $(T_1 - \vara)v$, and $v$ is cyclic, as desired.
\end{proof}

Note that the structure of this proof is parallel to the structure of Theorem \ref{Irreducibility Theorem}:
we project down to the analogous representation on fewer letters, and we inductively lift irreducibility from this smaller representation to our original one.

Now we may begin considering restrictions of Fibonacci representations as follows.
\begin{lemma}\label{Fib branching}
  The following branching rules hold:
  \begin{align*}
    \emph{$V^{m-1}_{\vp\vp}$} \cong \emph{$\Res V_{\vs\vp}^m$} &\cong \emph{$V^{m-1}_{\vs\vs} \oplus V^{m-1}_{\vs\vp}$},\\
    \emph{$V_{\vp\vs}^{m-1} \cong \Res V_{\vs\vs}^m$} &\cong \emph{$V_{\vs\vp}^{m-1}$}.
  \end{align*}
\end{lemma}
\begin{proof}
  By the results of Appendix \ref{Algebra}, any two restrictions to distinct subalgebras of $\SH$ generated each by $m-2$ characters are isomorphic.
  Using this fact, the leftmost isomorphism on each line follows by considering the restrictions to the subalgebra of $\SH$ generated by $\cbr{T_2,\dots,T_{m-1}}$.
  Further, the rightmost isomorphism on each line follows by considering the restrictions to the subalgebra of $\SH$ generated by $\cbr{T_1,\dots,T_{m-2}}$.
\end{proof}
\begin{corollary}\label{ss irreducibility}
  The representation \emph{$V^m_{\vs\vs}$} is irreducible.\qed
\end{corollary}
\begin{proof}
  By Lemma \ref{Fib branching} and Proposition \ref{Fib irreducibility}, we have $\Res V^m_{\vs\vs} \cong V^{m-1}_{\vs\vp}$ is irreducible, implying that $V^m_{\vs\vs}$ is also irreducible.
\end{proof}
Recall that we have the decomposition
\[
  V^m \cong V^m_{\vs\vs} \oplus V^m_{\vs\vp} \oplus V^m_{\vp\vs} \oplus V^m_{\vp\vp}.
\]
Using this, we may now decompose $V$ into a direct sum of irreducible representations.
\begin{corollary}\label{V decomposition into V}
  The representation $V^m$ decomposes into a direct sum of irreducible representations as follows:
  \[
    \emph{$V^m \cong 3V^m_{\vs\vp} \oplus 2V^m_{\vs\vs}$.}
  \]\qed
\end{corollary}

\begin{remark}
  As in \cite{Shor}, we may be consider a representation $\widetilde V^m$ of the braid group on $m$ strands via Definition \ref{Fib definition}. 
  In fact, Proposition \ref{Fib irreducibility}, Lemma \ref{Fib branching}, and Corollaries \ref{ss irreducibility} and \ref{V decomposition into V} hold in reference to $\widetilde V^m$ by analogous arguments.
\end{remark}

Now we may use these in order to characterize $V$ via irreducible quotients of Specht modules.
\begin{theorem}\label{Fibonacci Theorem}
  We have the following isomorphisms:
     \begin{align*}  
      V^{2n}_{\vs\vs} &\cong D^{(n,n)},\\ 
      V^{2n-1}_{\vs\vs} &\cong D^{(n+1,n-2)},\\
      V^{2n}_{\vs\vp} &\cong D^{(n+1,n-1)},\\
      V^{2n-1}_{\vs\vp} &\cong D^{(n,n-1)}.
     \end{align*}
\end{theorem}
\begin{proof}
  We will prove this by induction on $m = 2n$;
  we have already proven the base case $V^{2}$ via Propoition \ref{2 correspondence}, so suppose that we have proven these isomorphisms for $V^{2n-2}$.
  We will prove the isomorphisms for $V^{2n-1}$ and $V^{2n}$.

  By Proposition \ref{Fib irreducibility}, $V^{2n-1}_{\vs\vs} \cong D^{\lambda}$ and $V^{2n-1}_{\vs\vp} \cong D^{\mu}$ for some partitions $\lambda,\mu \vdash 2n - 1$.
  We will show that $\lambda = (n+1,n-2)$ and $\mu = (n,n-1)$.
  
  First, by Lemma \ref{Fib branching} and the inductive hypothesis, we have \[\Res \; D^{\lambda} \cong D^{(n,n-2)} \cong \Res \; D^{(n+1,n-2)}\] and \[\Res \; D^{\mu} \cong D^{(n,n-2)} \oplus D^{(n-1,n-1)} \cong \Res \; D^{(n,n-1)}.\]
  
  By irreducibility of $\Res D^{\lambda}$, the only normal number in $\lambda$ is 1 \cite{Kleshchev,Brundan}.
  Further, the only diagrams which can be transformed into $(n,n-2)$ by removing a cell are $(n+1,n-2)$, $(n,n-1)$, and $(n,n-2,1)$ as illustrated in Figure \ref{OddRes};
  we have already seen that $D^{(n,n-1)}$ does not have irreducible restriction via Corollary \ref{D Restrictions}, so we are left with $(n+1,n-2)$ and $\varsigma := (n,n-2,1)$.
  We may directly check that $\varsigma$ doesn't satisfy this, as we have
  \begin{align*} 
    \beta_\varsigma(1,2) &= 3 - 2 + (n-2) = n-1,\\
    \beta_\varsigma(1,3) &= 3 - 1 + n = n+2,\\
    \beta_\varsigma(2,3) &= 2 - 1 + 3 = 4.
   \end{align*} 
  At least one of $\beta_\varsigma(1,2)$ and $\beta_\varsigma(1,3)$ is nonzero, since $\beta_\varsigma(1,3) - \beta_\varsigma(1,2) = 3 \not\equiv 0 \pmod e$.
  Hence at least one of 2 or 3 is normal in $\varsigma$, and $\lambda = (n+1,n-2)$.

  For $\mu$, we immediately see from Figure \ref{OddRes} that the only option is $(n,n-1)$.
  \begin{figure}
  \[
      \cbr{
        \begin{gathered}
        \ydiagram{6,3}. \; \;
        \ydiagram{5,4}, \; \;
        \ydiagram{5,3,1}
      \end{gathered} 
    }\longrightarrow \begin{gathered}
      \ydiagram{5,3}
  \end{gathered}\]
      \[\cbr{
        \begin{gathered}
          \ydiagram{5,4}
        \end{gathered}
      }\longrightarrow
      \begin{gathered}
        \ydiagram{5,3},\;\;
        \ydiagram{4,4}.
      \end{gathered}
    \]
      \caption{
      Illustration of the partitions of $9$ which can, via removal of cells, yield $(n,n-2)$ alone, or both $(n,n-2)$ and $(n-1,n-1)$.
    }\label{OddRes}
\end{figure}  
  
  We can perform a similar argument for the $V^{2n}$ case;
  for $V^{2n}_{\vs\vs} \cong D^{\lambda'}$ and $V^{2n}_{\vs\vp} \cong D^{\mu'}$, we have
  \[\Res \; D^{\lambda'} \cong D^{(n,n-1)} \cong \Res \; D^{(n,n)}\] and 
  \[\Res \; D^{\mu'} \cong D^{(n,n-1)} \oplus D^{(n+1,n-2)} \cong \Res \; D^{(n+1,n-1)}.\]
  
  Through a similar process, we see that $\mu' = (n+1,n-1)$.
  We narrow down $\lambda'$ to one of $(n,n)$ or $\varpi := (n,n-1,1)$, and note that
  \begin{align*} 
    \beta_\varpi(1,2) &= 3 - 2 + (n-1) = n,\\
    \beta_\varpi(1,3) &= 3 - 1 + n = n+2,\\
    \beta_\varpi(2,3) &= 2 - 1 + 2 = 3.
  \end{align*}
  and hence at least one of 2 or 3 is normal, $\Res D^{\varpi}$ is not irreducible, and $\lambda' = (n,n)$, finishing our proof.
\end{proof}
\begin{corollary}
  We have the following isomorphisms of representations:
  \begin{align*} 
    V^{2n} &\cong 3D^{(n+1,n-1)} \oplus 2D^{(n,n)},\\
    V^{2n - 1} &\cong 3D^{(n,n-1)} \oplus 2D^{(n+1,n-2).}
  \end{align*}
\end{corollary}
\begin{proof}
  This follows from Corollary \ref{V decomposition into V} and Theorem \ref{Fibonacci Theorem}.
\end{proof}

Hence we have entirely characterized Shor--Jordan's Fibonacci representation \cite{Shor} as a direct sum of irreducible quotients of Specht modules, and we have given graphical realizations of $D^{(n+r,n)}$ for $e = 5$ and $r \leq 3$.

\newpage
\appendix
\section{Compatibility of Representations with the Relations}\label{Compatibility Section}
In general, we defined the representations $V := V^{2n + r}$ and $M := M_{2n + r}$ for the free algebra on generators $\cbr{T_1,\dots,T_{2n + r - 1}}$.
Recall that we may give a presentation of $\SH$ having generators $T_i$ and relations
\begin{align}
  (T_i - q)(T_i + 1) &= 0 \label{quadratic}\\
  T_iT_{i+1}T_i &= T_{i+1}T_iT_{i+1} \label{braid1}\\ 
  T_iT_j &= T_jT_i \hspace{40pt} \abs{i - j} > 1. \label{braid2}
\end{align}
We call \eqref{quadratic} the \emph{quadratic relation} and \eqref{braid1}, \eqref{braid2} the \emph{braid relations}.
It is easily seen that a representation of $\SH$ is equivalent to a representation of the free algebra $k\langle T_i \rangle$ which is compatible with the relations.
We will begin this appendix by proving a more careful definition of the representation $M_{2n + r}^r$.
We will prove in the following sections that $V$ and $M$ are compatible with the Hecke algebra relations.


\subsection{Explicit definition of crossingless matchings}\label{Explicit Definition}
We will give a more careful definition of the crossingless matchings representation here.
\begin{definition}
  A \emph{crossingless matching on $2n+r$ indices with $r$ anchors} is a partition of $\cbr{1,\dots,2n+r}$ into $n$ parts of size $2$ and $r$ of size 1 such that no two parts $(a,a')$ and $(b,b')$ satisfy $a < b < a' < b'$, and no parts $(c), (a,a')$ satisfy $a < c < a'$.
  We will call these arcs and anchors, respectively.
  Then, define $M^r_{2n+r}$ to be the $k$-vector space with basis the set of crossingless matchings on $2n+r$ indices with $r$ anchors.
  If basis element $w_j$ contains arc $(a,b)$, say $w_j(a) := b$ and $w_k(b) := a$.

  In order to endow $M^r_{2n + r}$ with an $\SH$-action, consider some basis element $w_j$ and some element $(1 + T_i)$ of $\SH$.
  The elements $\cbr{1} \cup \cbr{1 + T_i | 1 \leq i < 2n + r}$ generate $\SH$, so it is sufficient to define the action of $1 + T_i$ on $w_j$.

  If $w_j$ has arc $(i,i+1)$, define $(1 + T_i)w_j := (1 + q)w_j$.
  If $w_j$ has anchors $w_j(i) = i$ and $w_j(i+1) = i+1$, define $(1 + T_i)w_j := 0$.
  If $w_j$ has anchor $w_j(i) = i$ and arc $w_j(i+1) = b$, define $(1 + T_i)w_j := q^{1/2}w_l$, where $w_l(i) = i+1$, $w_l(b) = b$, and all other arcs agree with $w_j$.
  If $w_j$ has arcs $w_j(i) = a$ and $w_j(i+1) = b$, then define $(1 + T_i)w_j := q^{1/2}w_l$, where $w_l(i) = i+1$,$w(a) = b$, and all other acts agree with $w_j$. 
  We verify that this is well-defined in Appendix \ref{Cross Relations}.
\end{definition}

We may alternately sharpen our topological definition;
\begin{definition}
Fix $2n + r$ distinct points $a_1,\dots,a_{2n + r}$ points along $\RR \times \cbr{0} \subset \RR^2$ and $r$ distinct points $b_1,\dots,b_r$ along $\RR \times \cbr{1}$.
Then, define $M_{2n + r}^r$ to have basis given by the isotopy classes of $n + r$ paths connecting the points $a_1,\dots,a_{2n+r},b_1,\dots,b_r$ such that no distinct $b_i,b_j$ are connected by a path. 

We will take some basis element $w_j \in M_{2n +r}^r$ and define the action $(1 + T_i)w_j$.
To do so, map $w_j$ through the natural embedding $\RR \times \brk{0,1} \hookrightarrow \RR \times \brk{\frac{1}{2},1}$, and form the figure $w_j^i$ by adjoining the lines connecting $a_l$ and $a_l + \prn{0,\frac{1}{2}}$ for all $l \neq i,i+1$ as well as paths from $a_i$ to $a_{i+1}$ and $a_i + \prn{0,\frac{1}{2}}$ to $a_{i+1} + \prn{0,\frac{1}{2}}$.
This has either 0 or 1 path components which do not intersect $\RR \times \cbr{0,1}$;
these form ``loops.''

Take the figure $\tilde w_j^i$ without this component.
If $\tilde w_j^i$ is not isotopic to some $w_l$, then define $(1 + T_i)w_j := 0$.
If $\tilde w_j^i$ is isotopic to some $w_l$, define $(1 + T_i)w_j := (1 + q)w_l$ if $w_j^i$ has a loop and $(1 + T_i)w_j := q^{1/2}w_l$ otherwise.
This process is illustrated in Figure \ref{Action}.
\end{definition}
  
Let the length of an arc $(i,j)$ be $j - i + 1$.
Note that the crossingless matchings on $2n$ indices with no anchors can all be identified with a list of $n$ integers describing the lengths of the arcs from left to right;
The basis on $M_5^1$ induced by the quotient $M_6^0 \twoheadrightarrow M_{5}^1$ is illustrated in Figure \ref{S5 Basis}, and we call this the \emph{increasing lexicographic basis}. 

\begin{remark}
  This definition gives a graphical calculus for working with our module.
  It should be clear that, if $w_j^i$ has a loop, then $w_l(i) = i+1$ and $w_l = w_j$.
  Further, this easily defines an arbitrary composition:
  \[
    (1 + T_{i_1})\cdots(1 + T_{i_\ell})w_j = q^{(\ell - t)\frac{1}{2}}(1 + q)^tw_l
  \]
  if the figure we make via $(1+T_{i_1})\cdots(1+T_{i_\ell})$ is isotopic to $w_l$ after removing $t$ loops.
\end{remark}

We now verify that this is well-defined as a representation of $\SH$.

\subsection{Compatibility for the crossingless matchings representations}
\label{Cross Relations}
We verify the relations on the crossingless matchings representation $M$.
Take some basis vector $w_i \in M$.
We will first check \eqref{quadratic} by case work:
\begin{itemize}
  \item Suppose there is an arc $(i,i+1)$.
    Then, \[(T_i-q)(T_i + 1)w = (1 + q)\prn{(1 + T_i)w - (1 + q)w} = 0,\] giving \eqref{quadratic}.
  \item Suppose there is no arc $(i,i+1)$ and indices $i,i+1$ do not both have anchors;
    then $(T_i +  1)w = q^{1/2}w'$ for some basis vector $w'$ having arc $(i,i+1)$, and the computation follows as above for \eqref{quadratic}.
  \item Suppose $i,i+1$ are anchors;
    then $(T_i + 1)w = 0$, giving \eqref{quadratic}.
\end{itemize}
  
\vspace{5pt}
Now we verify \eqref{braid1}.
Let $h := (1 + T_i)(1 + T_{i+1})(1+T_i)$, and let $g := (1 + T_{i+1})(1 + T_i)(1 + T_{i+1})$.
Note the following expansion:  
  \begin{align*}
      hw
      &= 1 + 2T_i + T_i^2 + T_{i+1} + T_iT_{i+1} + T_{i+1}T_i + T_iT_{i+1}T_i\\
      &= 1 + (1+q)T_i + T_{i+1} + T_iT_{i+1} + T_{i+1}T_i + T_iT_{i+1}T_i.
    \end{align*}
    This equality, with $i$ and $i+1$ interchanged, holds for $g$.
    Hence we have
    \[
      (h-g)w = q(T_i - T_{i+1}) + T_iT_{i+1}T_i - T_{i+1}T_iT_{i+1}.
    \]
    Hence we may equivalently check that $(h-g)w = q(T_i - T_{i+1})$.
    In fact, $hw = q(1 + T_i)$ and $gw = q(1 + T_{i+1})$ by Figure \ref{braid1arc}, giving compatibility.
    \begin{figure}
\[
  \begin{gathered}\EmptyNAction{5}{2/1,3/2,2/3}\end{gathered}
  \longsquigrightarrow
  \begin{gathered}\EmptyNAction{5}{2/1}\end{gathered}
  \hspace{30pt}
  \begin{gathered}\EmptyNAction{5}{3/1,2/2,3/3}\end{gathered}
  \longsquigrightarrow
  \begin{gathered}\EmptyNAction{5}{3/1}\end{gathered}
\]
  \caption{
    The above give visual intuition for isotopies giving rise to equalities between $(1 + T_i)(1 + T_{i-1})(1 + T_i)w_j$ and $q(1 + T_i)$, and between $(1 + T_{i+1})(1 + T_i)(1 + T_{i+1})w_j$ and $q(1 + T_i)$.
  }
  \label{braid1arc}
\end{figure}

\begin{figure}
\[  
  \begin{gathered}\EmptyNAction{7}{2/1,5/2}\end{gathered}
  \longsquigrightarrow
  \begin{gathered}\EmptyNAction{7}{5/1,2/2}\end{gathered}
\]
  \caption{
    The above give visual intuition for isotopies giving rise to equalities between $(1 + T_i)(1+T_j)w_l$ and $(1 + T_j)(1 + T_i)w_l$.}
    \label{braid2arc}
\end{figure}


Lastly, we have the equation
\[
  (1 + T_i)(1 + T_j) - (1 + T_j)(1 + T_i) = T_iT_j - T_jT_i
\]
and hence we simply need to verify that $(1 + T_i)$ and $(1 + T_j)$ commute, as shown in Figure \ref{braid2arc}
 
\subsection{Compatibility for the fibonacci representations} 
\label{Fib Relations}
We verify the relations on the Fibonacci representation $V$.
Note that \eqref{braid2} follows easily from the ``local'' nature of $V$, and the others may be verified explicitly on strings of length 3 and 4.
By considering the coefficients in order of \eqref{Fib Action}, the quadratic relation \eqref{quadratic} gives the following polynomials in $q$:
\begin{equation}
  \begin{split}
    (\vara - q)(\vara + 1) &= 0,\\
    (\varb - q)(\varb + 1) &= 0,\\
    \varc\vard + \vard\vare &= (q-1)\vard,\\
    \varc^2 + \vard^2 &= (q -1)\varc + q,\\
    \vare^2 + \vard^2 &= (q -1)\vare + q
  \end{split}
\end{equation}
The first two of these are easily verified.
Since $\vard \neq 0$, the third is equivalently given by
\[
  (q - 1) = \varc + \vare = \tau(q\tau - 1 + q - \tau) = (\tau^2 + \tau)(q - 1)
\]
or that $\prn{\tau^2 + \tau - 1}(q-1) = 0$.
One may verify that \[\tau^2 + \tau - 1 = q^6 + 2q^5 + q^4 + q^3 + q^2 - 1 = (-1+q+q^2)\brk{5}_q = 0.\]

The fourth is given by the quadratic
\[
  \tau^2\brk{(q\tau-1)^2 - \tau(q+1)} = \tau(q-1)(q\tau - 1) + q
\]
or equivalently,
\[
  (\tau^2 + \tau - 1)\brk{q\prn{qt^2 + 1} + t} = 0.
\]

The fifth is similarly given by
\[
  (\tau^2 + \tau - 1)\brk{q\prn{qt + 1} + t^2} = 0. 
\]
All of these vanish for $e = 5$, giving compatibility with \eqref{quadratic}.

\vspace{7pt}
We now verify \eqref{braid1}.
We may order the basis for $V^4$ as follows:
\[
  \cbr{(\vp\vp\vp\vp),(\vs\vp\vp\vs),(\vp\vp\vp\vs),(\vs\vp\vp\vp),(\vs\vp\vs\vp),(\vp\vs\vp\vs),(\vp\vp\vs\vp),(\vp\vs\vp\vp)}.
\]
Then, in verifying the braid relation \eqref{braid1} in this order, we encounter the following quadratics (with tautologies and repetitions omitted):
\begin{align*}
    \vara\vare^2 + \varb\vard^2 &= \vara^2 \vare\\
    \vara\vard\vare + \varb\varc\vard &= \vara\varb\vard\\
    \varb\varc^2 + \vara\vard^2 &= \varb^2\varc\\
    \vara\varc^2 + \vard^2\vare &= \vara^2\varc\\
    \vard\vare^2 + \vara\varc\vard &= \vara\vard\vare
\end{align*}
Substituting in $\tau$ and dividing by $\delta$ whenever possible, these are equivalent to the vanishing of the following polynomials in $q$:
\begin{align*}
  -q (1 + q) (1 + q^2 + q^3) (2 + q + 3 q^2 + 2 q^3) \brk{5}_q &= 0\\
  (1 + 2 q + q^3 + q^4) \brk{5}_q &= 0\\
  (1+q)^2 (1+q^2+q^3) (1+3q^3 - q^4 + q^6)\brk{5}_q &= 0\\
  (1+q)^2 (1+q^2+q^3) (1+5q+5q^2+3q^3+3q^4+3q^5+q^6)\brk{5}_q &= 0\\
  (1+q) (1+q^2+q^3) (-1+2q+q^2+q^3+q^4)\brk{5}_q &= 0.
\end{align*}
Notably, each of these vanish when $e = 5$, giving compatibility with \eqref{braid1}.
  
\section{Restrictions to conjugate subalgebras}\label{Algebra}
Throughout the text, for some representation $V$, we refer to $\Res_{\SH(S_{l})}^{\SH(S_m)} V$ without specifying exactly which subalgebra $\SH(S_{l})$.
For instance, in section \ref{Fibonacci Section}, we explicitly state that the subrepresentations $V_{\vs\vp} \oplus V_{\vs\vs}$ and $V_{\vp\vp}$ are isomorphic because they both may be characterized by such a restriction.
We will verify that this is justified, using a more general fact about resctrictions to conjugate subalgebras.
\begin{proposition}
  Suppose $B,B'$ are subalgebras of a $k$-algebra $A$ with $B = uB'u^{-1}$ for some unit $u \in A^\times$, and let $V$ be a left $A$-module.
  Let $\phi:V \rightarrow V$ be the linear automorphism specified by $v \mapsto uv$.
  Then, the following commutes for any $b \in B$:
  \[
    \begin{tikzcd}
      V \arrow[r,"\phi"] \arrow[d,"b"] & V \arrow[d,"ubu^{-1}"]\\
      V \arrow[r,"\phi"] & V
    \end{tikzcd}
  \]
  Hence, through the identification of $B$ and $B'$ via conjugation by $u$, we have $\Res_{B}^A V \cong \Res_{B'}^A V$
\end{proposition}
\begin{proof}
  It suffices to note that $(ubu^{-1})uv = ubv$.
\end{proof}

\begin{corollary}
  Suppose $\SH',\SH''$ are two subalgebras of $\SH(S_m)$ generated by $l$ simple reflections and $V$ is a representation of $\SH$.
  Then, $\Res_{\SH'}^{\SH} V \cong \Res_{\SH''}^\SH V$.
\end{corollary}
\begin{proof}
  Let $\SH'$ and $\SH''$ be the subalgebras of $\SH(S_m)$ generated by the reflections $\cbr{T_{i_1},\dots,T_{i_l}}$ and $\cbr{T_{i_1},\dots,T_{i_{j-1}},T_{i_j + 1},T_{i_{j+1}},\dots,T_{i_l}}$ for $1 \leq i_1 < \dots < i_{j-1} < i_j + 1 < i_{j+1} < \dots < i_l \leq n$.
  It is sufficient to prove that $\SH'$ and $\SH''$ are conjugate;
  then transitivity gives conjugacy of any $S_l \subset S_m$, and the previous proposition gives isomorphisms of the representations.
 
  We will show that $\SH'' = T_{i_j}\SH'T_{i_j}^{-1}$.
  It suffices to show that $T_{i_j}T_wT_{i_j}^{-1} \in \SH''$ for $w$ a word generated by simple transpositions $s_{i_1},\dots,s_{i_{l}} \in S_m$.
  First, note that $l(w) < l(s_{i_j}w)$, implying $T_{i_j}T_w = T_{s_{i_j}w}$ by \cite[Leb.~1.12]{Mathas-book}.
  Further, by the same lemma, we have
  \begin{align*}
    T_{s_{i_j}w}T_{i_j}^{-1} 
    &= q^{-1}\prn{T_{s_{i_j}w}T_{i_j} + (1-q)T_{s_{i_j}w}}\\
    &= q^{-1}\prn{T_{qs_{i_j}ws_{i_j}} + (q-1)T_{s_{i_j}w} + (1-q)T_{s_{i_j}w}}\\
    &= T_{s_{i_j}ws_{i_j}}
  \end{align*}
  which is in $\SH''$.
\end{proof}

\begin{figure}
  \[
    \ENRGrid{13/7}{2/1,5/1,8/1,11/1,1/2,4/2,7/2,10/2}
  \]\[
    \ENRGrid{13/7}{3/1,7/1,11/1,2/2,6/2,10/2,1/3,5/3,9/3}
  \]
  \caption{
    Illustration of the modules $M_{2n + r}^r$ having sign submodules for $p = \infty$ and $e = 3,4$ respectively.
    The value $n + r + 1$ is filled in the squares, and modules having sign submodules are colored magenta.
    For $p = \infty$ and $2n + r \leq 14$, it has been verified, through a combination of theorems here and empirical computations, that $K_{2n + r}^r$ is nontrivial if and only if $e | n+r+1$ and $e > n$.
  }\label{Empirical Kernel}
\end{figure}


\section{Heuristics}\label{Empirics Section}
In this section we aim both to support Conjectures \ref{Preimage of K} and \ref{Sharp kernel} and to provide transition matrices wherever possible between our two graphical representations \cite{Github}.
The computations were made via a combination of the \texttt{Python} \cite{Python} and \texttt{Magma} \cite{Magma} languages.

\begin{remark}
  The data on composition series and transition matrices $M/K \rightarrow V$ were computed using our implementation \cite{Github} of the Hecke algebra in the \texttt{Magma} language.
  This stores the algebra via basis elements and structure constants, and hence the memory required for the structure grows with $(m!)^3$.
  This is prohibitively large when $m > 7$.
  An implementation of the Hecke algebra as a quotient of a free algebra is possible, but potentially difficult to work with via the Magma language.
\end{remark}

For $M$, the data throughout this section are specified with respect to the basis on $M_{2n + r}^r$ induced by the increasing lexicographic basis on $M_{2n + 2r}^0$ and the quotient $M_{2n + 2r}^r \twoheadrightarrow M_{2n + r}^r$ as defined in Appendix \ref{Explicit Definition}.
For $V$, the data are specified with respect to the basis on $V^m$ given by increasing lexicographic order $\vs < \vp$.

For the following data, set $e := 5$.
The following data define the isomorphisms \[\varphi_{2n + r}^r: M_{2n + r}^r / K_{2n + r}^r \xrightarrow{\sim} V_{s}^{2n + r},\] where $s = \vs\vs$ if $r \in \cbr{0,3}$ and $s = \vs\vp$ otherwise.
All of such computations use $q$ a primitive 5th root of unity in the algebraic extension of the Cyclotomic field $\QQ(\zeta_{10})$ by $\sqrt\tau$ \cite{Github}.
This data covers all cases $2n + r \leq 6$, and they include the cases $n = r = 2$ and $n = 1, r = 3$ where $K_{2n + r}^r \neq 0$.
\def\tho{q^{3/2} + 1}
\def\tht{\brk{4}_{q^{1/2}}}
\def\to{q^{1/2}\prn{q^{1/2} + 1}}
\begin{align*}
  \varphi_6^0, \varphi_5^1 &= \begin{bmatrix}
    0 & 0 & -\tho & 0 & 0\\
    0 & -\tho & \tht & 0 & 0\\
    0 & 0 & \tht & 0 & -\tho\\
    -\tht & \tht & \to & 0 & \tht\\
    \tht & 0 & \tht & -\tht & 0
  \end{bmatrix}
\end{align*}
\begin{align*}
  \varphi_6^2 &= \begin{bmatrix}
    0 & 0 & - q - 1 & 0 & 0 & 0 & 0 & 0\\
    0 & -q - 1 & -q^{1/2} & 0 & 0 & 0 & 0 & 0\\
    0 & q + 1 & 0 & q^{1/2} & 0 & 0 & 0 & 0\\
    0 & 0 & -q^{1/2} & 0 & 0 & 0 & -q^2 - 1 & 0\\
    q^{1/2} & -q^{1/2} & -\brk{3}_{q^{1/2}} & 0 & 0 & 0 & -q^{1/2} & 0\\
   -q^{1/2} & q^{1/2} & 0 & \brk{3}_{q^{1/2}} &0 & 0 & 0 & q^{1/2}\\
   -q^{1/2} & 0 & -q^{1/2} & 0 & q^{1/2} & 0 & 0 & 0\\
    q^{1/2} & 0 & 0 & 0 & -q^{1/2} & -\brk{3}_{q^{1/2}} & 0 & -q^{1/2}
  \end{bmatrix}
\end{align*}
\begin{align*}
  \varphi_5^3 &= \begin{bmatrix}
    -\to & \tht & q^{1/2}\\
    q^{1/2} & q^{3/2} & 0\\
    0 & -q - 1
  \end{bmatrix}  \hspace{30pt} 
  \varphi_4^0 = \begin{bmatrix}
      0 & -\tht\\
    -1 & 1
  \end{bmatrix}\\
  \varphi_3^1 &= \begin{bmatrix}
    0 & -\tht\\
    \to & -\to
  \end{bmatrix} \hspace{27pt}
  \varphi_4^2 = \begin{bmatrix}
    0 & \tht & 0\\
    1 & -1 & 0\\
    -1 & 0 & 1
  \end{bmatrix}
\end{align*}

We give in Figure \ref{Empirical Kernel} some data supporting a conjecture concerning sign subrepresentations of $M_{2n + r}^r$.
The computations to support this were done over $\CC$ with $q$ a primitive 5th root of unity \cite{Github}.

It is known that, for small $e$ and $2n + r$, each Specht module $S^{(n+r,n)}$ has a composition series of length at most 2 \cite[Appendix B]{Mathas-book}.
Heuristically, $M_{2n + r}^r$ satisfies this as well;
when $M_{2n + r}^r$ has such a composition series, denote the series by
\begin{equation}\label{U composition series}
  0 \subset U_{2n + r}^r \subset M_{2n + r}.
\end{equation}
In the following data, we specify the inclusion maps map $\iota_{e,2n+r}^r:U_{2n + r}^r \hookrightarrow M_{2n + r}^r $, which conjecturally illustrates the inclusion of the first composition factor of $S^{(n+r,n)}$ into $S^{(n+r,n)}$ for all $2n + r \leq 7$.
We begin by defining the following constants.
\def\qqo{\varkappa}
\def\tto{\eth}
\def\vtv{\varrho}
\def\fft{\wp}
\def\fto{\vartheta}
\setcounter{MaxMatrixCols}{20}
\begin{align*}
  \qqo &:= q^{1/2}\prn{q - 1},\\
  \tto &:= q^{3/2} + q + 1,\\
  \vtv &:= q^{1/2}\prn{q - 2},\\
  \fft &:= q^{5/2} + q^2 + q^{3/2} + q + 1,\\ 
  \fto &:= q^{5/2} + q + 1.
\end{align*}
Using these constants, we give the following data.
We omit the cases where $M_{2n + r}$ is irreducible;
this includes all cases where $e > 7$, as we then have $e > 2n + r \geq n + r + 1$.
\begin{align*} 
  \iota_{3,3}^1 = \iota_{3,4}^0 &=  
\begin{bmatrix}
  1 & 1
\end{bmatrix}^\intercal\\
\iota_{3,5}^1 = \iota_{3,6}^0&=  
\begin{bmatrix}
  1 & 0 & 0 & 0 & 0\\
  0 & 0 & 1 & 1 & 0\\
  0 & 1 & 0 & 1 & 0\\
  0 & 0 & 0 & 1 & 1
\end{bmatrix}^\intercal\\
\iota_{3,6}^4&=  
\begin{bmatrix}
  -1 & -1 & 0 & 1 & 1
\end{bmatrix}^\intercal\\
\iota_{3,7}^3&=  
\begin{bmatrix}
 1 & 0 &-1 &-1 & 1 & 0 & -1 & -1 & 0 & 0 & 0 & 0 & 1 & 1
\end{bmatrix}^\intercal
\end{align*}
\begin{align*}
\iota_{3,7}^1&=  
\begin{bmatrix}
 1 & 0 & 0 & 0 & 0 & 0 & 0 & 0 & 0 & 0 & 0 & 0 & 0 &-1\\
 0 & 1 & 0 & 0 & 0 & 0 & 0 & 0 & 0 & 0 & 0 & 0 & 0 & 1\\
 0 & 0 & 1 & 0 & 0 & 0 & 0 & 0 & 0 & 0 & 0 & 0 & 0 & 1\\
 0 & 0 & 0 & 1 & 0 & 0 & 0 & 0 & 0 & 0 & 0 & 0 & 0 &-1\\
 0 & 0 & 0 & 0 & 1 & 0 & 0 & 0 & 0 & 0 & 0 & 0 & 0 & 1\\
 0 & 0 & 0 & 0 & 0 & 1 & 0 & 0 & 0 & 0 & 0 & 0 & 0 & 1\\
 0 & 0 & 0 & 0 & 0 & 0 & 1 & 0 & 0 & 0 & 0 & 0 & 0 &-1\\
 0 & 0 & 0 & 0 & 0 & 0 & 0 & 1 & 0 & 0 & 0 & 0 & 0 &-1\\
 0 & 0 & 0 & 0 & 0 & 0 & 0 & 0 & 1 & 0 & 0 & 0 & 0 & 1\\
 0 & 0 & 0 & 0 & 0 & 0 & 0 & 0 & 0 & 1 & 0 & 0 & 0 & 1\\
 0 & 0 & 0 & 0 & 0 & 0 & 0 & 0 & 0 & 0 & 1 & 0 & 0 &-1\\
 0 & 0 & 0 & 0 & 0 & 0 & 0 & 0 & 0 & 0 & 0 & 1 & 0 &-1\\
 0 & 0 & 0 & 0 & 0 & 0 & 0 & 0 & 0 & 0 & 0 & 0 & 1 & 1
\end{bmatrix}^\intercal
\end{align*}
\begin{align*}
  \iota_{4,4}^2&=  
\begin{bmatrix}
  1 & \qqo & 1
\end{bmatrix}^\intercal\\
\iota_{4,4}^1&=  
\begin{bmatrix}
  1 & \frac{1}{2}\qqo & \frac{1}{2}\qqo & 1 & \frac{1}{2}\qqo
\end{bmatrix}^\intercal\\
\iota_{4,6}^2 &=
\begin{bmatrix}
  1& 0& 0& 0& 0& 0& \qqo & 1& 0\\
  0& 1& 0& 0& 0& -1& -1& -\qqo &0\\
  0& 0& 1& 0& 0& \qqo& 0 &1 &0\\
  0& 0& 0& 1& 0& 0& -1& -\qqo &-1\\
  0& 0& 0& 0& 1& \frac{1}{2}\qqo& \frac{1}{2}\qqo &1 &\frac{1}{2}\qqo
\end{bmatrix}^\intercal
\end{align*}
\begin{align*}
\iota_{4,6}^0 &=
\begin{bmatrix}
  \qqo & 1 & 1 & \qqo & 1
\end{bmatrix}^\intercal\\
\iota_{4,7}^1 &=
\begin{bmatrix}
1& 0& 0& 0& \frac{1}{2}\qqo& 0& -\frac{1}{2}& -\frac{1}{2}& 0& 0& 1& 1/2& \qqo& \frac{1}{2}\\
0& 1& 0& 0& -1& 0& \frac{1}{2}\qqo& \frac{1}{2}\qqo& 1& 0& \qqo& \frac{1}{2}\qqo& -1 & \frac{1}{2}\qqo\\
0& 0& 1& 0& -1& 0& 0& \qqo& 1& 0& -\qqo& 0& -1& 0\\
0& 0& 0& 1& \qqo& 0& 0& -1& -\qqo& 0& 1& -1& 0& 0\\
0& 0& 0& 0& 0& 1& \frac{1}{2}\qqo& \frac{1}{2}\qqo& 0& 0& 0& \frac{1}{2}\qqo& -1& \frac{1}{2}\qqo\\
0& 0& 0& 0& 0& 0& 0& 0& 0& 1& -\frac{1}{2}\qqo& \frac{1}{2}\qqo& 1& \frac{1}{2}\qqo
\end{bmatrix}^\intercal
\end{align*}
\begin{align*}
  \iota_{5,5}^3 &=
\begin{bmatrix}
  1 & \tto & \tto & 1
\end{bmatrix}^\intercal\\
\iota_{5,6}^2 &=
\begin{bmatrix}
  \tto& \tto& 1& 1& \tto+1& \tto & \tto & \tto & 1
\end{bmatrix}^\intercal\\
\iota_{5,7}^3 &=
\begin{bmatrix}
1& \tto& \tto& 1& 0& 0& 0& 0& 0& 0& 0& 0& 0& 0\\
0& 0& 0& 0& 1& \tto& \tto& 1& 0& 0& 0& 0& 0& 0\\
1& 0& 0& 0& 0& 0& 0& 0& \tto& \tto& 1& 0& 0& 0\\
\tto& \tto& 1& 0& 1& \tto+1& \tto& 0& \tto& \tto& 0& 1& 0& 0\\
-\tto-1& -\tto& -\tto& 0& -\tto& -\tto-1& -\tto-1& 0& -\tto-1& -\tto-1& 0& 0& 1& 0\\
0& 0& 1& 0& 0& 0& \tto& 0& 0& \tto& 0& 0& 0& 1
\end{bmatrix}^\intercal\\
\end{align*}
\begin{align*}
\iota_{5,7}^1 &=
\begin{bmatrix}
1& \qqo& \qqo& \qqo& -\tto& \qqo& -\tto& \qqo& -\tto& 1& \qqo& \qqo& \qqo& -\tto
\end{bmatrix}^\intercal\\
\iota_{6,6}^4 &=
\begin{bmatrix}
  1 & \vtv & 2 & \vtv & 1 
\end{bmatrix}^\intercal\\
\iota_{6,7}^3 &=
\begin{bmatrix}
  \vtv & 2 & \vtv & 1 & 1 & 2\vtv & 3 & \vtv & 2 & 2\vtv & 2 & 2 & \vtv & 1
\end{bmatrix}^\intercal\\
\iota_{6,7}^5 &=
\begin{bmatrix}
  1 & \fft & \fto & \fto & \fft & 1
\end{bmatrix}^\intercal
\end{align*}

\begin{remark}
  From this data, we may explicitly characterize the subrepresentation $\ima \iota_{e,n+r}^r \subset M_{2n + r}^r$;
  this in turn explicitly characterizes the quotient map $M_{2n + r}^r \twoheadrightarrow M_{2n + r}^r/\ima \iota_{e,n+r}^r$, fully determining the data of the composition series specified by the short exact sequence 
  \[
    0 \longrightarrow \ima \iota_{e,n+r}^r \longrightarrow M_{2n + r}^r \longrightarrow M_{2n + r}/\ima\iota_{e,n+r}^r \longrightarrow 0,
  \]
  i.e. \ref{U composition series}.
\end{remark}

A potential extension of this work is to implement a standard basis for the Specht module (e.g. the Murphy basis \cite{Murphy1,Murphy2}) in order to draw explicit isomorphisms between the Specht module and crossingless matchings representation, rather than the implicit isomorphisms proven in \ref{Correspondence Theorem}.

\bibliographystyle{plain} 
 \bibliography{../RepBib}
\end{document}
